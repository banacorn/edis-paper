\documentclass{llncs}

\usepackage{amsmath}
%\usepackage{amsfonts}
%\usepackage{amsthm}
\usepackage{url}
\usepackage{mathptmx}
\usepackage{doubleequals}
\usepackage{makeidx}  % allows for indexgeneration

\input{Preamble}
\newcommand{\hide}[1]{}
\newcommand{\todo}[1]{\noindent\textcolor{blue}{\ifmmode \text{[#1]}\else [#1] \fi}}

\newcommand{\Redis}{{\sf Redis}}
\newcommand{\Hedis}{{\sf Hedis}}
\newcommand{\Popcorn}{{\sf Popcorn}}


\begin{document}

\title{Preventing Runtime Errors of Redis at Compile Time}
\author{Ting-Yan Lai \inst{1}\inst{2}\and Tyng-Ruey Chuang \inst{1}
  \and Shin-Cheng Mu \inst{1}}
\institute{Institute of Information Science, Academia Sinica, Taiwan
\and Institute of Computer Science and Engineering,
National Chiao Tung University, Taiwan}

\maketitle

\begin{abstract}
Programmers often interact with database systems by sending queries through libraries or packages in some programming languages. People, however, make mistakes.
While some of the syntactic and semantic errors can be prevented by the
 language at compile time, or caught by the package at runtime, most semantic
 errors are just being ignored, causing problems at the database system.

In this paper, we demonstrate how to prevent those runtime errors at compile
 time, thus allowing users to write more reliable database queries without
 runtime overhead, by exploiting type-level programming techniques such as
 indexed monad, type-level literals and closed type families in Haskell.

The database system and the package we are targeting are \emph{Redis} and
 \emph{Hedis} respectively, and our implementation is available as \emph{Popcorn}
 on \emph{Hackage}.
\end{abstract}

%% ODER: format ==         = "\mathrel{==}"
%% ODER: format /=         = "\neq "
%
%
\makeatletter
\@ifundefined{lhs2tex.lhs2tex.sty.read}%
  {\@namedef{lhs2tex.lhs2tex.sty.read}{}%
   \newcommand\SkipToFmtEnd{}%
   \newcommand\EndFmtInput{}%
   \long\def\SkipToFmtEnd#1\EndFmtInput{}%
  }\SkipToFmtEnd

\newcommand\ReadOnlyOnce[1]{\@ifundefined{#1}{\@namedef{#1}{}}\SkipToFmtEnd}
\usepackage{amstext}
\usepackage{amssymb}
\usepackage{stmaryrd}
\DeclareFontFamily{OT1}{cmtex}{}
\DeclareFontShape{OT1}{cmtex}{m}{n}
  {<5><6><7><8>cmtex8
   <9>cmtex9
   <10><10.95><12><14.4><17.28><20.74><24.88>cmtex10}{}
\DeclareFontShape{OT1}{cmtex}{m}{it}
  {<-> ssub * cmtt/m/it}{}
\newcommand{\texfamily}{\fontfamily{cmtex}\selectfont}
\DeclareFontShape{OT1}{cmtt}{bx}{n}
  {<5><6><7><8>cmtt8
   <9>cmbtt9
   <10><10.95><12><14.4><17.28><20.74><24.88>cmbtt10}{}
\DeclareFontShape{OT1}{cmtex}{bx}{n}
  {<-> ssub * cmtt/bx/n}{}
\newcommand{\tex}[1]{\text{\texfamily#1}}	% NEU

\newcommand{\Sp}{\hskip.33334em\relax}


\newcommand{\Conid}[1]{\mathit{#1}}
\newcommand{\Varid}[1]{\mathit{#1}}
\newcommand{\anonymous}{\kern0.06em \vbox{\hrule\@width.5em}}
\newcommand{\plus}{\mathbin{+\!\!\!+}}
\newcommand{\bind}{\mathbin{>\!\!\!>\mkern-6.7mu=}}
\newcommand{\rbind}{\mathbin{=\mkern-6.7mu<\!\!\!<}}% suggested by Neil Mitchell
\newcommand{\sequ}{\mathbin{>\!\!\!>}}
\renewcommand{\leq}{\leqslant}
\renewcommand{\geq}{\geqslant}
\usepackage{polytable}

%mathindent has to be defined
\@ifundefined{mathindent}%
  {\newdimen\mathindent\mathindent\leftmargini}%
  {}%

\def\resethooks{%
  \global\let\SaveRestoreHook\empty
  \global\let\ColumnHook\empty}
\newcommand*{\savecolumns}[1][default]%
  {\g@addto@macro\SaveRestoreHook{\savecolumns[#1]}}
\newcommand*{\restorecolumns}[1][default]%
  {\g@addto@macro\SaveRestoreHook{\restorecolumns[#1]}}
\newcommand*{\aligncolumn}[2]%
  {\g@addto@macro\ColumnHook{\column{#1}{#2}}}

\resethooks

\newcommand{\onelinecommentchars}{\quad-{}- }
\newcommand{\commentbeginchars}{\enskip\{-}
\newcommand{\commentendchars}{-\}\enskip}

\newcommand{\visiblecomments}{%
  \let\onelinecomment=\onelinecommentchars
  \let\commentbegin=\commentbeginchars
  \let\commentend=\commentendchars}

\newcommand{\invisiblecomments}{%
  \let\onelinecomment=\empty
  \let\commentbegin=\empty
  \let\commentend=\empty}

\visiblecomments

\newlength{\blanklineskip}
\setlength{\blanklineskip}{0.66084ex}

\newcommand{\hsindent}[1]{\quad}% default is fixed indentation
\let\hspre\empty
\let\hspost\empty
\newcommand{\NB}{\textbf{NB}}
\newcommand{\Todo}[1]{$\langle$\textbf{To do:}~#1$\rangle$}

\EndFmtInput
\makeatother
%
%
%
%
%
%
% This package provides two environments suitable to take the place
% of hscode, called "plainhscode" and "arrayhscode". 
%
% The plain environment surrounds each code block by vertical space,
% and it uses \abovedisplayskip and \belowdisplayskip to get spacing
% similar to formulas. Note that if these dimensions are changed,
% the spacing around displayed math formulas changes as well.
% All code is indented using \leftskip.
%
% Changed 19.08.2004 to reflect changes in colorcode. Should work with
% CodeGroup.sty.
%
\ReadOnlyOnce{polycode.fmt}%
\makeatletter

\newcommand{\hsnewpar}[1]%
  {{\parskip=0pt\parindent=0pt\par\vskip #1\noindent}}

% can be used, for instance, to redefine the code size, by setting the
% command to \small or something alike
\newcommand{\hscodestyle}{}

% The command \sethscode can be used to switch the code formatting
% behaviour by mapping the hscode environment in the subst directive
% to a new LaTeX environment.

\newcommand{\sethscode}[1]%
  {\expandafter\let\expandafter\hscode\csname #1\endcsname
   \expandafter\let\expandafter\endhscode\csname end#1\endcsname}

% "compatibility" mode restores the non-polycode.fmt layout.

\newenvironment{compathscode}%
  {\par\noindent
   \advance\leftskip\mathindent
   \hscodestyle
   \let\\=\@normalcr
   \let\hspre\(\let\hspost\)%
   \pboxed}%
  {\endpboxed\)%
   \par\noindent
   \ignorespacesafterend}

\newcommand{\compaths}{\sethscode{compathscode}}

% "plain" mode is the proposed default.
% It should now work with \centering.
% This required some changes. The old version
% is still available for reference as oldplainhscode.

\newenvironment{plainhscode}%
  {\hsnewpar\abovedisplayskip
   \advance\leftskip\mathindent
   \hscodestyle
   \let\hspre\(\let\hspost\)%
   \pboxed}%
  {\endpboxed%
   \hsnewpar\belowdisplayskip
   \ignorespacesafterend}

\newenvironment{oldplainhscode}%
  {\hsnewpar\abovedisplayskip
   \advance\leftskip\mathindent
   \hscodestyle
   \let\\=\@normalcr
   \(\pboxed}%
  {\endpboxed\)%
   \hsnewpar\belowdisplayskip
   \ignorespacesafterend}

% Here, we make plainhscode the default environment.

\newcommand{\plainhs}{\sethscode{plainhscode}}
\newcommand{\oldplainhs}{\sethscode{oldplainhscode}}
\plainhs

% The arrayhscode is like plain, but makes use of polytable's
% parray environment which disallows page breaks in code blocks.

\newenvironment{arrayhscode}%
  {\hsnewpar\abovedisplayskip
   \advance\leftskip\mathindent
   \hscodestyle
   \let\\=\@normalcr
   \(\parray}%
  {\endparray\)%
   \hsnewpar\belowdisplayskip
   \ignorespacesafterend}

\newcommand{\arrayhs}{\sethscode{arrayhscode}}

% The mathhscode environment also makes use of polytable's parray 
% environment. It is supposed to be used only inside math mode 
% (I used it to typeset the type rules in my thesis).

\newenvironment{mathhscode}%
  {\parray}{\endparray}

\newcommand{\mathhs}{\sethscode{mathhscode}}

% texths is similar to mathhs, but works in text mode.

\newenvironment{texthscode}%
  {\(\parray}{\endparray\)}

\newcommand{\texths}{\sethscode{texthscode}}

% The framed environment places code in a framed box.

\def\codeframewidth{\arrayrulewidth}
\RequirePackage{calc}

\newenvironment{framedhscode}%
  {\parskip=\abovedisplayskip\par\noindent
   \hscodestyle
   \arrayrulewidth=\codeframewidth
   \tabular{@{}|p{\linewidth-2\arraycolsep-2\arrayrulewidth-2pt}|@{}}%
   \hline\framedhslinecorrect\\{-1.5ex}%
   \let\endoflinesave=\\
   \let\\=\@normalcr
   \(\pboxed}%
  {\endpboxed\)%
   \framedhslinecorrect\endoflinesave{.5ex}\hline
   \endtabular
   \parskip=\belowdisplayskip\par\noindent
   \ignorespacesafterend}

\newcommand{\framedhslinecorrect}[2]%
  {#1[#2]}

\newcommand{\framedhs}{\sethscode{framedhscode}}

% The inlinehscode environment is an experimental environment
% that can be used to typeset displayed code inline.

\newenvironment{inlinehscode}%
  {\(\def\column##1##2{}%
   \let\>\undefined\let\<\undefined\let\\\undefined
   \newcommand\>[1][]{}\newcommand\<[1][]{}\newcommand\\[1][]{}%
   \def\fromto##1##2##3{##3}%
   \def\nextline{}}{\) }%

\newcommand{\inlinehs}{\sethscode{inlinehscode}}

% The joincode environment is a separate environment that
% can be used to surround and thereby connect multiple code
% blocks.

\newenvironment{joincode}%
  {\let\orighscode=\hscode
   \let\origendhscode=\endhscode
   \def\endhscode{\def\hscode{\endgroup\def\@currenvir{hscode}\\}\begingroup}
   %\let\SaveRestoreHook=\empty
   %\let\ColumnHook=\empty
   %\let\resethooks=\empty
   \orighscode\def\hscode{\endgroup\def\@currenvir{hscode}}}%
  {\origendhscode
   \global\let\hscode=\orighscode
   \global\let\endhscode=\origendhscode}%

\makeatother
\EndFmtInput
%

\ReadOnlyOnce{Formatting.fmt}%
\makeatletter

\let\Varid\mathit
\let\Conid\mathsf

\def\commentbegin{\quad\{\ }
\def\commentend{\}}

\newcommand{\ty}[1]{\Conid{#1}}
\newcommand{\con}[1]{\Conid{#1}}
\newcommand{\id}[1]{\Varid{#1}}
\newcommand{\cl}[1]{\Varid{#1}}
\newcommand{\opsym}[1]{\mathrel{#1}}

\newcommand\Keyword[1]{\textbf{\textsf{#1}}}
\newcommand\Hide{\mathbin{\downarrow}}
\newcommand\Reveal{\mathbin{\uparrow}}


%% Paper-specific keywords


\makeatother
\EndFmtInput

\section{Introduction}
\label{sec:introduction}

\Redis{}\footnote{\url{https://redis.io}} is an open source, in-memory data structure store, often used as database, cache and message broker. A \Redis{} datatype can be think of as a set of key-value pairs, where each value is associated with a binary-safe string key to identify and manipulate with.
\Redis{} allows values of various types, including strings, hashes, lists, and sets, etc, to be stored, and provides a collection of of atomic \emph{commands} to manipulate these values.

For an example, consider the following sequence of commands, entered through the interactive interface of \Redis{}. The keys \texttt{some-set} and \texttt{another-set}
are both associated to a set. The two call to command \texttt{SADD} respectively
adds three and two values to the two sets, before \texttt{SINTER} takes their intersection:
\begin{tabbing}\tt
~redis\char62{}~SADD~some\char45{}set~a~b~c\\
\tt ~\char40{}integer\char41{}~3\\
\tt ~redis\char62{}~SADD~another\char45{}set~a~b\\
\tt ~\char40{}integer\char41{}~2\\
\tt ~redis\char62{}~SINTER~some\char45{}set~another\char45{}set\\
\tt ~1\char41{}~\char34{}a\char34{}\\
\tt ~2\char41{}~\char34{}b\char34{}
\end{tabbing}

Note that the keys \texttt{some-set} and \texttt{another-set}, if not existing before the call to \texttt{SADD}, are created on site. The call to
\texttt{SADD} returns the size of the set after completion of the command.

Many third party libraries provide interfaces that allow general purpose programming languages to access \Redis{} through its TCP protocol.
For Haskell, the most popular library is \Hedis{}\footnote{\url{https://hackage.haskell.org/package/hedis}}.
The following program implements the previous example:
\begin{hscode}\SaveRestoreHook
\column{B}{@{}>{\hspre}l<{\hspost}@{}}%
\column{5}{@{}>{\hspre}l<{\hspost}@{}}%
\column{E}{@{}>{\hspre}l<{\hspost}@{}}%
\>[B]{}\Varid{program}\mathbin{::}\Conid{Redis}\;(\Conid{Either}\;\Conid{Reply}\;[\mskip1.5mu \Conid{ByteString}\mskip1.5mu]){}\<[E]%
\\
\>[B]{}\Varid{program}\mathrel{=}\mathbf{do}{}\<[E]%
\\
\>[B]{}\hsindent{5}{}\<[5]%
\>[5]{}\Varid{sadd}\;\text{\tt \char34 some-set\char34}\;[\mskip1.5mu \text{\tt \char34 a\char34},\text{\tt \char34 b\char34}\mskip1.5mu]{}\<[E]%
\\
\>[B]{}\hsindent{5}{}\<[5]%
\>[5]{}\Varid{sadd}\;\text{\tt \char34 another-set\char34}\;[\mskip1.5mu \text{\tt \char34 a\char34},\text{\tt \char34 b\char34},\text{\tt \char34 c\char34}\mskip1.5mu]{}\<[E]%
\\
\>[B]{}\hsindent{5}{}\<[5]%
\>[5]{}\Varid{sinter}\;[\mskip1.5mu \text{\tt \char34 some-set\char34},\text{\tt \char34 another-set\char34}\mskip1.5mu]~~.{}\<[E]%
\ColumnHook
\end{hscode}\resethooks
The function \ensuremath{\Varid{sadd}\mathbin{::}\Conid{ByteString}\to [\mskip1.5mu \Conid{ByteString}\mskip1.5mu]\to \Conid{Redis}\;(\Conid{Either}\;\Conid{Reply}\;\Conid{Integer})} takes a key and a list of values as arguments, and returns
an \ensuremath{\Conid{Integer}} on success, or returns a \ensuremath{\Conid{Reply}}, a low-level representation of
replies from the Redis server, in case of failures. All wrapped in the monad
\ensuremath{\Conid{Redis}}, the context of command execution.\footnotemark

Note that keys and values, being nothing but binary strings in Redis, are
represented using Haskell \ensuremath{\Conid{ByteString}}. Values of other types must be encoded
as \ensuremath{\Conid{ByteString}}s before being written to the database, and decoded after being
read back.

\footnotetext{\Hedis{} provides another kind of context, \ensuremath{\Conid{RedisTx}}, for \emph{transactions}, united with \ensuremath{\Conid{Redis}} under the class \ensuremath{\Conid{RedisCtx}}. We
demonstrate only \ensuremath{\Conid{Redis}} in this paper.}

%\paragraph{Motivation}
%All binary strings are equal, but some binary strings are more equal than others.
%While everything in \Redis{} is essentially a binary string, these strings
%are treated differently.
\Redis{} supports many different kind of data
structures, such as strings, hashes, lists, etc. While they are all encoded as
binary strings before being written to the database, most commands only works
with data of certain types. In the following example, the key
\texttt{some-string} is associated to string \texttt{foo} --- the command
\texttt{SET} always associates a key to a string. The subsequent call to \texttt{SADD}, which adds a value to a set, thus causes a runtime error.
\begin{tabbing}\tt
~redis\char62{}~SET~some\char45{}string~foo\\
\tt ~OK\\
\tt ~redis\char62{}~SADD~some\char45{}string~bar\\
\tt ~\char40{}error\char41{}~WRONGTYPE~Operation~against~a~key~holding~the~wrong\\
\tt ~kind~of~value
\end{tabbing}

%\paragraph{Example 2} Even worse, not all strings are equal!
% The call \texttt{INCR some-string} parses the string associated with key
% \texttt{some-string} to an integer, increments it by one, and store it back as
% a string. If the string can not be parse as an integer, a runtime error
% is raised.
% \begin{verbatim}
% redis> SET some-string foo
% OK
% redis> INCR some-string
% (error) ERR value is not an integer or out of range
% \end{verbatim}

Being a simple wrapper on top of the TCP protocol of \Redis{}, \Hedis{}
inherits the problem. Executing following program yields the same error
wrapped in Haskell: \ensuremath{\Conid{Left}\;(\Conid{Error}} \texttt{"WRONGTYPE Operation against a
key holding the wrong kind of value"}\ensuremath{)}.
\begin{hscode}\SaveRestoreHook
\column{B}{@{}>{\hspre}l<{\hspost}@{}}%
\column{5}{@{}>{\hspre}l<{\hspost}@{}}%
\column{E}{@{}>{\hspre}l<{\hspost}@{}}%
\>[B]{}\Varid{program}\mathbin{::}\Conid{Redis}\;(\Conid{Either}\;\Conid{Reply}\;\Conid{Integer}){}\<[E]%
\\
\>[B]{}\Varid{program}\mathrel{=}\mathbf{do}{}\<[E]%
\\
\>[B]{}\hsindent{5}{}\<[5]%
\>[5]{}\Varid{set}\;\text{\tt \char34 some-string\char34}\;\text{\tt \char34 foo\char34}{}\<[E]%
\\
\>[B]{}\hsindent{5}{}\<[5]%
\>[5]{}\Varid{sadd}\;\text{\tt \char34 some-string\char34}\;[\mskip1.5mu \text{\tt \char34 a\char34}\mskip1.5mu]~~.{}\<[E]%
\ColumnHook
\end{hscode}\resethooks

% \paragraph{The Cause} Every key is associated with a value, and every value has
% its own type. But most commands in \Redis{} only work with a certain type of
%  value. When a command is used on a wrong type of key, a runtime error occurs.
%  The problems illustrated above arise from the absence of type checking, with
%  respects to \textbf{the type of a value that associates with a key}.
%  These problems could have been avoided, if we could know the type every key
%  associates with in advance, and prevent programs with invalid commands from
%  executing.
%
% \paragraph{Hedis as an embedded DSL}
% Haskell makes it easy to build and use {\em domain specific embedded languages} (DSELs), and \Hedis{} can be regarded as one of them. What makes \Hedis{} peculiar is that,
%  it has \emph{variable bindings} (between keys and values), but with very
%  little or no semantic checking, neither dynamically nor statically.
%
Such a programming model is certainly very error-prone. Working within Haskell,
a host language with a strong typing system, one naturally wishes to build a
a domain-specific embedded language (DSEL) that exploits the rich type system
of Haskell to ensure absence of \Redis{} type errors.

This paper discusses the techniques we used and experiences we learned from building such a language, nicknamed \Popcorn{}. We constructed an {\em indexed
monad}, on top of the monad \ensuremath{\Conid{Redis}}, which is indexed by a dictionary that
maintains the set of currently defined keys and their types. To represent
the dictionary, we need to encode variable binds with {\em type-level} lists
and strings. To summarize our contributions:
\begin{itemize}
\item We present \Popcorn{}, a statically typed domain-specific language embedded in Haskell and built on \Hedis{}.
% also makes Redis polymorphic by automatically converting back and forth from values of arbitrary types and boring ByteStrings.
%
\item We demonstrate how to model variable bindings of an embedded DSL using
 language extensions including type-level literals and data kinds.
%
\item We provide (yet another) an example of encoding effects and constraints of
in types, with indexed monad~\cite{indexedmonad}, closed type-families~\cite{closedtypefamilies} and constraints kinds~\cite{constraintskinds}.
\end{itemize}
\todo{Phrase this better.}

%% ODER: format ==         = "\mathrel{==}"
%% ODER: format /=         = "\neq "
%
%
\makeatletter
\@ifundefined{lhs2tex.lhs2tex.sty.read}%
  {\@namedef{lhs2tex.lhs2tex.sty.read}{}%
   \newcommand\SkipToFmtEnd{}%
   \newcommand\EndFmtInput{}%
   \long\def\SkipToFmtEnd#1\EndFmtInput{}%
  }\SkipToFmtEnd

\newcommand\ReadOnlyOnce[1]{\@ifundefined{#1}{\@namedef{#1}{}}\SkipToFmtEnd}
\usepackage{amstext}
\usepackage{amssymb}
\usepackage{stmaryrd}
\DeclareFontFamily{OT1}{cmtex}{}
\DeclareFontShape{OT1}{cmtex}{m}{n}
  {<5><6><7><8>cmtex8
   <9>cmtex9
   <10><10.95><12><14.4><17.28><20.74><24.88>cmtex10}{}
\DeclareFontShape{OT1}{cmtex}{m}{it}
  {<-> ssub * cmtt/m/it}{}
\newcommand{\texfamily}{\fontfamily{cmtex}\selectfont}
\DeclareFontShape{OT1}{cmtt}{bx}{n}
  {<5><6><7><8>cmtt8
   <9>cmbtt9
   <10><10.95><12><14.4><17.28><20.74><24.88>cmbtt10}{}
\DeclareFontShape{OT1}{cmtex}{bx}{n}
  {<-> ssub * cmtt/bx/n}{}
\newcommand{\tex}[1]{\text{\texfamily#1}}	% NEU

\newcommand{\Sp}{\hskip.33334em\relax}


\newcommand{\Conid}[1]{\mathit{#1}}
\newcommand{\Varid}[1]{\mathit{#1}}
\newcommand{\anonymous}{\kern0.06em \vbox{\hrule\@width.5em}}
\newcommand{\plus}{\mathbin{+\!\!\!+}}
\newcommand{\bind}{\mathbin{>\!\!\!>\mkern-6.7mu=}}
\newcommand{\rbind}{\mathbin{=\mkern-6.7mu<\!\!\!<}}% suggested by Neil Mitchell
\newcommand{\sequ}{\mathbin{>\!\!\!>}}
\renewcommand{\leq}{\leqslant}
\renewcommand{\geq}{\geqslant}
\usepackage{polytable}

%mathindent has to be defined
\@ifundefined{mathindent}%
  {\newdimen\mathindent\mathindent\leftmargini}%
  {}%

\def\resethooks{%
  \global\let\SaveRestoreHook\empty
  \global\let\ColumnHook\empty}
\newcommand*{\savecolumns}[1][default]%
  {\g@addto@macro\SaveRestoreHook{\savecolumns[#1]}}
\newcommand*{\restorecolumns}[1][default]%
  {\g@addto@macro\SaveRestoreHook{\restorecolumns[#1]}}
\newcommand*{\aligncolumn}[2]%
  {\g@addto@macro\ColumnHook{\column{#1}{#2}}}

\resethooks

\newcommand{\onelinecommentchars}{\quad-{}- }
\newcommand{\commentbeginchars}{\enskip\{-}
\newcommand{\commentendchars}{-\}\enskip}

\newcommand{\visiblecomments}{%
  \let\onelinecomment=\onelinecommentchars
  \let\commentbegin=\commentbeginchars
  \let\commentend=\commentendchars}

\newcommand{\invisiblecomments}{%
  \let\onelinecomment=\empty
  \let\commentbegin=\empty
  \let\commentend=\empty}

\visiblecomments

\newlength{\blanklineskip}
\setlength{\blanklineskip}{0.66084ex}

\newcommand{\hsindent}[1]{\quad}% default is fixed indentation
\let\hspre\empty
\let\hspost\empty
\newcommand{\NB}{\textbf{NB}}
\newcommand{\Todo}[1]{$\langle$\textbf{To do:}~#1$\rangle$}

\EndFmtInput
\makeatother
%
%
%
%
%
%
% This package provides two environments suitable to take the place
% of hscode, called "plainhscode" and "arrayhscode". 
%
% The plain environment surrounds each code block by vertical space,
% and it uses \abovedisplayskip and \belowdisplayskip to get spacing
% similar to formulas. Note that if these dimensions are changed,
% the spacing around displayed math formulas changes as well.
% All code is indented using \leftskip.
%
% Changed 19.08.2004 to reflect changes in colorcode. Should work with
% CodeGroup.sty.
%
\ReadOnlyOnce{polycode.fmt}%
\makeatletter

\newcommand{\hsnewpar}[1]%
  {{\parskip=0pt\parindent=0pt\par\vskip #1\noindent}}

% can be used, for instance, to redefine the code size, by setting the
% command to \small or something alike
\newcommand{\hscodestyle}{}

% The command \sethscode can be used to switch the code formatting
% behaviour by mapping the hscode environment in the subst directive
% to a new LaTeX environment.

\newcommand{\sethscode}[1]%
  {\expandafter\let\expandafter\hscode\csname #1\endcsname
   \expandafter\let\expandafter\endhscode\csname end#1\endcsname}

% "compatibility" mode restores the non-polycode.fmt layout.

\newenvironment{compathscode}%
  {\par\noindent
   \advance\leftskip\mathindent
   \hscodestyle
   \let\\=\@normalcr
   \let\hspre\(\let\hspost\)%
   \pboxed}%
  {\endpboxed\)%
   \par\noindent
   \ignorespacesafterend}

\newcommand{\compaths}{\sethscode{compathscode}}

% "plain" mode is the proposed default.
% It should now work with \centering.
% This required some changes. The old version
% is still available for reference as oldplainhscode.

\newenvironment{plainhscode}%
  {\hsnewpar\abovedisplayskip
   \advance\leftskip\mathindent
   \hscodestyle
   \let\hspre\(\let\hspost\)%
   \pboxed}%
  {\endpboxed%
   \hsnewpar\belowdisplayskip
   \ignorespacesafterend}

\newenvironment{oldplainhscode}%
  {\hsnewpar\abovedisplayskip
   \advance\leftskip\mathindent
   \hscodestyle
   \let\\=\@normalcr
   \(\pboxed}%
  {\endpboxed\)%
   \hsnewpar\belowdisplayskip
   \ignorespacesafterend}

% Here, we make plainhscode the default environment.

\newcommand{\plainhs}{\sethscode{plainhscode}}
\newcommand{\oldplainhs}{\sethscode{oldplainhscode}}
\plainhs

% The arrayhscode is like plain, but makes use of polytable's
% parray environment which disallows page breaks in code blocks.

\newenvironment{arrayhscode}%
  {\hsnewpar\abovedisplayskip
   \advance\leftskip\mathindent
   \hscodestyle
   \let\\=\@normalcr
   \(\parray}%
  {\endparray\)%
   \hsnewpar\belowdisplayskip
   \ignorespacesafterend}

\newcommand{\arrayhs}{\sethscode{arrayhscode}}

% The mathhscode environment also makes use of polytable's parray 
% environment. It is supposed to be used only inside math mode 
% (I used it to typeset the type rules in my thesis).

\newenvironment{mathhscode}%
  {\parray}{\endparray}

\newcommand{\mathhs}{\sethscode{mathhscode}}

% texths is similar to mathhs, but works in text mode.

\newenvironment{texthscode}%
  {\(\parray}{\endparray\)}

\newcommand{\texths}{\sethscode{texthscode}}

% The framed environment places code in a framed box.

\def\codeframewidth{\arrayrulewidth}
\RequirePackage{calc}

\newenvironment{framedhscode}%
  {\parskip=\abovedisplayskip\par\noindent
   \hscodestyle
   \arrayrulewidth=\codeframewidth
   \tabular{@{}|p{\linewidth-2\arraycolsep-2\arrayrulewidth-2pt}|@{}}%
   \hline\framedhslinecorrect\\{-1.5ex}%
   \let\endoflinesave=\\
   \let\\=\@normalcr
   \(\pboxed}%
  {\endpboxed\)%
   \framedhslinecorrect\endoflinesave{.5ex}\hline
   \endtabular
   \parskip=\belowdisplayskip\par\noindent
   \ignorespacesafterend}

\newcommand{\framedhslinecorrect}[2]%
  {#1[#2]}

\newcommand{\framedhs}{\sethscode{framedhscode}}

% The inlinehscode environment is an experimental environment
% that can be used to typeset displayed code inline.

\newenvironment{inlinehscode}%
  {\(\def\column##1##2{}%
   \let\>\undefined\let\<\undefined\let\\\undefined
   \newcommand\>[1][]{}\newcommand\<[1][]{}\newcommand\\[1][]{}%
   \def\fromto##1##2##3{##3}%
   \def\nextline{}}{\) }%

\newcommand{\inlinehs}{\sethscode{inlinehscode}}

% The joincode environment is a separate environment that
% can be used to surround and thereby connect multiple code
% blocks.

\newenvironment{joincode}%
  {\let\orighscode=\hscode
   \let\origendhscode=\endhscode
   \def\endhscode{\def\hscode{\endgroup\def\@currenvir{hscode}\\}\begingroup}
   %\let\SaveRestoreHook=\empty
   %\let\ColumnHook=\empty
   %\let\resethooks=\empty
   \orighscode\def\hscode{\endgroup\def\@currenvir{hscode}}}%
  {\origendhscode
   \global\let\hscode=\orighscode
   \global\let\endhscode=\origendhscode}%

\makeatother
\EndFmtInput
%

\ReadOnlyOnce{Formatting.fmt}%
\makeatletter

\let\Varid\mathit
\let\Conid\mathsf

\def\commentbegin{\quad\{\ }
\def\commentend{\}}

\newcommand{\ty}[1]{\Conid{#1}}
\newcommand{\con}[1]{\Conid{#1}}
\newcommand{\id}[1]{\Varid{#1}}
\newcommand{\cl}[1]{\Varid{#1}}
\newcommand{\opsym}[1]{\mathrel{#1}}

\newcommand\Keyword[1]{\textbf{\textsf{#1}}}
\newcommand\Hide{\mathbin{\downarrow}}
\newcommand\Reveal{\mathbin{\uparrow}}


%% Paper-specific keywords


\makeatother
\EndFmtInput

\section{Motivation}
\label{sec:motivation}

All binary strings are equal, but some binary strings are more equal than others.

Although everything in Redis is essentially a binary string, these strings are treated differently. Redis supports many different
kind of data structures, such as strings, hashes, lists, etc. While
they are all encoded as binary strings before being written to the databse, most commands, much like how the C language treats a piece of data, only works with data of certain types.

\paragraph{Problem 1} The command \text{SET}, by
its definition, associates a key to a string. In the following
example, the key \text{some-string} is associated
to string \text{foo}. Subsequent calls to
\text{SADD} causes runtime errors, since the value
of \text{some-string} is not a set, but a string.

\begin{tabbing}\tt
~redis\char62{}~SET~some\char45{}string~foo\\
\tt ~OK\\
\tt ~redis\char62{}~SADD~some\char45{}string~bar\\
\tt ~\char40{}error\char41{}~WRONGTYPE~Operation~against~a~key\\
\tt ~~holding~the~wrong~kind~of~value
\end{tabbing}

\paragraph{Problem 2} Even worse, not all strings are equal!
The call \text{INCR some-string} parses the string
associated with key \text{some-string}
to an integer, increments it by one, and store it back as a string.
If the string can not be parse as an integer, a runtime error
is raised.

\begin{tabbing}\tt
~redis\char62{}~SET~some\char45{}string~foo\\
\tt ~OK\\
\tt ~redis\char62{}~INCR~some\char45{}string\\
\tt ~\char40{}error\char41{}~ERR~value~is~not~an~integer~or~out\\
\tt ~~of~range
\end{tabbing}

\paragraph{In Hedis} Hedis, being only a simple wrapper on top
of the TCP protocol of Redis, inherits all the problems mentioned
above. The following program yields the same error as that in
the Redis client.

\begin{hscode}\SaveRestoreHook
\column{B}{@{}>{\hspre}l<{\hspost}@{}}%
\column{5}{@{}>{\hspre}l<{\hspost}@{}}%
\column{E}{@{}>{\hspre}l<{\hspost}@{}}%
\>[B]{}\Varid{program}\mathbin{::}\Conid{Redis}\;(\Conid{Either}\;\Conid{Reply}\;\Conid{Integer}){}\<[E]%
\\
\>[B]{}\Varid{program}\mathrel{=}\mathbf{do}{}\<[E]%
\\
\>[B]{}\hsindent{5}{}\<[5]%
\>[5]{}\Varid{set}\;\text{\tt \char34 some-string\char34}\;\text{\tt \char34 foo\char34}{}\<[E]%
\\
\>[B]{}\hsindent{5}{}\<[5]%
\>[5]{}\Varid{sadd}\;\text{\tt \char34 some-string\char34}\;[\mskip1.5mu \text{\tt \char34 a\char34}\mskip1.5mu]{}\<[E]%
\ColumnHook
\end{hscode}\resethooks
\begin{hscode}\SaveRestoreHook
\column{B}{@{}>{\hspre}l<{\hspost}@{}}%
\column{E}{@{}>{\hspre}l<{\hspost}@{}}%
\>[B]{}\Conid{Left}\;(\Conid{Error}\;\text{\tt \char34 WRONGTYPE~Operation~against~a\;~key~holding~the~wrong~kind~of~value\char34}{}\<[E]%
\\
\>[B]{}){}\<[E]%
\ColumnHook
\end{hscode}\resethooks

\paragraph{The Cause} Every key is associated with a value, and every value has
 it's own type. But most commands in Redis only work with a certain type of
 value. When a command is used on a wrong type of key, a runtime error occurs.
 The problems illustrated above arise from the absence of type checking, with
 respects to \textbf{the type of a value that associates with a key}.
 These problems could have been avoided, if we could know the type every key
 associates with in advance, and prevent programs with invalid commands from
 executing.

\subsection{Hedis as an embedded DSL}

Haskell makes it easy to build and use Domain Specific Languages (DSLs),
 and Hedis can be regarded as one of them. What makes Hedis peculiar is that,
 it has \emph{variable bindings} (between keys and values), but with very
 little or no semantic checking, neither dynamically nor statically.

We began with making Hedis a dynamically typechecked embedded DSL, and implemented a
 runtime type checker that keeps track of types of all the variable. But then we
 found that things can be a lot easier, by leveraging the host language's type
 checker. We encode variable bindings with \emph{type-level lists} and
 \emph{strings}, and control the effects on the bindings with
 \emph{indexed monad}. In contrast to the former approach, we \textbf{embedded}
 our type checker into Haskell's type system, without having to build a
 \textbf{standalone} one on the term level.

\subsection{Contributions}

To summarize our contributions:

\begin{itemize}
\item We make Hedis statically type-checked, without runtime overhead.
\item We demonstrates how to model variable bindings of an embedded DSL with
 language extensions like type-level literals and data kinds.
\item We provide (yet another) an example of encoding effects and constraints of
 an action in types, with indexed monad\cite{indexedmonad} and other language
 extensions such as closed type-families\cite{closedtypefamilies} and
 constraints kinds\cite{constraintskinds}.
\item Popcorn, a package we built for programmers. This package helps programmers
 to write more reliable Redis programs, and also makes Redis polymorphic by
 automatically converting back and forth from values of arbitrary types and
 boring ByteStrings.
\end{itemize}

%% ODER: format ==         = "\mathrel{==}"
%% ODER: format /=         = "\neq "
%
%
\makeatletter
\@ifundefined{lhs2tex.lhs2tex.sty.read}%
  {\@namedef{lhs2tex.lhs2tex.sty.read}{}%
   \newcommand\SkipToFmtEnd{}%
   \newcommand\EndFmtInput{}%
   \long\def\SkipToFmtEnd#1\EndFmtInput{}%
  }\SkipToFmtEnd

\newcommand\ReadOnlyOnce[1]{\@ifundefined{#1}{\@namedef{#1}{}}\SkipToFmtEnd}
\usepackage{amstext}
\usepackage{amssymb}
\usepackage{stmaryrd}
\DeclareFontFamily{OT1}{cmtex}{}
\DeclareFontShape{OT1}{cmtex}{m}{n}
  {<5><6><7><8>cmtex8
   <9>cmtex9
   <10><10.95><12><14.4><17.28><20.74><24.88>cmtex10}{}
\DeclareFontShape{OT1}{cmtex}{m}{it}
  {<-> ssub * cmtt/m/it}{}
\newcommand{\texfamily}{\fontfamily{cmtex}\selectfont}
\DeclareFontShape{OT1}{cmtt}{bx}{n}
  {<5><6><7><8>cmtt8
   <9>cmbtt9
   <10><10.95><12><14.4><17.28><20.74><24.88>cmbtt10}{}
\DeclareFontShape{OT1}{cmtex}{bx}{n}
  {<-> ssub * cmtt/bx/n}{}
\newcommand{\tex}[1]{\text{\texfamily#1}}	% NEU

\newcommand{\Sp}{\hskip.33334em\relax}


\newcommand{\Conid}[1]{\mathit{#1}}
\newcommand{\Varid}[1]{\mathit{#1}}
\newcommand{\anonymous}{\kern0.06em \vbox{\hrule\@width.5em}}
\newcommand{\plus}{\mathbin{+\!\!\!+}}
\newcommand{\bind}{\mathbin{>\!\!\!>\mkern-6.7mu=}}
\newcommand{\rbind}{\mathbin{=\mkern-6.7mu<\!\!\!<}}% suggested by Neil Mitchell
\newcommand{\sequ}{\mathbin{>\!\!\!>}}
\renewcommand{\leq}{\leqslant}
\renewcommand{\geq}{\geqslant}
\usepackage{polytable}

%mathindent has to be defined
\@ifundefined{mathindent}%
  {\newdimen\mathindent\mathindent\leftmargini}%
  {}%

\def\resethooks{%
  \global\let\SaveRestoreHook\empty
  \global\let\ColumnHook\empty}
\newcommand*{\savecolumns}[1][default]%
  {\g@addto@macro\SaveRestoreHook{\savecolumns[#1]}}
\newcommand*{\restorecolumns}[1][default]%
  {\g@addto@macro\SaveRestoreHook{\restorecolumns[#1]}}
\newcommand*{\aligncolumn}[2]%
  {\g@addto@macro\ColumnHook{\column{#1}{#2}}}

\resethooks

\newcommand{\onelinecommentchars}{\quad-{}- }
\newcommand{\commentbeginchars}{\enskip\{-}
\newcommand{\commentendchars}{-\}\enskip}

\newcommand{\visiblecomments}{%
  \let\onelinecomment=\onelinecommentchars
  \let\commentbegin=\commentbeginchars
  \let\commentend=\commentendchars}

\newcommand{\invisiblecomments}{%
  \let\onelinecomment=\empty
  \let\commentbegin=\empty
  \let\commentend=\empty}

\visiblecomments

\newlength{\blanklineskip}
\setlength{\blanklineskip}{0.66084ex}

\newcommand{\hsindent}[1]{\quad}% default is fixed indentation
\let\hspre\empty
\let\hspost\empty
\newcommand{\NB}{\textbf{NB}}
\newcommand{\Todo}[1]{$\langle$\textbf{To do:}~#1$\rangle$}

\EndFmtInput
\makeatother
%
%
%
%
%
%
% This package provides two environments suitable to take the place
% of hscode, called "plainhscode" and "arrayhscode". 
%
% The plain environment surrounds each code block by vertical space,
% and it uses \abovedisplayskip and \belowdisplayskip to get spacing
% similar to formulas. Note that if these dimensions are changed,
% the spacing around displayed math formulas changes as well.
% All code is indented using \leftskip.
%
% Changed 19.08.2004 to reflect changes in colorcode. Should work with
% CodeGroup.sty.
%
\ReadOnlyOnce{polycode.fmt}%
\makeatletter

\newcommand{\hsnewpar}[1]%
  {{\parskip=0pt\parindent=0pt\par\vskip #1\noindent}}

% can be used, for instance, to redefine the code size, by setting the
% command to \small or something alike
\newcommand{\hscodestyle}{}

% The command \sethscode can be used to switch the code formatting
% behaviour by mapping the hscode environment in the subst directive
% to a new LaTeX environment.

\newcommand{\sethscode}[1]%
  {\expandafter\let\expandafter\hscode\csname #1\endcsname
   \expandafter\let\expandafter\endhscode\csname end#1\endcsname}

% "compatibility" mode restores the non-polycode.fmt layout.

\newenvironment{compathscode}%
  {\par\noindent
   \advance\leftskip\mathindent
   \hscodestyle
   \let\\=\@normalcr
   \let\hspre\(\let\hspost\)%
   \pboxed}%
  {\endpboxed\)%
   \par\noindent
   \ignorespacesafterend}

\newcommand{\compaths}{\sethscode{compathscode}}

% "plain" mode is the proposed default.
% It should now work with \centering.
% This required some changes. The old version
% is still available for reference as oldplainhscode.

\newenvironment{plainhscode}%
  {\hsnewpar\abovedisplayskip
   \advance\leftskip\mathindent
   \hscodestyle
   \let\hspre\(\let\hspost\)%
   \pboxed}%
  {\endpboxed%
   \hsnewpar\belowdisplayskip
   \ignorespacesafterend}

\newenvironment{oldplainhscode}%
  {\hsnewpar\abovedisplayskip
   \advance\leftskip\mathindent
   \hscodestyle
   \let\\=\@normalcr
   \(\pboxed}%
  {\endpboxed\)%
   \hsnewpar\belowdisplayskip
   \ignorespacesafterend}

% Here, we make plainhscode the default environment.

\newcommand{\plainhs}{\sethscode{plainhscode}}
\newcommand{\oldplainhs}{\sethscode{oldplainhscode}}
\plainhs

% The arrayhscode is like plain, but makes use of polytable's
% parray environment which disallows page breaks in code blocks.

\newenvironment{arrayhscode}%
  {\hsnewpar\abovedisplayskip
   \advance\leftskip\mathindent
   \hscodestyle
   \let\\=\@normalcr
   \(\parray}%
  {\endparray\)%
   \hsnewpar\belowdisplayskip
   \ignorespacesafterend}

\newcommand{\arrayhs}{\sethscode{arrayhscode}}

% The mathhscode environment also makes use of polytable's parray 
% environment. It is supposed to be used only inside math mode 
% (I used it to typeset the type rules in my thesis).

\newenvironment{mathhscode}%
  {\parray}{\endparray}

\newcommand{\mathhs}{\sethscode{mathhscode}}

% texths is similar to mathhs, but works in text mode.

\newenvironment{texthscode}%
  {\(\parray}{\endparray\)}

\newcommand{\texths}{\sethscode{texthscode}}

% The framed environment places code in a framed box.

\def\codeframewidth{\arrayrulewidth}
\RequirePackage{calc}

\newenvironment{framedhscode}%
  {\parskip=\abovedisplayskip\par\noindent
   \hscodestyle
   \arrayrulewidth=\codeframewidth
   \tabular{@{}|p{\linewidth-2\arraycolsep-2\arrayrulewidth-2pt}|@{}}%
   \hline\framedhslinecorrect\\{-1.5ex}%
   \let\endoflinesave=\\
   \let\\=\@normalcr
   \(\pboxed}%
  {\endpboxed\)%
   \framedhslinecorrect\endoflinesave{.5ex}\hline
   \endtabular
   \parskip=\belowdisplayskip\par\noindent
   \ignorespacesafterend}

\newcommand{\framedhslinecorrect}[2]%
  {#1[#2]}

\newcommand{\framedhs}{\sethscode{framedhscode}}

% The inlinehscode environment is an experimental environment
% that can be used to typeset displayed code inline.

\newenvironment{inlinehscode}%
  {\(\def\column##1##2{}%
   \let\>\undefined\let\<\undefined\let\\\undefined
   \newcommand\>[1][]{}\newcommand\<[1][]{}\newcommand\\[1][]{}%
   \def\fromto##1##2##3{##3}%
   \def\nextline{}}{\) }%

\newcommand{\inlinehs}{\sethscode{inlinehscode}}

% The joincode environment is a separate environment that
% can be used to surround and thereby connect multiple code
% blocks.

\newenvironment{joincode}%
  {\let\orighscode=\hscode
   \let\origendhscode=\endhscode
   \def\endhscode{\def\hscode{\endgroup\def\@currenvir{hscode}\\}\begingroup}
   %\let\SaveRestoreHook=\empty
   %\let\ColumnHook=\empty
   %\let\resethooks=\empty
   \orighscode\def\hscode{\endgroup\def\@currenvir{hscode}}}%
  {\origendhscode
   \global\let\hscode=\orighscode
   \global\let\endhscode=\origendhscode}%

\makeatother
\EndFmtInput
%

\ReadOnlyOnce{Formatting.fmt}%
\makeatletter

\let\Varid\mathit
\let\Conid\mathsf

\def\commentbegin{\quad\{\ }
\def\commentend{\}}

\newcommand{\ty}[1]{\Conid{#1}}
\newcommand{\con}[1]{\Conid{#1}}
\newcommand{\id}[1]{\Varid{#1}}
\newcommand{\cl}[1]{\Varid{#1}}
\newcommand{\opsym}[1]{\mathrel{#1}}

\newcommand\Keyword[1]{\textbf{\textsf{#1}}}
\newcommand\Hide{\mathbin{\downarrow}}
\newcommand\Reveal{\mathbin{\uparrow}}


%% Paper-specific keywords


\makeatother
\EndFmtInput

\section{Type-Level Dictionaries}
\label{sec:type-level-dict}

One of the challenges of statically ensuring type correctness of \Redis{},
which also presents in other stateful languages, is that the type of the value
associated to a key can be altered after updating. To ensure type correctness,
we have to keep track of the (\Redis{}) types of all existing keys in a
{\em dictionary} --- conceptually, a list of pairs of keys and \Redis{} types.
Each \Redis{} command is embedded in \Popcorn{} as a monadic computation. The
monad, to be presented in Section~\ref{sec:indexed-monads}, is indexed by
the dictionaries before and after the computation. In a dependently typed
programming language (without the so-called ``phase distinction'' ---
separation between types and terms), this would pose no problem. In Haskell
however, the dictionaries, to index a monad, has to be a Haskell type as well.

In this section we describe how to construct a type-level dictionary, to be
used with the indexed monad in Section~\ref{sec:indexed-monads}. More operations
on the dictionary will be presented in Section~\ref{sec:type-level-fun}.

\subsection{Datatype Promotion}

% Normally, at the term level, we could express the datatype of dictionary with
% \emph{type synonym} like this.\footnotemark
%
% \begin{spec}
% type Key = String
% type Dictionary = [(Key, TypeRep)]
% \end{spec}
%
% \footnotetext{|TypeRep| supports term-level representations
%  of datatypes, available in |Data.Typeable|}

A datatype definition such as the one below:
\begin{hscode}\SaveRestoreHook
\column{B}{@{}>{\hspre}l<{\hspost}@{}}%
\column{E}{@{}>{\hspre}l<{\hspost}@{}}%
\>[B]{}\mathbf{data}\;\Conid{List}\;\Varid{a}\mathrel{=}\Conid{Nil}\mid \Conid{Cons}\;\Varid{a}\;(\Conid{List}\;\Varid{a})~~,{}\<[E]%
\ColumnHook
\end{hscode}\resethooks
is usually understood as having defined a type constructor \ensuremath{\Conid{List}}, and two value
constructors \ensuremath{\Conid{Nil}} and \ensuremath{\Conid{Cons}}. For all lifted types \ensuremath{\Varid{a}}, \ensuremath{\Conid{List}\;\Varid{a}} is also a
lifted type. Thus the {\em kind} of \ensuremath{\Conid{List}} is \ensuremath{\mathbin{*}\to \mathbin{*}}, where \ensuremath{\mathbin{*}} is the kind of
all {\em lifted types} in Haskell. The two value constructors respectively
have types \ensuremath{\Conid{Nil}\mathbin{::}\Conid{List}\;\Varid{a}} and \ensuremath{\Conid{Cons}\mathbin{::}\Varid{a}\to \Conid{List}\;\Varid{a}\to \Conid{List}\;\Varid{a}}.

The GHC extension \emph{data kinds}~\cite{promotion}, however, automatically
promotes certain ``suitable'' types to kinds. With the extension, the \ensuremath{\mathbf{data}}
definition has an alternative reading: for all kind \ensuremath{\Varid{k}}, \ensuremath{\Conid{List}\;\Varid{k}} is also a
kind. \ensuremath{\Conid{Nil}} is a type, whose kind is \ensuremath{\Conid{List}\;\Varid{k}} for some \ensuremath{\Varid{k}}. Given a type \ensuremath{\Varid{x}}
of kind \ensuremath{\Varid{k}} and a type \ensuremath{\Varid{xs}} of kind \ensuremath{\Conid{List}\;\Varid{k}}, \ensuremath{\Conid{Cons}\;\Varid{x}\;\Varid{xs}} is again a type of
kind \ensuremath{\Conid{List}\;\Varid{k}}. Formally, for kind \ensuremath{\Varid{k}}, \ensuremath{\Conid{Nil}\mathbin{::}\Conid{List}\;\Varid{k}} and \ensuremath{\Conid{Cons}\mathbin{::}\Varid{k}\to \Conid{List}\;\Varid{k}\to \Conid{List}\;\Varid{k}}. Whether a constructor is promoted can often be inferred
from the context. To be more specific, prefixing a contructor with
a single quote, such as in \ensuremath{\thinspace'\Conid{Nil}} and \ensuremath{\thinspace'\Conid{Cons}}, denotes that it is promoted.

The built-in lists in Haskell has two constructors \ensuremath{[\mskip1.5mu \mskip1.5mu]} and \ensuremath{(\mathbin{:})}, and
\ensuremath{[\mskip1.5mu \mathrm{1},\mathrm{2},\mathrm{3}\mskip1.5mu]} is take as a syntax sugar for \ensuremath{\mathrm{1}\mathbin{:}(\mathrm{2}\mathbin{:}(\mathrm{3}\mathbin{:}[\mskip1.5mu \mskip1.5mu]))}. The type and
data constructors can also be promoted. Given kind \ensuremath{\Varid{k}}, \ensuremath{[\mskip1.5mu \Varid{k}\mskip1.5mu]} is also a kind.
The type constructor 
For example, since \ensuremath{\Conid{Int}}, \ensuremath{\Conid{Char}}, etc., all have kind \ensuremath{\mathbin{*}}, \ensuremath{\Conid{Int}\mathbin{\thinspace':}(\Conid{Char}\mathbin{\thinspace':}(\Conid{Bool}\mathbin{\thinspace':}\thinspace'[]))} is a type having kind \ensuremath{[\mskip1.5mu \mathbin{*}\mskip1.5mu]} -- a list of types. The same list can be denoted by a syntax
sugar \ensuremath{\thinspace'[\!\;\Conid{Int},\Conid{Char},\Conid{Bool}\;\!]}.

sugar for \ensuremath{\mathrm{1}\mathbin{\thinspace':}(\mathrm{2}\mathbin{\thinspace':}(\mathrm{3}\mathbin{\thinspace':}\thinspace'[]))}
Haskell sugars lists \ensuremath{[\mskip1.5mu \mathrm{1},\mathrm{2},\mathrm{3}\mskip1.5mu]} and tuples
 \ensuremath{(\mathrm{1},\text{\tt 'a'})} with brackets and parentheses.
 We could also express promoted lists and tuples in types like this with
 a single quote prefixed. For example:
 \ensuremath{\thinspace'[\!\;\Conid{Int},\Conid{Char}\;\!]}, \ensuremath{\thinspace'(\!\;\Conid{Int},\Conid{Char}\;\!)}.

\subsection{Type-level literals}

Now we have type-level lists and tuples to construct the dictionary.
For keys, \ensuremath{\Conid{String}} also has a type-level correspondence:
\ensuremath{\Conid{Symbol}}.

\begin{hscode}\SaveRestoreHook
\column{B}{@{}>{\hspre}l<{\hspost}@{}}%
\column{E}{@{}>{\hspre}l<{\hspost}@{}}%
\>[B]{}\mathbf{data}\;\Conid{Symbol}{}\<[E]%
\ColumnHook
\end{hscode}\resethooks

Symbol is defined without a value constructor, because it's intended to be used
 as a promoted kind.

\begin{hscode}\SaveRestoreHook
\column{B}{@{}>{\hspre}l<{\hspost}@{}}%
\column{E}{@{}>{\hspre}l<{\hspost}@{}}%
\>[B]{}\text{\tt \char34 this~is~a~type-level~string~literal\char34}\mathbin{::}\Conid{Symbol}{}\<[E]%
\ColumnHook
\end{hscode}\resethooks
%
% Nonetheless, it's still useful to have a term-level value that links with a
%  Symbol, when we want to retrieve type-level information at runtime (but not the
%  other way around!).

\subsection{Putting Everything Together}

With all of these ingredients ready, let's build some dictionaries!

\begin{hscode}\SaveRestoreHook
\column{B}{@{}>{\hspre}l<{\hspost}@{}}%
\column{E}{@{}>{\hspre}l<{\hspost}@{}}%
\>[B]{}\mathbf{type}\;\Conid{DictEmpty}\mathrel{=}\text{\tt '[]\;type~Dict0~=~'}{}\<[E]%
\\
\>[B]{}[\mskip1.5mu \text{\tt '(\char34 key\char34 ,~Bool)~]\;type~Dict1~=~'}{}\<[E]%
\\
\>[B]{}[\mskip1.5mu \text{\tt '(\char34 A\char34 ,~Int),~'}\;(\text{\tt \char34 B\char34},\text{\tt \char34 A\char34})\mskip1.5mu]{}\<[E]%
\ColumnHook
\end{hscode}\resethooks

These dictionaries are defined with \emph{type synonym}, since they are
 \emph{types}, not \emph{terms}. If we ask \text{GHCi} what is the
 kind of \ensuremath{\Conid{Dict1}}, we will get \ensuremath{\Conid{Dict1}\mathbin{::}[\mskip1.5mu (\Conid{Symbol},\mathbin{*})\mskip1.5mu]}

The kind \ensuremath{\mathbin{*}} (pronounced ``star'') stands for the set of all
 concrete type expressions, such as \ensuremath{\Conid{Int}},
 \ensuremath{\Conid{Char}} or even a symbol \ensuremath{\text{\tt \char34 symbol\char34}},
 while \ensuremath{\Conid{Symbol}} is restricted to all symbols only.

%% ODER: format ==         = "\mathrel{==}"
%% ODER: format /=         = "\neq "
%
%
\makeatletter
\@ifundefined{lhs2tex.lhs2tex.sty.read}%
  {\@namedef{lhs2tex.lhs2tex.sty.read}{}%
   \newcommand\SkipToFmtEnd{}%
   \newcommand\EndFmtInput{}%
   \long\def\SkipToFmtEnd#1\EndFmtInput{}%
  }\SkipToFmtEnd

\newcommand\ReadOnlyOnce[1]{\@ifundefined{#1}{\@namedef{#1}{}}\SkipToFmtEnd}
\usepackage{amstext}
\usepackage{amssymb}
\usepackage{stmaryrd}
\DeclareFontFamily{OT1}{cmtex}{}
\DeclareFontShape{OT1}{cmtex}{m}{n}
  {<5><6><7><8>cmtex8
   <9>cmtex9
   <10><10.95><12><14.4><17.28><20.74><24.88>cmtex10}{}
\DeclareFontShape{OT1}{cmtex}{m}{it}
  {<-> ssub * cmtt/m/it}{}
\newcommand{\texfamily}{\fontfamily{cmtex}\selectfont}
\DeclareFontShape{OT1}{cmtt}{bx}{n}
  {<5><6><7><8>cmtt8
   <9>cmbtt9
   <10><10.95><12><14.4><17.28><20.74><24.88>cmbtt10}{}
\DeclareFontShape{OT1}{cmtex}{bx}{n}
  {<-> ssub * cmtt/bx/n}{}
\newcommand{\tex}[1]{\text{\texfamily#1}}	% NEU

\newcommand{\Sp}{\hskip.33334em\relax}


\newcommand{\Conid}[1]{\mathit{#1}}
\newcommand{\Varid}[1]{\mathit{#1}}
\newcommand{\anonymous}{\kern0.06em \vbox{\hrule\@width.5em}}
\newcommand{\plus}{\mathbin{+\!\!\!+}}
\newcommand{\bind}{\mathbin{>\!\!\!>\mkern-6.7mu=}}
\newcommand{\rbind}{\mathbin{=\mkern-6.7mu<\!\!\!<}}% suggested by Neil Mitchell
\newcommand{\sequ}{\mathbin{>\!\!\!>}}
\renewcommand{\leq}{\leqslant}
\renewcommand{\geq}{\geqslant}
\usepackage{polytable}

%mathindent has to be defined
\@ifundefined{mathindent}%
  {\newdimen\mathindent\mathindent\leftmargini}%
  {}%

\def\resethooks{%
  \global\let\SaveRestoreHook\empty
  \global\let\ColumnHook\empty}
\newcommand*{\savecolumns}[1][default]%
  {\g@addto@macro\SaveRestoreHook{\savecolumns[#1]}}
\newcommand*{\restorecolumns}[1][default]%
  {\g@addto@macro\SaveRestoreHook{\restorecolumns[#1]}}
\newcommand*{\aligncolumn}[2]%
  {\g@addto@macro\ColumnHook{\column{#1}{#2}}}

\resethooks

\newcommand{\onelinecommentchars}{\quad-{}- }
\newcommand{\commentbeginchars}{\enskip\{-}
\newcommand{\commentendchars}{-\}\enskip}

\newcommand{\visiblecomments}{%
  \let\onelinecomment=\onelinecommentchars
  \let\commentbegin=\commentbeginchars
  \let\commentend=\commentendchars}

\newcommand{\invisiblecomments}{%
  \let\onelinecomment=\empty
  \let\commentbegin=\empty
  \let\commentend=\empty}

\visiblecomments

\newlength{\blanklineskip}
\setlength{\blanklineskip}{0.66084ex}

\newcommand{\hsindent}[1]{\quad}% default is fixed indentation
\let\hspre\empty
\let\hspost\empty
\newcommand{\NB}{\textbf{NB}}
\newcommand{\Todo}[1]{$\langle$\textbf{To do:}~#1$\rangle$}

\EndFmtInput
\makeatother
%
%
%
%
%
%
% This package provides two environments suitable to take the place
% of hscode, called "plainhscode" and "arrayhscode". 
%
% The plain environment surrounds each code block by vertical space,
% and it uses \abovedisplayskip and \belowdisplayskip to get spacing
% similar to formulas. Note that if these dimensions are changed,
% the spacing around displayed math formulas changes as well.
% All code is indented using \leftskip.
%
% Changed 19.08.2004 to reflect changes in colorcode. Should work with
% CodeGroup.sty.
%
\ReadOnlyOnce{polycode.fmt}%
\makeatletter

\newcommand{\hsnewpar}[1]%
  {{\parskip=0pt\parindent=0pt\par\vskip #1\noindent}}

% can be used, for instance, to redefine the code size, by setting the
% command to \small or something alike
\newcommand{\hscodestyle}{}

% The command \sethscode can be used to switch the code formatting
% behaviour by mapping the hscode environment in the subst directive
% to a new LaTeX environment.

\newcommand{\sethscode}[1]%
  {\expandafter\let\expandafter\hscode\csname #1\endcsname
   \expandafter\let\expandafter\endhscode\csname end#1\endcsname}

% "compatibility" mode restores the non-polycode.fmt layout.

\newenvironment{compathscode}%
  {\par\noindent
   \advance\leftskip\mathindent
   \hscodestyle
   \let\\=\@normalcr
   \let\hspre\(\let\hspost\)%
   \pboxed}%
  {\endpboxed\)%
   \par\noindent
   \ignorespacesafterend}

\newcommand{\compaths}{\sethscode{compathscode}}

% "plain" mode is the proposed default.
% It should now work with \centering.
% This required some changes. The old version
% is still available for reference as oldplainhscode.

\newenvironment{plainhscode}%
  {\hsnewpar\abovedisplayskip
   \advance\leftskip\mathindent
   \hscodestyle
   \let\hspre\(\let\hspost\)%
   \pboxed}%
  {\endpboxed%
   \hsnewpar\belowdisplayskip
   \ignorespacesafterend}

\newenvironment{oldplainhscode}%
  {\hsnewpar\abovedisplayskip
   \advance\leftskip\mathindent
   \hscodestyle
   \let\\=\@normalcr
   \(\pboxed}%
  {\endpboxed\)%
   \hsnewpar\belowdisplayskip
   \ignorespacesafterend}

% Here, we make plainhscode the default environment.

\newcommand{\plainhs}{\sethscode{plainhscode}}
\newcommand{\oldplainhs}{\sethscode{oldplainhscode}}
\plainhs

% The arrayhscode is like plain, but makes use of polytable's
% parray environment which disallows page breaks in code blocks.

\newenvironment{arrayhscode}%
  {\hsnewpar\abovedisplayskip
   \advance\leftskip\mathindent
   \hscodestyle
   \let\\=\@normalcr
   \(\parray}%
  {\endparray\)%
   \hsnewpar\belowdisplayskip
   \ignorespacesafterend}

\newcommand{\arrayhs}{\sethscode{arrayhscode}}

% The mathhscode environment also makes use of polytable's parray 
% environment. It is supposed to be used only inside math mode 
% (I used it to typeset the type rules in my thesis).

\newenvironment{mathhscode}%
  {\parray}{\endparray}

\newcommand{\mathhs}{\sethscode{mathhscode}}

% texths is similar to mathhs, but works in text mode.

\newenvironment{texthscode}%
  {\(\parray}{\endparray\)}

\newcommand{\texths}{\sethscode{texthscode}}

% The framed environment places code in a framed box.

\def\codeframewidth{\arrayrulewidth}
\RequirePackage{calc}

\newenvironment{framedhscode}%
  {\parskip=\abovedisplayskip\par\noindent
   \hscodestyle
   \arrayrulewidth=\codeframewidth
   \tabular{@{}|p{\linewidth-2\arraycolsep-2\arrayrulewidth-2pt}|@{}}%
   \hline\framedhslinecorrect\\{-1.5ex}%
   \let\endoflinesave=\\
   \let\\=\@normalcr
   \(\pboxed}%
  {\endpboxed\)%
   \framedhslinecorrect\endoflinesave{.5ex}\hline
   \endtabular
   \parskip=\belowdisplayskip\par\noindent
   \ignorespacesafterend}

\newcommand{\framedhslinecorrect}[2]%
  {#1[#2]}

\newcommand{\framedhs}{\sethscode{framedhscode}}

% The inlinehscode environment is an experimental environment
% that can be used to typeset displayed code inline.

\newenvironment{inlinehscode}%
  {\(\def\column##1##2{}%
   \let\>\undefined\let\<\undefined\let\\\undefined
   \newcommand\>[1][]{}\newcommand\<[1][]{}\newcommand\\[1][]{}%
   \def\fromto##1##2##3{##3}%
   \def\nextline{}}{\) }%

\newcommand{\inlinehs}{\sethscode{inlinehscode}}

% The joincode environment is a separate environment that
% can be used to surround and thereby connect multiple code
% blocks.

\newenvironment{joincode}%
  {\let\orighscode=\hscode
   \let\origendhscode=\endhscode
   \def\endhscode{\def\hscode{\endgroup\def\@currenvir{hscode}\\}\begingroup}
   %\let\SaveRestoreHook=\empty
   %\let\ColumnHook=\empty
   %\let\resethooks=\empty
   \orighscode\def\hscode{\endgroup\def\@currenvir{hscode}}}%
  {\origendhscode
   \global\let\hscode=\orighscode
   \global\let\endhscode=\origendhscode}%

\makeatother
\EndFmtInput
%

\ReadOnlyOnce{Formatting.fmt}%
\makeatletter

\let\Varid\mathit
\let\Conid\mathsf

\def\commentbegin{\quad\{\ }
\def\commentend{\}}

\newcommand{\ty}[1]{\Conid{#1}}
\newcommand{\con}[1]{\Conid{#1}}
\newcommand{\id}[1]{\Varid{#1}}
\newcommand{\cl}[1]{\Varid{#1}}
\newcommand{\opsym}[1]{\mathrel{#1}}

\newcommand\Keyword[1]{\textbf{\textsf{#1}}}
\newcommand\Hide{\mathbin{\downarrow}}
\newcommand\Reveal{\mathbin{\uparrow}}


%% Paper-specific keywords


\makeatother
\EndFmtInput

\section{Indexed Monads}
\label{sec:indexed-monads}

Stateful computations are often reasoned in a Hoare-logic style: each command
is labelled by a \emph{precondition} and a \emph{postcondition}. If the former
is satisfied before the command is executed, the latter is guaranteed to hold
afterwards.

In Haskell, stateful computations are represented by monads. In order to
reason about their behaviors within the type system, we wish to label a state
monad with its pre and postcondition. An \emph{indexed
monad}~\cite{indexedmonad} (also called \emph{monadish} or
\emph{parameterised monad}) is a monad that, in addition to the type of value
it computes, takes two more type arguments representing an initial state and
a final state, to be interpreted like a Hoare triple~\cite{kleisli}:
\begin{hscode}\SaveRestoreHook
\column{B}{@{}>{\hspre}l<{\hspost}@{}}%
\column{5}{@{}>{\hspre}l<{\hspost}@{}}%
\column{E}{@{}>{\hspre}l<{\hspost}@{}}%
\>[B]{}\mathbf{class}\;\Conid{IMonad}\;\Varid{m}\;\mathbf{where}{}\<[E]%
\\
\>[B]{}\hsindent{5}{}\<[5]%
\>[5]{}\Varid{unit}\mathbin{::}\Varid{a}\to \Varid{m}\;\Varid{p}\;\Varid{p}\;\Varid{a}{}\<[E]%
\\
\>[B]{}\hsindent{5}{}\<[5]%
\>[5]{}\Varid{bind}\mathbin{::}\Varid{m}\;\Varid{p}\;\Varid{q}\;\Varid{a}\to (\Varid{a}\to \Varid{m}\;\Varid{q}\;\Varid{r}\;\Varid{b})\to \Varid{m}\;\Varid{p}\;\Varid{r}\;\Varid{b}~~.{}\<[E]%
\ColumnHook
\end{hscode}\resethooks
The intention is that a computation of type \ensuremath{\Varid{m}\;\Varid{p}\;\Varid{q}\;\Varid{a}} is a stateful computation
such that if it starts execution in a state satisfying \ensuremath{\Varid{p}} and terminates, it
yields a value of type \ensuremath{\Varid{a}}, and the new state satisfies \ensuremath{\Varid{q}}.
The operator \ensuremath{\Varid{unit}} lifts a pure computation to a stateful computation that
does not alter the state. In \ensuremath{\Varid{x}\mathbin{`\Varid{bind}`}\Varid{f}}, a computation \ensuremath{\Varid{x}\mathbin{::}\Varid{m}\;\Varid{p}\;\Varid{q}\;\Varid{a}} must
be chained before \ensuremath{\Varid{f}\mathbin{::}\Varid{a}\to \Varid{m}\;\Varid{q}\;\Varid{r}\;\Varid{b}}, which expects a value of type \ensuremath{\Varid{a}} and
a state satisfying \ensuremath{\Varid{q}} and, if terminates, ends in a state satisfying \ensuremath{\Varid{r}}.
The result is a monad \ensuremath{\Varid{m}\;\Varid{p}\;\Varid{r}\;\Varid{b}} --- a computation that, if executed in a state
satisfying \ensuremath{\Varid{p}} and terminates, yields a value \ensuremath{\Varid{b}} and a state satisfying \ensuremath{\Varid{r}}.
Indexed monads have been used ~\cite{typefun,staticresources} ... \todo{for what? Some discriptions here to properly cite them.}

We define a new indexed monad \ensuremath{\Conid{Popcorn}} which, at term level, merely wraps
\ensuremath{\Conid{Redis}} in an additional constructor. The purpose is to add the pre and
postconditions at type level:
\begin{hscode}\SaveRestoreHook
\column{B}{@{}>{\hspre}l<{\hspost}@{}}%
\column{5}{@{}>{\hspre}l<{\hspost}@{}}%
\column{E}{@{}>{\hspre}l<{\hspost}@{}}%
\>[B]{}\mathbf{newtype}\;\Conid{Popcorn}\;\Varid{p}\;\Varid{q}\;\Varid{a}\mathrel{=}\Conid{Popcorn}\;\{\mskip1.5mu \Varid{unPopcorn}\mathbin{::}\Conid{Redis}\;\Varid{a}\mskip1.5mu\}~~,{}\<[E]%
\\[\blanklineskip]%
\>[B]{}\mathbf{instance}\;\Conid{IMonad}\;\Conid{Popcorn}\;\mathbf{where}{}\<[E]%
\\
\>[B]{}\hsindent{5}{}\<[5]%
\>[5]{}\Varid{unit}\mathrel{=}\Conid{Popcorn}\mathbin{\cdot}\Varid{return}{}\<[E]%
\\
\>[B]{}\hsindent{5}{}\<[5]%
\>[5]{}\Varid{bind}\;\Varid{m}\;\Varid{f}\mathrel{=}\Conid{Popcorn}\;(\Varid{unPopcorn}\;\Varid{m}\bind \Varid{unPopcorn}\mathbin{\cdot}\Varid{f})~~.{}\<[E]%
\ColumnHook
\end{hscode}\resethooks
To execute a \ensuremath{\Conid{Popcorn}} program, simply apply it to \ensuremath{\Varid{unPopcorn}} to erase the additional type information and get back an ordinary \Hedis{} program.

\paragraph{\text{PING}: A First Example}
In \Redis{}, \text{PING} does nothing but replies with
 \text{PONG} if the connection is alive. In Hedis,
 \ensuremath{\Varid{ping}} has type:

\begin{hscode}\SaveRestoreHook
\column{B}{@{}>{\hspre}l<{\hspost}@{}}%
\column{E}{@{}>{\hspre}l<{\hspost}@{}}%
\>[B]{}\Varid{ping}\mathbin{::}\Conid{Redis}\;(\Conid{Either}\;\Conid{Reply}\;\Conid{Status}){}\<[E]%
\ColumnHook
\end{hscode}\resethooks

Now with \ensuremath{\Conid{Popcorn}}, we could make our own version of
\ensuremath{\Varid{ping}}\footnotemark

\footnotetext{\ensuremath{\Varid{ping}} from Hedis is qualified with
\ensuremath{\Conid{Hedis}} to prevent function name clashing in our code.}

\begin{hscode}\SaveRestoreHook
\column{B}{@{}>{\hspre}l<{\hspost}@{}}%
\column{E}{@{}>{\hspre}l<{\hspost}@{}}%
\>[B]{}\Varid{ping}\mathbin{::}\Conid{Popcorn}\;\Varid{xs}\;\Varid{xs}\;(\Conid{Either}\;\Conid{Reply}\;\Conid{Status}){}\<[E]%
\\
\>[B]{}\Varid{ping}\mathrel{=}\Conid{Popcorn}\;\Varid{\Conid{Hedis}.ping}{}\<[E]%
\ColumnHook
\end{hscode}\resethooks

The dictionary \ensuremath{\Varid{xs}} in the type remains unaffected after the
 action, because \ensuremath{\Varid{ping}} does not affect any key-type
 bindings. To encode other commands that modifies key-type bindings, we need
 type-level functions to annotate those effects on the dictionary.

%% ODER: format ==         = "\mathrel{==}"
%% ODER: format /=         = "\neq "
%
%
\makeatletter
\@ifundefined{lhs2tex.lhs2tex.sty.read}%
  {\@namedef{lhs2tex.lhs2tex.sty.read}{}%
   \newcommand\SkipToFmtEnd{}%
   \newcommand\EndFmtInput{}%
   \long\def\SkipToFmtEnd#1\EndFmtInput{}%
  }\SkipToFmtEnd

\newcommand\ReadOnlyOnce[1]{\@ifundefined{#1}{\@namedef{#1}{}}\SkipToFmtEnd}
\usepackage{amstext}
\usepackage{amssymb}
\usepackage{stmaryrd}
\DeclareFontFamily{OT1}{cmtex}{}
\DeclareFontShape{OT1}{cmtex}{m}{n}
  {<5><6><7><8>cmtex8
   <9>cmtex9
   <10><10.95><12><14.4><17.28><20.74><24.88>cmtex10}{}
\DeclareFontShape{OT1}{cmtex}{m}{it}
  {<-> ssub * cmtt/m/it}{}
\newcommand{\texfamily}{\fontfamily{cmtex}\selectfont}
\DeclareFontShape{OT1}{cmtt}{bx}{n}
  {<5><6><7><8>cmtt8
   <9>cmbtt9
   <10><10.95><12><14.4><17.28><20.74><24.88>cmbtt10}{}
\DeclareFontShape{OT1}{cmtex}{bx}{n}
  {<-> ssub * cmtt/bx/n}{}
\newcommand{\tex}[1]{\text{\texfamily#1}}	% NEU

\newcommand{\Sp}{\hskip.33334em\relax}


\newcommand{\Conid}[1]{\mathit{#1}}
\newcommand{\Varid}[1]{\mathit{#1}}
\newcommand{\anonymous}{\kern0.06em \vbox{\hrule\@width.5em}}
\newcommand{\plus}{\mathbin{+\!\!\!+}}
\newcommand{\bind}{\mathbin{>\!\!\!>\mkern-6.7mu=}}
\newcommand{\rbind}{\mathbin{=\mkern-6.7mu<\!\!\!<}}% suggested by Neil Mitchell
\newcommand{\sequ}{\mathbin{>\!\!\!>}}
\renewcommand{\leq}{\leqslant}
\renewcommand{\geq}{\geqslant}
\usepackage{polytable}

%mathindent has to be defined
\@ifundefined{mathindent}%
  {\newdimen\mathindent\mathindent\leftmargini}%
  {}%

\def\resethooks{%
  \global\let\SaveRestoreHook\empty
  \global\let\ColumnHook\empty}
\newcommand*{\savecolumns}[1][default]%
  {\g@addto@macro\SaveRestoreHook{\savecolumns[#1]}}
\newcommand*{\restorecolumns}[1][default]%
  {\g@addto@macro\SaveRestoreHook{\restorecolumns[#1]}}
\newcommand*{\aligncolumn}[2]%
  {\g@addto@macro\ColumnHook{\column{#1}{#2}}}

\resethooks

\newcommand{\onelinecommentchars}{\quad-{}- }
\newcommand{\commentbeginchars}{\enskip\{-}
\newcommand{\commentendchars}{-\}\enskip}

\newcommand{\visiblecomments}{%
  \let\onelinecomment=\onelinecommentchars
  \let\commentbegin=\commentbeginchars
  \let\commentend=\commentendchars}

\newcommand{\invisiblecomments}{%
  \let\onelinecomment=\empty
  \let\commentbegin=\empty
  \let\commentend=\empty}

\visiblecomments

\newlength{\blanklineskip}
\setlength{\blanklineskip}{0.66084ex}

\newcommand{\hsindent}[1]{\quad}% default is fixed indentation
\let\hspre\empty
\let\hspost\empty
\newcommand{\NB}{\textbf{NB}}
\newcommand{\Todo}[1]{$\langle$\textbf{To do:}~#1$\rangle$}

\EndFmtInput
\makeatother
%
%
%
%
%
%
% This package provides two environments suitable to take the place
% of hscode, called "plainhscode" and "arrayhscode". 
%
% The plain environment surrounds each code block by vertical space,
% and it uses \abovedisplayskip and \belowdisplayskip to get spacing
% similar to formulas. Note that if these dimensions are changed,
% the spacing around displayed math formulas changes as well.
% All code is indented using \leftskip.
%
% Changed 19.08.2004 to reflect changes in colorcode. Should work with
% CodeGroup.sty.
%
\ReadOnlyOnce{polycode.fmt}%
\makeatletter

\newcommand{\hsnewpar}[1]%
  {{\parskip=0pt\parindent=0pt\par\vskip #1\noindent}}

% can be used, for instance, to redefine the code size, by setting the
% command to \small or something alike
\newcommand{\hscodestyle}{}

% The command \sethscode can be used to switch the code formatting
% behaviour by mapping the hscode environment in the subst directive
% to a new LaTeX environment.

\newcommand{\sethscode}[1]%
  {\expandafter\let\expandafter\hscode\csname #1\endcsname
   \expandafter\let\expandafter\endhscode\csname end#1\endcsname}

% "compatibility" mode restores the non-polycode.fmt layout.

\newenvironment{compathscode}%
  {\par\noindent
   \advance\leftskip\mathindent
   \hscodestyle
   \let\\=\@normalcr
   \let\hspre\(\let\hspost\)%
   \pboxed}%
  {\endpboxed\)%
   \par\noindent
   \ignorespacesafterend}

\newcommand{\compaths}{\sethscode{compathscode}}

% "plain" mode is the proposed default.
% It should now work with \centering.
% This required some changes. The old version
% is still available for reference as oldplainhscode.

\newenvironment{plainhscode}%
  {\hsnewpar\abovedisplayskip
   \advance\leftskip\mathindent
   \hscodestyle
   \let\hspre\(\let\hspost\)%
   \pboxed}%
  {\endpboxed%
   \hsnewpar\belowdisplayskip
   \ignorespacesafterend}

\newenvironment{oldplainhscode}%
  {\hsnewpar\abovedisplayskip
   \advance\leftskip\mathindent
   \hscodestyle
   \let\\=\@normalcr
   \(\pboxed}%
  {\endpboxed\)%
   \hsnewpar\belowdisplayskip
   \ignorespacesafterend}

% Here, we make plainhscode the default environment.

\newcommand{\plainhs}{\sethscode{plainhscode}}
\newcommand{\oldplainhs}{\sethscode{oldplainhscode}}
\plainhs

% The arrayhscode is like plain, but makes use of polytable's
% parray environment which disallows page breaks in code blocks.

\newenvironment{arrayhscode}%
  {\hsnewpar\abovedisplayskip
   \advance\leftskip\mathindent
   \hscodestyle
   \let\\=\@normalcr
   \(\parray}%
  {\endparray\)%
   \hsnewpar\belowdisplayskip
   \ignorespacesafterend}

\newcommand{\arrayhs}{\sethscode{arrayhscode}}

% The mathhscode environment also makes use of polytable's parray 
% environment. It is supposed to be used only inside math mode 
% (I used it to typeset the type rules in my thesis).

\newenvironment{mathhscode}%
  {\parray}{\endparray}

\newcommand{\mathhs}{\sethscode{mathhscode}}

% texths is similar to mathhs, but works in text mode.

\newenvironment{texthscode}%
  {\(\parray}{\endparray\)}

\newcommand{\texths}{\sethscode{texthscode}}

% The framed environment places code in a framed box.

\def\codeframewidth{\arrayrulewidth}
\RequirePackage{calc}

\newenvironment{framedhscode}%
  {\parskip=\abovedisplayskip\par\noindent
   \hscodestyle
   \arrayrulewidth=\codeframewidth
   \tabular{@{}|p{\linewidth-2\arraycolsep-2\arrayrulewidth-2pt}|@{}}%
   \hline\framedhslinecorrect\\{-1.5ex}%
   \let\endoflinesave=\\
   \let\\=\@normalcr
   \(\pboxed}%
  {\endpboxed\)%
   \framedhslinecorrect\endoflinesave{.5ex}\hline
   \endtabular
   \parskip=\belowdisplayskip\par\noindent
   \ignorespacesafterend}

\newcommand{\framedhslinecorrect}[2]%
  {#1[#2]}

\newcommand{\framedhs}{\sethscode{framedhscode}}

% The inlinehscode environment is an experimental environment
% that can be used to typeset displayed code inline.

\newenvironment{inlinehscode}%
  {\(\def\column##1##2{}%
   \let\>\undefined\let\<\undefined\let\\\undefined
   \newcommand\>[1][]{}\newcommand\<[1][]{}\newcommand\\[1][]{}%
   \def\fromto##1##2##3{##3}%
   \def\nextline{}}{\) }%

\newcommand{\inlinehs}{\sethscode{inlinehscode}}

% The joincode environment is a separate environment that
% can be used to surround and thereby connect multiple code
% blocks.

\newenvironment{joincode}%
  {\let\orighscode=\hscode
   \let\origendhscode=\endhscode
   \def\endhscode{\def\hscode{\endgroup\def\@currenvir{hscode}\\}\begingroup}
   %\let\SaveRestoreHook=\empty
   %\let\ColumnHook=\empty
   %\let\resethooks=\empty
   \orighscode\def\hscode{\endgroup\def\@currenvir{hscode}}}%
  {\origendhscode
   \global\let\hscode=\orighscode
   \global\let\endhscode=\origendhscode}%

\makeatother
\EndFmtInput
%

\ReadOnlyOnce{Formatting.fmt}%
\makeatletter

\let\Varid\mathit
\let\Conid\mathsf

\def\commentbegin{\quad\{\ }
\def\commentend{\}}

\newcommand{\ty}[1]{\Conid{#1}}
\newcommand{\con}[1]{\Conid{#1}}
\newcommand{\id}[1]{\Varid{#1}}
\newcommand{\cl}[1]{\Varid{#1}}
\newcommand{\opsym}[1]{\mathrel{#1}}

\newcommand\Keyword[1]{\textbf{\textsf{#1}}}
\newcommand\Hide{\mathbin{\downarrow}}
\newcommand\Reveal{\mathbin{\uparrow}}


%% Paper-specific keywords


\makeatother
\EndFmtInput

\section{Type-level Functions}
\label{sec:type-level-fun}

\subsection{Closed Type Families}

Type families have a wide variety of applications. They can appear inside type
 classes\cite{tfclass,tfsynonym}, or at toplevel. Toplevel type families
 can be used to compute over types, they come in two forms: open\cite{tfopen}
 and closed \cite{tfclosed}.

We choose \emph{closed type families}, because it allows overlapping instances,
 and we need none of the extensibility provided by open type families.
For example, consider both term-level and type-level \ensuremath{\mathrel{\wedge}}:

\begin{hscode}\SaveRestoreHook
\column{B}{@{}>{\hspre}l<{\hspost}@{}}%
\column{5}{@{}>{\hspre}l<{\hspost}@{}}%
\column{6}{@{}>{\hspre}l<{\hspost}@{}}%
\column{14}{@{}>{\hspre}l<{\hspost}@{}}%
\column{19}{@{}>{\hspre}l<{\hspost}@{}}%
\column{E}{@{}>{\hspre}l<{\hspost}@{}}%
\>[B]{}(\mathrel{\wedge})\mathbin{::}\Conid{Bool}\to \Conid{Bool}\to \Conid{Bool}{}\<[E]%
\\
\>[B]{}\Conid{True}\mathrel{\wedge}\Conid{True}\mathrel{=}\Conid{True}{}\<[E]%
\\
\>[B]{}\Varid{a}{}\<[6]%
\>[6]{}\mathrel{\wedge}\Varid{b}{}\<[14]%
\>[14]{}\mathrel{=}\Conid{False}{}\<[E]%
\\[\blanklineskip]%
\>[B]{}\mathbf{type}\;\Varid{family}\;\Conid{And}\;(\Varid{a}\mathbin{::}\Conid{Bool})\;(\Varid{b}\mathbin{::}\Conid{Bool})\mathbin{::}\Conid{Bool}\;\mathbf{where}{}\<[E]%
\\
\>[B]{}\hsindent{5}{}\<[5]%
\>[5]{}\Conid{And}\;\Conid{True}\;\Conid{True}\mathrel{=}\Conid{True}{}\<[E]%
\\
\>[B]{}\hsindent{5}{}\<[5]%
\>[5]{}\Conid{And}\;\Varid{a}\;{}\<[14]%
\>[14]{}\Varid{b}{}\<[19]%
\>[19]{}\mathrel{=}\Conid{False}{}\<[E]%
\ColumnHook
\end{hscode}\resethooks

The first instance of \ensuremath{\Conid{And}} could be subsumed under a more
 general instance, \ensuremath{\Conid{And}\;\Varid{a}\;\Varid{b}}.
But the closedness allows these instances to be resolved in order, just like
 how cases are resolved in term-level functions. Also notice that how much
 \ensuremath{\Conid{And}} resembles to it's term-level sibling.

\subsection{Functions on Type-Level Dictionaries}

With closed type families, we could define functions on the type level.
 Let's begin with dictionary lookup.

\begin{hscode}\SaveRestoreHook
\column{B}{@{}>{\hspre}l<{\hspost}@{}}%
\column{5}{@{}>{\hspre}l<{\hspost}@{}}%
\column{29}{@{}>{\hspre}l<{\hspost}@{}}%
\column{E}{@{}>{\hspre}l<{\hspost}@{}}%
\>[B]{}\mathbf{type}\;\Varid{family}\;\Conid{Get}{}\<[E]%
\\
\>[B]{}\hsindent{5}{}\<[5]%
\>[5]{}(\Varid{xs}\mathbin{::}[\mskip1.5mu (\Conid{Symbol},\mathbin{*})\mskip1.5mu]){}\<[29]%
\>[29]{}\mbox{\onelinecomment  dictionary}{}\<[E]%
\\
\>[B]{}\hsindent{5}{}\<[5]%
\>[5]{}(\Varid{s}\mathbin{::}\Conid{Symbol}){}\<[29]%
\>[29]{}\mbox{\onelinecomment  key}{}\<[E]%
\\
\>[B]{}\hsindent{5}{}\<[5]%
\>[5]{}\mathbin{::}\mathbin{*}\mathbf{where}\;{}\<[29]%
\>[29]{}\mbox{\onelinecomment  type}{}\<[E]%
\\[\blanklineskip]%
\>[B]{}\hsindent{5}{}\<[5]%
\>[5]{}\Conid{Get}\;(\mathbin{’}(\Varid{s},\Varid{x})\mathbin{’:}\Varid{xs})\;\Varid{s}\mathrel{=}\Varid{x}{}\<[E]%
\\
\>[B]{}\hsindent{5}{}\<[5]%
\>[5]{}\Conid{Get}\;(\mathbin{’}(\Varid{t},\Varid{x})\mathbin{’:}\Varid{xs})\;\Varid{s}\mathrel{=}\Conid{Get}\;\Varid{xs}\;\Varid{s}{}\<[E]%
\ColumnHook
\end{hscode}\resethooks

Another benefit of closed type families is that type-level equality can be
expressed by unifying type variables with the same name.
\ensuremath{\Conid{Get}} takes two type arguments, a dictionary and a symbol.
If the key we are looking for unifies with the symbol of an entry, then
 \ensuremath{\Conid{Get}} returns the corresponding type, else it keeps
 searching down the rest of the dictionary.

\ensuremath{\Conid{Get}\;\mbox{\textquotesingle}[\!\;\mbox{\textquotesingle}(\!\;\text{\tt \char34 A\char34},\Conid{Int}\;\Conid{CLoseTPar}\;\!]\;\text{\tt \char34 A\char34}} evaluates to
\ensuremath{\Conid{Int}}.

But \ensuremath{\Conid{Get}\;\mbox{\textquotesingle}[\!\;\mbox{\textquotesingle}(\!\;\text{\tt \char34 A\char34},\Conid{Int}\;\Conid{CLoseTPar}\;\!]\;\text{\tt \char34 B\char34}} would get stuck.
That's because \ensuremath{\Conid{Get}} is a partial function on types,
 and these types are computed at compile-time. It wouldn't make
 much sense for a type checker to crash and throw a ``Non-exhaustive'' error or
 be non-terminating.

We could make \ensuremath{\Conid{Get}} total, as we would at the term level,
 with \ensuremath{\Conid{Maybe}}.

\begin{hscode}\SaveRestoreHook
\column{B}{@{}>{\hspre}l<{\hspost}@{}}%
\column{5}{@{}>{\hspre}l<{\hspost}@{}}%
\column{25}{@{}>{\hspre}l<{\hspost}@{}}%
\column{29}{@{}>{\hspre}l<{\hspost}@{}}%
\column{E}{@{}>{\hspre}l<{\hspost}@{}}%
\>[B]{}\mathbf{type}\;\Varid{family}\;\Conid{Get}{}\<[E]%
\\
\>[B]{}\hsindent{5}{}\<[5]%
\>[5]{}(\Varid{xs}\mathbin{::}[\mskip1.5mu (\Conid{Symbol},\mathbin{*})\mskip1.5mu]){}\<[29]%
\>[29]{}\mbox{\onelinecomment  dictionary}{}\<[E]%
\\
\>[B]{}\hsindent{5}{}\<[5]%
\>[5]{}(\Varid{s}\mathbin{::}\Conid{Symbol}){}\<[29]%
\>[29]{}\mbox{\onelinecomment  key}{}\<[E]%
\\
\>[B]{}\hsindent{5}{}\<[5]%
\>[5]{}\mathbin{::}\Conid{Maybe}\mathbin{*}\mathbf{where}\;{}\<[29]%
\>[29]{}\mbox{\onelinecomment  type}{}\<[E]%
\\[\blanklineskip]%
\>[B]{}\hsindent{5}{}\<[5]%
\>[5]{}\Conid{Get}\mathbin{’}[\mskip1.5mu \mskip1.5mu]\;{}\<[25]%
\>[25]{}\Varid{s}\mathrel{=}\Conid{Nothing}{}\<[E]%
\\
\>[B]{}\hsindent{5}{}\<[5]%
\>[5]{}\Conid{Get}\;(\mathbin{’}(\Varid{s},\Varid{x})\mathbin{’:}\Varid{xs})\;\Varid{s}\mathrel{=}\Conid{Just}\;\Varid{x}{}\<[E]%
\\
\>[B]{}\hsindent{5}{}\<[5]%
\>[5]{}\Conid{Get}\;(\mathbin{’}(\Varid{t},\Varid{x})\mathbin{’:}\Varid{xs})\;\Varid{s}\mathrel{=}\Conid{Get}\;\Varid{xs}\;\Varid{s}{}\<[E]%
\ColumnHook
\end{hscode}\resethooks
%
Other dictionary-related functions are defined in a similar fashion.

\begin{hscode}\SaveRestoreHook
\column{B}{@{}>{\hspre}l<{\hspost}@{}}%
\column{5}{@{}>{\hspre}l<{\hspost}@{}}%
\column{9}{@{}>{\hspre}l<{\hspost}@{}}%
\column{25}{@{}>{\hspre}l<{\hspost}@{}}%
\column{27}{@{}>{\hspre}l<{\hspost}@{}}%
\column{28}{@{}>{\hspre}l<{\hspost}@{}}%
\column{29}{@{}>{\hspre}l<{\hspost}@{}}%
\column{E}{@{}>{\hspre}l<{\hspost}@{}}%
\>[B]{}\mbox{\onelinecomment  inserts or updates an entry}{}\<[E]%
\\
\>[B]{}\mathbf{type}\;\Varid{family}\;\Conid{Set}{}\<[E]%
\\
\>[B]{}\hsindent{5}{}\<[5]%
\>[5]{}(\Varid{xs}\mathbin{::}[\mskip1.5mu (\Conid{Symbol},\mathbin{*})\mskip1.5mu]){}\<[29]%
\>[29]{}\mbox{\onelinecomment  old dictionary}{}\<[E]%
\\
\>[B]{}\hsindent{5}{}\<[5]%
\>[5]{}(\Varid{s}\mathbin{::}\Conid{Symbol}){}\<[29]%
\>[29]{}\mbox{\onelinecomment  key}{}\<[E]%
\\
\>[B]{}\hsindent{5}{}\<[5]%
\>[5]{}(\Varid{x}\mathbin{::}\mathbin{*}){}\<[29]%
\>[29]{}\mbox{\onelinecomment  type}{}\<[E]%
\\
\>[B]{}\hsindent{5}{}\<[5]%
\>[5]{}\mathbin{::}[\mskip1.5mu (\Conid{Symbol},\mathbin{*})\mskip1.5mu]\;\mathbf{where}\;{}\<[29]%
\>[29]{}\mbox{\onelinecomment  new dictionary}{}\<[E]%
\\[\blanklineskip]%
\>[B]{}\hsindent{5}{}\<[5]%
\>[5]{}\Conid{Set}\mathbin{’}[\mskip1.5mu \mskip1.5mu]\;{}\<[25]%
\>[25]{}\Varid{s}\;\Varid{x}\mathrel{=}\mathbin{’}[\mskip1.5mu \mathbin{’}(\Varid{s},\Varid{x})\mskip1.5mu]{}\<[E]%
\\
\>[B]{}\hsindent{5}{}\<[5]%
\>[5]{}\Conid{Set}\;(\mathbin{’}(\Varid{s},\Varid{y})\mathbin{’:}\Varid{xs})\;\Varid{s}\;\Varid{x}\mathrel{=}(\mathbin{’}(\Varid{s},\Varid{x})\mathbin{’:}\Varid{xs}){}\<[E]%
\\
\>[B]{}\hsindent{5}{}\<[5]%
\>[5]{}\Conid{Set}\;(\mathbin{’}(\Varid{t},\Varid{y})\mathbin{’:}\Varid{xs})\;\Varid{s}\;\Varid{x}\mathrel{=}{}\<[E]%
\\
\>[5]{}\hsindent{4}{}\<[9]%
\>[9]{}\mathbin{’}(\Varid{t},\Varid{y})\mathbin{’:}(\Conid{Set}\;\Varid{xs}\;\Varid{s}\;\Varid{x}){}\<[E]%
\\[\blanklineskip]%
\>[B]{}\mbox{\onelinecomment  removes an entry}{}\<[E]%
\\
\>[B]{}\mathbf{type}\;\Varid{family}\;\Conid{Del}{}\<[E]%
\\
\>[B]{}\hsindent{5}{}\<[5]%
\>[5]{}(\Varid{xs}\mathbin{::}[\mskip1.5mu (\Conid{Symbol},\mathbin{*})\mskip1.5mu]){}\<[29]%
\>[29]{}\mbox{\onelinecomment  old dictionary}{}\<[E]%
\\
\>[B]{}\hsindent{5}{}\<[5]%
\>[5]{}(\Varid{s}\mathbin{::}\Conid{Symbol}){}\<[29]%
\>[29]{}\mbox{\onelinecomment  key}{}\<[E]%
\\
\>[B]{}\hsindent{5}{}\<[5]%
\>[5]{}\mathbin{::}[\mskip1.5mu (\Conid{Symbol},\mathbin{*})\mskip1.5mu]\;\mathbf{where}\;{}\<[29]%
\>[29]{}\mbox{\onelinecomment  new dictionary}{}\<[E]%
\\[\blanklineskip]%
\>[B]{}\hsindent{5}{}\<[5]%
\>[5]{}\Conid{Del}\mathbin{’}[\mskip1.5mu \mskip1.5mu]\;\Varid{s}{}\<[27]%
\>[27]{}\mathrel{=}\mathbin{’}[\mskip1.5mu \mskip1.5mu]{}\<[E]%
\\
\>[B]{}\hsindent{5}{}\<[5]%
\>[5]{}\Conid{Del}\;(\mathbin{’}(\Varid{s},\Varid{y})\mathbin{’:}\Varid{xs})\;\Varid{s}\mathrel{=}\Varid{xs}{}\<[E]%
\\
\>[B]{}\hsindent{5}{}\<[5]%
\>[5]{}\Conid{Del}\;(\mathbin{’}(\Varid{t},\Varid{y})\mathbin{’:}\Varid{xs})\;\Varid{s}\mathrel{=}\mathbin{’}(\Varid{t},\Varid{y})\mathbin{’:}(\Conid{Del}\;\Varid{xs}\;\Varid{s}){}\<[E]%
\\[\blanklineskip]%
\>[B]{}\mbox{\onelinecomment  membership}{}\<[E]%
\\
\>[B]{}\mathbf{type}\;\Varid{family}\;\Conid{Member}{}\<[E]%
\\
\>[B]{}\hsindent{5}{}\<[5]%
\>[5]{}(\Varid{xs}\mathbin{::}[\mskip1.5mu (\Conid{Symbol},\mathbin{*})\mskip1.5mu]){}\<[29]%
\>[29]{}\mbox{\onelinecomment  dictionary}{}\<[E]%
\\
\>[B]{}\hsindent{5}{}\<[5]%
\>[5]{}(\Varid{s}\mathbin{::}\Conid{Symbol}){}\<[29]%
\>[29]{}\mbox{\onelinecomment  key}{}\<[E]%
\\
\>[B]{}\hsindent{5}{}\<[5]%
\>[5]{}\mathbin{::}\Conid{Bool}\;\mathbf{where}\;{}\<[29]%
\>[29]{}\mbox{\onelinecomment  exists?}{}\<[E]%
\\[\blanklineskip]%
\>[B]{}\hsindent{5}{}\<[5]%
\>[5]{}\Conid{Member}\mathbin{’}[\mskip1.5mu \mskip1.5mu]\;{}\<[28]%
\>[28]{}\Varid{s}\mathrel{=}\Conid{False}{}\<[E]%
\\
\>[B]{}\hsindent{5}{}\<[5]%
\>[5]{}\Conid{Member}\;(\mathbin{’}(\Varid{s},\Varid{x})\mathbin{’:}\Varid{xs})\;\Varid{s}\mathrel{=}\Conid{True}{}\<[E]%
\\
\>[B]{}\hsindent{5}{}\<[5]%
\>[5]{}\Conid{Member}\;(\mathbin{’}(\Varid{t},\Varid{x})\mathbin{’:}\Varid{xs})\;\Varid{s}\mathrel{=}\Conid{Member}\;\Varid{xs}\;\Varid{s}{}\<[E]%
\ColumnHook
\end{hscode}\resethooks

\subsection{Proxies and Singleton Types}

Now we could annotate the effects of a command in types. \text{DEL}
 removes a key from the current database, regardless of its type.

\begin{hscode}\SaveRestoreHook
\column{B}{@{}>{\hspre}l<{\hspost}@{}}%
\column{5}{@{}>{\hspre}l<{\hspost}@{}}%
\column{E}{@{}>{\hspre}l<{\hspost}@{}}%
\>[B]{}\Varid{del}\mathbin{::}\Conid{KnownSymbol}\;\Varid{s}{}\<[E]%
\\
\>[B]{}\hsindent{5}{}\<[5]%
\>[5]{}\Rightarrow \Conid{Proxy}\;\Varid{s}{}\<[E]%
\\
\>[B]{}\hsindent{5}{}\<[5]%
\>[5]{}\to \Conid{Popcorn}\;\Varid{xs}\;(\Conid{Del}\;\Varid{xs}\;\Varid{s})\;(\Conid{Either}\;\Conid{Reply}\;\Conid{Integer}){}\<[E]%
\\
\>[B]{}\Varid{del}\;\Varid{key}\mathrel{=}\Conid{Popcorn}\mathbin{\$}\Varid{\Conid{Hedis}.del}\;(\Varid{encodeKey}\;\Varid{key}){}\<[E]%
\ColumnHook
\end{hscode}\resethooks

\ensuremath{\Conid{KnownSymbol}} is a class that gives the string associated
 with a concrete type-level symbol, which can be retrieved with
 \ensuremath{\Varid{symbolVal}}.\footnotemark
 Where \ensuremath{\Varid{encodeKey}} converts \ensuremath{\Conid{Proxy}\;\Varid{s}} to
 \ensuremath{\Conid{ByteString}}.
\footnotetext{They are defined in \ensuremath{\Conid{\Conid{GHC}.TypeLits}}.}

\begin{hscode}\SaveRestoreHook
\column{B}{@{}>{\hspre}l<{\hspost}@{}}%
\column{E}{@{}>{\hspre}l<{\hspost}@{}}%
\>[B]{}\Varid{encodeKey}\mathbin{::}\Conid{KnownSymbol}\;\Varid{s}\Rightarrow \Conid{Proxy}\;\Varid{s}\to \Conid{ByteString}{}\<[E]%
\\
\>[B]{}\Varid{encodeKey}\mathrel{=}\Varid{encode}\mathbin{\cdot}\Varid{symbolVal}{}\<[E]%
\ColumnHook
\end{hscode}\resethooks

Since Haskell has a \emph{phase distinction, phasedistinction}, types are
 erased before runtime. It's impossible to obtain information directly from
 types, we can only do this indirectly, with
 \emph{singleton types, singletons}.

A singleton type is a type that has only one instance, and the instance can be
 think of as the representative of the type at the realm of runtime values.

\ensuremath{\Conid{Proxy}}, as its name would suggest, can be used as
 singletons. It's a phantom type that could be indexed with any type.

\begin{hscode}\SaveRestoreHook
\column{B}{@{}>{\hspre}l<{\hspost}@{}}%
\column{E}{@{}>{\hspre}l<{\hspost}@{}}%
\>[B]{}\mathbf{data}\;\Conid{Proxy}\;\Varid{t}\mathrel{=}\Conid{Proxy}{}\<[E]%
\ColumnHook
\end{hscode}\resethooks

In the type of \ensuremath{\Varid{del}}, the type variable
 \ensuremath{\Varid{s}} is a \ensuremath{\Conid{Symbol}} that is decided by
 the argument of type \ensuremath{\Conid{Proxy}\;\Varid{s}}.
 To use \ensuremath{\Varid{del}}, we would have to apply it with a clumsy
 term-level proxy like this:

\begin{hscode}\SaveRestoreHook
\column{B}{@{}>{\hspre}l<{\hspost}@{}}%
\column{E}{@{}>{\hspre}l<{\hspost}@{}}%
\>[B]{}\Varid{del}\;(\Conid{Proxy}\mathbin{::}\Conid{Proxy}\;\text{\tt \char34 A\char34}){}\<[E]%
\ColumnHook
\end{hscode}\resethooks

%% ODER: format ==         = "\mathrel{==}"
%% ODER: format /=         = "\neq "
%
%
\makeatletter
\@ifundefined{lhs2tex.lhs2tex.sty.read}%
  {\@namedef{lhs2tex.lhs2tex.sty.read}{}%
   \newcommand\SkipToFmtEnd{}%
   \newcommand\EndFmtInput{}%
   \long\def\SkipToFmtEnd#1\EndFmtInput{}%
  }\SkipToFmtEnd

\newcommand\ReadOnlyOnce[1]{\@ifundefined{#1}{\@namedef{#1}{}}\SkipToFmtEnd}
\usepackage{amstext}
\usepackage{amssymb}
\usepackage{stmaryrd}
\DeclareFontFamily{OT1}{cmtex}{}
\DeclareFontShape{OT1}{cmtex}{m}{n}
  {<5><6><7><8>cmtex8
   <9>cmtex9
   <10><10.95><12><14.4><17.28><20.74><24.88>cmtex10}{}
\DeclareFontShape{OT1}{cmtex}{m}{it}
  {<-> ssub * cmtt/m/it}{}
\newcommand{\texfamily}{\fontfamily{cmtex}\selectfont}
\DeclareFontShape{OT1}{cmtt}{bx}{n}
  {<5><6><7><8>cmtt8
   <9>cmbtt9
   <10><10.95><12><14.4><17.28><20.74><24.88>cmbtt10}{}
\DeclareFontShape{OT1}{cmtex}{bx}{n}
  {<-> ssub * cmtt/bx/n}{}
\newcommand{\tex}[1]{\text{\texfamily#1}}	% NEU

\newcommand{\Sp}{\hskip.33334em\relax}


\newcommand{\Conid}[1]{\mathit{#1}}
\newcommand{\Varid}[1]{\mathit{#1}}
\newcommand{\anonymous}{\kern0.06em \vbox{\hrule\@width.5em}}
\newcommand{\plus}{\mathbin{+\!\!\!+}}
\newcommand{\bind}{\mathbin{>\!\!\!>\mkern-6.7mu=}}
\newcommand{\rbind}{\mathbin{=\mkern-6.7mu<\!\!\!<}}% suggested by Neil Mitchell
\newcommand{\sequ}{\mathbin{>\!\!\!>}}
\renewcommand{\leq}{\leqslant}
\renewcommand{\geq}{\geqslant}
\usepackage{polytable}

%mathindent has to be defined
\@ifundefined{mathindent}%
  {\newdimen\mathindent\mathindent\leftmargini}%
  {}%

\def\resethooks{%
  \global\let\SaveRestoreHook\empty
  \global\let\ColumnHook\empty}
\newcommand*{\savecolumns}[1][default]%
  {\g@addto@macro\SaveRestoreHook{\savecolumns[#1]}}
\newcommand*{\restorecolumns}[1][default]%
  {\g@addto@macro\SaveRestoreHook{\restorecolumns[#1]}}
\newcommand*{\aligncolumn}[2]%
  {\g@addto@macro\ColumnHook{\column{#1}{#2}}}

\resethooks

\newcommand{\onelinecommentchars}{\quad-{}- }
\newcommand{\commentbeginchars}{\enskip\{-}
\newcommand{\commentendchars}{-\}\enskip}

\newcommand{\visiblecomments}{%
  \let\onelinecomment=\onelinecommentchars
  \let\commentbegin=\commentbeginchars
  \let\commentend=\commentendchars}

\newcommand{\invisiblecomments}{%
  \let\onelinecomment=\empty
  \let\commentbegin=\empty
  \let\commentend=\empty}

\visiblecomments

\newlength{\blanklineskip}
\setlength{\blanklineskip}{0.66084ex}

\newcommand{\hsindent}[1]{\quad}% default is fixed indentation
\let\hspre\empty
\let\hspost\empty
\newcommand{\NB}{\textbf{NB}}
\newcommand{\Todo}[1]{$\langle$\textbf{To do:}~#1$\rangle$}

\EndFmtInput
\makeatother
%
%
%
%
%
%
% This package provides two environments suitable to take the place
% of hscode, called "plainhscode" and "arrayhscode". 
%
% The plain environment surrounds each code block by vertical space,
% and it uses \abovedisplayskip and \belowdisplayskip to get spacing
% similar to formulas. Note that if these dimensions are changed,
% the spacing around displayed math formulas changes as well.
% All code is indented using \leftskip.
%
% Changed 19.08.2004 to reflect changes in colorcode. Should work with
% CodeGroup.sty.
%
\ReadOnlyOnce{polycode.fmt}%
\makeatletter

\newcommand{\hsnewpar}[1]%
  {{\parskip=0pt\parindent=0pt\par\vskip #1\noindent}}

% can be used, for instance, to redefine the code size, by setting the
% command to \small or something alike
\newcommand{\hscodestyle}{}

% The command \sethscode can be used to switch the code formatting
% behaviour by mapping the hscode environment in the subst directive
% to a new LaTeX environment.

\newcommand{\sethscode}[1]%
  {\expandafter\let\expandafter\hscode\csname #1\endcsname
   \expandafter\let\expandafter\endhscode\csname end#1\endcsname}

% "compatibility" mode restores the non-polycode.fmt layout.

\newenvironment{compathscode}%
  {\par\noindent
   \advance\leftskip\mathindent
   \hscodestyle
   \let\\=\@normalcr
   \let\hspre\(\let\hspost\)%
   \pboxed}%
  {\endpboxed\)%
   \par\noindent
   \ignorespacesafterend}

\newcommand{\compaths}{\sethscode{compathscode}}

% "plain" mode is the proposed default.
% It should now work with \centering.
% This required some changes. The old version
% is still available for reference as oldplainhscode.

\newenvironment{plainhscode}%
  {\hsnewpar\abovedisplayskip
   \advance\leftskip\mathindent
   \hscodestyle
   \let\hspre\(\let\hspost\)%
   \pboxed}%
  {\endpboxed%
   \hsnewpar\belowdisplayskip
   \ignorespacesafterend}

\newenvironment{oldplainhscode}%
  {\hsnewpar\abovedisplayskip
   \advance\leftskip\mathindent
   \hscodestyle
   \let\\=\@normalcr
   \(\pboxed}%
  {\endpboxed\)%
   \hsnewpar\belowdisplayskip
   \ignorespacesafterend}

% Here, we make plainhscode the default environment.

\newcommand{\plainhs}{\sethscode{plainhscode}}
\newcommand{\oldplainhs}{\sethscode{oldplainhscode}}
\plainhs

% The arrayhscode is like plain, but makes use of polytable's
% parray environment which disallows page breaks in code blocks.

\newenvironment{arrayhscode}%
  {\hsnewpar\abovedisplayskip
   \advance\leftskip\mathindent
   \hscodestyle
   \let\\=\@normalcr
   \(\parray}%
  {\endparray\)%
   \hsnewpar\belowdisplayskip
   \ignorespacesafterend}

\newcommand{\arrayhs}{\sethscode{arrayhscode}}

% The mathhscode environment also makes use of polytable's parray 
% environment. It is supposed to be used only inside math mode 
% (I used it to typeset the type rules in my thesis).

\newenvironment{mathhscode}%
  {\parray}{\endparray}

\newcommand{\mathhs}{\sethscode{mathhscode}}

% texths is similar to mathhs, but works in text mode.

\newenvironment{texthscode}%
  {\(\parray}{\endparray\)}

\newcommand{\texths}{\sethscode{texthscode}}

% The framed environment places code in a framed box.

\def\codeframewidth{\arrayrulewidth}
\RequirePackage{calc}

\newenvironment{framedhscode}%
  {\parskip=\abovedisplayskip\par\noindent
   \hscodestyle
   \arrayrulewidth=\codeframewidth
   \tabular{@{}|p{\linewidth-2\arraycolsep-2\arrayrulewidth-2pt}|@{}}%
   \hline\framedhslinecorrect\\{-1.5ex}%
   \let\endoflinesave=\\
   \let\\=\@normalcr
   \(\pboxed}%
  {\endpboxed\)%
   \framedhslinecorrect\endoflinesave{.5ex}\hline
   \endtabular
   \parskip=\belowdisplayskip\par\noindent
   \ignorespacesafterend}

\newcommand{\framedhslinecorrect}[2]%
  {#1[#2]}

\newcommand{\framedhs}{\sethscode{framedhscode}}

% The inlinehscode environment is an experimental environment
% that can be used to typeset displayed code inline.

\newenvironment{inlinehscode}%
  {\(\def\column##1##2{}%
   \let\>\undefined\let\<\undefined\let\\\undefined
   \newcommand\>[1][]{}\newcommand\<[1][]{}\newcommand\\[1][]{}%
   \def\fromto##1##2##3{##3}%
   \def\nextline{}}{\) }%

\newcommand{\inlinehs}{\sethscode{inlinehscode}}

% The joincode environment is a separate environment that
% can be used to surround and thereby connect multiple code
% blocks.

\newenvironment{joincode}%
  {\let\orighscode=\hscode
   \let\origendhscode=\endhscode
   \def\endhscode{\def\hscode{\endgroup\def\@currenvir{hscode}\\}\begingroup}
   %\let\SaveRestoreHook=\empty
   %\let\ColumnHook=\empty
   %\let\resethooks=\empty
   \orighscode\def\hscode{\endgroup\def\@currenvir{hscode}}}%
  {\origendhscode
   \global\let\hscode=\orighscode
   \global\let\endhscode=\origendhscode}%

\makeatother
\EndFmtInput
%

\ReadOnlyOnce{Formatting.fmt}%
\makeatletter

\let\Varid\mathit
\let\Conid\mathsf

\def\commentbegin{\quad\{\ }
\def\commentend{\}}

\newcommand{\ty}[1]{\Conid{#1}}
\newcommand{\con}[1]{\Conid{#1}}
\newcommand{\id}[1]{\Varid{#1}}
\newcommand{\cl}[1]{\Varid{#1}}
\newcommand{\opsym}[1]{\mathrel{#1}}

\newcommand\Keyword[1]{\textbf{\textsf{#1}}}
\newcommand\Hide{\mathbin{\downarrow}}
\newcommand\Reveal{\mathbin{\uparrow}}


%% Paper-specific keywords


\makeatother
\EndFmtInput

\section{Making Redis Polymorphic}
\label{sec:polymorphic-redis}

Redis supports many different datatypes, these datatypes as can be viewed as
 \emph{containers} of strings. For example, lists (of strings),
 sets (of strings), and strings themselves.

\subsection{Denoting Containers}
Most Redis commands only work with a certain type of these containers. To
 annotate what container a key is associated with, we introduce these types for
 the universe of containers.

\begin{hscode}\SaveRestoreHook
\column{B}{@{}>{\hspre}l<{\hspost}@{}}%
\column{E}{@{}>{\hspre}l<{\hspost}@{}}%
\>[B]{}\mathbf{data}\;\Conid{Strings}{}\<[E]%
\\
\>[B]{}\mathbf{data}\;\Conid{Lists}{}\<[E]%
\\
\>[B]{}\mathbf{data}\;\Conid{Sets}{}\<[E]%
\\
\>[B]{}\mathbin{...}{}\<[E]%
\ColumnHook
\end{hscode}\resethooks

\text{SET} stores a string, regardless the datatype the key was
 associated with. Now we could implement \text{SET} like this:

\begin{hscode}\SaveRestoreHook
\column{B}{@{}>{\hspre}l<{\hspost}@{}}%
\column{5}{@{}>{\hspre}l<{\hspost}@{}}%
\column{25}{@{}>{\hspre}l<{\hspost}@{}}%
\column{E}{@{}>{\hspre}l<{\hspost}@{}}%
\>[B]{}\Varid{set}\mathbin{::}\Conid{KnownSymbol}\;\Varid{s}{}\<[E]%
\\
\>[B]{}\hsindent{5}{}\<[5]%
\>[5]{}\Rightarrow \Conid{Proxy}\;\Varid{s}{}\<[E]%
\\
\>[B]{}\hsindent{5}{}\<[5]%
\>[5]{}\to \Conid{ByteString}{}\<[25]%
\>[25]{}\mbox{\onelinecomment  data to store}{}\<[E]%
\\
\>[B]{}\hsindent{5}{}\<[5]%
\>[5]{}\to \Conid{Edis}\;\Varid{xs}\;(\Conid{Set}\;\Varid{xs}\;\Varid{s}\;\Conid{Strings})\;(\Conid{Either}\;\Conid{Reply}\;\Conid{Status}){}\<[E]%
\\
\>[B]{}\Varid{set}\;\Varid{key}\;\Varid{val}\mathrel{=}\Conid{Edis}\mathbin{\$}\Varid{\Conid{Hedis}.set}\;(\Varid{encodeKey}\;\Varid{key})\;\Varid{val}{}\<[E]%
\ColumnHook
\end{hscode}\resethooks

After \text{SET}, the key will be associated with
 \ensuremath{\Conid{Strings}} in the dictionary, indicating that it's a string.

\subsection{Automatic Data Serialization}

But in the real world, raw binary strings are hardly useful, people would
 usually serialize their data into strings before storing them, and deserialize
 them back when in need.

Instead of letting users writing these boilerplates, we can do these
 serializations/deserializations for them. With the help from
 \text{cereal}, a binary serialization library.
 \text{cereal} comes with these two functions:

\begin{hscode}\SaveRestoreHook
\column{B}{@{}>{\hspre}l<{\hspost}@{}}%
\column{8}{@{}>{\hspre}l<{\hspost}@{}}%
\column{E}{@{}>{\hspre}l<{\hspost}@{}}%
\>[B]{}\Varid{encode}\mathbin{::}\Conid{Serialize}\;\Varid{a}{}\<[E]%
\\
\>[B]{}\hsindent{8}{}\<[8]%
\>[8]{}\Rightarrow \Varid{a}\to \Conid{ByteString}\;\Conid{Source}{}\<[E]%
\\[\blanklineskip]%
\>[B]{}\Varid{decode}\mathbin{::}\Conid{Serialize}\;\Varid{a}{}\<[E]%
\\
\>[B]{}\hsindent{8}{}\<[8]%
\>[8]{}\Rightarrow \Conid{ByteString}\to \Conid{Either}\;\Conid{String}\;\Varid{a}{}\<[E]%
\ColumnHook
\end{hscode}\resethooks

Which would do all the works for us, as long as the datatype it's handling is
 an instance of class \ensuremath{\Conid{Serialize}}.\footnotemark

\footnotetext{The methods of \ensuremath{\Conid{Serialize}} will have default
 generic implementations for all datatypes with some language extensions
 enabled, no sweat!}

\subsection{Extending container types}
We rename container types and extend it with an extra type argument,
 to indicate what kind of encoded value it's holding.

\begin{hscode}\SaveRestoreHook
\column{B}{@{}>{\hspre}l<{\hspost}@{}}%
\column{E}{@{}>{\hspre}l<{\hspost}@{}}%
\>[B]{}\mathbf{data}\;\Conid{StringOf}\;\Varid{x}{}\<[E]%
\\
\>[B]{}\mathbf{data}\;\Conid{ListOf}\;\Varid{x}{}\<[E]%
\\
\>[B]{}\mathbf{data}\;\Conid{SetOf}\;\Varid{x}{}\<[E]%
\\
\>[B]{}\mathbin{...}{}\<[E]%
\ColumnHook
\end{hscode}\resethooks

\ensuremath{\Varid{set}} reimplemented with extended container types:

\begin{hscode}\SaveRestoreHook
\column{B}{@{}>{\hspre}l<{\hspost}@{}}%
\column{5}{@{}>{\hspre}l<{\hspost}@{}}%
\column{16}{@{}>{\hspre}l<{\hspost}@{}}%
\column{E}{@{}>{\hspre}l<{\hspost}@{}}%
\>[B]{}\Varid{set}\mathbin{::}(\Conid{KnownSymbol}\;\Varid{s},\Conid{Serialize}\;\Varid{x}){}\<[E]%
\\
\>[B]{}\hsindent{5}{}\<[5]%
\>[5]{}\Rightarrow \Conid{Proxy}\;\Varid{s}{}\<[E]%
\\
\>[B]{}\hsindent{5}{}\<[5]%
\>[5]{}\to \Varid{x}{}\<[16]%
\>[16]{}\mbox{\onelinecomment  can be anything, as long as it's serializable}{}\<[E]%
\\
\>[B]{}\hsindent{5}{}\<[5]%
\>[5]{}\to \Conid{Edis}\;\Varid{xs}\;(\Conid{Set}\;\Varid{xs}\;\Varid{s}\;(\Conid{StringOf}\;\Varid{x}))\;(\Conid{Either}\;\Conid{Reply}\;\Conid{Status}){}\<[E]%
\\
\>[B]{}\Varid{set}\;\Varid{key}\;\Varid{val}\mathrel{=}\Conid{Edis}\mathbin{\$}\Varid{\Conid{Hedis}.set}\;(\Varid{encodeKey}\;\Varid{key})\;(\Varid{encode}\;\Varid{val}){}\<[E]%
\ColumnHook
\end{hscode}\resethooks

For example, if we execute \ensuremath{\Varid{set}\;(\Conid{Proxy}\mathbin{::}\Conid{Proxy}\;\text{\tt \char34 A\char34})\;\Conid{True}},
 a new entry \text{'}\ensuremath{(\text{\tt \char34 A\char34},\Conid{StringOf}\;\Conid{Bool})} will
 be inserted to the dictionary.

\subsection{Handling \text{INCR}}

Commands such as \text{INCR} and \text{INCRBYFLOAT}, are
 not only container-specific, they also have some requirements on what types of
 value they could operate with.

We could handle this by mapping Redis's strings of integers and floats to
 Haskell's \text{Integer} and \text{Double}.

\begin{hscode}\SaveRestoreHook
\column{B}{@{}>{\hspre}l<{\hspost}@{}}%
\column{5}{@{}>{\hspre}l<{\hspost}@{}}%
\column{6}{@{}>{\hspre}l<{\hspost}@{}}%
\column{7}{@{}>{\hspre}l<{\hspost}@{}}%
\column{E}{@{}>{\hspre}l<{\hspost}@{}}%
\>[B]{}\Varid{incr}\mathbin{::}(\Conid{KnownSymbol}\;\Varid{s}{}\<[E]%
\\
\>[B]{}\hsindent{7}{}\<[7]%
\>[7]{},\Conid{Get}\;\Varid{xs}\;\Varid{s}\mathord{\sim}\Conid{Just}\;(\Conid{StringOf}\;\Conid{Integer})){}\<[E]%
\\
\>[B]{}\hsindent{6}{}\<[6]%
\>[6]{}\Rightarrow \Conid{Proxy}\;\Varid{s}{}\<[E]%
\\
\>[B]{}\hsindent{6}{}\<[6]%
\>[6]{}\to \Conid{Edis}\;\Varid{xs}\;\Varid{xs}\;(\Conid{Either}\;\Conid{Reply}\;\Conid{Integer}){}\<[E]%
\\
\>[B]{}\Varid{incr}\;\Varid{key}\mathrel{=}\Conid{Edis}\mathbin{\$}\Varid{\Conid{Hedis}.incr}\;(\Varid{encodeKey}\;\Varid{key}){}\<[E]%
\\[\blanklineskip]%
\>[B]{}\Varid{incrbyfloat}\mathbin{::}(\Conid{KnownSymbol}\;\Varid{s}{}\<[E]%
\\
\>[B]{}\hsindent{6}{}\<[6]%
\>[6]{},\Conid{Get}\;\Varid{xs}\;\Varid{s}\mathord{\sim}\Conid{Just}\;(\Conid{StringOf}\;\Conid{Double})){}\<[E]%
\\
\>[B]{}\hsindent{5}{}\<[5]%
\>[5]{}\Rightarrow \Conid{Proxy}\;\Varid{s}{}\<[E]%
\\
\>[B]{}\hsindent{5}{}\<[5]%
\>[5]{}\to \Conid{Double}{}\<[E]%
\\
\>[B]{}\hsindent{5}{}\<[5]%
\>[5]{}\to \Conid{Edis}\;\Varid{xs}\;\Varid{xs}\;(\Conid{Either}\;\Conid{Reply}\;\Conid{Double}){}\<[E]%
\\
\>[B]{}\Varid{incrbyfloat}\;\Varid{key}\;\Varid{n}\mathrel{=}{}\<[E]%
\\
\>[B]{}\hsindent{5}{}\<[5]%
\>[5]{}\Conid{Edis}\mathbin{\$}\Varid{\Conid{Hedis}.incrbyfloat}\;(\Varid{encodeKey}\;\Varid{key})\;\Varid{n}{}\<[E]%
\ColumnHook
\end{hscode}\resethooks

%% ODER: format ==         = "\mathrel{==}"
%% ODER: format /=         = "\neq "
%
%
\makeatletter
\@ifundefined{lhs2tex.lhs2tex.sty.read}%
  {\@namedef{lhs2tex.lhs2tex.sty.read}{}%
   \newcommand\SkipToFmtEnd{}%
   \newcommand\EndFmtInput{}%
   \long\def\SkipToFmtEnd#1\EndFmtInput{}%
  }\SkipToFmtEnd

\newcommand\ReadOnlyOnce[1]{\@ifundefined{#1}{\@namedef{#1}{}}\SkipToFmtEnd}
\usepackage{amstext}
\usepackage{amssymb}
\usepackage{stmaryrd}
\DeclareFontFamily{OT1}{cmtex}{}
\DeclareFontShape{OT1}{cmtex}{m}{n}
  {<5><6><7><8>cmtex8
   <9>cmtex9
   <10><10.95><12><14.4><17.28><20.74><24.88>cmtex10}{}
\DeclareFontShape{OT1}{cmtex}{m}{it}
  {<-> ssub * cmtt/m/it}{}
\newcommand{\texfamily}{\fontfamily{cmtex}\selectfont}
\DeclareFontShape{OT1}{cmtt}{bx}{n}
  {<5><6><7><8>cmtt8
   <9>cmbtt9
   <10><10.95><12><14.4><17.28><20.74><24.88>cmbtt10}{}
\DeclareFontShape{OT1}{cmtex}{bx}{n}
  {<-> ssub * cmtt/bx/n}{}
\newcommand{\tex}[1]{\text{\texfamily#1}}	% NEU

\newcommand{\Sp}{\hskip.33334em\relax}


\newcommand{\Conid}[1]{\mathit{#1}}
\newcommand{\Varid}[1]{\mathit{#1}}
\newcommand{\anonymous}{\kern0.06em \vbox{\hrule\@width.5em}}
\newcommand{\plus}{\mathbin{+\!\!\!+}}
\newcommand{\bind}{\mathbin{>\!\!\!>\mkern-6.7mu=}}
\newcommand{\rbind}{\mathbin{=\mkern-6.7mu<\!\!\!<}}% suggested by Neil Mitchell
\newcommand{\sequ}{\mathbin{>\!\!\!>}}
\renewcommand{\leq}{\leqslant}
\renewcommand{\geq}{\geqslant}
\usepackage{polytable}

%mathindent has to be defined
\@ifundefined{mathindent}%
  {\newdimen\mathindent\mathindent\leftmargini}%
  {}%

\def\resethooks{%
  \global\let\SaveRestoreHook\empty
  \global\let\ColumnHook\empty}
\newcommand*{\savecolumns}[1][default]%
  {\g@addto@macro\SaveRestoreHook{\savecolumns[#1]}}
\newcommand*{\restorecolumns}[1][default]%
  {\g@addto@macro\SaveRestoreHook{\restorecolumns[#1]}}
\newcommand*{\aligncolumn}[2]%
  {\g@addto@macro\ColumnHook{\column{#1}{#2}}}

\resethooks

\newcommand{\onelinecommentchars}{\quad-{}- }
\newcommand{\commentbeginchars}{\enskip\{-}
\newcommand{\commentendchars}{-\}\enskip}

\newcommand{\visiblecomments}{%
  \let\onelinecomment=\onelinecommentchars
  \let\commentbegin=\commentbeginchars
  \let\commentend=\commentendchars}

\newcommand{\invisiblecomments}{%
  \let\onelinecomment=\empty
  \let\commentbegin=\empty
  \let\commentend=\empty}

\visiblecomments

\newlength{\blanklineskip}
\setlength{\blanklineskip}{0.66084ex}

\newcommand{\hsindent}[1]{\quad}% default is fixed indentation
\let\hspre\empty
\let\hspost\empty
\newcommand{\NB}{\textbf{NB}}
\newcommand{\Todo}[1]{$\langle$\textbf{To do:}~#1$\rangle$}

\EndFmtInput
\makeatother
%
%
%
%
%
%
% This package provides two environments suitable to take the place
% of hscode, called "plainhscode" and "arrayhscode". 
%
% The plain environment surrounds each code block by vertical space,
% and it uses \abovedisplayskip and \belowdisplayskip to get spacing
% similar to formulas. Note that if these dimensions are changed,
% the spacing around displayed math formulas changes as well.
% All code is indented using \leftskip.
%
% Changed 19.08.2004 to reflect changes in colorcode. Should work with
% CodeGroup.sty.
%
\ReadOnlyOnce{polycode.fmt}%
\makeatletter

\newcommand{\hsnewpar}[1]%
  {{\parskip=0pt\parindent=0pt\par\vskip #1\noindent}}

% can be used, for instance, to redefine the code size, by setting the
% command to \small or something alike
\newcommand{\hscodestyle}{}

% The command \sethscode can be used to switch the code formatting
% behaviour by mapping the hscode environment in the subst directive
% to a new LaTeX environment.

\newcommand{\sethscode}[1]%
  {\expandafter\let\expandafter\hscode\csname #1\endcsname
   \expandafter\let\expandafter\endhscode\csname end#1\endcsname}

% "compatibility" mode restores the non-polycode.fmt layout.

\newenvironment{compathscode}%
  {\par\noindent
   \advance\leftskip\mathindent
   \hscodestyle
   \let\\=\@normalcr
   \let\hspre\(\let\hspost\)%
   \pboxed}%
  {\endpboxed\)%
   \par\noindent
   \ignorespacesafterend}

\newcommand{\compaths}{\sethscode{compathscode}}

% "plain" mode is the proposed default.
% It should now work with \centering.
% This required some changes. The old version
% is still available for reference as oldplainhscode.

\newenvironment{plainhscode}%
  {\hsnewpar\abovedisplayskip
   \advance\leftskip\mathindent
   \hscodestyle
   \let\hspre\(\let\hspost\)%
   \pboxed}%
  {\endpboxed%
   \hsnewpar\belowdisplayskip
   \ignorespacesafterend}

\newenvironment{oldplainhscode}%
  {\hsnewpar\abovedisplayskip
   \advance\leftskip\mathindent
   \hscodestyle
   \let\\=\@normalcr
   \(\pboxed}%
  {\endpboxed\)%
   \hsnewpar\belowdisplayskip
   \ignorespacesafterend}

% Here, we make plainhscode the default environment.

\newcommand{\plainhs}{\sethscode{plainhscode}}
\newcommand{\oldplainhs}{\sethscode{oldplainhscode}}
\plainhs

% The arrayhscode is like plain, but makes use of polytable's
% parray environment which disallows page breaks in code blocks.

\newenvironment{arrayhscode}%
  {\hsnewpar\abovedisplayskip
   \advance\leftskip\mathindent
   \hscodestyle
   \let\\=\@normalcr
   \(\parray}%
  {\endparray\)%
   \hsnewpar\belowdisplayskip
   \ignorespacesafterend}

\newcommand{\arrayhs}{\sethscode{arrayhscode}}

% The mathhscode environment also makes use of polytable's parray 
% environment. It is supposed to be used only inside math mode 
% (I used it to typeset the type rules in my thesis).

\newenvironment{mathhscode}%
  {\parray}{\endparray}

\newcommand{\mathhs}{\sethscode{mathhscode}}

% texths is similar to mathhs, but works in text mode.

\newenvironment{texthscode}%
  {\(\parray}{\endparray\)}

\newcommand{\texths}{\sethscode{texthscode}}

% The framed environment places code in a framed box.

\def\codeframewidth{\arrayrulewidth}
\RequirePackage{calc}

\newenvironment{framedhscode}%
  {\parskip=\abovedisplayskip\par\noindent
   \hscodestyle
   \arrayrulewidth=\codeframewidth
   \tabular{@{}|p{\linewidth-2\arraycolsep-2\arrayrulewidth-2pt}|@{}}%
   \hline\framedhslinecorrect\\{-1.5ex}%
   \let\endoflinesave=\\
   \let\\=\@normalcr
   \(\pboxed}%
  {\endpboxed\)%
   \framedhslinecorrect\endoflinesave{.5ex}\hline
   \endtabular
   \parskip=\belowdisplayskip\par\noindent
   \ignorespacesafterend}

\newcommand{\framedhslinecorrect}[2]%
  {#1[#2]}

\newcommand{\framedhs}{\sethscode{framedhscode}}

% The inlinehscode environment is an experimental environment
% that can be used to typeset displayed code inline.

\newenvironment{inlinehscode}%
  {\(\def\column##1##2{}%
   \let\>\undefined\let\<\undefined\let\\\undefined
   \newcommand\>[1][]{}\newcommand\<[1][]{}\newcommand\\[1][]{}%
   \def\fromto##1##2##3{##3}%
   \def\nextline{}}{\) }%

\newcommand{\inlinehs}{\sethscode{inlinehscode}}

% The joincode environment is a separate environment that
% can be used to surround and thereby connect multiple code
% blocks.

\newenvironment{joincode}%
  {\let\orighscode=\hscode
   \let\origendhscode=\endhscode
   \def\endhscode{\def\hscode{\endgroup\def\@currenvir{hscode}\\}\begingroup}
   %\let\SaveRestoreHook=\empty
   %\let\ColumnHook=\empty
   %\let\resethooks=\empty
   \orighscode\def\hscode{\endgroup\def\@currenvir{hscode}}}%
  {\origendhscode
   \global\let\hscode=\orighscode
   \global\let\endhscode=\origendhscode}%

\makeatother
\EndFmtInput
%

\ReadOnlyOnce{Formatting.fmt}%
\makeatletter

\let\Varid\mathit
\let\Conid\mathsf

\def\commentbegin{\quad\{\ }
\def\commentend{\}}

\newcommand{\ty}[1]{\Conid{#1}}
\newcommand{\con}[1]{\Conid{#1}}
\newcommand{\id}[1]{\Varid{#1}}
\newcommand{\cl}[1]{\Varid{#1}}
\newcommand{\opsym}[1]{\mathrel{#1}}

\newcommand\Keyword[1]{\textbf{\textsf{#1}}}
\newcommand\Hide{\mathbin{\downarrow}}
\newcommand\Reveal{\mathbin{\uparrow}}


%% Paper-specific keywords


\makeatother
\EndFmtInput

\section{Imposing constraints}
\label{sec:constraints}

To rule out programs with undesired properties, certain constraints must be
imposed, on what arguments they can take, or what preconditions they must hold.

Consider the following example: \text{LLEN} returns the length of
 the list associated with a key, else raises a type error.

\begin{tabbing}\tt
~redis\char62{}~LPUSH~some\char45{}list~bar\\
\tt ~\char40{}integer\char41{}~1\\
\tt ~redis\char62{}~LLEN~some\char45{}list\\
\tt ~\char40{}integer\char41{}~1\\
\tt ~redis\char62{}~SET~some\char45{}string~foo\\
\tt ~OK\\
\tt ~redis\char62{}~LLEN~some\char45{}string\\
\tt ~\char40{}error\char41{}~WRONGTYPE~Operation~against~a~key\\
\tt ~holding~the~wrong~kind~of~value
\end{tabbing}

Such constraint could be expressed in types with
\emph{equality constraints}\cite{typeeq}.

\begin{hscode}\SaveRestoreHook
\column{B}{@{}>{\hspre}l<{\hspost}@{}}%
\column{6}{@{}>{\hspre}l<{\hspost}@{}}%
\column{7}{@{}>{\hspre}l<{\hspost}@{}}%
\column{E}{@{}>{\hspre}l<{\hspost}@{}}%
\>[B]{}\Varid{llen}\mathbin{::}(\Conid{KnownSymbol}\;\Varid{s}{}\<[E]%
\\
\>[B]{}\hsindent{7}{}\<[7]%
\>[7]{},\Conid{Get}\;\Varid{xs}\;\Varid{s}\mathord{\sim}\Conid{Just}\;(\Conid{ListOf}\;\Varid{x})){}\<[E]%
\\
\>[B]{}\hsindent{6}{}\<[6]%
\>[6]{}\Rightarrow \Conid{Proxy}\;\Varid{s}{}\<[E]%
\\
\>[B]{}\hsindent{6}{}\<[6]%
\>[6]{}\to \Conid{Edis}\;\Varid{xs}\;\Varid{xs}\;(\Conid{Either}\;\Conid{Reply}\;\Conid{Integer}){}\<[E]%
\\
\>[B]{}\Varid{llen}\;\Varid{key}\mathrel{=}{}\<[E]%
\\
\>[B]{}\hsindent{6}{}\<[6]%
\>[6]{}\Conid{Edis}\mathbin{\$}\Varid{\Conid{Hedis}.llen}\;(\Varid{encodeKey}\;\Varid{key}){}\<[E]%
\ColumnHook
\end{hscode}\resethooks

Where \ensuremath{(\mathord{\sim})} denotes that \ensuremath{\Conid{Get}\;\Varid{xs}\;\Varid{s}}
and \ensuremath{\Conid{Just}\;(\Conid{ListOf}\;\Varid{x})} needs to be the same.

The semantics of \text{LLEN} defined above is actually not
complete. \text{LLEN} also accepts keys that do not exist, and
 replies with \text{0}.

\begin{tabbing}\tt
~redis\char62{}~LLEN~nonexistent\\
\tt ~\char40{}integer\char41{}~0
\end{tabbing}

In other words, we require that the key to be associated with a list,
 \textbf{unless} it doesn't exist at all.

\subsection{Expressing Constraint Disjunctions}

Unfortunately, expressing disjunctions in constraints is much more difficult
 than expressing conjunctions, since the latter could be easily done by placing
 constraints in a tuple (at the left side of \ensuremath{\Rightarrow }).

There are at least three ways to express type-level constraints
\cite{singletons}. Luckily we could express constraint disjunctions with type
 families in a modular way.

The semantics we want could be expressed informally like this:
\ensuremath{\Conid{Get}\;\Varid{xs}\;\Varid{s}} $\equiv$ \ensuremath{\Conid{Just}\;(\Conid{ListOf}\;\Varid{x})}
$\vee$ $\neg$ \ensuremath{(\Conid{Member}\;\Varid{xs}\;\Varid{s})}

We could achieve this simply by translating the semantics we want to the
 domain of Boolean, with type-level boolean functions such as
\ensuremath{(\mathrel{\wedge})},
\ensuremath{(\mid )}, \ensuremath{\Conid{Not}},
\ensuremath{(\doubleequals)}, etc.\footnotemark To avoid

\footnotetext{Available in \text{Data.Type.Bool} and
 \text{Data.Type.Equality}}

\begin{hscode}\SaveRestoreHook
\column{B}{@{}>{\hspre}l<{\hspost}@{}}%
\column{E}{@{}>{\hspre}l<{\hspost}@{}}%
\>[B]{}\Conid{Get}\;\Varid{xs}\;\Varid{s}\doubleequals\Conid{Just}\;(\Conid{ListOf}\;\Varid{x})\mathrel{\vee}\Conid{Not}\;(\Conid{Member}\;\Varid{xs}\;\Varid{s}){}\<[E]%
\ColumnHook
\end{hscode}\resethooks

To avoid addressing the type of value (as it may not exist at all), we defined
 an auxiliary predicate \ensuremath{\Conid{IsList}\mathbin{::}\Conid{Maybe}\mathbin{*}\to \Conid{Bool}} to
 replace the former part.

\begin{hscode}\SaveRestoreHook
\column{B}{@{}>{\hspre}l<{\hspost}@{}}%
\column{E}{@{}>{\hspre}l<{\hspost}@{}}%
\>[B]{}\Conid{IsList}\;(\Conid{Get}\;\Varid{xs}\;\Varid{s})\mathrel{\vee}\Conid{Not}\;(\Conid{Member}\;\Varid{xs}\;\Varid{s}){}\<[E]%
\ColumnHook
\end{hscode}\resethooks

The type expression above has kind \ensuremath{\Conid{Bool}}, we could make it
 a type constraint by asserting equality.

\begin{hscode}\SaveRestoreHook
\column{B}{@{}>{\hspre}l<{\hspost}@{}}%
\column{E}{@{}>{\hspre}l<{\hspost}@{}}%
\>[B]{}(\Conid{IsList}\;(\Conid{Get}\;\Varid{xs}\;\Varid{s})\mathrel{\vee}\Conid{Not}\;(\Conid{Member}\;\Varid{xs}\;\Varid{s}))\mathord{\sim}\Conid{True}{}\<[E]%
\ColumnHook
\end{hscode}\resethooks

With \emph{constraint kind}, a recent addition to GHC, type constraints now has
 its own kind: \ensuremath{\Conid{Constraint}}. That means type constraints
 are not restricted to the left side of a \ensuremath{\Rightarrow } anymore,
 they could appear in anywhere that accepts something of kind
 \ensuremath{\Conid{Constraint}}, and any type that has kind
 \ensuremath{\Conid{Constraint}} can also be used as a type constraint.
 \footnote{See \url{https://downloads.haskell.org/~ghc/7.4.1/docs/html/users_guide/constraint-kind.html}.}

As many other list-related commands also have this ``List or nothing'' semantics,
 we could abstract the lengthy type constraint above and give it an alias with
 type synonym.

\begin{hscode}\SaveRestoreHook
\column{B}{@{}>{\hspre}l<{\hspost}@{}}%
\column{5}{@{}>{\hspre}l<{\hspost}@{}}%
\column{E}{@{}>{\hspre}l<{\hspost}@{}}%
\>[B]{}\Conid{ListOrNX}\;\Varid{xs}\;\Varid{s}\mathrel{=}{}\<[E]%
\\
\>[B]{}\hsindent{5}{}\<[5]%
\>[5]{}(\Conid{IsList}\;(\Conid{Get}\;\Varid{xs}\;\Varid{s})\mathrel{\vee}\Conid{Not}\;(\Conid{Member}\;\Varid{xs}\;\Varid{s}))\mathord{\sim}\Conid{True}{}\<[E]%
\ColumnHook
\end{hscode}\resethooks

The complete implementation of \text{LLEN} with
\ensuremath{\Conid{ListOrNX}} would become:

\begin{hscode}\SaveRestoreHook
\column{B}{@{}>{\hspre}l<{\hspost}@{}}%
\column{9}{@{}>{\hspre}l<{\hspost}@{}}%
\column{E}{@{}>{\hspre}l<{\hspost}@{}}%
\>[B]{}\Varid{llen}\mathbin{::}(\Conid{KnownSymbol}\;\Varid{s},\Conid{ListOrNX}\;\Varid{xs}\;\Varid{s}){}\<[E]%
\\
\>[B]{}\hsindent{9}{}\<[9]%
\>[9]{}\Rightarrow \Conid{Proxy}\;\Varid{s}{}\<[E]%
\\
\>[B]{}\hsindent{9}{}\<[9]%
\>[9]{}\to \Conid{Edis}\;\Varid{xs}\;\Varid{xs}\;(\Conid{Either}\;\Conid{Reply}\;\Conid{Integer}){}\<[E]%
\\
\>[B]{}\Varid{llen}\;\Varid{key}\mathrel{=}\Conid{Edis}\mathbin{\$}\Varid{\Conid{Hedis}.llen}\;(\Varid{encodeKey}\;\Varid{key}){}\<[E]%
\ColumnHook
\end{hscode}\resethooks

\input{sections/Assertions}
\input{sections/Discussions}
\section{Related Work}
\label{sec:related}

While Redis can be viewed as a non-relational database system (although Redis
 seems reluctant to admit this in recent years), there also are similar goals on
 relational database systems, trying to achieve a safer database interface by
 making them statically checked.
 \emph{HaskellDB}\cite{haskelldb, haskelldbimproved} expresses quires
 with relational algebra-like combinators, wrapped in phantom types. Or as
 examples, at first to demonstrate the power of dependent types in
 Agda\cite{pi}, and then with singletons in Haskell\cite{singletons}.


\section{Conclusions}
\label{sec:conclusions}

By exploiting various recent extensions and type-level programming techniques,
we have designed a domain-specific embedded language \Edis{} which, at
term-level, is simply a wrapper of \Hedis{}, but enforces typing disciplines that keeps track of available keys and their types and allows a program to be
constructed only if it does not throw a runtime type error. We believe that it
is a neat case study of application of type-level programming.



% With more and more extensions added to the language, Haskell is gradually
%  becoming a dependently-typed language,\footnote{Why not be dependently typed? \url{http://stackoverflow.com/questions/12961651/why-not-be-dependently-typed}}
%  but it's not that dreadful as many (including us) would have thought.


\bibliographystyle{splncs03}
\bibliography{cites}

\end{document}
