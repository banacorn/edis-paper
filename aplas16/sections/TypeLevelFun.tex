%% ODER: format ==         = "\mathrel{==}"
%% ODER: format /=         = "\neq "
%
%
\makeatletter
\@ifundefined{lhs2tex.lhs2tex.sty.read}%
  {\@namedef{lhs2tex.lhs2tex.sty.read}{}%
   \newcommand\SkipToFmtEnd{}%
   \newcommand\EndFmtInput{}%
   \long\def\SkipToFmtEnd#1\EndFmtInput{}%
  }\SkipToFmtEnd

\newcommand\ReadOnlyOnce[1]{\@ifundefined{#1}{\@namedef{#1}{}}\SkipToFmtEnd}
\usepackage{amstext}
\usepackage{amssymb}
\usepackage{stmaryrd}
\DeclareFontFamily{OT1}{cmtex}{}
\DeclareFontShape{OT1}{cmtex}{m}{n}
  {<5><6><7><8>cmtex8
   <9>cmtex9
   <10><10.95><12><14.4><17.28><20.74><24.88>cmtex10}{}
\DeclareFontShape{OT1}{cmtex}{m}{it}
  {<-> ssub * cmtt/m/it}{}
\newcommand{\texfamily}{\fontfamily{cmtex}\selectfont}
\DeclareFontShape{OT1}{cmtt}{bx}{n}
  {<5><6><7><8>cmtt8
   <9>cmbtt9
   <10><10.95><12><14.4><17.28><20.74><24.88>cmbtt10}{}
\DeclareFontShape{OT1}{cmtex}{bx}{n}
  {<-> ssub * cmtt/bx/n}{}
\newcommand{\tex}[1]{\text{\texfamily#1}}	% NEU

\newcommand{\Sp}{\hskip.33334em\relax}


\newcommand{\Conid}[1]{\mathit{#1}}
\newcommand{\Varid}[1]{\mathit{#1}}
\newcommand{\anonymous}{\kern0.06em \vbox{\hrule\@width.5em}}
\newcommand{\plus}{\mathbin{+\!\!\!+}}
\newcommand{\bind}{\mathbin{>\!\!\!>\mkern-6.7mu=}}
\newcommand{\rbind}{\mathbin{=\mkern-6.7mu<\!\!\!<}}% suggested by Neil Mitchell
\newcommand{\sequ}{\mathbin{>\!\!\!>}}
\renewcommand{\leq}{\leqslant}
\renewcommand{\geq}{\geqslant}
\usepackage{polytable}

%mathindent has to be defined
\@ifundefined{mathindent}%
  {\newdimen\mathindent\mathindent\leftmargini}%
  {}%

\def\resethooks{%
  \global\let\SaveRestoreHook\empty
  \global\let\ColumnHook\empty}
\newcommand*{\savecolumns}[1][default]%
  {\g@addto@macro\SaveRestoreHook{\savecolumns[#1]}}
\newcommand*{\restorecolumns}[1][default]%
  {\g@addto@macro\SaveRestoreHook{\restorecolumns[#1]}}
\newcommand*{\aligncolumn}[2]%
  {\g@addto@macro\ColumnHook{\column{#1}{#2}}}

\resethooks

\newcommand{\onelinecommentchars}{\quad-{}- }
\newcommand{\commentbeginchars}{\enskip\{-}
\newcommand{\commentendchars}{-\}\enskip}

\newcommand{\visiblecomments}{%
  \let\onelinecomment=\onelinecommentchars
  \let\commentbegin=\commentbeginchars
  \let\commentend=\commentendchars}

\newcommand{\invisiblecomments}{%
  \let\onelinecomment=\empty
  \let\commentbegin=\empty
  \let\commentend=\empty}

\visiblecomments

\newlength{\blanklineskip}
\setlength{\blanklineskip}{0.66084ex}

\newcommand{\hsindent}[1]{\quad}% default is fixed indentation
\let\hspre\empty
\let\hspost\empty
\newcommand{\NB}{\textbf{NB}}
\newcommand{\Todo}[1]{$\langle$\textbf{To do:}~#1$\rangle$}

\EndFmtInput
\makeatother
%
%
%
%
%
%
% This package provides two environments suitable to take the place
% of hscode, called "plainhscode" and "arrayhscode". 
%
% The plain environment surrounds each code block by vertical space,
% and it uses \abovedisplayskip and \belowdisplayskip to get spacing
% similar to formulas. Note that if these dimensions are changed,
% the spacing around displayed math formulas changes as well.
% All code is indented using \leftskip.
%
% Changed 19.08.2004 to reflect changes in colorcode. Should work with
% CodeGroup.sty.
%
\ReadOnlyOnce{polycode.fmt}%
\makeatletter

\newcommand{\hsnewpar}[1]%
  {{\parskip=0pt\parindent=0pt\par\vskip #1\noindent}}

% can be used, for instance, to redefine the code size, by setting the
% command to \small or something alike
\newcommand{\hscodestyle}{}

% The command \sethscode can be used to switch the code formatting
% behaviour by mapping the hscode environment in the subst directive
% to a new LaTeX environment.

\newcommand{\sethscode}[1]%
  {\expandafter\let\expandafter\hscode\csname #1\endcsname
   \expandafter\let\expandafter\endhscode\csname end#1\endcsname}

% "compatibility" mode restores the non-polycode.fmt layout.

\newenvironment{compathscode}%
  {\par\noindent
   \advance\leftskip\mathindent
   \hscodestyle
   \let\\=\@normalcr
   \let\hspre\(\let\hspost\)%
   \pboxed}%
  {\endpboxed\)%
   \par\noindent
   \ignorespacesafterend}

\newcommand{\compaths}{\sethscode{compathscode}}

% "plain" mode is the proposed default.
% It should now work with \centering.
% This required some changes. The old version
% is still available for reference as oldplainhscode.

\newenvironment{plainhscode}%
  {\hsnewpar\abovedisplayskip
   \advance\leftskip\mathindent
   \hscodestyle
   \let\hspre\(\let\hspost\)%
   \pboxed}%
  {\endpboxed%
   \hsnewpar\belowdisplayskip
   \ignorespacesafterend}

\newenvironment{oldplainhscode}%
  {\hsnewpar\abovedisplayskip
   \advance\leftskip\mathindent
   \hscodestyle
   \let\\=\@normalcr
   \(\pboxed}%
  {\endpboxed\)%
   \hsnewpar\belowdisplayskip
   \ignorespacesafterend}

% Here, we make plainhscode the default environment.

\newcommand{\plainhs}{\sethscode{plainhscode}}
\newcommand{\oldplainhs}{\sethscode{oldplainhscode}}
\plainhs

% The arrayhscode is like plain, but makes use of polytable's
% parray environment which disallows page breaks in code blocks.

\newenvironment{arrayhscode}%
  {\hsnewpar\abovedisplayskip
   \advance\leftskip\mathindent
   \hscodestyle
   \let\\=\@normalcr
   \(\parray}%
  {\endparray\)%
   \hsnewpar\belowdisplayskip
   \ignorespacesafterend}

\newcommand{\arrayhs}{\sethscode{arrayhscode}}

% The mathhscode environment also makes use of polytable's parray 
% environment. It is supposed to be used only inside math mode 
% (I used it to typeset the type rules in my thesis).

\newenvironment{mathhscode}%
  {\parray}{\endparray}

\newcommand{\mathhs}{\sethscode{mathhscode}}

% texths is similar to mathhs, but works in text mode.

\newenvironment{texthscode}%
  {\(\parray}{\endparray\)}

\newcommand{\texths}{\sethscode{texthscode}}

% The framed environment places code in a framed box.

\def\codeframewidth{\arrayrulewidth}
\RequirePackage{calc}

\newenvironment{framedhscode}%
  {\parskip=\abovedisplayskip\par\noindent
   \hscodestyle
   \arrayrulewidth=\codeframewidth
   \tabular{@{}|p{\linewidth-2\arraycolsep-2\arrayrulewidth-2pt}|@{}}%
   \hline\framedhslinecorrect\\{-1.5ex}%
   \let\endoflinesave=\\
   \let\\=\@normalcr
   \(\pboxed}%
  {\endpboxed\)%
   \framedhslinecorrect\endoflinesave{.5ex}\hline
   \endtabular
   \parskip=\belowdisplayskip\par\noindent
   \ignorespacesafterend}

\newcommand{\framedhslinecorrect}[2]%
  {#1[#2]}

\newcommand{\framedhs}{\sethscode{framedhscode}}

% The inlinehscode environment is an experimental environment
% that can be used to typeset displayed code inline.

\newenvironment{inlinehscode}%
  {\(\def\column##1##2{}%
   \let\>\undefined\let\<\undefined\let\\\undefined
   \newcommand\>[1][]{}\newcommand\<[1][]{}\newcommand\\[1][]{}%
   \def\fromto##1##2##3{##3}%
   \def\nextline{}}{\) }%

\newcommand{\inlinehs}{\sethscode{inlinehscode}}

% The joincode environment is a separate environment that
% can be used to surround and thereby connect multiple code
% blocks.

\newenvironment{joincode}%
  {\let\orighscode=\hscode
   \let\origendhscode=\endhscode
   \def\endhscode{\def\hscode{\endgroup\def\@currenvir{hscode}\\}\begingroup}
   %\let\SaveRestoreHook=\empty
   %\let\ColumnHook=\empty
   %\let\resethooks=\empty
   \orighscode\def\hscode{\endgroup\def\@currenvir{hscode}}}%
  {\origendhscode
   \global\let\hscode=\orighscode
   \global\let\endhscode=\origendhscode}%

\makeatother
\EndFmtInput
%

\ReadOnlyOnce{Formatting.fmt}%
\makeatletter

\let\Varid\mathit
\let\Conid\mathsf

\def\commentbegin{\quad\{\ }
\def\commentend{\}}

\newcommand{\ty}[1]{\Conid{#1}}
\newcommand{\con}[1]{\Conid{#1}}
\newcommand{\id}[1]{\Varid{#1}}
\newcommand{\cl}[1]{\Varid{#1}}
\newcommand{\opsym}[1]{\mathrel{#1}}

\newcommand\Keyword[1]{\textbf{\textsf{#1}}}
\newcommand\Hide{\mathbin{\downarrow}}
\newcommand\Reveal{\mathbin{\uparrow}}


%% Paper-specific keywords


\makeatother
\EndFmtInput

\section{Type-level Functions}
\label{sec:type-level-fun}

\subsection{Closed Type Families}

Type families have a wide variety of applications. They can appear inside type
 classes\cite{tfclass,tfsynonym}, or at toplevel. Toplevel type families
 can be used to compute over types, they come in two forms: open\cite{tfopen}
 and closed \cite{tfclosed}.

We choose \emph{closed type families}, because it allows overlapping instances,
 and we need none of the extensibility provided by open type families.
For example, consider both term-level and type-level \ensuremath{\mathrel{\wedge}}:

\begin{hscode}\SaveRestoreHook
\column{B}{@{}>{\hspre}l<{\hspost}@{}}%
\column{5}{@{}>{\hspre}l<{\hspost}@{}}%
\column{6}{@{}>{\hspre}l<{\hspost}@{}}%
\column{14}{@{}>{\hspre}l<{\hspost}@{}}%
\column{19}{@{}>{\hspre}l<{\hspost}@{}}%
\column{E}{@{}>{\hspre}l<{\hspost}@{}}%
\>[B]{}(\mathrel{\wedge})\mathbin{::}\Conid{Bool}\to \Conid{Bool}\to \Conid{Bool}{}\<[E]%
\\
\>[B]{}\Conid{True}\mathrel{\wedge}\Conid{True}\mathrel{=}\Conid{True}{}\<[E]%
\\
\>[B]{}\Varid{a}{}\<[6]%
\>[6]{}\mathrel{\wedge}\Varid{b}{}\<[14]%
\>[14]{}\mathrel{=}\Conid{False}{}\<[E]%
\\[\blanklineskip]%
\>[B]{}\mathbf{type}\;\Varid{family}\;\Conid{And}\;(\Varid{a}\mathbin{::}\Conid{Bool})\;(\Varid{b}\mathbin{::}\Conid{Bool})\mathbin{::}\Conid{Bool}\;\mathbf{where}{}\<[E]%
\\
\>[B]{}\hsindent{5}{}\<[5]%
\>[5]{}\Conid{And}\;\Conid{True}\;\Conid{True}\mathrel{=}\Conid{True}{}\<[E]%
\\
\>[B]{}\hsindent{5}{}\<[5]%
\>[5]{}\Conid{And}\;\Varid{a}\;{}\<[14]%
\>[14]{}\Varid{b}{}\<[19]%
\>[19]{}\mathrel{=}\Conid{False}{}\<[E]%
\ColumnHook
\end{hscode}\resethooks

The first instance of \ensuremath{\Conid{And}} could be subsumed under a more
 general instance, \ensuremath{\Conid{And}\;\Varid{a}\;\Varid{b}}.
But the closedness allows these instances to be resolved in order, just like
 how cases are resolved in term-level functions. Also notice that how much
 \ensuremath{\Conid{And}} resembles to it's term-level sibling.

\subsection{Functions on Type-Level Dictionaries}

With closed type families, we could define functions on the type level.
 Let's begin with dictionary lookup.

\begin{hscode}\SaveRestoreHook
\column{B}{@{}>{\hspre}l<{\hspost}@{}}%
\column{5}{@{}>{\hspre}l<{\hspost}@{}}%
\column{29}{@{}>{\hspre}l<{\hspost}@{}}%
\column{E}{@{}>{\hspre}l<{\hspost}@{}}%
\>[B]{}\mathbf{type}\;\Varid{family}\;\Conid{Get}{}\<[E]%
\\
\>[B]{}\hsindent{5}{}\<[5]%
\>[5]{}(\Varid{xs}\mathbin{::}[\mskip1.5mu (\Conid{Symbol},\mathbin{*})\mskip1.5mu]){}\<[29]%
\>[29]{}\mbox{\onelinecomment  dictionary}{}\<[E]%
\\
\>[B]{}\hsindent{5}{}\<[5]%
\>[5]{}(\Varid{s}\mathbin{::}\Conid{Symbol}){}\<[29]%
\>[29]{}\mbox{\onelinecomment  key}{}\<[E]%
\\
\>[B]{}\hsindent{5}{}\<[5]%
\>[5]{}\mathbin{::}\mathbin{*}\mathbf{where}\;{}\<[29]%
\>[29]{}\mbox{\onelinecomment  type}{}\<[E]%
\\[\blanklineskip]%
\>[B]{}\hsindent{5}{}\<[5]%
\>[5]{}\Conid{Get}\;(\mathbin{’}(\Varid{s},\Varid{x})\mathbin{’:}\Varid{xs})\;\Varid{s}\mathrel{=}\Varid{x}{}\<[E]%
\\
\>[B]{}\hsindent{5}{}\<[5]%
\>[5]{}\Conid{Get}\;(\mathbin{’}(\Varid{t},\Varid{x})\mathbin{’:}\Varid{xs})\;\Varid{s}\mathrel{=}\Conid{Get}\;\Varid{xs}\;\Varid{s}{}\<[E]%
\ColumnHook
\end{hscode}\resethooks

Another benefit of closed type families is that type-level equality can be
expressed by unifying type variables with the same name.
\ensuremath{\Conid{Get}} takes two type arguments, a dictionary and a symbol.
If the key we are looking for unifies with the symbol of an entry, then
 \ensuremath{\Conid{Get}} returns the corresponding type, else it keeps
 searching down the rest of the dictionary.

\ensuremath{\Conid{Get}\;\mbox{\textquotesingle}[\!\;\mbox{\textquotesingle}(\!\;\text{\tt \char34 A\char34},\Conid{Int}\;\Conid{CLoseTPar}\;\!]\;\text{\tt \char34 A\char34}} evaluates to
\ensuremath{\Conid{Int}}.

But \ensuremath{\Conid{Get}\;\mbox{\textquotesingle}[\!\;\mbox{\textquotesingle}(\!\;\text{\tt \char34 A\char34},\Conid{Int}\;\Conid{CLoseTPar}\;\!]\;\text{\tt \char34 B\char34}} would get stuck.
That's because \ensuremath{\Conid{Get}} is a partial function on types,
 and these types are computed at compile-time. It wouldn't make
 much sense for a type checker to crash and throw a ``Non-exhaustive'' error or
 be non-terminating.

We could make \ensuremath{\Conid{Get}} total, as we would at the term level,
 with \ensuremath{\Conid{Maybe}}.

\begin{hscode}\SaveRestoreHook
\column{B}{@{}>{\hspre}l<{\hspost}@{}}%
\column{5}{@{}>{\hspre}l<{\hspost}@{}}%
\column{25}{@{}>{\hspre}l<{\hspost}@{}}%
\column{29}{@{}>{\hspre}l<{\hspost}@{}}%
\column{E}{@{}>{\hspre}l<{\hspost}@{}}%
\>[B]{}\mathbf{type}\;\Varid{family}\;\Conid{Get}{}\<[E]%
\\
\>[B]{}\hsindent{5}{}\<[5]%
\>[5]{}(\Varid{xs}\mathbin{::}[\mskip1.5mu (\Conid{Symbol},\mathbin{*})\mskip1.5mu]){}\<[29]%
\>[29]{}\mbox{\onelinecomment  dictionary}{}\<[E]%
\\
\>[B]{}\hsindent{5}{}\<[5]%
\>[5]{}(\Varid{s}\mathbin{::}\Conid{Symbol}){}\<[29]%
\>[29]{}\mbox{\onelinecomment  key}{}\<[E]%
\\
\>[B]{}\hsindent{5}{}\<[5]%
\>[5]{}\mathbin{::}\Conid{Maybe}\mathbin{*}\mathbf{where}\;{}\<[29]%
\>[29]{}\mbox{\onelinecomment  type}{}\<[E]%
\\[\blanklineskip]%
\>[B]{}\hsindent{5}{}\<[5]%
\>[5]{}\Conid{Get}\mathbin{’}[\mskip1.5mu \mskip1.5mu]\;{}\<[25]%
\>[25]{}\Varid{s}\mathrel{=}\Conid{Nothing}{}\<[E]%
\\
\>[B]{}\hsindent{5}{}\<[5]%
\>[5]{}\Conid{Get}\;(\mathbin{’}(\Varid{s},\Varid{x})\mathbin{’:}\Varid{xs})\;\Varid{s}\mathrel{=}\Conid{Just}\;\Varid{x}{}\<[E]%
\\
\>[B]{}\hsindent{5}{}\<[5]%
\>[5]{}\Conid{Get}\;(\mathbin{’}(\Varid{t},\Varid{x})\mathbin{’:}\Varid{xs})\;\Varid{s}\mathrel{=}\Conid{Get}\;\Varid{xs}\;\Varid{s}{}\<[E]%
\ColumnHook
\end{hscode}\resethooks
%
Other dictionary-related functions are defined in a similar fashion.

\begin{hscode}\SaveRestoreHook
\column{B}{@{}>{\hspre}l<{\hspost}@{}}%
\column{5}{@{}>{\hspre}l<{\hspost}@{}}%
\column{9}{@{}>{\hspre}l<{\hspost}@{}}%
\column{25}{@{}>{\hspre}l<{\hspost}@{}}%
\column{27}{@{}>{\hspre}l<{\hspost}@{}}%
\column{28}{@{}>{\hspre}l<{\hspost}@{}}%
\column{29}{@{}>{\hspre}l<{\hspost}@{}}%
\column{E}{@{}>{\hspre}l<{\hspost}@{}}%
\>[B]{}\mbox{\onelinecomment  inserts or updates an entry}{}\<[E]%
\\
\>[B]{}\mathbf{type}\;\Varid{family}\;\Conid{Set}{}\<[E]%
\\
\>[B]{}\hsindent{5}{}\<[5]%
\>[5]{}(\Varid{xs}\mathbin{::}[\mskip1.5mu (\Conid{Symbol},\mathbin{*})\mskip1.5mu]){}\<[29]%
\>[29]{}\mbox{\onelinecomment  old dictionary}{}\<[E]%
\\
\>[B]{}\hsindent{5}{}\<[5]%
\>[5]{}(\Varid{s}\mathbin{::}\Conid{Symbol}){}\<[29]%
\>[29]{}\mbox{\onelinecomment  key}{}\<[E]%
\\
\>[B]{}\hsindent{5}{}\<[5]%
\>[5]{}(\Varid{x}\mathbin{::}\mathbin{*}){}\<[29]%
\>[29]{}\mbox{\onelinecomment  type}{}\<[E]%
\\
\>[B]{}\hsindent{5}{}\<[5]%
\>[5]{}\mathbin{::}[\mskip1.5mu (\Conid{Symbol},\mathbin{*})\mskip1.5mu]\;\mathbf{where}\;{}\<[29]%
\>[29]{}\mbox{\onelinecomment  new dictionary}{}\<[E]%
\\[\blanklineskip]%
\>[B]{}\hsindent{5}{}\<[5]%
\>[5]{}\Conid{Set}\mathbin{’}[\mskip1.5mu \mskip1.5mu]\;{}\<[25]%
\>[25]{}\Varid{s}\;\Varid{x}\mathrel{=}\mathbin{’}[\mskip1.5mu \mathbin{’}(\Varid{s},\Varid{x})\mskip1.5mu]{}\<[E]%
\\
\>[B]{}\hsindent{5}{}\<[5]%
\>[5]{}\Conid{Set}\;(\mathbin{’}(\Varid{s},\Varid{y})\mathbin{’:}\Varid{xs})\;\Varid{s}\;\Varid{x}\mathrel{=}(\mathbin{’}(\Varid{s},\Varid{x})\mathbin{’:}\Varid{xs}){}\<[E]%
\\
\>[B]{}\hsindent{5}{}\<[5]%
\>[5]{}\Conid{Set}\;(\mathbin{’}(\Varid{t},\Varid{y})\mathbin{’:}\Varid{xs})\;\Varid{s}\;\Varid{x}\mathrel{=}{}\<[E]%
\\
\>[5]{}\hsindent{4}{}\<[9]%
\>[9]{}\mathbin{’}(\Varid{t},\Varid{y})\mathbin{’:}(\Conid{Set}\;\Varid{xs}\;\Varid{s}\;\Varid{x}){}\<[E]%
\\[\blanklineskip]%
\>[B]{}\mbox{\onelinecomment  removes an entry}{}\<[E]%
\\
\>[B]{}\mathbf{type}\;\Varid{family}\;\Conid{Del}{}\<[E]%
\\
\>[B]{}\hsindent{5}{}\<[5]%
\>[5]{}(\Varid{xs}\mathbin{::}[\mskip1.5mu (\Conid{Symbol},\mathbin{*})\mskip1.5mu]){}\<[29]%
\>[29]{}\mbox{\onelinecomment  old dictionary}{}\<[E]%
\\
\>[B]{}\hsindent{5}{}\<[5]%
\>[5]{}(\Varid{s}\mathbin{::}\Conid{Symbol}){}\<[29]%
\>[29]{}\mbox{\onelinecomment  key}{}\<[E]%
\\
\>[B]{}\hsindent{5}{}\<[5]%
\>[5]{}\mathbin{::}[\mskip1.5mu (\Conid{Symbol},\mathbin{*})\mskip1.5mu]\;\mathbf{where}\;{}\<[29]%
\>[29]{}\mbox{\onelinecomment  new dictionary}{}\<[E]%
\\[\blanklineskip]%
\>[B]{}\hsindent{5}{}\<[5]%
\>[5]{}\Conid{Del}\mathbin{’}[\mskip1.5mu \mskip1.5mu]\;\Varid{s}{}\<[27]%
\>[27]{}\mathrel{=}\mathbin{’}[\mskip1.5mu \mskip1.5mu]{}\<[E]%
\\
\>[B]{}\hsindent{5}{}\<[5]%
\>[5]{}\Conid{Del}\;(\mathbin{’}(\Varid{s},\Varid{y})\mathbin{’:}\Varid{xs})\;\Varid{s}\mathrel{=}\Varid{xs}{}\<[E]%
\\
\>[B]{}\hsindent{5}{}\<[5]%
\>[5]{}\Conid{Del}\;(\mathbin{’}(\Varid{t},\Varid{y})\mathbin{’:}\Varid{xs})\;\Varid{s}\mathrel{=}\mathbin{’}(\Varid{t},\Varid{y})\mathbin{’:}(\Conid{Del}\;\Varid{xs}\;\Varid{s}){}\<[E]%
\\[\blanklineskip]%
\>[B]{}\mbox{\onelinecomment  membership}{}\<[E]%
\\
\>[B]{}\mathbf{type}\;\Varid{family}\;\Conid{Member}{}\<[E]%
\\
\>[B]{}\hsindent{5}{}\<[5]%
\>[5]{}(\Varid{xs}\mathbin{::}[\mskip1.5mu (\Conid{Symbol},\mathbin{*})\mskip1.5mu]){}\<[29]%
\>[29]{}\mbox{\onelinecomment  dictionary}{}\<[E]%
\\
\>[B]{}\hsindent{5}{}\<[5]%
\>[5]{}(\Varid{s}\mathbin{::}\Conid{Symbol}){}\<[29]%
\>[29]{}\mbox{\onelinecomment  key}{}\<[E]%
\\
\>[B]{}\hsindent{5}{}\<[5]%
\>[5]{}\mathbin{::}\Conid{Bool}\;\mathbf{where}\;{}\<[29]%
\>[29]{}\mbox{\onelinecomment  exists?}{}\<[E]%
\\[\blanklineskip]%
\>[B]{}\hsindent{5}{}\<[5]%
\>[5]{}\Conid{Member}\mathbin{’}[\mskip1.5mu \mskip1.5mu]\;{}\<[28]%
\>[28]{}\Varid{s}\mathrel{=}\Conid{False}{}\<[E]%
\\
\>[B]{}\hsindent{5}{}\<[5]%
\>[5]{}\Conid{Member}\;(\mathbin{’}(\Varid{s},\Varid{x})\mathbin{’:}\Varid{xs})\;\Varid{s}\mathrel{=}\Conid{True}{}\<[E]%
\\
\>[B]{}\hsindent{5}{}\<[5]%
\>[5]{}\Conid{Member}\;(\mathbin{’}(\Varid{t},\Varid{x})\mathbin{’:}\Varid{xs})\;\Varid{s}\mathrel{=}\Conid{Member}\;\Varid{xs}\;\Varid{s}{}\<[E]%
\ColumnHook
\end{hscode}\resethooks

\subsection{Proxies and Singleton Types}

Now we could annotate the effects of a command in types. \text{DEL}
 removes a key from the current database, regardless of its type.

\begin{hscode}\SaveRestoreHook
\column{B}{@{}>{\hspre}l<{\hspost}@{}}%
\column{5}{@{}>{\hspre}l<{\hspost}@{}}%
\column{E}{@{}>{\hspre}l<{\hspost}@{}}%
\>[B]{}\Varid{del}\mathbin{::}\Conid{KnownSymbol}\;\Varid{s}{}\<[E]%
\\
\>[B]{}\hsindent{5}{}\<[5]%
\>[5]{}\Rightarrow \Conid{Proxy}\;\Varid{s}{}\<[E]%
\\
\>[B]{}\hsindent{5}{}\<[5]%
\>[5]{}\to \Conid{Popcorn}\;\Varid{xs}\;(\Conid{Del}\;\Varid{xs}\;\Varid{s})\;(\Conid{Either}\;\Conid{Reply}\;\Conid{Integer}){}\<[E]%
\\
\>[B]{}\Varid{del}\;\Varid{key}\mathrel{=}\Conid{Popcorn}\mathbin{\$}\Varid{\Conid{Hedis}.del}\;(\Varid{encodeKey}\;\Varid{key}){}\<[E]%
\ColumnHook
\end{hscode}\resethooks

\ensuremath{\Conid{KnownSymbol}} is a class that gives the string associated
 with a concrete type-level symbol, which can be retrieved with
 \ensuremath{\Varid{symbolVal}}.\footnotemark
 Where \ensuremath{\Varid{encodeKey}} converts \ensuremath{\Conid{Proxy}\;\Varid{s}} to
 \ensuremath{\Conid{ByteString}}.
\footnotetext{They are defined in \ensuremath{\Conid{\Conid{GHC}.TypeLits}}.}

\begin{hscode}\SaveRestoreHook
\column{B}{@{}>{\hspre}l<{\hspost}@{}}%
\column{E}{@{}>{\hspre}l<{\hspost}@{}}%
\>[B]{}\Varid{encodeKey}\mathbin{::}\Conid{KnownSymbol}\;\Varid{s}\Rightarrow \Conid{Proxy}\;\Varid{s}\to \Conid{ByteString}{}\<[E]%
\\
\>[B]{}\Varid{encodeKey}\mathrel{=}\Varid{encode}\mathbin{\cdot}\Varid{symbolVal}{}\<[E]%
\ColumnHook
\end{hscode}\resethooks

Since Haskell has a \emph{phase distinction, phasedistinction}, types are
 erased before runtime. It's impossible to obtain information directly from
 types, we can only do this indirectly, with
 \emph{singleton types, singletons}.

A singleton type is a type that has only one instance, and the instance can be
 think of as the representative of the type at the realm of runtime values.

\ensuremath{\Conid{Proxy}}, as its name would suggest, can be used as
 singletons. It's a phantom type that could be indexed with any type.

\begin{hscode}\SaveRestoreHook
\column{B}{@{}>{\hspre}l<{\hspost}@{}}%
\column{E}{@{}>{\hspre}l<{\hspost}@{}}%
\>[B]{}\mathbf{data}\;\Conid{Proxy}\;\Varid{t}\mathrel{=}\Conid{Proxy}{}\<[E]%
\ColumnHook
\end{hscode}\resethooks

In the type of \ensuremath{\Varid{del}}, the type variable
 \ensuremath{\Varid{s}} is a \ensuremath{\Conid{Symbol}} that is decided by
 the argument of type \ensuremath{\Conid{Proxy}\;\Varid{s}}.
 To use \ensuremath{\Varid{del}}, we would have to apply it with a clumsy
 term-level proxy like this:

\begin{hscode}\SaveRestoreHook
\column{B}{@{}>{\hspre}l<{\hspost}@{}}%
\column{E}{@{}>{\hspre}l<{\hspost}@{}}%
\>[B]{}\Varid{del}\;(\Conid{Proxy}\mathbin{::}\Conid{Proxy}\;\text{\tt \char34 A\char34}){}\<[E]%
\ColumnHook
\end{hscode}\resethooks
