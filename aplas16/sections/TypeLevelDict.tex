%% ODER: format ==         = "\mathrel{==}"
%% ODER: format /=         = "\neq "
%
%
\makeatletter
\@ifundefined{lhs2tex.lhs2tex.sty.read}%
  {\@namedef{lhs2tex.lhs2tex.sty.read}{}%
   \newcommand\SkipToFmtEnd{}%
   \newcommand\EndFmtInput{}%
   \long\def\SkipToFmtEnd#1\EndFmtInput{}%
  }\SkipToFmtEnd

\newcommand\ReadOnlyOnce[1]{\@ifundefined{#1}{\@namedef{#1}{}}\SkipToFmtEnd}
\usepackage{amstext}
\usepackage{amssymb}
\usepackage{stmaryrd}
\DeclareFontFamily{OT1}{cmtex}{}
\DeclareFontShape{OT1}{cmtex}{m}{n}
  {<5><6><7><8>cmtex8
   <9>cmtex9
   <10><10.95><12><14.4><17.28><20.74><24.88>cmtex10}{}
\DeclareFontShape{OT1}{cmtex}{m}{it}
  {<-> ssub * cmtt/m/it}{}
\newcommand{\texfamily}{\fontfamily{cmtex}\selectfont}
\DeclareFontShape{OT1}{cmtt}{bx}{n}
  {<5><6><7><8>cmtt8
   <9>cmbtt9
   <10><10.95><12><14.4><17.28><20.74><24.88>cmbtt10}{}
\DeclareFontShape{OT1}{cmtex}{bx}{n}
  {<-> ssub * cmtt/bx/n}{}
\newcommand{\tex}[1]{\text{\texfamily#1}}	% NEU

\newcommand{\Sp}{\hskip.33334em\relax}


\newcommand{\Conid}[1]{\mathit{#1}}
\newcommand{\Varid}[1]{\mathit{#1}}
\newcommand{\anonymous}{\kern0.06em \vbox{\hrule\@width.5em}}
\newcommand{\plus}{\mathbin{+\!\!\!+}}
\newcommand{\bind}{\mathbin{>\!\!\!>\mkern-6.7mu=}}
\newcommand{\rbind}{\mathbin{=\mkern-6.7mu<\!\!\!<}}% suggested by Neil Mitchell
\newcommand{\sequ}{\mathbin{>\!\!\!>}}
\renewcommand{\leq}{\leqslant}
\renewcommand{\geq}{\geqslant}
\usepackage{polytable}

%mathindent has to be defined
\@ifundefined{mathindent}%
  {\newdimen\mathindent\mathindent\leftmargini}%
  {}%

\def\resethooks{%
  \global\let\SaveRestoreHook\empty
  \global\let\ColumnHook\empty}
\newcommand*{\savecolumns}[1][default]%
  {\g@addto@macro\SaveRestoreHook{\savecolumns[#1]}}
\newcommand*{\restorecolumns}[1][default]%
  {\g@addto@macro\SaveRestoreHook{\restorecolumns[#1]}}
\newcommand*{\aligncolumn}[2]%
  {\g@addto@macro\ColumnHook{\column{#1}{#2}}}

\resethooks

\newcommand{\onelinecommentchars}{\quad-{}- }
\newcommand{\commentbeginchars}{\enskip\{-}
\newcommand{\commentendchars}{-\}\enskip}

\newcommand{\visiblecomments}{%
  \let\onelinecomment=\onelinecommentchars
  \let\commentbegin=\commentbeginchars
  \let\commentend=\commentendchars}

\newcommand{\invisiblecomments}{%
  \let\onelinecomment=\empty
  \let\commentbegin=\empty
  \let\commentend=\empty}

\visiblecomments

\newlength{\blanklineskip}
\setlength{\blanklineskip}{0.66084ex}

\newcommand{\hsindent}[1]{\quad}% default is fixed indentation
\let\hspre\empty
\let\hspost\empty
\newcommand{\NB}{\textbf{NB}}
\newcommand{\Todo}[1]{$\langle$\textbf{To do:}~#1$\rangle$}

\EndFmtInput
\makeatother
%
%
%
%
%
%
% This package provides two environments suitable to take the place
% of hscode, called "plainhscode" and "arrayhscode". 
%
% The plain environment surrounds each code block by vertical space,
% and it uses \abovedisplayskip and \belowdisplayskip to get spacing
% similar to formulas. Note that if these dimensions are changed,
% the spacing around displayed math formulas changes as well.
% All code is indented using \leftskip.
%
% Changed 19.08.2004 to reflect changes in colorcode. Should work with
% CodeGroup.sty.
%
\ReadOnlyOnce{polycode.fmt}%
\makeatletter

\newcommand{\hsnewpar}[1]%
  {{\parskip=0pt\parindent=0pt\par\vskip #1\noindent}}

% can be used, for instance, to redefine the code size, by setting the
% command to \small or something alike
\newcommand{\hscodestyle}{}

% The command \sethscode can be used to switch the code formatting
% behaviour by mapping the hscode environment in the subst directive
% to a new LaTeX environment.

\newcommand{\sethscode}[1]%
  {\expandafter\let\expandafter\hscode\csname #1\endcsname
   \expandafter\let\expandafter\endhscode\csname end#1\endcsname}

% "compatibility" mode restores the non-polycode.fmt layout.

\newenvironment{compathscode}%
  {\par\noindent
   \advance\leftskip\mathindent
   \hscodestyle
   \let\\=\@normalcr
   \let\hspre\(\let\hspost\)%
   \pboxed}%
  {\endpboxed\)%
   \par\noindent
   \ignorespacesafterend}

\newcommand{\compaths}{\sethscode{compathscode}}

% "plain" mode is the proposed default.
% It should now work with \centering.
% This required some changes. The old version
% is still available for reference as oldplainhscode.

\newenvironment{plainhscode}%
  {\hsnewpar\abovedisplayskip
   \advance\leftskip\mathindent
   \hscodestyle
   \let\hspre\(\let\hspost\)%
   \pboxed}%
  {\endpboxed%
   \hsnewpar\belowdisplayskip
   \ignorespacesafterend}

\newenvironment{oldplainhscode}%
  {\hsnewpar\abovedisplayskip
   \advance\leftskip\mathindent
   \hscodestyle
   \let\\=\@normalcr
   \(\pboxed}%
  {\endpboxed\)%
   \hsnewpar\belowdisplayskip
   \ignorespacesafterend}

% Here, we make plainhscode the default environment.

\newcommand{\plainhs}{\sethscode{plainhscode}}
\newcommand{\oldplainhs}{\sethscode{oldplainhscode}}
\plainhs

% The arrayhscode is like plain, but makes use of polytable's
% parray environment which disallows page breaks in code blocks.

\newenvironment{arrayhscode}%
  {\hsnewpar\abovedisplayskip
   \advance\leftskip\mathindent
   \hscodestyle
   \let\\=\@normalcr
   \(\parray}%
  {\endparray\)%
   \hsnewpar\belowdisplayskip
   \ignorespacesafterend}

\newcommand{\arrayhs}{\sethscode{arrayhscode}}

% The mathhscode environment also makes use of polytable's parray 
% environment. It is supposed to be used only inside math mode 
% (I used it to typeset the type rules in my thesis).

\newenvironment{mathhscode}%
  {\parray}{\endparray}

\newcommand{\mathhs}{\sethscode{mathhscode}}

% texths is similar to mathhs, but works in text mode.

\newenvironment{texthscode}%
  {\(\parray}{\endparray\)}

\newcommand{\texths}{\sethscode{texthscode}}

% The framed environment places code in a framed box.

\def\codeframewidth{\arrayrulewidth}
\RequirePackage{calc}

\newenvironment{framedhscode}%
  {\parskip=\abovedisplayskip\par\noindent
   \hscodestyle
   \arrayrulewidth=\codeframewidth
   \tabular{@{}|p{\linewidth-2\arraycolsep-2\arrayrulewidth-2pt}|@{}}%
   \hline\framedhslinecorrect\\{-1.5ex}%
   \let\endoflinesave=\\
   \let\\=\@normalcr
   \(\pboxed}%
  {\endpboxed\)%
   \framedhslinecorrect\endoflinesave{.5ex}\hline
   \endtabular
   \parskip=\belowdisplayskip\par\noindent
   \ignorespacesafterend}

\newcommand{\framedhslinecorrect}[2]%
  {#1[#2]}

\newcommand{\framedhs}{\sethscode{framedhscode}}

% The inlinehscode environment is an experimental environment
% that can be used to typeset displayed code inline.

\newenvironment{inlinehscode}%
  {\(\def\column##1##2{}%
   \let\>\undefined\let\<\undefined\let\\\undefined
   \newcommand\>[1][]{}\newcommand\<[1][]{}\newcommand\\[1][]{}%
   \def\fromto##1##2##3{##3}%
   \def\nextline{}}{\) }%

\newcommand{\inlinehs}{\sethscode{inlinehscode}}

% The joincode environment is a separate environment that
% can be used to surround and thereby connect multiple code
% blocks.

\newenvironment{joincode}%
  {\let\orighscode=\hscode
   \let\origendhscode=\endhscode
   \def\endhscode{\def\hscode{\endgroup\def\@currenvir{hscode}\\}\begingroup}
   %\let\SaveRestoreHook=\empty
   %\let\ColumnHook=\empty
   %\let\resethooks=\empty
   \orighscode\def\hscode{\endgroup\def\@currenvir{hscode}}}%
  {\origendhscode
   \global\let\hscode=\orighscode
   \global\let\endhscode=\origendhscode}%

\makeatother
\EndFmtInput
%

\ReadOnlyOnce{Formatting.fmt}%
\makeatletter

\let\Varid\mathit
\let\Conid\mathsf

\def\commentbegin{\quad\{\ }
\def\commentend{\}}

\newcommand{\ty}[1]{\Conid{#1}}
\newcommand{\con}[1]{\Conid{#1}}
\newcommand{\id}[1]{\Varid{#1}}
\newcommand{\cl}[1]{\Varid{#1}}
\newcommand{\opsym}[1]{\mathrel{#1}}

\newcommand\Keyword[1]{\textbf{\textsf{#1}}}
\newcommand\Hide{\mathbin{\downarrow}}
\newcommand\Reveal{\mathbin{\uparrow}}


%% Paper-specific keywords


\makeatother
\EndFmtInput

\section{Type-Level Dictionaries}
\label{sec:type-level-dict}

One of the challenges of statically ensuring type correctness of \Redis{},
which also presents in other stateful languages, is that the type of the value
associated to a key can be altered after updating. To ensure type correctness,
we have to keep track of the (\Redis{}) types of all existing keys in a
{\em dictionary} --- conceptually, a list of pairs of keys and \Redis{} types.
Each \Redis{} command is embedded in \Popcorn{} as a monadic computation. The
monad, to be presented in Section~\ref{sec:indexed-monads}, is indexed by
the dictionaries before and after the computation. In a dependently typed
programming language (without the so-called ``phase distinction'' ---
separation between types and terms), this would pose no problem. In Haskell
however, the dictionaries, to index a monad, has to be a Haskell type as well.

In this section we describe how to construct a type-level dictionary, to be
used with the indexed monad in Section~\ref{sec:indexed-monads}. More operations
on the dictionary will be presented in Section~\ref{sec:type-level-fun}.

Haskell maintains the distinction between values, types, and kinds: values are
categorized by types, and types are categorized by kinds. The kinds are relatively simple: \ensuremath{\mathbin{*}} is the kind of all {\em lifted} types, while type
constructors have kinds such as \ensuremath{\mathbin{*}\to \mathbin{*}}, \ensuremath{\mathbin{*}\to \mathbin{*}\to \mathbin{*}}, etc. Consider the
datatype definitions below:
\begin{hscode}\SaveRestoreHook
\column{B}{@{}>{\hspre}l<{\hspost}@{}}%
\column{E}{@{}>{\hspre}l<{\hspost}@{}}%
\>[B]{}\mathbf{data}\;\Conid{Nat}\mathrel{=}\Conid{Zero}\mid \Conid{Suc}\;\Conid{Nat}~~,\qquad\;\mathbf{data}\;[\mskip1.5mu \Varid{a}\mskip1.5mu]\mathrel{=}[\mskip1.5mu \mskip1.5mu]\mid \Varid{a}\mathbin{:}[\mskip1.5mu \Varid{a}\mskip1.5mu]~~.{}\<[E]%
\ColumnHook
\end{hscode}\resethooks
The lefthand side is usually seen as having defined a type \ensuremath{\Conid{Nat}\mathbin{::}\mathbin{*}},
and two value constructors \ensuremath{\Conid{Zero}\mathbin{::}\Conid{Nat}} and \ensuremath{\Conid{Suc}\mathbin{::}\Conid{Nat}\to \Conid{Nat}}. The righthand
side is how Haskell lists are understood. The {\em kind} of \ensuremath{[\mskip1.5mu \mathbin{\cdot}\mskip1.5mu]} is \ensuremath{\mathbin{*}\to \mathbin{*}},
since it takes a lifted type \ensuremath{\Varid{a}} to a lifted type \ensuremath{[\mskip1.5mu \Varid{a}\mskip1.5mu]}. The two value constructors respectively have types \ensuremath{[\mskip1.5mu \mskip1.5mu]\mathbin{::}[\mskip1.5mu \Varid{a}\mskip1.5mu]} and \ensuremath{(\mathbin{:})\mathbin{::}\Varid{a}\to [\mskip1.5mu \Varid{a}\mskip1.5mu]\to [\mskip1.5mu \Varid{a}\mskip1.5mu]}, for all type \ensuremath{\Varid{a}}.

The GHC extension \emph{data kinds}~\cite{promotion}, however, automatically
promotes certain ``suitable'' types to kinds.\footnote{It is only informally
described in the GHC manual what types are ``suitable''.} With the extension,
the \ensuremath{\mathbf{data}} definitions above has an alternative reading: \ensuremath{\Conid{Nat}} is a new kind,
\ensuremath{\Conid{Zero}\mathbin{::}\Conid{Nat}} is a type having kind \ensuremath{\Conid{Nat}}, and \ensuremath{\Conid{Suc}\mathbin{::}\Conid{Nat}\to \Conid{Nat}} is a type
constructor, taking a type in kind \ensuremath{\Conid{Nat}} to another type in \ensuremath{\Conid{Nat}}. Whether a
constructor is promoted can often be inferred from the context. To be more specific, prefixing a constructor with a single quote, such as in \ensuremath{\mbox{\textquotesingle}\Conid{Zero}} and
\ensuremath{\mbox{\textquotesingle}\Conid{Suc}}, denotes that it is promoted.

The situation of lists is similar: for all kind \ensuremath{\Varid{k}}, \ensuremath{[\mskip1.5mu \Varid{k}\mskip1.5mu]} is also a kind. For
all kind \ensuremath{\Varid{k}}, \ensuremath{[\mskip1.5mu \mskip1.5mu]\mathbin{::}[\mskip1.5mu \Varid{k}\mskip1.5mu]} is a type. Given a type \ensuremath{\Varid{x}}
of kind \ensuremath{\Varid{k}} and a type \ensuremath{\Varid{xs}} of kind \ensuremath{[\mskip1.5mu \Varid{k}\mskip1.5mu]}, \ensuremath{\Varid{x}\mathbin{:}\Varid{xs}} is again a type of
kind \ensuremath{[\mskip1.5mu \Varid{k}\mskip1.5mu]}. Formally, \ensuremath{(\mathbin{:})\mathbin{::}\Varid{k}\to [\mskip1.5mu \Varid{k}\mskip1.5mu]\to [\mskip1.5mu \Varid{k}\mskip1.5mu]}. For example,
\ensuremath{\Conid{Int}\mathbin{\mbox{\textquotesingle}\!:}(\Conid{Char}\mathbin{\mbox{\textquotesingle}\!:}(\Conid{Bool}\mathbin{\mbox{\textquotesingle}\!:}\mbox{\textquotesingle}[]))} is a type having kind \ensuremath{[\mskip1.5mu \mathbin{*}\mskip1.5mu]} --- a list of
(lifted) types. The optional quote denotes that the constructors are promoted.
The same list can be denoted by a syntax sugar \ensuremath{\mbox{\textquotesingle}[\Conid{Int},\Conid{Char},\Conid{Bool}]}.

Tuples are also promoted. Thus we may put two types in a pair to form another
type, such as in \ensuremath{\mbox{\textquotesingle}(\Conid{Int},\Conid{Char})}, a type having kind \ensuremath{(\mathbin{*},\mathbin{*})}.

Strings in Haskell are nothing but lists of \ensuremath{\Conid{Char}}s. Regarding promotion,
however, a \ensuremath{\Conid{String}} can be promoted to a type having kind \ensuremath{\Conid{Symbol}}. \ensuremath{\Conid{Symbol}} is
a type without a constructor: \ensuremath{\mathbf{data}\;\Conid{Symbol}},
intended to be used as a promoted kind. In the expression:
\begin{hscode}\SaveRestoreHook
\column{B}{@{}>{\hspre}l<{\hspost}@{}}%
\column{E}{@{}>{\hspre}l<{\hspost}@{}}%
\>[B]{}\text{\tt \char34 this~is~a~type-level~string~literal\char34}\mathbin{::}\Conid{Symbol}~~,{}\<[E]%
\ColumnHook
\end{hscode}\resethooks
the string on the lefthand side of \ensuremath{(\mathbin{::})} is a type, having kind \ensuremath{\Conid{Symbol}}.

With all of these ingredients, we are ready to build our dictionaries, or
type-level associate lists:
\begin{hscode}\SaveRestoreHook
\column{B}{@{}>{\hspre}l<{\hspost}@{}}%
\column{E}{@{}>{\hspre}l<{\hspost}@{}}%
\>[B]{}\mathbf{type}\;\Conid{DictEmpty}\mathrel{=}\mbox{\textquotesingle}[]~~,{}\<[E]%
\\
\>[B]{}\mathbf{type}\;\Conid{Dict0}\mathrel{=}\mbox{\textquotesingle}[\mbox{\textquotesingle}(\text{\tt \char34 key\char34},\Conid{Bool})]~~,{}\<[E]%
\\
\>[B]{}\mathbf{type}\;\Conid{Dict1}\mathrel{=}\mbox{\textquotesingle}[\mbox{\textquotesingle}(\text{\tt \char34 A\char34},\Conid{Int}),\mbox{\textquotesingle}(\text{\tt \char34 B\char34},\text{\tt \char34 A\char34})]~~.{}\<[E]%
\ColumnHook
\end{hscode}\resethooks
All the entities defined above are types, where \ensuremath{\Conid{Dict0}} and \ensuremath{\Conid{Dict1}}
have kind \ensuremath{[\mskip1.5mu (\Conid{Symbol},\mathbin{*})\mskip1.5mu]}.
