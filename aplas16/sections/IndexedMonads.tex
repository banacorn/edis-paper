%% ODER: format ==         = "\mathrel{==}"
%% ODER: format /=         = "\neq "
%
%
\makeatletter
\@ifundefined{lhs2tex.lhs2tex.sty.read}%
  {\@namedef{lhs2tex.lhs2tex.sty.read}{}%
   \newcommand\SkipToFmtEnd{}%
   \newcommand\EndFmtInput{}%
   \long\def\SkipToFmtEnd#1\EndFmtInput{}%
  }\SkipToFmtEnd

\newcommand\ReadOnlyOnce[1]{\@ifundefined{#1}{\@namedef{#1}{}}\SkipToFmtEnd}
\usepackage{amstext}
\usepackage{amssymb}
\usepackage{stmaryrd}
\DeclareFontFamily{OT1}{cmtex}{}
\DeclareFontShape{OT1}{cmtex}{m}{n}
  {<5><6><7><8>cmtex8
   <9>cmtex9
   <10><10.95><12><14.4><17.28><20.74><24.88>cmtex10}{}
\DeclareFontShape{OT1}{cmtex}{m}{it}
  {<-> ssub * cmtt/m/it}{}
\newcommand{\texfamily}{\fontfamily{cmtex}\selectfont}
\DeclareFontShape{OT1}{cmtt}{bx}{n}
  {<5><6><7><8>cmtt8
   <9>cmbtt9
   <10><10.95><12><14.4><17.28><20.74><24.88>cmbtt10}{}
\DeclareFontShape{OT1}{cmtex}{bx}{n}
  {<-> ssub * cmtt/bx/n}{}
\newcommand{\tex}[1]{\text{\texfamily#1}}	% NEU

\newcommand{\Sp}{\hskip.33334em\relax}


\newcommand{\Conid}[1]{\mathit{#1}}
\newcommand{\Varid}[1]{\mathit{#1}}
\newcommand{\anonymous}{\kern0.06em \vbox{\hrule\@width.5em}}
\newcommand{\plus}{\mathbin{+\!\!\!+}}
\newcommand{\bind}{\mathbin{>\!\!\!>\mkern-6.7mu=}}
\newcommand{\rbind}{\mathbin{=\mkern-6.7mu<\!\!\!<}}% suggested by Neil Mitchell
\newcommand{\sequ}{\mathbin{>\!\!\!>}}
\renewcommand{\leq}{\leqslant}
\renewcommand{\geq}{\geqslant}
\usepackage{polytable}

%mathindent has to be defined
\@ifundefined{mathindent}%
  {\newdimen\mathindent\mathindent\leftmargini}%
  {}%

\def\resethooks{%
  \global\let\SaveRestoreHook\empty
  \global\let\ColumnHook\empty}
\newcommand*{\savecolumns}[1][default]%
  {\g@addto@macro\SaveRestoreHook{\savecolumns[#1]}}
\newcommand*{\restorecolumns}[1][default]%
  {\g@addto@macro\SaveRestoreHook{\restorecolumns[#1]}}
\newcommand*{\aligncolumn}[2]%
  {\g@addto@macro\ColumnHook{\column{#1}{#2}}}

\resethooks

\newcommand{\onelinecommentchars}{\quad-{}- }
\newcommand{\commentbeginchars}{\enskip\{-}
\newcommand{\commentendchars}{-\}\enskip}

\newcommand{\visiblecomments}{%
  \let\onelinecomment=\onelinecommentchars
  \let\commentbegin=\commentbeginchars
  \let\commentend=\commentendchars}

\newcommand{\invisiblecomments}{%
  \let\onelinecomment=\empty
  \let\commentbegin=\empty
  \let\commentend=\empty}

\visiblecomments

\newlength{\blanklineskip}
\setlength{\blanklineskip}{0.66084ex}

\newcommand{\hsindent}[1]{\quad}% default is fixed indentation
\let\hspre\empty
\let\hspost\empty
\newcommand{\NB}{\textbf{NB}}
\newcommand{\Todo}[1]{$\langle$\textbf{To do:}~#1$\rangle$}

\EndFmtInput
\makeatother
%
%
%
%
%
%
% This package provides two environments suitable to take the place
% of hscode, called "plainhscode" and "arrayhscode". 
%
% The plain environment surrounds each code block by vertical space,
% and it uses \abovedisplayskip and \belowdisplayskip to get spacing
% similar to formulas. Note that if these dimensions are changed,
% the spacing around displayed math formulas changes as well.
% All code is indented using \leftskip.
%
% Changed 19.08.2004 to reflect changes in colorcode. Should work with
% CodeGroup.sty.
%
\ReadOnlyOnce{polycode.fmt}%
\makeatletter

\newcommand{\hsnewpar}[1]%
  {{\parskip=0pt\parindent=0pt\par\vskip #1\noindent}}

% can be used, for instance, to redefine the code size, by setting the
% command to \small or something alike
\newcommand{\hscodestyle}{}

% The command \sethscode can be used to switch the code formatting
% behaviour by mapping the hscode environment in the subst directive
% to a new LaTeX environment.

\newcommand{\sethscode}[1]%
  {\expandafter\let\expandafter\hscode\csname #1\endcsname
   \expandafter\let\expandafter\endhscode\csname end#1\endcsname}

% "compatibility" mode restores the non-polycode.fmt layout.

\newenvironment{compathscode}%
  {\par\noindent
   \advance\leftskip\mathindent
   \hscodestyle
   \let\\=\@normalcr
   \let\hspre\(\let\hspost\)%
   \pboxed}%
  {\endpboxed\)%
   \par\noindent
   \ignorespacesafterend}

\newcommand{\compaths}{\sethscode{compathscode}}

% "plain" mode is the proposed default.
% It should now work with \centering.
% This required some changes. The old version
% is still available for reference as oldplainhscode.

\newenvironment{plainhscode}%
  {\hsnewpar\abovedisplayskip
   \advance\leftskip\mathindent
   \hscodestyle
   \let\hspre\(\let\hspost\)%
   \pboxed}%
  {\endpboxed%
   \hsnewpar\belowdisplayskip
   \ignorespacesafterend}

\newenvironment{oldplainhscode}%
  {\hsnewpar\abovedisplayskip
   \advance\leftskip\mathindent
   \hscodestyle
   \let\\=\@normalcr
   \(\pboxed}%
  {\endpboxed\)%
   \hsnewpar\belowdisplayskip
   \ignorespacesafterend}

% Here, we make plainhscode the default environment.

\newcommand{\plainhs}{\sethscode{plainhscode}}
\newcommand{\oldplainhs}{\sethscode{oldplainhscode}}
\plainhs

% The arrayhscode is like plain, but makes use of polytable's
% parray environment which disallows page breaks in code blocks.

\newenvironment{arrayhscode}%
  {\hsnewpar\abovedisplayskip
   \advance\leftskip\mathindent
   \hscodestyle
   \let\\=\@normalcr
   \(\parray}%
  {\endparray\)%
   \hsnewpar\belowdisplayskip
   \ignorespacesafterend}

\newcommand{\arrayhs}{\sethscode{arrayhscode}}

% The mathhscode environment also makes use of polytable's parray 
% environment. It is supposed to be used only inside math mode 
% (I used it to typeset the type rules in my thesis).

\newenvironment{mathhscode}%
  {\parray}{\endparray}

\newcommand{\mathhs}{\sethscode{mathhscode}}

% texths is similar to mathhs, but works in text mode.

\newenvironment{texthscode}%
  {\(\parray}{\endparray\)}

\newcommand{\texths}{\sethscode{texthscode}}

% The framed environment places code in a framed box.

\def\codeframewidth{\arrayrulewidth}
\RequirePackage{calc}

\newenvironment{framedhscode}%
  {\parskip=\abovedisplayskip\par\noindent
   \hscodestyle
   \arrayrulewidth=\codeframewidth
   \tabular{@{}|p{\linewidth-2\arraycolsep-2\arrayrulewidth-2pt}|@{}}%
   \hline\framedhslinecorrect\\{-1.5ex}%
   \let\endoflinesave=\\
   \let\\=\@normalcr
   \(\pboxed}%
  {\endpboxed\)%
   \framedhslinecorrect\endoflinesave{.5ex}\hline
   \endtabular
   \parskip=\belowdisplayskip\par\noindent
   \ignorespacesafterend}

\newcommand{\framedhslinecorrect}[2]%
  {#1[#2]}

\newcommand{\framedhs}{\sethscode{framedhscode}}

% The inlinehscode environment is an experimental environment
% that can be used to typeset displayed code inline.

\newenvironment{inlinehscode}%
  {\(\def\column##1##2{}%
   \let\>\undefined\let\<\undefined\let\\\undefined
   \newcommand\>[1][]{}\newcommand\<[1][]{}\newcommand\\[1][]{}%
   \def\fromto##1##2##3{##3}%
   \def\nextline{}}{\) }%

\newcommand{\inlinehs}{\sethscode{inlinehscode}}

% The joincode environment is a separate environment that
% can be used to surround and thereby connect multiple code
% blocks.

\newenvironment{joincode}%
  {\let\orighscode=\hscode
   \let\origendhscode=\endhscode
   \def\endhscode{\def\hscode{\endgroup\def\@currenvir{hscode}\\}\begingroup}
   %\let\SaveRestoreHook=\empty
   %\let\ColumnHook=\empty
   %\let\resethooks=\empty
   \orighscode\def\hscode{\endgroup\def\@currenvir{hscode}}}%
  {\origendhscode
   \global\let\hscode=\orighscode
   \global\let\endhscode=\origendhscode}%

\makeatother
\EndFmtInput
%

\ReadOnlyOnce{Formatting.fmt}%
\makeatletter

\let\Varid\mathit
\let\Conid\mathsf

\def\commentbegin{\quad\{\ }
\def\commentend{\}}

\newcommand{\ty}[1]{\Conid{#1}}
\newcommand{\con}[1]{\Conid{#1}}
\newcommand{\id}[1]{\Varid{#1}}
\newcommand{\cl}[1]{\Varid{#1}}
\newcommand{\opsym}[1]{\mathrel{#1}}

\newcommand\Keyword[1]{\textbf{\textsf{#1}}}
\newcommand\Hide{\mathbin{\downarrow}}
\newcommand\Reveal{\mathbin{\uparrow}}


%% Paper-specific keywords


\makeatother
\EndFmtInput

\section{Indexed Monads}
\label{sec:indexed-monads}

Stateful computations are often reasoned using Hoare logic. A {\em Hoare triple}
$\{P\} S \{Q\}$ denotes such a proposition: if the statement $S$ is executed in
a state satisfying prediate $P$, when it terminates, the state must satisfy
predicate $Q$. Predicates $P$ and $Q$ are respectively is called the
\emph{precondition} and the \emph{postcondition} of the Hoare triple.

In Haskell, stateful computations are represented by monads. In order to
reason about their behaviors within the type system, we wish to label a state
monad with its pre and postcondition. An \emph{indexed monad}~%
\cite{indexedmonad} (also called \emph{monadish} or \emph{parameterised monad})
is a monad that, in addition to the type of value it computes, takes two more
type arguments representing an initial state and a final state, to be
interpreted like a Hoare triple~\cite{kleisli}:
\begin{hscode}\SaveRestoreHook
\column{B}{@{}>{\hspre}l<{\hspost}@{}}%
\column{5}{@{}>{\hspre}l<{\hspost}@{}}%
\column{E}{@{}>{\hspre}l<{\hspost}@{}}%
\>[B]{}\mathbf{class}\;\Conid{IMonad}\;\Varid{m}\;\mathbf{where}{}\<[E]%
\\
\>[B]{}\hsindent{5}{}\<[5]%
\>[5]{}\Varid{unit}\mathbin{::}\Varid{a}\to \Varid{m}\;\Varid{p}\;\Varid{p}\;\Varid{a}{}\<[E]%
\\
\>[B]{}\hsindent{5}{}\<[5]%
\>[5]{}\Varid{bind}\mathbin{::}\Varid{m}\;\Varid{p}\;\Varid{q}\;\Varid{a}\to (\Varid{a}\to \Varid{m}\;\Varid{q}\;\Varid{r}\;\Varid{b})\to \Varid{m}\;\Varid{p}\;\Varid{r}\;\Varid{b}~~.{}\<[E]%
\ColumnHook
\end{hscode}\resethooks
The intention is that a computation of type \ensuremath{\Varid{m}\;\Varid{p}\;\Varid{q}\;\Varid{a}} is a stateful computation
such that, if it starts execution in a state satisfying \ensuremath{\Varid{p}} and terminates, it
yields a value of type \ensuremath{\Varid{a}}, and the new state satisfies \ensuremath{\Varid{q}}. The operator \ensuremath{\Varid{unit}}
lifts a pure computation to a stateful computation that does not alter the
state. In \ensuremath{\Varid{x}\mathbin{`\Varid{bind}`}\Varid{f}}, a computation \ensuremath{\Varid{x}\mathbin{::}\Varid{m}\;\Varid{p}\;\Varid{q}\;\Varid{a}} is followed by
\ensuremath{\Varid{f}\mathbin{::}\Varid{a}\to \Varid{m}\;\Varid{q}\;\Varid{r}\;\Varid{b}} --- the postcondition of \ensuremath{\Varid{x}} matches the precondition of
the computation returned by \ensuremath{\Varid{f}}. The result is a monad \ensuremath{\Varid{m}\;\Varid{p}\;\Varid{r}\;\Varid{b}}.
Indexed monads have been used ~\cite{typefun,staticresources} ... \todo{for what? Some discriptions here to properly cite them.}

We define a new indexed monad \ensuremath{\Conid{Edis}} which, at term level, merely wraps
\ensuremath{\Conid{Redis}} in an additional constructor. The purpose is to add the
pre/postconditions at type level:
\begin{hscode}\SaveRestoreHook
\column{B}{@{}>{\hspre}l<{\hspost}@{}}%
\column{5}{@{}>{\hspre}l<{\hspost}@{}}%
\column{E}{@{}>{\hspre}l<{\hspost}@{}}%
\>[B]{}\mathbf{newtype}\;\Conid{Edis}\;\Varid{p}\;\Varid{q}\;\Varid{a}\mathrel{=}\Conid{Edis}\;\{\mskip1.5mu \Varid{unEdis}\mathbin{::}\Conid{Redis}\;\Varid{a}\mskip1.5mu\}~~,{}\<[E]%
\\[\blanklineskip]%
\>[B]{}\mathbf{instance}\;\Conid{IMonad}\;\Conid{Edis}\;\mathbf{where}{}\<[E]%
\\
\>[B]{}\hsindent{5}{}\<[5]%
\>[5]{}\Varid{unit}\mathrel{=}\Conid{Edis}\mathbin{\cdot}\Varid{return}{}\<[E]%
\\
\>[B]{}\hsindent{5}{}\<[5]%
\>[5]{}\Varid{bind}\;\Varid{m}\;\Varid{f}\mathrel{=}\Conid{Edis}\;(\Varid{unEdis}\;\Varid{m}\bind \Varid{unEdis}\mathbin{\cdot}\Varid{f})~~.{}\<[E]%
\ColumnHook
\end{hscode}\resethooks
At term level, the \ensuremath{\Varid{unit}} and \ensuremath{\Varid{bind}} methods are not interesting: they merely
make calls to \ensuremath{\Varid{return}} and \ensuremath{(\bind )} of \ensuremath{\Conid{Redis}}, and extracts and re-apply the constructor \ensuremath{\Conid{Edis}} when necessary. With \ensuremath{\Conid{Edis}} being a \ensuremath{\mathbf{newtype}}, they
can be optimized away in runtime. The interesting bits happen in compile type,
on the added type information.

The properties of the state we care about are the set of currently allocated
keys and their associated types. We will present, in Section~\ref{sec:type-level-dict}, techniques that allow us to specify
properties such as ``the keys in the database are \ensuremath{\text{\tt \char34 A\char34}}, \ensuremath{\text{\tt \char34 B\char34}}, and \ensuremath{\text{\tt \char34 C\char34}},
respectively associated to values of type \ensuremath{\Conid{Int}}, \ensuremath{\Conid{Char}}, and \ensuremath{\Conid{Bool}}.''
For now, however, let us look at the simplest \Redis{} command.

The command \text{PING} in \Redis{} does nothing but replies a message
\text{PONG} if the connection is alive. In \Hedis{}, \ensuremath{\Varid{ping}} has type
\ensuremath{\Conid{Redis}\;(\Conid{Either}\;\Conid{Reply}\;\Conid{Status})}. The \Edis{} version of \ensuremath{\Varid{ping}} simply
applys an additional constructor (functions from \Hedis{} are qualified with
\ensuremath{\Conid{Hedis}} to prevent name clashing):
\begin{hscode}\SaveRestoreHook
\column{B}{@{}>{\hspre}l<{\hspost}@{}}%
\column{E}{@{}>{\hspre}l<{\hspost}@{}}%
\>[B]{}\Varid{ping}\mathbin{::}\Conid{Edis}\;\Varid{xs}\;\Varid{xs}\;(\Conid{Either}\;\Conid{Reply}\;\Conid{Status}){}\<[E]%
\\
\>[B]{}\Varid{ping}\mathrel{=}\Conid{Edis}\;\Varid{\Conid{Hedis}.ping}~~.{}\<[E]%
\ColumnHook
\end{hscode}\resethooks
Since \ensuremath{\Varid{ping}} does not alter the database, the postcondition and precondition
are the same. Commands that are more interesting will be introduced after
we present our type-level encoding of constraints on states.
