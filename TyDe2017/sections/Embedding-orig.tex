%% ODER: format ==         = "\mathrel{==}"
%% ODER: format /=         = "\neq "
%
%
\makeatletter
\@ifundefined{lhs2tex.lhs2tex.sty.read}%
  {\@namedef{lhs2tex.lhs2tex.sty.read}{}%
   \newcommand\SkipToFmtEnd{}%
   \newcommand\EndFmtInput{}%
   \long\def\SkipToFmtEnd#1\EndFmtInput{}%
  }\SkipToFmtEnd

\newcommand\ReadOnlyOnce[1]{\@ifundefined{#1}{\@namedef{#1}{}}\SkipToFmtEnd}
\usepackage{amstext}
\usepackage{amssymb}
\usepackage{stmaryrd}
\DeclareFontFamily{OT1}{cmtex}{}
\DeclareFontShape{OT1}{cmtex}{m}{n}
  {<5><6><7><8>cmtex8
   <9>cmtex9
   <10><10.95><12><14.4><17.28><20.74><24.88>cmtex10}{}
\DeclareFontShape{OT1}{cmtex}{m}{it}
  {<-> ssub * cmtt/m/it}{}
\newcommand{\texfamily}{\fontfamily{cmtex}\selectfont}
\DeclareFontShape{OT1}{cmtt}{bx}{n}
  {<5><6><7><8>cmtt8
   <9>cmbtt9
   <10><10.95><12><14.4><17.28><20.74><24.88>cmbtt10}{}
\DeclareFontShape{OT1}{cmtex}{bx}{n}
  {<-> ssub * cmtt/bx/n}{}
\newcommand{\tex}[1]{\text{\texfamily#1}}	% NEU

\newcommand{\Sp}{\hskip.33334em\relax}


\newcommand{\Conid}[1]{\mathit{#1}}
\newcommand{\Varid}[1]{\mathit{#1}}
\newcommand{\anonymous}{\kern0.06em \vbox{\hrule\@width.5em}}
\newcommand{\plus}{\mathbin{+\!\!\!+}}
\newcommand{\bind}{\mathbin{>\!\!\!>\mkern-6.7mu=}}
\newcommand{\rbind}{\mathbin{=\mkern-6.7mu<\!\!\!<}}% suggested by Neil Mitchell
\newcommand{\sequ}{\mathbin{>\!\!\!>}}
\renewcommand{\leq}{\leqslant}
\renewcommand{\geq}{\geqslant}
\usepackage{polytable}

%mathindent has to be defined
\@ifundefined{mathindent}%
  {\newdimen\mathindent\mathindent\leftmargini}%
  {}%

\def\resethooks{%
  \global\let\SaveRestoreHook\empty
  \global\let\ColumnHook\empty}
\newcommand*{\savecolumns}[1][default]%
  {\g@addto@macro\SaveRestoreHook{\savecolumns[#1]}}
\newcommand*{\restorecolumns}[1][default]%
  {\g@addto@macro\SaveRestoreHook{\restorecolumns[#1]}}
\newcommand*{\aligncolumn}[2]%
  {\g@addto@macro\ColumnHook{\column{#1}{#2}}}

\resethooks

\newcommand{\onelinecommentchars}{\quad-{}- }
\newcommand{\commentbeginchars}{\enskip\{-}
\newcommand{\commentendchars}{-\}\enskip}

\newcommand{\visiblecomments}{%
  \let\onelinecomment=\onelinecommentchars
  \let\commentbegin=\commentbeginchars
  \let\commentend=\commentendchars}

\newcommand{\invisiblecomments}{%
  \let\onelinecomment=\empty
  \let\commentbegin=\empty
  \let\commentend=\empty}

\visiblecomments

\newlength{\blanklineskip}
\setlength{\blanklineskip}{0.66084ex}

\newcommand{\hsindent}[1]{\quad}% default is fixed indentation
\let\hspre\empty
\let\hspost\empty
\newcommand{\NB}{\textbf{NB}}
\newcommand{\Todo}[1]{$\langle$\textbf{To do:}~#1$\rangle$}

\EndFmtInput
\makeatother
%
%
%
%
%
%
% This package provides two environments suitable to take the place
% of hscode, called "plainhscode" and "arrayhscode". 
%
% The plain environment surrounds each code block by vertical space,
% and it uses \abovedisplayskip and \belowdisplayskip to get spacing
% similar to formulas. Note that if these dimensions are changed,
% the spacing around displayed math formulas changes as well.
% All code is indented using \leftskip.
%
% Changed 19.08.2004 to reflect changes in colorcode. Should work with
% CodeGroup.sty.
%
\ReadOnlyOnce{polycode.fmt}%
\makeatletter

\newcommand{\hsnewpar}[1]%
  {{\parskip=0pt\parindent=0pt\par\vskip #1\noindent}}

% can be used, for instance, to redefine the code size, by setting the
% command to \small or something alike
\newcommand{\hscodestyle}{}

% The command \sethscode can be used to switch the code formatting
% behaviour by mapping the hscode environment in the subst directive
% to a new LaTeX environment.

\newcommand{\sethscode}[1]%
  {\expandafter\let\expandafter\hscode\csname #1\endcsname
   \expandafter\let\expandafter\endhscode\csname end#1\endcsname}

% "compatibility" mode restores the non-polycode.fmt layout.

\newenvironment{compathscode}%
  {\par\noindent
   \advance\leftskip\mathindent
   \hscodestyle
   \let\\=\@normalcr
   \let\hspre\(\let\hspost\)%
   \pboxed}%
  {\endpboxed\)%
   \par\noindent
   \ignorespacesafterend}

\newcommand{\compaths}{\sethscode{compathscode}}

% "plain" mode is the proposed default.
% It should now work with \centering.
% This required some changes. The old version
% is still available for reference as oldplainhscode.

\newenvironment{plainhscode}%
  {\hsnewpar\abovedisplayskip
   \advance\leftskip\mathindent
   \hscodestyle
   \let\hspre\(\let\hspost\)%
   \pboxed}%
  {\endpboxed%
   \hsnewpar\belowdisplayskip
   \ignorespacesafterend}

\newenvironment{oldplainhscode}%
  {\hsnewpar\abovedisplayskip
   \advance\leftskip\mathindent
   \hscodestyle
   \let\\=\@normalcr
   \(\pboxed}%
  {\endpboxed\)%
   \hsnewpar\belowdisplayskip
   \ignorespacesafterend}

% Here, we make plainhscode the default environment.

\newcommand{\plainhs}{\sethscode{plainhscode}}
\newcommand{\oldplainhs}{\sethscode{oldplainhscode}}
\plainhs

% The arrayhscode is like plain, but makes use of polytable's
% parray environment which disallows page breaks in code blocks.

\newenvironment{arrayhscode}%
  {\hsnewpar\abovedisplayskip
   \advance\leftskip\mathindent
   \hscodestyle
   \let\\=\@normalcr
   \(\parray}%
  {\endparray\)%
   \hsnewpar\belowdisplayskip
   \ignorespacesafterend}

\newcommand{\arrayhs}{\sethscode{arrayhscode}}

% The mathhscode environment also makes use of polytable's parray 
% environment. It is supposed to be used only inside math mode 
% (I used it to typeset the type rules in my thesis).

\newenvironment{mathhscode}%
  {\parray}{\endparray}

\newcommand{\mathhs}{\sethscode{mathhscode}}

% texths is similar to mathhs, but works in text mode.

\newenvironment{texthscode}%
  {\(\parray}{\endparray\)}

\newcommand{\texths}{\sethscode{texthscode}}

% The framed environment places code in a framed box.

\def\codeframewidth{\arrayrulewidth}
\RequirePackage{calc}

\newenvironment{framedhscode}%
  {\parskip=\abovedisplayskip\par\noindent
   \hscodestyle
   \arrayrulewidth=\codeframewidth
   \tabular{@{}|p{\linewidth-2\arraycolsep-2\arrayrulewidth-2pt}|@{}}%
   \hline\framedhslinecorrect\\{-1.5ex}%
   \let\endoflinesave=\\
   \let\\=\@normalcr
   \(\pboxed}%
  {\endpboxed\)%
   \framedhslinecorrect\endoflinesave{.5ex}\hline
   \endtabular
   \parskip=\belowdisplayskip\par\noindent
   \ignorespacesafterend}

\newcommand{\framedhslinecorrect}[2]%
  {#1[#2]}

\newcommand{\framedhs}{\sethscode{framedhscode}}

% The inlinehscode environment is an experimental environment
% that can be used to typeset displayed code inline.

\newenvironment{inlinehscode}%
  {\(\def\column##1##2{}%
   \let\>\undefined\let\<\undefined\let\\\undefined
   \newcommand\>[1][]{}\newcommand\<[1][]{}\newcommand\\[1][]{}%
   \def\fromto##1##2##3{##3}%
   \def\nextline{}}{\) }%

\newcommand{\inlinehs}{\sethscode{inlinehscode}}

% The joincode environment is a separate environment that
% can be used to surround and thereby connect multiple code
% blocks.

\newenvironment{joincode}%
  {\let\orighscode=\hscode
   \let\origendhscode=\endhscode
   \def\endhscode{\def\hscode{\endgroup\def\@currenvir{hscode}\\}\begingroup}
   %\let\SaveRestoreHook=\empty
   %\let\ColumnHook=\empty
   %\let\resethooks=\empty
   \orighscode\def\hscode{\endgroup\def\@currenvir{hscode}}}%
  {\origendhscode
   \global\let\hscode=\orighscode
   \global\let\endhscode=\origendhscode}%

\makeatother
\EndFmtInput
%

\ReadOnlyOnce{Formatting.fmt}%
\makeatletter

\let\Varid\mathit
\let\Conid\mathsf

\def\commentbegin{\quad\{\ }
\def\commentend{\}}

\newcommand{\ty}[1]{\Conid{#1}}
\newcommand{\con}[1]{\Conid{#1}}
\newcommand{\id}[1]{\Varid{#1}}
\newcommand{\cl}[1]{\Varid{#1}}
\newcommand{\opsym}[1]{\mathrel{#1}}

\newcommand\Keyword[1]{\textbf{\textsf{#1}}}
\newcommand\Hide{\mathbin{\downarrow}}
\newcommand\Reveal{\mathbin{\uparrow}}


%% Paper-specific keywords


\makeatother
\EndFmtInput

\section{Embedding \Hedis{} Commands}
\label{sec:embedding-commands}

Having the indexed monads and type-level dictionaries, in this section we
present our embedding of \Hedis{} commands into \Edis{}, while introducing
necessary concepts when they are used.

\subsection{Proxies and Singleton Types}
\label{sec:proxy-key}

The \Hedis{} function \ensuremath{\Varid{del}\mathbin{::}[\mskip1.5mu \Conid{ByteString}\mskip1.5mu]\to \Conid{Redis}\;(\Conid{R} \uplus \Conid{Integer})} takes a list of keys (encoded to \ensuremath{\Conid{ByteString}}) and removes the entries having those
keys in the database. For some reason to be explained later, we consider an \Edis{}
counterpart that takes only one key. A first attempt may lead to something like
the following:
\begin{hscode}\SaveRestoreHook
\column{B}{@{}>{\hspre}l<{\hspost}@{}}%
\column{E}{@{}>{\hspre}l<{\hspost}@{}}%
\>[B]{}\Varid{del}\mathbin{::}\Conid{String}\to \Conid{Edis}\;\Varid{xs}\;(\Conid{Del}\;\Varid{xs}~\mathbin{?})\;(\Conid{R} \uplus \Conid{Integer}){}\<[E]%
\\
\>[B]{}\Varid{del}\;\Varid{key}\mathrel{=}\Conid{Edis}\;(\Varid{\Conid{Hedis}.del}\;[\mskip1.5mu \Varid{encode}\;\Varid{key}\mskip1.5mu])~~,{}\<[E]%
\ColumnHook
\end{hscode}\resethooks
where the function \ensuremath{\Varid{encode}} converts \ensuremath{\Conid{String}} to \ensuremath{\Conid{ByteString}}. At term-level,
our \ensuremath{\Varid{del}} merely calls \ensuremath{\Varid{\Conid{Hedis}.del}}. At type-level, if the status of the database
before \ensuremath{\Varid{del}} is called meets the constraint represented by the dictionary
\ensuremath{\Varid{xs}}, the status afterwards should meet the constraint \ensuremath{\Conid{Del}\;\Varid{xs}~\mathbin{?}}. The question, however, is what to fill in place of the question mark. It cannot be
\ensuremath{\Conid{Del}\;\Varid{xs}\;\Varid{key}}, since \ensuremath{\Varid{key}} is a runtime value and not a type. How do we smuggle
a runtime value to type-level?

In a language with phase distinction like Haskell, it is certainly impossible
to pass the value of \ensuremath{\Varid{key}} to the type checker if it truly is a runtime value,
for example, a string read from the user. If the value of \ensuremath{\Varid{key}} can be
determined statically, however, {\em singleton types} can be used to represent a
type as a value, thus build a connection between the two realms.

A singleton type is a type that has only one term. When the term is built, it
carries a type that can be inspected by the type checker. The term can be
thought of as a representative of its type at the realm of runtime values. For
our purpose, we will use the following type \ensuremath{\Conid{Proxy}}:
\begin{hscode}\SaveRestoreHook
\column{B}{@{}>{\hspre}l<{\hspost}@{}}%
\column{E}{@{}>{\hspre}l<{\hspost}@{}}%
\>[B]{}\mathbf{data}\;\Conid{Proxy}\;\Varid{t}\mathrel{=}\Conid{Proxy}~~.{}\<[E]%
\ColumnHook
\end{hscode}\resethooks
For every type \ensuremath{\Varid{t}}, \ensuremath{\Conid{Proxy}\;\Varid{t}} is a type that has only one term: \ensuremath{\Conid{Proxy}}.%
\footnote{While giving the same name to both the type and the term can be very
confusing, it is unfortunately a common practice in the Haskell community.}
To call \ensuremath{\Varid{del}}, instead of passing a key as a \ensuremath{\Conid{String}} value, we give it a proxy
with a specified type:
\begin{hscode}\SaveRestoreHook
\column{B}{@{}>{\hspre}l<{\hspost}@{}}%
\column{E}{@{}>{\hspre}l<{\hspost}@{}}%
\>[B]{}\Varid{del}\;(\Conid{Proxy}\mathbin{::}\Conid{Proxy}\;\text{\tt \char34 A\char34})~~,{}\<[E]%
\ColumnHook
\end{hscode}\resethooks
where \ensuremath{\text{\tt \char34 A\char34}} is not a value, but a string lifted to a type (of kind \ensuremath{\Conid{Symbol}}).
Now that the type checker has access to the key, the type of \ensuremath{\Varid{del}} can be
\ensuremath{\Conid{Proxy}\;\Varid{k}\to \Conid{Edis}\;\Varid{xs}\;(\Conid{Del}\;\Varid{xs}\;\Varid{k})\;(\Conid{R} \uplus \Conid{Integer})}.

The next problem is that, \ensuremath{\Varid{del}}, at term level, gets only a value constructor
\ensuremath{\Conid{Proxy}} without further information, while it needs to pass a \ensuremath{\Conid{ByteString}} key
to \ensuremath{\Varid{\Conid{Hedis}.del}}. Every concrete string literal lifted to a type, for example,
\ensuremath{\text{\tt \char34 A\char34}}, belongs to a type class \ensuremath{\Conid{KnownSymbol}}. For all type \ensuremath{\Varid{k}} in \ensuremath{\Conid{KnownSymbol}},
the function \ensuremath{\Varid{symbolVal}}:\begin{hscode}\SaveRestoreHook
\column{B}{@{}>{\hspre}l<{\hspost}@{}}%
\column{3}{@{}>{\hspre}l<{\hspost}@{}}%
\column{E}{@{}>{\hspre}l<{\hspost}@{}}%
\>[3]{}\Varid{symbolVal}\mathbin{::}\Conid{KnownSymbol}\;\Varid{k}\Rightarrow \Varid{proxy}\;\Varid{k}\to \Conid{String}~~,{}\<[E]%
\ColumnHook
\end{hscode}\resethooks
retrieves the string associated with a type-level literal that is known at
compile time. In summary, \ensuremath{\Varid{del}} can be implemented as:
\begin{hscode}\SaveRestoreHook
\column{B}{@{}>{\hspre}l<{\hspost}@{}}%
\column{6}{@{}>{\hspre}l<{\hspost}@{}}%
\column{45}{@{}>{\hspre}l<{\hspost}@{}}%
\column{E}{@{}>{\hspre}l<{\hspost}@{}}%
\>[B]{}\Varid{del}{}\<[6]%
\>[6]{}\mathbin{::}\Conid{KnownSymbol}\;\Varid{k}{}\<[E]%
\\
\>[6]{}\Rightarrow \Conid{Proxy}\;\Varid{k}\to \Conid{Edis}\;\Varid{xs}\;(\Conid{Del}\;\Varid{xs}\;\Varid{k})\;(\Conid{R} \uplus \Conid{Integer}){}\<[E]%
\\
\>[B]{}\Varid{del}\;\Varid{key}\mathrel{=}\Conid{Edis}\;(\Varid{\Conid{Hedis}.del}\;[\mskip1.5mu \Varid{encodeKey}\;\Varid{key}\mskip1.5mu]){}\<[45]%
\>[45]{}~~,{}\<[E]%
\ColumnHook
\end{hscode}\resethooks
where \ensuremath{\Varid{encodeKey}\mathrel{=}\Varid{encode}\mathbin{\cdot}\Varid{symbolVal}}.

A final note: the function \ensuremath{\Varid{encode}}, from the Haskell library {\sc cereal},
helps to convert certain datatypes that are {\em serializable} into \ensuremath{\Conid{ByteString}}.
The function and its dual \ensuremath{\Varid{decode}} will be used more later.
\begin{hscode}\SaveRestoreHook
\column{B}{@{}>{\hspre}l<{\hspost}@{}}%
\column{9}{@{}>{\hspre}l<{\hspost}@{}}%
\column{E}{@{}>{\hspre}l<{\hspost}@{}}%
\>[B]{}\Varid{encode}{}\<[9]%
\>[9]{}\mathbin{::}\Conid{Serialize}\;\Varid{a}\Rightarrow \Varid{a}\to \Conid{ByteString}~~,{}\<[E]%
\\
\>[B]{}\Varid{decode}{}\<[9]%
\>[9]{}\mathbin{::}\Conid{Serialize}\;\Varid{a}\Rightarrow \Conid{ByteString}\to \Conid{Either}\;\Conid{String}\;\Varid{a}~~.{}\<[E]%
\ColumnHook
\end{hscode}\resethooks

\subsection{Automatic Serialization}
\label{sec:polymorphic-redis}

As mentioned before, while \Redis{} provide a number of container types
including lists, sets, and hash, etc., the primitive type is string. \Hedis{}
programmers manually convert data of other types to strings before saving them
into the data store. In \Edis{}, we wish to save some of the effort for the
programmers, as well as keeping a careful record of the intended types of the
strings in the data store.

To keep track of intended types of strings in the data store, we define the
following types (that have no terms):
\begin{hscode}\SaveRestoreHook
\column{B}{@{}>{\hspre}l<{\hspost}@{}}%
\column{16}{@{}>{\hspre}l<{\hspost}@{}}%
\column{E}{@{}>{\hspre}l<{\hspost}@{}}%
\>[B]{}\mathbf{data}\;\Conid{StringOf}{}\<[16]%
\>[16]{}\mathbin{::}\mathbin{*}\to \mathbin{*}~~,{}\<[E]%
\\
\>[B]{}\mathbf{data}\;\Conid{ListOf}{}\<[16]%
\>[16]{}\mathbin{::}\mathbin{*}\to \mathbin{*}~~,{}\<[E]%
\\
\>[B]{}\mathbf{data}\;\Conid{SetOf}{}\<[16]%
\>[16]{}\mathbin{::}\mathbin{*}\to \mathbin{*}~~...{}\<[E]%
\ColumnHook
\end{hscode}\resethooks
If a key is associated with, for example, \ensuremath{\Conid{StringOf}\;\Conid{Int}} in our dictionary, we
mean that its value in the data store was serialized from an \ensuremath{\Conid{Int}} and should be
used as an \ensuremath{\Conid{Int}}. Types \ensuremath{\Conid{ListOf}\;\Varid{a}} and \ensuremath{\Conid{SetOf}\;\Varid{a}}, respectively, denotes that the
value is a list or a set of type \ensuremath{\Varid{a}}.

While the \ensuremath{\Varid{set}} command in \Hedis{} always writes a string to the data store,
the corresponding \ensuremath{\Varid{set}} in \Redis{} applies to any serializable type (those
in the class \ensuremath{\Conid{Serialize}}), and performs the encoding for the user:
\begin{hscode}\SaveRestoreHook
\column{B}{@{}>{\hspre}l<{\hspost}@{}}%
\column{6}{@{}>{\hspre}l<{\hspost}@{}}%
\column{E}{@{}>{\hspre}l<{\hspost}@{}}%
\>[B]{}\Varid{set}{}\<[6]%
\>[6]{}\mathbin{::}(\Conid{KnownSymbol}\;\Varid{k},\Conid{Serialize}\;\Varid{a}){}\<[E]%
\\
\>[6]{}\Rightarrow \Conid{Proxy}\;\Varid{k}\to \Varid{a}\to \Conid{Edis}\;\Varid{xs}\;(\Conid{Set}\;\Varid{xs}\;\Varid{k}\;(\Conid{StringOf}\;\Varid{a}))\;(\Conid{Either}\;\Conid{Reply}\;\Conid{Status}){}\<[E]%
\\
\>[B]{}\Varid{set}\;\Varid{key}\;\Varid{val}\mathrel{=}\Conid{Edis}\;(\Varid{\Conid{Hedis}.set}\;(\Varid{encodeKey}\;\Varid{key})\;(\Varid{encode}\;\Varid{val}))~~,{}\<[E]%
\ColumnHook
\end{hscode}\resethooks
For example, executing \ensuremath{\Varid{set}\;(\Conid{Proxy}\mathbin{::}\Conid{Proxy}\;\text{\tt \char34 A\char34})\;\Conid{True}} updates the dictionary
with an entry \ensuremath{\mbox{\textquotesingle}(\text{\tt \char34 A\char34},\Conid{StringOf}\;\Conid{Bool})}. If \ensuremath{\text{\tt \char34 A\char34}} is not in the dictionary,
this entry is added; otherwise the old type of \ensuremath{\text{\tt \char34 A\char34}} is updated to
\ensuremath{\Conid{StringOf}\;\Conid{Bool}}.

\Redis{} command \texttt{INCR} reads the (string) value of the given key, parses
it as an integer, and increments it by one, before storing it back. The command
\texttt{INCRBYFLOAT} increments the floating point value of a key by a given
amount. They are defined in \Edis{} below:
\begin{hscode}\SaveRestoreHook
\column{B}{@{}>{\hspre}l<{\hspost}@{}}%
\column{7}{@{}>{\hspre}l<{\hspost}@{}}%
\column{14}{@{}>{\hspre}l<{\hspost}@{}}%
\column{E}{@{}>{\hspre}l<{\hspost}@{}}%
\>[B]{}\Varid{incr}{}\<[7]%
\>[7]{}\mathbin{::}(\Conid{KnownSymbol}\;\Varid{k},\Conid{Get}\;\Varid{xs}\;\Varid{k}\mathrel{\sim}\Conid{StringOf}\;\Conid{Integer}){}\<[E]%
\\
\>[7]{}\Rightarrow \Conid{Proxy}\;\Varid{k}\to \Conid{Edis}\;\Varid{xs}\;\Varid{xs}\;(\Conid{R} \uplus \Conid{Integer}){}\<[E]%
\\
\>[B]{}\Varid{incr}\;\Varid{key}\mathrel{=}\Conid{Edis}\;(\Varid{\Conid{Hedis}.incr}\;(\Varid{encodeKey}\;\Varid{key}))~~,{}\<[E]%
\\[\blanklineskip]%
\>[B]{}\Varid{incrbyfloat}{}\<[14]%
\>[14]{}\mathbin{::}(\Conid{KnownSymbol}\;\Varid{k},\Conid{Get}\;\Varid{xs}\;\Varid{k}\mathrel{\sim}\Conid{StringOf}\;\Conid{Double}){}\<[E]%
\\
\>[14]{}\Rightarrow \Conid{Proxy}\;\Varid{k}\to \Conid{Double}\to \Conid{Edis}\;\Varid{xs}\;\Varid{xs}\;(\Conid{R} \uplus \Conid{Double}){}\<[E]%
\\
\>[B]{}\Varid{incrbyfloat}\;\Varid{key}\;\Varid{eps}\mathrel{=}\Conid{Edis}\;(\Varid{\Conid{Hedis}.incrbyfloat}\;(\Varid{encodeKey}\;\Varid{key})\;\Varid{eps})~~.{}\<[E]%
\ColumnHook
\end{hscode}\resethooks
Notice the use of (\ensuremath{\mathrel{\sim}}), \emph{equality constraints}~\cite{typeeq}, to enforce
that the intended type of the value of \ensuremath{\Varid{k}} must respectively be \ensuremath{\Conid{Integer}} and
\ensuremath{\Conid{Double}}. The function \ensuremath{\Varid{incr}} is only allowed to be called in a context where
the type checker is able to reduce \ensuremath{\Conid{Get}\;\Varid{xs}\;\Varid{k}} to \ensuremath{\Conid{StringOf}\;\Conid{Integer}} ---
recall that when \ensuremath{\Varid{k}} is not in \ensuremath{\Varid{xs}}, \ensuremath{\Conid{Get}\;\Varid{xs}\;\Varid{k}} cannot be fully reduced. The
type of \ensuremath{\Varid{incrbyfloat}} works in a similar way.

\subsection{Disjunctive Constraints}
\label{sec:disjunctive-constraints}

Recall, from Section \ref{sec:introduction}, that commands \texttt{LPUSH key
val} and \texttt{LLEN key} return normally either when \ensuremath{\Varid{key}} presents in the
data store and is assigned a list, or when \ensuremath{\Varid{key}} does not present at all.
What we wish to have in their constraint is thus a predicate equivalent to \ensuremath{\Conid{Get}\;\Varid{xs}\;\Varid{k}\doubleequals\Conid{ListOf}\;\Varid{a}\mathrel{\vee}\neg \;(\Conid{Member}\;\Varid{xs}\;\Varid{k})}. In fact, many \Redis{} commands
are invokable under such ``well-typed, or non-existent'' precondition.

To impose a conjunctive constraint \ensuremath{\Conid{P}\mathrel{\wedge}\Conid{Q}}, one may simply put them both in the
type: \ensuremath{(\Conid{P},\Conid{Q})\Rightarrow \mathbin{...}}. Expressing disjunctive constraints is only slightly
harder, thanks to our type-level functions. We may thus write the predicate as:
\begin{hscode}\SaveRestoreHook
\column{B}{@{}>{\hspre}l<{\hspost}@{}}%
\column{E}{@{}>{\hspre}l<{\hspost}@{}}%
\>[B]{}\Conid{Get}\;\Varid{xs}\;\Varid{k}\mathrel{\sim}\Conid{ListOf}\;\Varid{a}\mathrel{\Conid{`Or`}}\Conid{Not}\;(\Conid{Member}\;\Varid{xs}\;\Varid{k})~~.{}\<[E]%
\ColumnHook
\end{hscode}\resethooks
To avoid referring to \ensuremath{\Varid{a}}, which might not exist, we define an auxiliary predicate \ensuremath{\Conid{IsList}\mathbin{::}\mathbin{*}\to \Conid{Bool}} such that \ensuremath{\Conid{IsList}\;\Varid{t}} reduces to \ensuremath{\mbox{\textquotesingle}\Conid{True}}
only if \ensuremath{\Varid{t}\mathrel{=}\Conid{ListOf}\;\Varid{a}}. As many \Redis{} commands are invokable only under such
``well-typed, or non-existent'' precondition, we give names to such constraints,
as seen in Figure~\ref{fig:xxxOrNX}.

\begin{figure}[t]
\begin{hscode}\SaveRestoreHook
\column{B}{@{}>{\hspre}l<{\hspost}@{}}%
\column{3}{@{}>{\hspre}l<{\hspost}@{}}%
\column{5}{@{}>{\hspre}l<{\hspost}@{}}%
\column{14}{@{}>{\hspre}l<{\hspost}@{}}%
\column{18}{@{}>{\hspre}l<{\hspost}@{}}%
\column{22}{@{}>{\hspre}l<{\hspost}@{}}%
\column{24}{@{}>{\hspre}l<{\hspost}@{}}%
\column{28}{@{}>{\hspre}l<{\hspost}@{}}%
\column{E}{@{}>{\hspre}l<{\hspost}@{}}%
\>[B]{}\mathbf{type}\;\Varid{family}\;\Conid{IsList}\;(\Varid{t}\mathbin{::}\mathbin{*})\mathbin{::}\Conid{Bool}\;\mathbf{where}{}\<[E]%
\\
\>[B]{}\hsindent{5}{}\<[5]%
\>[5]{}\Conid{IsList}\;(\Conid{ListOf}\;\Varid{a}){}\<[24]%
\>[24]{}\mathrel{=}\mbox{\textquotesingle}\Conid{True}{}\<[E]%
\\
\>[B]{}\hsindent{5}{}\<[5]%
\>[5]{}\Conid{IsList}\;\Varid{t}{}\<[24]%
\>[24]{}\mathrel{=}\mbox{\textquotesingle}\Conid{False}{}\<[E]%
\\
\>[B]{}\mathbf{type}\;\Varid{family}\;\Conid{IsSet}\;(\Varid{t}\mathbin{::}\mathbin{*})\mathbin{::}\Conid{Bool}\;\mathbf{where}{}\<[E]%
\\
\>[B]{}\hsindent{5}{}\<[5]%
\>[5]{}\Conid{IsSet}\;(\Conid{SetOf}\;\Varid{a}){}\<[22]%
\>[22]{}\mathrel{=}\mbox{\textquotesingle}\Conid{True}{}\<[E]%
\\
\>[B]{}\hsindent{5}{}\<[5]%
\>[5]{}\Conid{IsSet}\;\Varid{t}{}\<[22]%
\>[22]{}\mathrel{=}\mbox{\textquotesingle}\Conid{False}{}\<[E]%
\\
\>[B]{}\mathbf{type}\;\Varid{family}\;\Conid{IsString}\;(\Varid{t}\mathbin{::}\mathbin{*})\mathbin{::}\Conid{Bool}\;\mathbf{where}{}\<[E]%
\\
\>[B]{}\hsindent{5}{}\<[5]%
\>[5]{}\Conid{IsString}\;(\Conid{StringOf}\;\Varid{a}){}\<[28]%
\>[28]{}\mathrel{=}\mbox{\textquotesingle}\Conid{True}{}\<[E]%
\\
\>[B]{}\hsindent{5}{}\<[5]%
\>[5]{}\Conid{IsString}\;\Varid{t}{}\<[28]%
\>[28]{}\mathrel{=}\mbox{\textquotesingle}\Conid{False}{}\<[E]%
\\[\blanklineskip]%
\>[B]{}\mathbf{type}\;\Conid{ListOrNX}\;{}\<[18]%
\>[18]{}\Varid{xs}\;\Varid{k}\mathrel{=}{}\<[E]%
\\
\>[B]{}\hsindent{3}{}\<[3]%
\>[3]{}(\Conid{IsList}\;{}\<[14]%
\>[14]{}(\Conid{Get}\;\Varid{xs}\;\Varid{k})\mathrel{\Conid{`Or`}}\Conid{Not}\;(\Conid{Member}\;\Varid{xs}\;\Varid{k}))\mathrel{\sim}\mbox{\textquotesingle}\Conid{True}{}\<[E]%
\\
\>[B]{}\mathbf{type}\;\Conid{SetOrNX}\;{}\<[18]%
\>[18]{}\Varid{xs}\;\Varid{k}\mathrel{=}{}\<[E]%
\\
\>[B]{}\hsindent{3}{}\<[3]%
\>[3]{}(\Conid{IsSet}\;{}\<[14]%
\>[14]{}(\Conid{Get}\;\Varid{xs}\;\Varid{k})\mathrel{\Conid{`Or`}}\Conid{Not}\;(\Conid{Member}\;\Varid{xs}\;\Varid{k}))\mathrel{\sim}\mbox{\textquotesingle}\Conid{True}{}\<[E]%
\\
\>[B]{}\mathbf{type}\;\Conid{StringOrNX}\;{}\<[18]%
\>[18]{}\Varid{xs}\;\Varid{k}\mathrel{=}{}\<[E]%
\\
\>[B]{}\hsindent{3}{}\<[3]%
\>[3]{}(\Conid{IsString}\;{}\<[14]%
\>[14]{}(\Conid{Get}\;\Varid{xs}\;\Varid{k})\mathrel{\Conid{`Or`}}\Conid{Not}\;(\Conid{Member}\;\Varid{xs}\;\Varid{k}))\mathrel{\sim}\mbox{\textquotesingle}\Conid{True}{}\<[E]%
\ColumnHook
\end{hscode}\resethooks
\caption{The ``well-typed, or non-existent'' constraints.}
\label{fig:xxxOrNX}
\end{figure}

The \Edis{} counterpart of \texttt{LPUSH} and \texttt{LLEN} are therefore:
\begin{hscode}\SaveRestoreHook
\column{B}{@{}>{\hspre}l<{\hspost}@{}}%
\column{7}{@{}>{\hspre}l<{\hspost}@{}}%
\column{8}{@{}>{\hspre}l<{\hspost}@{}}%
\column{E}{@{}>{\hspre}l<{\hspost}@{}}%
\>[B]{}\Varid{lpush}{}\<[8]%
\>[8]{}\mathbin{::}(\Conid{KnownSymbol}\;\Varid{k},\Conid{Serialize}\;\Varid{a},\Conid{ListOrNX}\;\Varid{xs}\;\Varid{k}){}\<[E]%
\\
\>[8]{}\Rightarrow \Conid{Proxy}\;\Varid{k}\to \Varid{a}\to \Conid{Edis}\;\Varid{xs}\;(\Conid{Set}\;\Varid{xs}\;\Varid{k}\;(\Conid{ListOf}\;\Varid{a}))\;(\Conid{R} \uplus \Conid{Integer}){}\<[E]%
\\
\>[B]{}\Varid{lpush}\;\Varid{key}\;\Varid{val}\mathrel{=}\Conid{Edis}\;(\Varid{\Conid{Hedis}.lpush}\;(\Varid{encodeKey}\;\Varid{key})\;[\mskip1.5mu \Varid{encode}\;\Varid{val}\mskip1.5mu])~~,{}\<[E]%
\\[\blanklineskip]%
\>[B]{}\Varid{llen}{}\<[7]%
\>[7]{}\mathbin{::}(\Conid{KnownSymbol}\;\Varid{k},\Conid{ListOrNX}\;\Varid{xs}\;\Varid{k}){}\<[E]%
\\
\>[7]{}\Rightarrow \Conid{Proxy}\;\Varid{k}\to \Conid{Edis}\;\Varid{xs}\;\Varid{xs}\;(\Conid{R} \uplus \Conid{Integer}){}\<[E]%
\\
\>[B]{}\Varid{llen}\;\Varid{key}\mathrel{=}\Conid{Edis}\;(\Varid{\Conid{Hedis}.llen}\;(\Varid{encodeKey}\;\Varid{key}))~~.{}\<[E]%
\ColumnHook
\end{hscode}\resethooks
Similarly, the type of \ensuremath{\Varid{sadd}}, a function we have talked about a lot,
is given below:
\begin{hscode}\SaveRestoreHook
\column{B}{@{}>{\hspre}l<{\hspost}@{}}%
\column{7}{@{}>{\hspre}l<{\hspost}@{}}%
\column{E}{@{}>{\hspre}l<{\hspost}@{}}%
\>[B]{}\Varid{sadd}{}\<[7]%
\>[7]{}\mathbin{::}(\Conid{KnownSymbol}\;\Varid{k},\Conid{Serialize}\;\Varid{a},\Conid{SetOrNX}\;\Varid{xs}\;\Varid{k}){}\<[E]%
\\
\>[7]{}\Rightarrow \Conid{Proxy}\;\Varid{k}\to \Varid{a}\to \Conid{Edis}\;\Varid{xs}\;(\Conid{Set}\;\Varid{xs}\;\Varid{k}\;(\Conid{SetOf}\;\Varid{a}))\;(\Conid{R} \uplus \Conid{Integer}){}\<[E]%
\\
\>[B]{}\Varid{sadd}\;\Varid{key}\;\Varid{val}\mathrel{=}\Conid{Edis}\;(\Varid{\Conid{Hedis}.sadd}\;(\Varid{encodeKey}\;\Varid{key})\;[\mskip1.5mu \Varid{encode}\;\Varid{val}\mskip1.5mu])~~,{}\<[E]%
\ColumnHook
\end{hscode}\resethooks

To see a command with a more complex type, consider \ensuremath{\Varid{setnx}}, which
uses the type-level function \ensuremath{\Conid{If}} defined in Section \ref{sec:type-fun}:
\begin{hscode}\SaveRestoreHook
\column{B}{@{}>{\hspre}l<{\hspost}@{}}%
\column{8}{@{}>{\hspre}l<{\hspost}@{}}%
\column{36}{@{}>{\hspre}l<{\hspost}@{}}%
\column{E}{@{}>{\hspre}l<{\hspost}@{}}%
\>[B]{}\Varid{setnx}{}\<[8]%
\>[8]{}\mathbin{::}(\Conid{KnownSymbol}\;\Varid{k},\Conid{Serialize}\;\Varid{a}){}\<[E]%
\\
\>[8]{}\Rightarrow \Conid{Proxy}\;\Varid{k}\to \Varid{a}\to \Conid{Edis}\;\Varid{xs}\;{}\<[36]%
\>[36]{}(\Conid{If}\;(\Conid{Member}\;\Varid{xs}\;\Varid{k})\;\Varid{xs}\;(\Conid{Set}\;\Varid{xs}\;\Varid{k}\;(\Conid{StringOf}\;\Varid{a})))\;{}\<[E]%
\\
\>[36]{}(\Conid{Either}\;\Conid{Reply}\;\Conid{Bool}){}\<[E]%
\\
\>[B]{}\Varid{setnx}\;\Varid{key}\;\Varid{val}\mathrel{=}\Conid{Edis}\;(\Varid{\Conid{Hedis}.setnx}\;(\Varid{encodeKey}\;\Varid{key})\;(\Varid{encode}\;\Varid{val}))~~.{}\<[E]%
\ColumnHook
\end{hscode}\resethooks
From the type one can see that \ensuremath{\Varid{setnx}\;\Varid{key}\;\Varid{val}} creates a new entry \ensuremath{(\Varid{key},\Varid{val})}
in the data store only if \ensuremath{\Varid{key}} is fresh. The type of \ensuremath{\Varid{setnx}} computes a
postcondition for static checking, as well as serving as a good documentation
for its semantics.

\subsection{Hashes}

{\em Hash} is a useful datatype supported by \Redis{}. While the \Redis{} data
store can be seen as a set of key/value pairs, a hash is itself a set of
field/value pairs. The following commands assigns a hash to key \texttt{user}.
The fields are \texttt{name}, \texttt{birthyear}, and \texttt{verified},
respectively with values \texttt{banacorn}, \texttt{1992}, and \texttt{1}.
\begin{Verbatim}[xleftmargin=.4in]
redis> hmset user name banacorn birthyear 1992 verified 1
OK
redis> hget user name
"banacorn"
redis> hget user birthyear
"1992"
\end{Verbatim}

For a hash to be useful, we should allow the fields to have different types. To
keep track of types of fields in a hash, \ensuremath{\Conid{HashOf}} takes a list of \ensuremath{(\Conid{Symbol},\mathbin{*})}
pairs:
\begin{hscode}\SaveRestoreHook
\column{B}{@{}>{\hspre}l<{\hspost}@{}}%
\column{E}{@{}>{\hspre}l<{\hspost}@{}}%
\>[B]{}\mathbf{data}\;\Conid{HashOf}\mathbin{::}[\mskip1.5mu (\Conid{Symbol},\mathbin{*})\mskip1.5mu]\to \mathbin{*}~~.{}\<[E]%
\ColumnHook
\end{hscode}\resethooks
By having an entry \ensuremath{(\Varid{k},\Conid{HashOF}\;\Varid{ys})} in a dictionary, we denote that the value of
key \ensuremath{\Varid{k}} is a hash whose fields and their types are specified by \ensuremath{\Varid{ys}}, which is
also a dictionary.

\begin{figure}[t]
\begin{hscode}\SaveRestoreHook
\column{B}{@{}>{\hspre}l<{\hspost}@{}}%
\column{5}{@{}>{\hspre}l<{\hspost}@{}}%
\column{22}{@{}>{\hspre}l<{\hspost}@{}}%
\column{34}{@{}>{\hspre}l<{\hspost}@{}}%
\column{35}{@{}>{\hspre}l<{\hspost}@{}}%
\column{36}{@{}>{\hspre}l<{\hspost}@{}}%
\column{43}{@{}>{\hspre}l<{\hspost}@{}}%
\column{44}{@{}>{\hspre}l<{\hspost}@{}}%
\column{45}{@{}>{\hspre}l<{\hspost}@{}}%
\column{76}{@{}>{\hspre}l<{\hspost}@{}}%
\column{77}{@{}>{\hspre}l<{\hspost}@{}}%
\column{81}{@{}>{\hspre}l<{\hspost}@{}}%
\column{E}{@{}>{\hspre}l<{\hspost}@{}}%
\>[B]{}\mathbf{type}\;\Varid{family}\;\Conid{GetHash}\;(\Varid{xs}\mathbin{::}[\mskip1.5mu (\Conid{Symbol},\mathbin{*})\mskip1.5mu])\;(\Varid{k}\mathbin{::}\Conid{Symbol})\;(\Varid{f}\mathbin{::}\Conid{Symbol})\mathbin{::}\mathbin{*}\mathbf{where}{}\<[E]%
\\
\>[B]{}\hsindent{5}{}\<[5]%
\>[5]{}\Conid{GetHash}\;(\mbox{\textquotesingle}(\Varid{k},\Conid{HashOf}\;\Varid{hs}){}\<[35]%
\>[35]{}\mathbin{\mbox{\textquotesingle}\!:}\Varid{xs})\;{}\<[44]%
\>[44]{}\Varid{k}\;\Varid{f}\mathrel{=}\Conid{Get}\;\Varid{hs}\;\Varid{f}{}\<[E]%
\\
\>[B]{}\hsindent{5}{}\<[5]%
\>[5]{}\Conid{GetHash}\;(\mbox{\textquotesingle}(\Varid{l},\Varid{y}){}\<[35]%
\>[35]{}\mathbin{\mbox{\textquotesingle}\!:}\Varid{xs})\;{}\<[44]%
\>[44]{}\Varid{k}\;\Varid{f}\mathrel{=}\Conid{GetHash}\;\Varid{xs}\;\Varid{k}\;\Varid{f}{}\<[E]%
\\[\blanklineskip]%
\>[B]{}\mathbf{type}\;\Varid{family}\;\Conid{SetHash}\;{}\<[22]%
\>[22]{}(\Varid{xs}\mathbin{::}[\mskip1.5mu (\Conid{Symbol},\mathbin{*})\mskip1.5mu])\;(\Varid{k}\mathbin{::}\Conid{Symbol})\;{}\<[E]%
\\
\>[22]{}(\Varid{f}\mathbin{::}\Conid{Symbol})\;(\Varid{a}\mathbin{::}\mathbin{*})\mathbin{::}[\mskip1.5mu (\Conid{Symbol},\mathbin{*})\mskip1.5mu]\;\mathbf{where}{}\<[E]%
\\
\>[B]{}\hsindent{5}{}\<[5]%
\>[5]{}\Conid{SetHash}\;\mbox{\textquotesingle}[~]\;{}\<[44]%
\>[44]{}\Varid{k}\;\Varid{f}\;\Varid{a}\mathrel{=}\mbox{\textquotesingle}(\Varid{k},\Conid{HashOf}\;(\Conid{Set}\;\mbox{\textquotesingle}[~]\;\Varid{f}\;\Varid{a}))\mathbin{\mbox{\textquotesingle}\!:}\mbox{\textquotesingle}[~]{}\<[E]%
\\
\>[B]{}\hsindent{5}{}\<[5]%
\>[5]{}\Conid{SetHash}\;(\mbox{\textquotesingle}(\Varid{k},\Conid{HashOf}\;\Varid{hs}){}\<[35]%
\>[35]{}\mathbin{\mbox{\textquotesingle}\!:}\Varid{xs})\;{}\<[44]%
\>[44]{}\Varid{k}\;\Varid{f}\;\Varid{a}\mathrel{=}\mbox{\textquotesingle}(\Varid{k},\Conid{HashOf}\;(\Conid{Set}\;\Varid{hs}\;{}\<[77]%
\>[77]{}\Varid{f}\;\Varid{a}))\mathbin{\mbox{\textquotesingle}\!:}\Varid{xs}{}\<[E]%
\\
\>[B]{}\hsindent{5}{}\<[5]%
\>[5]{}\Conid{SetHash}\;(\mbox{\textquotesingle}(\Varid{l},\Varid{y}){}\<[35]%
\>[35]{}\mathbin{\mbox{\textquotesingle}\!:}\Varid{xs})\;{}\<[44]%
\>[44]{}\Varid{k}\;\Varid{f}\;\Varid{a}\mathrel{=}\mbox{\textquotesingle}(\Varid{l},\Varid{y}){}\<[81]%
\>[81]{}\mathbin{\mbox{\textquotesingle}\!:}\Conid{SetHash}\;\Varid{xs}\;\Varid{k}\;\Varid{f}\;\Varid{a}{}\<[E]%
\\[\blanklineskip]%
\>[B]{}\mathbf{type}\;\Varid{family}\;\Conid{DelHash}\;(\Varid{xs}\mathbin{::}[\mskip1.5mu (\Conid{Symbol},\mathbin{*})\mskip1.5mu])\;(\Varid{k}\mathbin{::}\Conid{Symbol})\;(\Varid{f}\mathbin{::}\Conid{Symbol})\mathbin{::}[\mskip1.5mu (\Conid{Symbol},\mathbin{*})\mskip1.5mu]\;\mathbf{where}{}\<[E]%
\\
\>[B]{}\hsindent{5}{}\<[5]%
\>[5]{}\Conid{DelHash}\;\mbox{\textquotesingle}[~]\;{}\<[43]%
\>[43]{}\Varid{k}\;\Varid{f}\mathrel{=}\mbox{\textquotesingle}[~]{}\<[E]%
\\
\>[B]{}\hsindent{5}{}\<[5]%
\>[5]{}\Conid{DelHash}\;(\mbox{\textquotesingle}(\Varid{k},\Conid{HashOf}\;\Varid{hs})\mathbin{\mbox{\textquotesingle}\!:}\Varid{xs})\;{}\<[43]%
\>[43]{}\Varid{k}\;\Varid{f}\mathrel{=}\mbox{\textquotesingle}(\Varid{k},\Conid{HashOf}\;(\Conid{Del}\;\Varid{hs}\;\Varid{f}))\mathbin{\mbox{\textquotesingle}\!:}\Varid{xs}{}\<[E]%
\\
\>[B]{}\hsindent{5}{}\<[5]%
\>[5]{}\Conid{DelHash}\;(\mbox{\textquotesingle}(\Varid{l},\Varid{y}){}\<[34]%
\>[34]{}\mathbin{\mbox{\textquotesingle}\!:}\Varid{xs})\;{}\<[43]%
\>[43]{}\Varid{k}\;\Varid{f}\mathrel{=}\mbox{\textquotesingle}(\Varid{l},\Varid{y}){}\<[76]%
\>[76]{}\mathbin{\mbox{\textquotesingle}\!:}\Conid{DelHash}\;\Varid{xs}\;\Varid{k}\;\Varid{f}{}\<[E]%
\\[\blanklineskip]%
\>[B]{}\mathbf{type}\;\Varid{family}\;\Conid{MemHash}\;(\Varid{xs}\mathbin{::}[\mskip1.5mu (\Conid{Symbol},\mathbin{*})\mskip1.5mu])\;(\Varid{k}\mathbin{::}\Conid{Symbol})\;(\Varid{f}\mathbin{::}\Conid{Symbol})\mathbin{::}\Conid{Bool}\;\mathbf{where}{}\<[E]%
\\
\>[B]{}\hsindent{5}{}\<[5]%
\>[5]{}\Conid{MemHash}\;\mbox{\textquotesingle}[~]\;{}\<[45]%
\>[45]{}\Varid{k}\;\Varid{f}\mathrel{=}\mbox{\textquotesingle}\Conid{False}{}\<[E]%
\\
\>[B]{}\hsindent{5}{}\<[5]%
\>[5]{}\Conid{MemHash}\;(\mbox{\textquotesingle}(\Varid{k},\Conid{HashOf}\;\Varid{hs}){}\<[36]%
\>[36]{}\mathbin{\mbox{\textquotesingle}\!:}\Varid{xs})\;{}\<[45]%
\>[45]{}\Varid{k}\;\Varid{f}\mathrel{=}\Conid{Member}\;\Varid{hs}\;\Varid{f}{}\<[E]%
\\
\>[B]{}\hsindent{5}{}\<[5]%
\>[5]{}\Conid{MemHash}\;(\mbox{\textquotesingle}(\Varid{k},\Varid{x}){}\<[36]%
\>[36]{}\mathbin{\mbox{\textquotesingle}\!:}\Varid{xs})\;{}\<[45]%
\>[45]{}\Varid{k}\;\Varid{f}\mathrel{=}\mbox{\textquotesingle}\Conid{False}{}\<[E]%
\\
\>[B]{}\hsindent{5}{}\<[5]%
\>[5]{}\Conid{MemHash}\;(\mbox{\textquotesingle}(\Varid{l},\Varid{y}){}\<[36]%
\>[36]{}\mathbin{\mbox{\textquotesingle}\!:}\Varid{xs})\;{}\<[45]%
\>[45]{}\Varid{k}\;\Varid{f}\mathrel{=}\Conid{MemHash}\;\Varid{xs}\;\Varid{k}\;\Varid{f}{}\<[E]%
\ColumnHook
\end{hscode}\resethooks
\caption{Type-level operations for dictionaries with hashes.}
\label{fig:xxxHash}
\end{figure}

Figure \ref{fig:xxxHash} presents some operations we need on dictionaries when
dealing with hashes. Let \ensuremath{\Varid{xs}} be a dictionary, \ensuremath{\Conid{GetHash}\;\Varid{xs}\;\Varid{k}\;\Varid{f}} returns the type
of field \ensuremath{\Varid{f}} in the hash assigned to key \ensuremath{\Varid{k}}, if both \ensuremath{\Varid{k}} and \ensuremath{\Varid{f}} exists.
\ensuremath{\Conid{SetHash}\;\Varid{xs}\;\Varid{k}\;\Varid{f}\;\Varid{a}} assigns the type \ensuremath{\Varid{a}} to the field \ensuremath{\Varid{f}} of hash \ensuremath{\Varid{k}}; if either
\ensuremath{\Varid{f}} or \ensuremath{\Varid{k}} does not exist, the hash/field is created. \ensuremath{\Conid{Del}\;\Varid{xs}\;\Varid{k}\;\Varid{f}} removes a
field, while \ensuremath{\Conid{MemHash}\;\Varid{xs}\;\Varid{k}\;\Varid{f}} checks whether the key \ensuremath{\Varid{k}} exists in \ensuremath{\Varid{xs}}, and its
value is a hash having field \ensuremath{\Varid{f}}. Their definitions make use of functions \ensuremath{\Conid{Get}},
\ensuremath{\Conid{Set}}, and \ensuremath{\Conid{Member}} defined for dictionaries.

Once those type-level functions are defined, embedding of \Hedis{} commands for
hashes is more or less routine. For example, functions \ensuremath{\Varid{hset}} and \ensuremath{\Varid{hget}}
are shown below. Note that, instead of \ensuremath{\Varid{hmset}} (available in \Hedis{}), we
provide a function \ensuremath{\Varid{hset}} that assigns fields and values one pair at a time.
\begin{hscode}\SaveRestoreHook
\column{B}{@{}>{\hspre}l<{\hspost}@{}}%
\column{3}{@{}>{\hspre}l<{\hspost}@{}}%
\column{7}{@{}>{\hspre}l<{\hspost}@{}}%
\column{13}{@{}>{\hspre}l<{\hspost}@{}}%
\column{E}{@{}>{\hspre}l<{\hspost}@{}}%
\>[B]{}\Varid{hset}{}\<[7]%
\>[7]{}\mathbin{::}(\Conid{KnownSymbol}\;\Varid{k},\Conid{KnownSymbol}\;\Varid{f},\Conid{Serialize}\;\Varid{a},\Conid{HashOrNX}\;\Varid{xs}\;\Varid{k}){}\<[E]%
\\
\>[7]{}\Rightarrow \Conid{Proxy}\;\Varid{k}\to \Conid{Proxy}\;\Varid{f}\to \Varid{a}{}\<[E]%
\\
\>[7]{}\to \Conid{Edis}\;\Varid{xs}\;(\Conid{SetHash}\;\Varid{xs}\;\Varid{k}\;\Varid{f}\;(\Conid{StringOf}\;\Varid{a}))\;(\Conid{R} \uplus \Conid{Bool}){}\<[E]%
\\
\>[B]{}\Varid{hset}\;\Varid{key}\;\Varid{field}\;\Varid{val}\mathrel{=}{}\<[E]%
\\
\>[B]{}\hsindent{3}{}\<[3]%
\>[3]{}\Conid{Edis}\;(\Varid{\Conid{Hedis}.hset}\;(\Varid{encodeKey}\;\Varid{key})\;(\Varid{encodeKey}\;\Varid{field})\;(\Varid{encode}\;\Varid{val}))~~,{}\<[E]%
\\[\blanklineskip]%
\>[B]{}\Varid{hget}{}\<[7]%
\>[7]{}\mathbin{::}({}\<[13]%
\>[13]{}\Conid{KnownSymbol}\;\Varid{k},\Conid{KnownSymbol}\;\Varid{f},\Conid{Serialize}\;\Varid{a},{}\<[E]%
\\
\>[13]{}\Conid{StringOf}\;\Varid{a}\mathrel{\sim}\Conid{GetHash}\;\Varid{xs}\;\Varid{k}\;\Varid{f}){}\<[E]%
\\
\>[7]{}\Rightarrow \Conid{Proxy}\;\Varid{k}\to \Conid{Proxy}\;\Varid{f}\to \Conid{Edis}\;\Varid{xs}\;\Varid{xs}\;(\Conid{R} \uplus \Conid{Maybe}\;\Varid{a}){}\<[E]%
\\
\>[B]{}\Varid{hget}\;\Varid{key}\;\Varid{field}\mathrel{=}{}\<[E]%
\\
\>[B]{}\hsindent{3}{}\<[3]%
\>[3]{}\Conid{Edis}\;(\Varid{\Conid{Hedis}.hget}\;(\Varid{encodeKey}\;\Varid{key})\;(\Varid{encodeKey}\;\Varid{field})\bind \Varid{decodeAsMaybe})~~,{}\<[E]%
\ColumnHook
\end{hscode}\resethooks
where \ensuremath{\Varid{decodeAsMaybe}\mathbin{::}\Conid{Serialize}\;\Varid{a}\Rightarrow (\Conid{R} \uplus \Conid{Maybe}\;\Conid{ByteString})\to \Conid{Redis}\;(\Conid{R} \uplus \Conid{Maybe}\;\Varid{a})}, using the function \ensuremath{\Varid{decode}}
mentioned in Section \ref{sec:proxy-key}, parses the \ensuremath{\Conid{ByteString}} in
\ensuremath{\Conid{R} \uplus \Conid{Maybe}\;\anonymous } to type \ensuremath{\Varid{a}}. The definition is a bit tedious but
routine.

Note that, instead of \ensuremath{\Varid{hmset}} (available in \Hedis{}), we
provide a function \ensuremath{\Varid{hset}} that assigns fields and values one pair at at time.
We will talk about difficulties of implementing \ensuremath{\Varid{hmset}} in
Section~\ref{sec:discussions}.

\subsection{Assertions}
\label{sec:assertions}

% Consider the following scenario: We want to retrieve the value of some existing
% key, say, |"ab initio"|, with the function\footnote{The semantics of |get| in
% \Redis{} is actually more forgiving. See Section~\ref{sec:discussions}.}:
% \begin{spec}
% get  :: (KnownSymbol k, Serialize a, StringOf a ~ Get xs k)
%      => Proxy k -> Edis xs xs (EitherReply (Maybe a)) {-"~~."-}
% \end{spec}
%
% However, Haskell would complain that, the key |"ab initio"| was not
% before seen in the dictionary. So we could never write programs to retrieve the
% value of |"ab initio"|, unless its creation was witnessed by the type checker.


Finally, the creation/update behavior of \Redis{} functions is, in our opinion,
very error-prone. It might be preferable if we can explicit declare some new
keys, after ensuring that they do not already exist (in our types). This can be done below:
% It might be preferable if we can explicitly declare some new keys, under the
%  precondition that they do not already exist in our types. This can be done as
%  follows:
\begin{hscode}\SaveRestoreHook
\column{B}{@{}>{\hspre}l<{\hspost}@{}}%
\column{10}{@{}>{\hspre}l<{\hspost}@{}}%
\column{E}{@{}>{\hspre}l<{\hspost}@{}}%
\>[B]{}\Varid{declare}{}\<[10]%
\>[10]{}\mathbin{::}(\Conid{KnownSymbol}\;\Varid{k},\Conid{Member}\;\Varid{xs}\;\Varid{k}\mathrel{\sim}\Conid{False}){}\<[E]%
\\
\>[10]{}\Rightarrow \Conid{Proxy}\;\Varid{k}\to \Conid{Proxy}\;\Varid{a}\to \Conid{Edis}\;\Varid{xs}\;(\Conid{Set}\;\Varid{xs}\;\Varid{k}\;\Varid{a})\;(){}\<[E]%
\\
\>[B]{}\Varid{declare}\;\Varid{key}\;\Varid{typ}\mathrel{=}\Conid{Edis}\;(\Varid{return}\;())~~.{}\<[E]%
\ColumnHook
\end{hscode}\resethooks
% renounce :: (KnownSymbol s, Member xs s ~ True)
%         => Proxy s -> Edis xs (Del xs s) ()
% renounce s = Edis $ return () {-"~~."-}
The command \ensuremath{\Varid{declare}\;\Varid{key}\;\Varid{typ}}, where \ensuremath{\Varid{typ}} is the proxy of \ensuremath{\Varid{a}}, adds a fresh
 \ensuremath{\Varid{key}} with type \ensuremath{\Varid{a}} into the dictionary. Notice that \ensuremath{\Varid{declare}} does nothing at
 term level, but simply returns \ensuremath{()}, since it only has effects on types.

% The key is not actually created yet.
% The declaration, however, ensures that if |key| is actually created, perhaps
% by one of those ``well-typed or non-existent'' command, its type must be |a|.
% The command |start| initializes the dictionary to the empty list:

Another command for type level assertion, \ensuremath{\Varid{start}}, initializes the dictionary to
 the empty list, comes in handy when starting a series of \Edis{} commands:

\begin{hscode}\SaveRestoreHook
\column{B}{@{}>{\hspre}l<{\hspost}@{}}%
\column{E}{@{}>{\hspre}l<{\hspost}@{}}%
\>[B]{}\Varid{start}\mathbin{::}\Conid{Edis}\;\mbox{\textquotesingle}[~]\;\mbox{\textquotesingle}[~]\;(){}\<[E]%
\\
\>[B]{}\Varid{start}\mathrel{=}\Conid{Edis}\;(\Varid{return}\;())~~.{}\<[E]%
\ColumnHook
\end{hscode}\resethooks

\subsection{A Larger Example}

We present a larger example as a summary. The task is to store a queue of
messages in \Redis{}. Messages are represented by a \ensuremath{\Conid{ByteString}} and an
\ensuremath{\Conid{Integer}} identifier:
\begin{hscode}\SaveRestoreHook
\column{B}{@{}>{\hspre}l<{\hspost}@{}}%
\column{35}{@{}>{\hspre}l<{\hspost}@{}}%
\column{E}{@{}>{\hspre}l<{\hspost}@{}}%
\>[B]{}\mathbf{data}\;\Conid{Message}\mathrel{=}\Conid{Msg}\;\Conid{ByteString}\;\Conid{Integer}\;\mathbf{deriving}\;(\Conid{Show},\Conid{Generic})~~,{}\<[E]%
\\
\>[B]{}\mathbf{instance}\;\Conid{Serialize}\;\Conid{Message}\;\mathbf{where}\;{}\<[35]%
\>[35]{}~~.{}\<[E]%
\ColumnHook
\end{hscode}\resethooks

In the data store, the queue is represented by a list. Before pushing a message
into the queue, we increment \ensuremath{\Varid{counter}}, a key storing a counter, and use it as the
identifier of the message:
\begin{hscode}\SaveRestoreHook
\column{B}{@{}>{\hspre}l<{\hspost}@{}}%
\column{5}{@{}>{\hspre}l<{\hspost}@{}}%
\column{7}{@{}>{\hspre}l<{\hspost}@{}}%
\column{13}{@{}>{\hspre}l<{\hspost}@{}}%
\column{E}{@{}>{\hspre}l<{\hspost}@{}}%
\>[B]{}\Varid{push}{}\<[7]%
\>[7]{}\mathbin{::}(\Conid{StringOfIntegerOrNX}\;\Varid{xs}\;\text{\tt \char34 counter\char34},\Conid{ListOrNX}\;\Varid{xs}\;\text{\tt \char34 queue\char34}){}\<[E]%
\\
\>[7]{}\Rightarrow \Conid{ByteString}\to \Conid{Edis}\;\Varid{xs}\;(\Conid{Set}\;\Varid{xs}\;\text{\tt \char34 queue\char34}\;(\Conid{ListOf}\;\Conid{Message}))\;(\Conid{R} \uplus \Conid{Integer}){}\<[E]%
\\
\>[B]{}\Varid{push}\;\Varid{msg}\mathrel{=}{}\<[13]%
\>[13]{}\Varid{incr}\;\Varid{kCounter}{}\<[E]%
\\
\>[B]{}\hsindent{5}{}\<[5]%
\>[5]{}\mathbin{`\Varid{bind}`}{}\<[13]%
\>[13]{}\lambda \Varid{i}\to \Varid{lpush}\;\Varid{kQueue}\;(\Conid{Msg}\;\Varid{msg}\;(\Varid{fromRight}\;\Varid{i}))~~,{}\<[E]%
\ColumnHook
\end{hscode}\resethooks
%​
where \ensuremath{\Varid{fromRight}\mathbin{::}\Conid{Either}\;\Varid{a}\;\Varid{b}\to \Varid{b}} extracts the value wrapped by constructor
\ensuremath{\Conid{Right}}, and the constraint \ensuremath{\Conid{StringOfIntegerOrNX}\;\Varid{xs}\;\Varid{k}} holds if either \ensuremath{\Varid{k}}
presents in \ensuremath{\Varid{xs}} and is converted from an \ensuremath{\Conid{Integer}}, or \ensuremath{\Varid{k}} does not
present in \ensuremath{\Varid{xs}}. For brevity, the proxies are given names: \\
\noindent{\centering %\small
\begin{minipage}[b]{0.4\linewidth}
\begin{hscode}\SaveRestoreHook
\column{B}{@{}>{\hspre}l<{\hspost}@{}}%
\column{14}{@{}>{\hspre}l<{\hspost}@{}}%
\column{E}{@{}>{\hspre}l<{\hspost}@{}}%
\>[B]{}\Varid{kCounter}\mathbin{::}{}\<[14]%
\>[14]{}\Conid{Proxy}\;\text{\tt \char34 counter\char34}{}\<[E]%
\\
\>[B]{}\Varid{kCounter}\mathrel{=}{}\<[14]%
\>[14]{}\Conid{Proxy}~~,{}\<[E]%
\ColumnHook
\end{hscode}\resethooks
\end{minipage}
\begin{minipage}[b]{0.4\linewidth}
\begin{hscode}\SaveRestoreHook
\column{B}{@{}>{\hspre}l<{\hspost}@{}}%
\column{12}{@{}>{\hspre}l<{\hspost}@{}}%
\column{E}{@{}>{\hspre}l<{\hspost}@{}}%
\>[B]{}\Varid{kQueue}\mathbin{::}{}\<[12]%
\>[12]{}\Conid{Proxy}\;\text{\tt \char34 queue\char34}{}\<[E]%
\\
\>[B]{}\Varid{kQueue}\mathrel{=}{}\<[12]%
\>[12]{}\Conid{Proxy}~~.{}\<[E]%
\ColumnHook
\end{hscode}\resethooks
\end{minipage}}\\
To pop a message we use the function \ensuremath{\Varid{rpop}} which, given a key associated with
a list, extracts the rightmost element of the list
\begin{hscode}\SaveRestoreHook
\column{B}{@{}>{\hspre}l<{\hspost}@{}}%
\column{6}{@{}>{\hspre}l<{\hspost}@{}}%
\column{7}{@{}>{\hspre}l<{\hspost}@{}}%
\column{E}{@{}>{\hspre}l<{\hspost}@{}}%
\>[B]{}\Varid{pop}{}\<[6]%
\>[6]{}\mathbin{::}(\Conid{Get}\;\Varid{xs}\;\text{\tt \char34 queue\char34}\mathrel{\sim}\Conid{ListOf}\;\Conid{Message}){}\<[E]%
\\
\>[6]{}\Rightarrow \Conid{Edis}\;\Varid{xs}\;\Varid{xs}\;(\Conid{R} \uplus \Conid{Maybe}\;\Conid{Message}){}\<[E]%
\\
\>[B]{}\Varid{pop}\mathrel{=}\Varid{rpop}\;\Varid{kQueue}~~,{}\<[E]%
\\[\blanklineskip]%
\>[B]{}\Varid{rpop}{}\<[7]%
\>[7]{}\mathbin{::}(\Conid{KnownSymbol}\;\Varid{k},\Conid{Serialize}\;\Varid{a},\Conid{Get}\;\Varid{xs}\;\Varid{k}\mathrel{\sim}\Conid{ListOf}\;\Varid{a}){}\<[E]%
\\
\>[7]{}\Rightarrow \Conid{Proxy}\;\Varid{k}\to \Conid{Edis}\;\Varid{xs}\;\Varid{xs}\;(\Conid{R} \uplus \Conid{Maybe}\;\Varid{a}){}\<[E]%
\\
\>[B]{}\Varid{rpop}\;\Varid{key}\mathrel{=}\Conid{Edis}\;(\Varid{\Conid{Hedis}.rpop}\;(\Varid{encodeKey}\;\Varid{key})\bind \Varid{decodeAsMaybe})~~.{}\<[E]%
\ColumnHook
\end{hscode}\resethooks
To get things going, the main program builds a connection with the \Redis{}
server, runs an embedded program \ensuremath{\Varid{prog}}, and prints the result, which in this
example is \ensuremath{\text{\tt \char34 hello\char34}}:
\begin{hscode}\SaveRestoreHook
\column{B}{@{}>{\hspre}l<{\hspost}@{}}%
\column{4}{@{}>{\hspre}c<{\hspost}@{}}%
\column{4E}{@{}l@{}}%
\column{9}{@{}>{\hspre}l<{\hspost}@{}}%
\column{12}{@{}>{\hspre}l<{\hspost}@{}}%
\column{20}{@{}>{\hspre}l<{\hspost}@{}}%
\column{27}{@{}>{\hspre}l<{\hspost}@{}}%
\column{E}{@{}>{\hspre}l<{\hspost}@{}}%
\>[B]{}\Varid{main}\mathbin{::}\Conid{IO}\;(){}\<[E]%
\\
\>[B]{}\Varid{main}\mathrel{=}\mathbf{do}\;{}\<[12]%
\>[12]{}\Varid{conn}{}\<[20]%
\>[20]{}\leftarrow \Varid{connect}\;\Varid{defaultConnectInfo}{}\<[E]%
\\
\>[12]{}\Varid{result}{}\<[20]%
\>[20]{}\leftarrow \Varid{runRedis}\;\Varid{conn}\;(\Varid{unEdis}\;(\Varid{start}\mathbin{>\!\!\!>\!\!\!>}\Varid{prog})){}\<[E]%
\\
\>[12]{}\Varid{print}\;\Varid{result}~~,{}\<[E]%
\\[\blanklineskip]%
\>[B]{}\Varid{prog}\mathrel{=}{}\<[9]%
\>[9]{}\Varid{declare}\;\Varid{kCounter}\;{}\<[27]%
\>[27]{}(\Conid{Proxy}\mathbin{::}\Conid{Proxy}\;\Conid{Integer}){}\<[E]%
\\
\>[B]{}\hsindent{4}{}\<[4]%
\>[4]{}\mathbin{>\!\!\!>\!\!\!>}{}\<[4E]%
\>[9]{}\Varid{declare}\;\Varid{kQueue}\;{}\<[27]%
\>[27]{}(\Conid{Proxy}\mathbin{::}\Conid{Proxy}\;(\Conid{ListOf}\;\Conid{Message})){}\<[E]%
\\
\>[B]{}\hsindent{4}{}\<[4]%
\>[4]{}\mathbin{>\!\!\!>\!\!\!>}{}\<[4E]%
\>[9]{}\Varid{push}\;\text{\tt \char34 hello\char34}{}\<[E]%
\\
\>[B]{}\hsindent{4}{}\<[4]%
\>[4]{}\mathbin{>\!\!\!>\!\!\!>}{}\<[4E]%
\>[9]{}\Varid{push}\;\text{\tt \char34 world\char34}{}\<[E]%
\\
\>[B]{}\hsindent{4}{}\<[4]%
\>[4]{}\mathbin{>\!\!\!>\!\!\!>}{}\<[4E]%
\>[9]{}\Varid{pop}~~,{}\<[E]%
\ColumnHook
\end{hscode}\resethooks
where the monadic sequencing operator \ensuremath{(\mathbin{>\!\!\!>\!\!\!>})} is defined by:
\begin{hscode}\SaveRestoreHook
\column{B}{@{}>{\hspre}l<{\hspost}@{}}%
\column{E}{@{}>{\hspre}l<{\hspost}@{}}%
\>[B]{}(\mathbin{>\!\!\!>\!\!\!>})\mathbin{::}\Conid{IMonad}\;\Varid{m}\Rightarrow \Varid{m}\;\Varid{p}\;\Varid{q}\;\Varid{a}\to \Varid{m}\;\Varid{q}\;\Varid{r}\;\Varid{b}\to \Varid{m}\;\Varid{p}\;\Varid{r}\;\Varid{b}{}\<[E]%
\\
\>[B]{}\Varid{m}_{1}\mathbin{>\!\!\!>\!\!\!>}\Varid{m}_{2}\mathrel{=}\Varid{m}_{1}\mathbin{`\Varid{bind}`}(\lambda \anonymous \to \Varid{m}_{2})~~.{}\<[E]%
\ColumnHook
\end{hscode}\resethooks
Use of \ensuremath{\Varid{declare}} in \ensuremath{\Varid{prog}} ensures that neither \ensuremath{\text{\tt \char34 counter\char34}} nor \ensuremath{\text{\tt \char34 queue\char34}} exist
before the execution of \ensuremath{\Varid{prog}}, while \ensuremath{\Varid{start}} in \ensuremath{\Varid{main}} guarantees that the
program is given a fresh run without previously defined keys at all. Haskell
is able to infer the type of \ensuremath{\Varid{prog}}:
\begin{hscode}\SaveRestoreHook
\column{B}{@{}>{\hspre}l<{\hspost}@{}}%
\column{15}{@{}>{\hspre}l<{\hspost}@{}}%
\column{E}{@{}>{\hspre}l<{\hspost}@{}}%
\>[B]{}\Varid{prog}\mathbin{::}\Conid{Edis}\;{}\<[15]%
\>[15]{}\mbox{\textquotesingle}[~]\;\mbox{\textquotesingle}[\mbox{\textquotesingle}(\text{\tt \char34 counter\char34},\Conid{Integer}),\mbox{\textquotesingle}(\text{\tt \char34 queue\char34},\Conid{ListOf}\;\Conid{Message})]\;{}\<[E]%
\\
\>[15]{}(\Conid{R} \uplus \Conid{Maybe}\;\Conid{Message})~~.{}\<[E]%
\ColumnHook
\end{hscode}\resethooks


% The following program increases the value of |"A"| as an integer, push the result of the increment to list |"L"|, and then pops it out:
% \begin{spec}
% main :: IO ()
% main = do
%     conn    <- connect defaultConnectInfo
%     result  <- runRedis conn $ unEdis $ start
%         `bind` \ _ ->  declare (Proxy :: Proxy "A") (Proxy :: Proxy Integer)
%         `bind` \ _ ->  incr (Proxy :: Proxy "A")
%         `bind` \n ->  case n of
%             Left  err  -> lpush (Proxy :: Proxy "L") 0
%             Right n    -> lpush (Proxy :: Proxy "L") n
%         `bind` \ _ ->  lpop     (Proxy :: Proxy "L")
%     print result
% \end{spec}

% The syntax is pretty heavy, like the old days when there's no
%  \emph{do-notation}\cite{history}. But if we don't need any variable bindings
%  between operations, we could compose these commands with a sequencing operator
%  |(>>>)|.


% \begin{spec}
% program = start
%     >>> declare (Proxy :: Proxy "A") (Proxy :: Proxy Integer)
%     >>> incr    (Proxy :: Proxy "A")
%     >>> lpush   (Proxy :: Proxy "L") 0
%     >>> lpop    (Proxy :: Proxy "L")
% \end{spec}
