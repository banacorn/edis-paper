%% ODER: format ==         = "\mathrel{==}"
%% ODER: format /=         = "\neq "
%
%
\makeatletter
\@ifundefined{lhs2tex.lhs2tex.sty.read}%
  {\@namedef{lhs2tex.lhs2tex.sty.read}{}%
   \newcommand\SkipToFmtEnd{}%
   \newcommand\EndFmtInput{}%
   \long\def\SkipToFmtEnd#1\EndFmtInput{}%
  }\SkipToFmtEnd

\newcommand\ReadOnlyOnce[1]{\@ifundefined{#1}{\@namedef{#1}{}}\SkipToFmtEnd}
\usepackage{amstext}
\usepackage{amssymb}
\usepackage{stmaryrd}
\DeclareFontFamily{OT1}{cmtex}{}
\DeclareFontShape{OT1}{cmtex}{m}{n}
  {<5><6><7><8>cmtex8
   <9>cmtex9
   <10><10.95><12><14.4><17.28><20.74><24.88>cmtex10}{}
\DeclareFontShape{OT1}{cmtex}{m}{it}
  {<-> ssub * cmtt/m/it}{}
\newcommand{\texfamily}{\fontfamily{cmtex}\selectfont}
\DeclareFontShape{OT1}{cmtt}{bx}{n}
  {<5><6><7><8>cmtt8
   <9>cmbtt9
   <10><10.95><12><14.4><17.28><20.74><24.88>cmbtt10}{}
\DeclareFontShape{OT1}{cmtex}{bx}{n}
  {<-> ssub * cmtt/bx/n}{}
\newcommand{\tex}[1]{\text{\texfamily#1}}	% NEU

\newcommand{\Sp}{\hskip.33334em\relax}


\newcommand{\Conid}[1]{\mathit{#1}}
\newcommand{\Varid}[1]{\mathit{#1}}
\newcommand{\anonymous}{\kern0.06em \vbox{\hrule\@width.5em}}
\newcommand{\plus}{\mathbin{+\!\!\!+}}
\newcommand{\bind}{\mathbin{>\!\!\!>\mkern-6.7mu=}}
\newcommand{\rbind}{\mathbin{=\mkern-6.7mu<\!\!\!<}}% suggested by Neil Mitchell
\newcommand{\sequ}{\mathbin{>\!\!\!>}}
\renewcommand{\leq}{\leqslant}
\renewcommand{\geq}{\geqslant}
\usepackage{polytable}

%mathindent has to be defined
\@ifundefined{mathindent}%
  {\newdimen\mathindent\mathindent\leftmargini}%
  {}%

\def\resethooks{%
  \global\let\SaveRestoreHook\empty
  \global\let\ColumnHook\empty}
\newcommand*{\savecolumns}[1][default]%
  {\g@addto@macro\SaveRestoreHook{\savecolumns[#1]}}
\newcommand*{\restorecolumns}[1][default]%
  {\g@addto@macro\SaveRestoreHook{\restorecolumns[#1]}}
\newcommand*{\aligncolumn}[2]%
  {\g@addto@macro\ColumnHook{\column{#1}{#2}}}

\resethooks

\newcommand{\onelinecommentchars}{\quad-{}- }
\newcommand{\commentbeginchars}{\enskip\{-}
\newcommand{\commentendchars}{-\}\enskip}

\newcommand{\visiblecomments}{%
  \let\onelinecomment=\onelinecommentchars
  \let\commentbegin=\commentbeginchars
  \let\commentend=\commentendchars}

\newcommand{\invisiblecomments}{%
  \let\onelinecomment=\empty
  \let\commentbegin=\empty
  \let\commentend=\empty}

\visiblecomments

\newlength{\blanklineskip}
\setlength{\blanklineskip}{0.66084ex}

\newcommand{\hsindent}[1]{\quad}% default is fixed indentation
\let\hspre\empty
\let\hspost\empty
\newcommand{\NB}{\textbf{NB}}
\newcommand{\Todo}[1]{$\langle$\textbf{To do:}~#1$\rangle$}

\EndFmtInput
\makeatother
%
%
%
%
%
%
% This package provides two environments suitable to take the place
% of hscode, called "plainhscode" and "arrayhscode". 
%
% The plain environment surrounds each code block by vertical space,
% and it uses \abovedisplayskip and \belowdisplayskip to get spacing
% similar to formulas. Note that if these dimensions are changed,
% the spacing around displayed math formulas changes as well.
% All code is indented using \leftskip.
%
% Changed 19.08.2004 to reflect changes in colorcode. Should work with
% CodeGroup.sty.
%
\ReadOnlyOnce{polycode.fmt}%
\makeatletter

\newcommand{\hsnewpar}[1]%
  {{\parskip=0pt\parindent=0pt\par\vskip #1\noindent}}

% can be used, for instance, to redefine the code size, by setting the
% command to \small or something alike
\newcommand{\hscodestyle}{}

% The command \sethscode can be used to switch the code formatting
% behaviour by mapping the hscode environment in the subst directive
% to a new LaTeX environment.

\newcommand{\sethscode}[1]%
  {\expandafter\let\expandafter\hscode\csname #1\endcsname
   \expandafter\let\expandafter\endhscode\csname end#1\endcsname}

% "compatibility" mode restores the non-polycode.fmt layout.

\newenvironment{compathscode}%
  {\par\noindent
   \advance\leftskip\mathindent
   \hscodestyle
   \let\\=\@normalcr
   \let\hspre\(\let\hspost\)%
   \pboxed}%
  {\endpboxed\)%
   \par\noindent
   \ignorespacesafterend}

\newcommand{\compaths}{\sethscode{compathscode}}

% "plain" mode is the proposed default.
% It should now work with \centering.
% This required some changes. The old version
% is still available for reference as oldplainhscode.

\newenvironment{plainhscode}%
  {\hsnewpar\abovedisplayskip
   \advance\leftskip\mathindent
   \hscodestyle
   \let\hspre\(\let\hspost\)%
   \pboxed}%
  {\endpboxed%
   \hsnewpar\belowdisplayskip
   \ignorespacesafterend}

\newenvironment{oldplainhscode}%
  {\hsnewpar\abovedisplayskip
   \advance\leftskip\mathindent
   \hscodestyle
   \let\\=\@normalcr
   \(\pboxed}%
  {\endpboxed\)%
   \hsnewpar\belowdisplayskip
   \ignorespacesafterend}

% Here, we make plainhscode the default environment.

\newcommand{\plainhs}{\sethscode{plainhscode}}
\newcommand{\oldplainhs}{\sethscode{oldplainhscode}}
\plainhs

% The arrayhscode is like plain, but makes use of polytable's
% parray environment which disallows page breaks in code blocks.

\newenvironment{arrayhscode}%
  {\hsnewpar\abovedisplayskip
   \advance\leftskip\mathindent
   \hscodestyle
   \let\\=\@normalcr
   \(\parray}%
  {\endparray\)%
   \hsnewpar\belowdisplayskip
   \ignorespacesafterend}

\newcommand{\arrayhs}{\sethscode{arrayhscode}}

% The mathhscode environment also makes use of polytable's parray 
% environment. It is supposed to be used only inside math mode 
% (I used it to typeset the type rules in my thesis).

\newenvironment{mathhscode}%
  {\parray}{\endparray}

\newcommand{\mathhs}{\sethscode{mathhscode}}

% texths is similar to mathhs, but works in text mode.

\newenvironment{texthscode}%
  {\(\parray}{\endparray\)}

\newcommand{\texths}{\sethscode{texthscode}}

% The framed environment places code in a framed box.

\def\codeframewidth{\arrayrulewidth}
\RequirePackage{calc}

\newenvironment{framedhscode}%
  {\parskip=\abovedisplayskip\par\noindent
   \hscodestyle
   \arrayrulewidth=\codeframewidth
   \tabular{@{}|p{\linewidth-2\arraycolsep-2\arrayrulewidth-2pt}|@{}}%
   \hline\framedhslinecorrect\\{-1.5ex}%
   \let\endoflinesave=\\
   \let\\=\@normalcr
   \(\pboxed}%
  {\endpboxed\)%
   \framedhslinecorrect\endoflinesave{.5ex}\hline
   \endtabular
   \parskip=\belowdisplayskip\par\noindent
   \ignorespacesafterend}

\newcommand{\framedhslinecorrect}[2]%
  {#1[#2]}

\newcommand{\framedhs}{\sethscode{framedhscode}}

% The inlinehscode environment is an experimental environment
% that can be used to typeset displayed code inline.

\newenvironment{inlinehscode}%
  {\(\def\column##1##2{}%
   \let\>\undefined\let\<\undefined\let\\\undefined
   \newcommand\>[1][]{}\newcommand\<[1][]{}\newcommand\\[1][]{}%
   \def\fromto##1##2##3{##3}%
   \def\nextline{}}{\) }%

\newcommand{\inlinehs}{\sethscode{inlinehscode}}

% The joincode environment is a separate environment that
% can be used to surround and thereby connect multiple code
% blocks.

\newenvironment{joincode}%
  {\let\orighscode=\hscode
   \let\origendhscode=\endhscode
   \def\endhscode{\def\hscode{\endgroup\def\@currenvir{hscode}\\}\begingroup}
   %\let\SaveRestoreHook=\empty
   %\let\ColumnHook=\empty
   %\let\resethooks=\empty
   \orighscode\def\hscode{\endgroup\def\@currenvir{hscode}}}%
  {\origendhscode
   \global\let\hscode=\orighscode
   \global\let\endhscode=\origendhscode}%

\makeatother
\EndFmtInput
%

\ReadOnlyOnce{Formatting.fmt}%
\makeatletter

\let\Varid\mathit
\let\Conid\mathsf

\def\commentbegin{\quad\{\ }
\def\commentend{\}}

\newcommand{\ty}[1]{\Conid{#1}}
\newcommand{\con}[1]{\Conid{#1}}
\newcommand{\id}[1]{\Varid{#1}}
\newcommand{\cl}[1]{\Varid{#1}}
\newcommand{\opsym}[1]{\mathrel{#1}}

\newcommand\Keyword[1]{\textbf{\textsf{#1}}}
\newcommand\Hide{\mathbin{\downarrow}}
\newcommand\Reveal{\mathbin{\uparrow}}


%% Paper-specific keywords


\makeatother
\EndFmtInput

\section{Type-Level Dictionaries}
\label{sec:type-level-dict}

One of the challenges of statically ensuring type correctness of stateful
languages is that the type of the value of a key can be altered by updating.
In \Redis{}, one may delete an existing key and create it again by assigning to
it a value of a different type. To ensure type correctness, we keep track of the
types of all existing keys in a {\em dictionary} --- an associate list, or a
list of pairs of keys and some encoding of types. For example, we may use the
dictionary \ensuremath{[\mskip1.5mu (\text{\tt \char34 A\char34},\Conid{Int}),(\text{\tt \char34 B\char34},\Conid{Char}),(\text{\tt \char34 C\char34},\Conid{Bool})\mskip1.5mu]} to represent a
predicate, or a constraint, stating that ``the keys in the data store are \ensuremath{\text{\tt \char34 A\char34}},
\ensuremath{\text{\tt \char34 B\char34}}, and \ensuremath{\text{\tt \char34 C\char34}}, respectively assigned values of type \ensuremath{\Conid{Int}}, \ensuremath{\Conid{Char}}, and
\ensuremath{\Conid{Bool}}.'' (This representation will be refined in the next section.)

The dictionary above mixes values (strings such as \ensuremath{\text{\tt \char34 A\char34}}, \ensuremath{\text{\tt \char34 B\char34}}) and types.
Furthermore, as mentioned in Section~\ref{sec:indexed-monads}, the
dictionaries will be parameters to the indexed monad \ensuremath{\Conid{Edis}}. In a dependently
typed programming language (without the so-called ``phase distinction'' ---
separation between types and terms), this would pose no problem. In Haskell
however, the dictionaries, to index a monad, has to be a type as well.

In this section we describe how to construct a type-level dictionary, to be
used with the indexed monad in Section~\ref{sec:indexed-monads}.

\subsection{Datatype Promotion}

Haskell maintains the distinction between values, types, and kinds: values are
categorized by types, and types are categorized by kinds. The kinds are
relatively simple: \ensuremath{\mathbin{*}} is the kind of all {\em lifted} types, while type
constructors have kinds such as \ensuremath{\mathbin{*}\to \mathbin{*}}, or \ensuremath{\mathbin{*}\to \mathbin{*}\to \mathbin{*}}, etc.\footnotemark\,
Consider the datatype definitions below:
\begin{hscode}\SaveRestoreHook
\column{B}{@{}>{\hspre}l<{\hspost}@{}}%
\column{E}{@{}>{\hspre}l<{\hspost}@{}}%
\>[B]{}\mathbf{data}\;\Conid{Nat}\mathrel{=}\Conid{Zero}\mid \Conid{Suc}\;\Conid{Nat}~~,\qquad\;\mathbf{data}\;[\mskip1.5mu \Varid{a}\mskip1.5mu]\mathrel{=}[\mskip1.5mu \mskip1.5mu]\mid \Varid{a}\mathbin{:}[\mskip1.5mu \Varid{a}\mskip1.5mu]~~.{}\<[E]%
\ColumnHook
\end{hscode}\resethooks
The left-hand side is usually seen as having defined a type \ensuremath{\Conid{Nat}\mathbin{::}\mathbin{*}},
and two value constructors \ensuremath{\Conid{Zero}\mathbin{::}\Conid{Nat}} and \ensuremath{\Conid{Suc}\mathbin{::}\Conid{Nat}\to \Conid{Nat}}. The right-hand
side is how Haskell lists are understood. The {\em kind} of \ensuremath{[\mskip1.5mu \mathbin{\cdot}\mskip1.5mu]} is \ensuremath{\mathbin{*}\to \mathbin{*}},
since it takes a lifted type \ensuremath{\Varid{a}} to a lifted type \ensuremath{[\mskip1.5mu \Varid{a}\mskip1.5mu]}. The two value
constructors respectively have types \ensuremath{[\mskip1.5mu \mskip1.5mu]\mathbin{::}[\mskip1.5mu \Varid{a}\mskip1.5mu]} and \ensuremath{(\mathbin{:})\mathbin{::}\Varid{a}\to [\mskip1.5mu \Varid{a}\mskip1.5mu]\to [\mskip1.5mu \Varid{a}\mskip1.5mu]}, for all type \ensuremath{\Varid{a}}.

\footnotetext{In Haskell, the opposite of \emph{lifted} types are \emph{unboxed}
types, which are not represented by a pointer to a heap object, and cannot be
stored in a polymorphic data type.}

The GHC extension \emph{data kinds}~\cite{promotion}, however, automatically
promotes certain ``suitable'' types to kinds.\footnote{It is only informally
described in the GHC manual what types are ``suitable''.} With the extension,
the \ensuremath{\mathbf{data}} definitions above has an alternative reading: \ensuremath{\Conid{Nat}} is a new kind,
\ensuremath{\Conid{Zero}\mathbin{::}\Conid{Nat}} is a type having kind \ensuremath{\Conid{Nat}}, and \ensuremath{\Conid{Suc}\mathbin{::}\Conid{Nat}\to \Conid{Nat}} is a type
constructor, taking a type in kind \ensuremath{\Conid{Nat}} to another type in \ensuremath{\Conid{Nat}}. When one
sees a constructor in an expression, whether it is promoted can often be
inferred from the context. When one needs to be more specific, prefixing a
constructor with a single quote, such as in \ensuremath{\mbox{\textquotesingle}\Conid{Zero}} and \ensuremath{\mbox{\textquotesingle}\Conid{Suc}}, denotes that it
is promoted.

The situation of lists is similar: for all kind \ensuremath{\Varid{k}}, \ensuremath{[\mskip1.5mu \Varid{k}\mskip1.5mu]} is also a kind. For
all kind \ensuremath{\Varid{k}}, \ensuremath{[\mskip1.5mu \mskip1.5mu]} is a type of kind \ensuremath{[\mskip1.5mu \Varid{k}\mskip1.5mu]}. Given a type \ensuremath{\Varid{a}} of kind \ensuremath{\Varid{k}} and a
type \ensuremath{\Varid{as}} of kind \ensuremath{[\mskip1.5mu \Varid{k}\mskip1.5mu]}, \ensuremath{\Varid{a}\mathbin{:}\Varid{as}} is again a type of kind \ensuremath{[\mskip1.5mu \Varid{k}\mskip1.5mu]}. Formally,
\ensuremath{(\mathbin{:})\mathbin{::}\Varid{k}\to [\mskip1.5mu \Varid{k}\mskip1.5mu]\to [\mskip1.5mu \Varid{k}\mskip1.5mu]}. For example, \ensuremath{\Conid{Int}\mathrel{\,\mbox{\textquotesingle}\!:}(\Conid{Char}\mathrel{\,\mbox{\textquotesingle}\!:}(\Conid{Bool}\mathrel{\,\mbox{\textquotesingle}\!:}\mbox{\textquotesingle}[~]))} is a
type having kind \ensuremath{[\mskip1.5mu \mathbin{*}\mskip1.5mu]} --- it is a list of (lifted) types. The optional quote
denotes that the constructors are promoted. The same list can be denoted by a
syntax sugar \ensuremath{\mbox{\textquotesingle}[\Conid{Int},\Conid{Char},\Conid{Bool}]}.

Tuples are also promoted. Thus we may put two types in a pair to form another
type, such as in \ensuremath{\mbox{\textquotesingle}(\Conid{Int},\Conid{Char})}, a type having kind \ensuremath{(\mathbin{*},\mathbin{*})}.

Strings in Haskell are nothing but lists of \ensuremath{\Conid{Char}}s. Regarding promotion,
however, a string can be promoted to a type having kind \ensuremath{\Conid{Symbol}}. In the expression:
\begin{hscode}\SaveRestoreHook
\column{B}{@{}>{\hspre}l<{\hspost}@{}}%
\column{E}{@{}>{\hspre}l<{\hspost}@{}}%
\>[B]{}\text{\tt \char34 this~is~a~type-level~string~literal\char34}\mathbin{::}\Conid{Symbol}~~,{}\<[E]%
\ColumnHook
\end{hscode}\resethooks
the string on the left-hand side of \ensuremath{(\mathbin{::})} is a type whose kind is \ensuremath{\Conid{Symbol}}.

With all of these ingredients, we are ready to build our dictionaries, or
type-level associate lists:
\begin{hscode}\SaveRestoreHook
\column{B}{@{}>{\hspre}l<{\hspost}@{}}%
\column{E}{@{}>{\hspre}l<{\hspost}@{}}%
\>[B]{}\mathbf{type}\;\Conid{DictEmpty}\mathrel{=}\mbox{\textquotesingle}[~]~~,{}\<[E]%
\\
\>[B]{}\mathbf{type}\;\Conid{Dict0}\mathrel{=}\mbox{\textquotesingle}[\mbox{\textquotesingle}(\text{\tt \char34 key\char34},\Conid{Bool})]~~,{}\<[E]%
\\
\>[B]{}\mathbf{type}\;\Conid{Dict1}\mathrel{=}\mbox{\textquotesingle}[\mbox{\textquotesingle}(\text{\tt \char34 A\char34},\Conid{Int}),\mbox{\textquotesingle}(\text{\tt \char34 B\char34},\text{\tt \char34 A\char34})]~~.{}\<[E]%
\ColumnHook
\end{hscode}\resethooks
All the entities defined above are types, where \ensuremath{\Conid{Dict0}} has kind \ensuremath{[\mskip1.5mu (\Conid{Symbol},\mathbin{*})\mskip1.5mu]}. In \ensuremath{\Conid{Dict1}}, while \ensuremath{\Conid{Int}} has kind \ensuremath{\mathbin{*}} and \ensuremath{\text{\tt \char34 A\char34}} has kind \ensuremath{\Conid{Symbol}}, the former kind subsumes the later. Thus \ensuremath{\Conid{Dict1}} also has kind \ensuremath{[\mskip1.5mu (\Conid{Symbol},\mathbin{*})\mskip1.5mu]}.

\subsection{Type-Level Functions}
\label{sec:type-fun}

Now that we can represent dictionaries as types, the next step is to define
operations on them. A function that inserts an entry to a dictionary, for
example, is a function from a type to a type. While it was shown that it is
possible to simulate type-level functions using Haskell type
classes~\cite{McBride:02:Faking}, in recent versions of GHC, {\em indexed type
families}, or \emph{type families} for short, are considered a cleaner solution.

For example, compare disjunction \ensuremath{(\mathrel{\vee})} and its type-level
counterpart:\\
\noindent{\centering %\small
\begin{minipage}[b]{0.35\linewidth}
\begin{hscode}\SaveRestoreHook
\column{B}{@{}>{\hspre}l<{\hspost}@{}}%
\column{7}{@{}>{\hspre}c<{\hspost}@{}}%
\column{7E}{@{}l@{}}%
\column{11}{@{}>{\hspre}l<{\hspost}@{}}%
\column{14}{@{}>{\hspre}l<{\hspost}@{}}%
\column{E}{@{}>{\hspre}l<{\hspost}@{}}%
\>[B]{}(\mathrel{\vee})\mathbin{::}\Conid{Bool}\to \Conid{Bool}\to \Conid{Bool}{}\<[E]%
\\
\>[B]{}\Conid{True}{}\<[7]%
\>[7]{}\mathrel{\vee}{}\<[7E]%
\>[11]{}\Varid{b}{}\<[14]%
\>[14]{}\mathrel{=}\Conid{True}{}\<[E]%
\\
\>[B]{}\Varid{a}{}\<[7]%
\>[7]{}\mathrel{\vee}{}\<[7E]%
\>[11]{}\Varid{b}{}\<[14]%
\>[14]{}\mathrel{=}\Varid{b}~~,{}\<[E]%
\ColumnHook
\end{hscode}\resethooks
\end{minipage}
\begin{minipage}[b]{0.55\linewidth}
\begin{hscode}\SaveRestoreHook
\column{B}{@{}>{\hspre}l<{\hspost}@{}}%
\column{3}{@{}>{\hspre}l<{\hspost}@{}}%
\column{10}{@{}>{\hspre}l<{\hspost}@{}}%
\column{16}{@{}>{\hspre}l<{\hspost}@{}}%
\column{24}{@{}>{\hspre}l<{\hspost}@{}}%
\column{E}{@{}>{\hspre}l<{\hspost}@{}}%
\>[B]{}\mathbf{type}\;\Varid{family}\;\Conid{Or}\;(\Varid{a}\mathbin{::}\Conid{Bool})\;(\Varid{b}\mathbin{::}\Conid{Bool})\mathbin{::}\Conid{Bool}{}\<[E]%
\\
\>[B]{}\hsindent{3}{}\<[3]%
\>[3]{}\mathbf{where}\;{}\<[10]%
\>[10]{}\mbox{\textquotesingle}\Conid{True}{}\<[16]%
\>[16]{}\mathrel{\Conid{`Or`}}\Varid{b}{}\<[24]%
\>[24]{}\mathrel{=}\mbox{\textquotesingle}\Conid{True}{}\<[E]%
\\
\>[10]{}\Varid{a}{}\<[16]%
\>[16]{}\mathrel{\Conid{`Or`}}\Varid{b}{}\<[24]%
\>[24]{}\mathrel{=}\Varid{b}~~.{}\<[E]%
\ColumnHook
\end{hscode}\resethooks
\end{minipage}
}\\
The left-hand side is a typical definition of \ensuremath{(\mathrel{\vee})} by pattern matching.
On the right-hand side, \ensuremath{\Conid{Bool}} is not a type, but a type lifted to a kind,
while \ensuremath{\Conid{True}} and \ensuremath{\Conid{False}} are types of kind \ensuremath{\Conid{Bool}}. The declaration says that
\ensuremath{\Conid{Or}} is a family of types, indexed by two parameters \ensuremath{\Varid{a}} and \ensuremath{\Varid{b}} of kind \ensuremath{\Conid{Bool}}.
The type with index \ensuremath{\mbox{\textquotesingle}\Conid{True}} and \ensuremath{\Varid{b}} is \ensuremath{\mbox{\textquotesingle}\Conid{True}}, and all other indices lead to \ensuremath{\Varid{b}}.
For our purpose, we can read \ensuremath{\Conid{Or}} as a function from types to types ---
observe how it resembles the term-level \ensuremath{(\mathrel{\vee})}. We present two more type-level
functions about \ensuremath{\Conid{Bool}} --- negation, and conditional, that we will use later:\\
\noindent{\centering %\small
\begin{minipage}[b]{0.35\linewidth}
\begin{hscode}\SaveRestoreHook
\column{B}{@{}>{\hspre}l<{\hspost}@{}}%
\column{3}{@{}>{\hspre}l<{\hspost}@{}}%
\column{14}{@{}>{\hspre}l<{\hspost}@{}}%
\column{E}{@{}>{\hspre}l<{\hspost}@{}}%
\>[B]{}\mathbf{type}\;\Varid{family}\;\Conid{Not}\;\Varid{a}\;\mathbf{where}{}\<[E]%
\\
\>[B]{}\hsindent{3}{}\<[3]%
\>[3]{}\Conid{Not}\;\mbox{\textquotesingle}\Conid{False}{}\<[14]%
\>[14]{}\mathrel{=}\mbox{\textquotesingle}\Conid{True}{}\<[E]%
\\
\>[B]{}\hsindent{3}{}\<[3]%
\>[3]{}\Conid{Not}\;\mbox{\textquotesingle}\Conid{True}{}\<[14]%
\>[14]{}\mathrel{=}\mbox{\textquotesingle}\Conid{False}~~,{}\<[E]%
\ColumnHook
\end{hscode}\resethooks
\end{minipage}
\begin{minipage}[b]{0.55\linewidth}
\begin{hscode}\SaveRestoreHook
\column{B}{@{}>{\hspre}l<{\hspost}@{}}%
\column{3}{@{}>{\hspre}l<{\hspost}@{}}%
\column{12}{@{}>{\hspre}l<{\hspost}@{}}%
\column{17}{@{}>{\hspre}l<{\hspost}@{}}%
\column{E}{@{}>{\hspre}l<{\hspost}@{}}%
\>[B]{}\mathbf{type}\;\Varid{family}\;\Conid{If}\;(\Varid{c}\mathbin{::}\Conid{Bool})\;(\Varid{t}\mathbin{::}\Varid{a})\;(\Varid{f}\mathbin{::}\Varid{a})\mathbin{::}\Varid{a}\;\mathbf{where}{}\<[E]%
\\
\>[B]{}\hsindent{3}{}\<[3]%
\>[3]{}\Conid{If}\;\mbox{\textquotesingle}\Conid{True}\;{}\<[12]%
\>[12]{}\Varid{tru}\;{}\<[17]%
\>[17]{}\Varid{fls}\mathrel{=}\Varid{tru}{}\<[E]%
\\
\>[B]{}\hsindent{3}{}\<[3]%
\>[3]{}\Conid{If}\;\mbox{\textquotesingle}\Conid{False}\;\Varid{tru}\;{}\<[17]%
\>[17]{}\Varid{fls}\mathrel{=}\Varid{fls}~~.{}\<[E]%
\ColumnHook
\end{hscode}\resethooks
\end{minipage}
}

As a remark, type families in Haskell come in many flavors. One can define families of \ensuremath{\mathbf{data}}, as well as families of \ensuremath{\mathbf{type}} synonym. They can appear
inside type classes~\cite{tfclass,tfsynonym} or at toplevel. Top-level type families can be open~\cite{tfopen} or closed~\cite{tfclosed}. The flavor we
chose is top-level, closed type synonym family, since it allows overlapping
instances, and since we need none of the extensibility provided by open type
families. Notice that the instance \ensuremath{\mbox{\textquotesingle}\Conid{True}\mathrel{\Conid{`Or`}}\Varid{b}} could be subsumed under the
more general instance, \ensuremath{\Varid{a}\mathrel{\Conid{`Or`}}\Varid{b}}. In a closed type family we may resolve the
overlapping in order, just like how cases overlapping is resolved in term-level
functions.

Now we can define operations on type-level dictionaries. Let us begin
with:
\begin{hscode}\SaveRestoreHook
\column{B}{@{}>{\hspre}l<{\hspost}@{}}%
\column{5}{@{}>{\hspre}l<{\hspost}@{}}%
\column{20}{@{}>{\hspre}l<{\hspost}@{}}%
\column{33}{@{}>{\hspre}c<{\hspost}@{}}%
\column{33E}{@{}l@{}}%
\column{36}{@{}>{\hspre}l<{\hspost}@{}}%
\column{E}{@{}>{\hspre}l<{\hspost}@{}}%
\>[B]{}\mathbf{type}\;\Varid{family}\;\Conid{Get}\;(\Varid{xs}\mathbin{::}[\mskip1.5mu (\Conid{Symbol},\mathbin{*})\mskip1.5mu])\;(\Varid{k}\mathbin{::}\Conid{Symbol})\mathbin{::}\mathbin{*}\mathbf{where}{}\<[E]%
\\
\>[B]{}\hsindent{5}{}\<[5]%
\>[5]{}\Conid{Get}\;(\mbox{\textquotesingle}(\Varid{k},{}\<[20]%
\>[20]{}\Varid{x})\mathrel{\,\mbox{\textquotesingle}\!:}\Varid{xs})\;\Varid{k}{}\<[33]%
\>[33]{}\mathrel{=}{}\<[33E]%
\>[36]{}\Varid{x}{}\<[E]%
\\
\>[B]{}\hsindent{5}{}\<[5]%
\>[5]{}\Conid{Get}\;(\mbox{\textquotesingle}(\Varid{t},{}\<[20]%
\>[20]{}\Varid{x})\mathrel{\,\mbox{\textquotesingle}\!:}\Varid{xs})\;\Varid{k}{}\<[33]%
\>[33]{}\mathrel{=}{}\<[33E]%
\>[36]{}\Conid{Get}\;\Varid{xs}\;\Varid{k}~~.{}\<[E]%
\ColumnHook
\end{hscode}\resethooks
\ensuremath{\Conid{Get}\;\Varid{xs}\;\Varid{k}} returns the entry associated with key \ensuremath{\Varid{k}} in the
dictionary \ensuremath{\Varid{xs}}. Notice, in the first case, how type-level equality can be
expressed by unifying type variables with the same name. Note also that \ensuremath{\Conid{Get}} is
a partial function on types: while \ensuremath{\Conid{Get}\;\mbox{\textquotesingle}[\mbox{\textquotesingle}(\text{\tt \char34 A\char34},\Conid{Int})]\;\text{\tt \char34 A\char34}} evaluates
to \ensuremath{\Conid{Int}}, when \ensuremath{\Conid{Get}\;\mbox{\textquotesingle}[\mbox{\textquotesingle}(\text{\tt \char34 A\char34},\Conid{Int})]\;\text{\tt \char34 B\char34}} appears in a type expression,
there are no applicable rules to reduce it. The expression thus stays un-reduced.

% For our applications it is more convenient to make |Get| total, as we would at
% the term level, by having it return a |Maybe|:
% \begin{spec}
% type family Get (xs :: [(Symbol, *)]) (k :: Symbol) :: Maybe * where
%     Get NIL                 k = Nothing
%     Get (TPar(k, x) :- xs)  k = Just x
%     Get (TPar(t, x) :- xs)  k = Get xs k {-"~~."-}
% \end{spec}
%
Some other dictionary-related functions are defined
in Figure \ref{fig:dict-operations}. The function \ensuremath{\Conid{Set}} either updates an
existing entry or inserts a new entry, \ensuremath{\Conid{Del}} removes an entry matching
a given key, while \ensuremath{\Conid{Member}} checks whether a given key exists in the
dictionary.

\begin{figure}[t]
\begin{hscode}\SaveRestoreHook
\column{B}{@{}>{\hspre}l<{\hspost}@{}}%
\column{5}{@{}>{\hspre}l<{\hspost}@{}}%
\column{18}{@{}>{\hspre}l<{\hspost}@{}}%
\column{29}{@{}>{\hspre}l<{\hspost}@{}}%
\column{30}{@{}>{\hspre}l<{\hspost}@{}}%
\column{33}{@{}>{\hspre}l<{\hspost}@{}}%
\column{48}{@{}>{\hspre}l<{\hspost}@{}}%
\column{E}{@{}>{\hspre}l<{\hspost}@{}}%
\>[B]{}\mbox{\onelinecomment  inserts or updates an entry}{}\<[E]%
\\
\>[B]{}\mathbf{type}\;\Varid{family}\;\Conid{Set}\;{}\<[18]%
\>[18]{}(\Varid{xs}\mathbin{::}[\mskip1.5mu (\Conid{Symbol},\mathbin{*})\mskip1.5mu])\;(\Varid{k}\mathbin{::}\Conid{Symbol})\;(\Varid{x}\mathbin{::}\mathbin{*})\mathbin{::}[\mskip1.5mu (\Conid{Symbol},\mathbin{*})\mskip1.5mu]\;\mathbf{where}{}\<[E]%
\\
\>[B]{}\hsindent{5}{}\<[5]%
\>[5]{}\Conid{Set}\;\mbox{\textquotesingle}[~]\;{}\<[29]%
\>[29]{}\Varid{k}\;\Varid{x}\mathrel{=}\mbox{\textquotesingle}[\mbox{\textquotesingle}(\Varid{k},\Varid{x})]{}\<[E]%
\\
\>[B]{}\hsindent{5}{}\<[5]%
\>[5]{}\Conid{Set}\;(\mbox{\textquotesingle}(\Varid{k},\Varid{y})\mathrel{\,\mbox{\textquotesingle}\!:}\Varid{xs})\;{}\<[29]%
\>[29]{}\Varid{k}\;\Varid{x}\mathrel{=}\mbox{\textquotesingle}(\Varid{k},\Varid{x}){}\<[48]%
\>[48]{}\mathrel{\,\mbox{\textquotesingle}\!:}\Varid{xs}{}\<[E]%
\\
\>[B]{}\hsindent{5}{}\<[5]%
\>[5]{}\Conid{Set}\;(\mbox{\textquotesingle}(\Varid{t},\Varid{y})\mathrel{\,\mbox{\textquotesingle}\!:}\Varid{xs})\;{}\<[29]%
\>[29]{}\Varid{k}\;\Varid{x}\mathrel{=}\mbox{\textquotesingle}(\Varid{t},\Varid{y}){}\<[48]%
\>[48]{}\mathrel{\,\mbox{\textquotesingle}\!:}\Conid{Set}\;\Varid{xs}\;\Varid{k}\;\Varid{x}{}\<[E]%
\\[\blanklineskip]%
\>[B]{}\mbox{\onelinecomment  removes an entry}{}\<[E]%
\\
\>[B]{}\mathbf{type}\;\Varid{family}\;\Conid{Del}\;{}\<[18]%
\>[18]{}(\Varid{xs}\mathbin{::}[\mskip1.5mu (\Conid{Symbol},\mathbin{*})\mskip1.5mu])\;(\Varid{k}\mathbin{::}\Conid{Symbol})\mathbin{::}[\mskip1.5mu (\Conid{Symbol},\mathbin{*})\mskip1.5mu]\;\mathbf{where}{}\<[E]%
\\
\>[B]{}\hsindent{5}{}\<[5]%
\>[5]{}\Conid{Del}\;\Conid{Nil}\;{}\<[30]%
\>[30]{}\Varid{k}\mathrel{=}\Conid{Nil}{}\<[E]%
\\
\>[B]{}\hsindent{5}{}\<[5]%
\>[5]{}\Conid{Del}\;(\mbox{\textquotesingle}(\Varid{k},\Varid{y})\mathrel{\,\mbox{\textquotesingle}\!:}\Varid{xs})\;{}\<[30]%
\>[30]{}\Varid{k}\mathrel{=}\Varid{xs}{}\<[E]%
\\
\>[B]{}\hsindent{5}{}\<[5]%
\>[5]{}\Conid{Del}\;(\mbox{\textquotesingle}(\Varid{t},\Varid{y})\mathrel{\,\mbox{\textquotesingle}\!:}\Varid{xs})\;{}\<[30]%
\>[30]{}\Varid{k}\mathrel{=}\mbox{\textquotesingle}(\Varid{t},\Varid{y})\mathrel{\,\mbox{\textquotesingle}\!:}\Conid{Del}\;\Varid{xs}\;\Varid{k}{}\<[E]%
\\[\blanklineskip]%
\>[B]{}\mbox{\onelinecomment  membership}{}\<[E]%
\\
\>[B]{}\mathbf{type}\;\Varid{family}\;\Conid{Member}\;(\Varid{xs}\mathbin{::}[\mskip1.5mu (\Conid{Symbol},\mathbin{*})\mskip1.5mu])\;(\Varid{k}\mathbin{::}\Conid{Symbol})\mathbin{::}\Conid{Bool}\;\mathbf{where}{}\<[E]%
\\
\>[B]{}\hsindent{5}{}\<[5]%
\>[5]{}\Conid{Member}\;\Conid{Nil}\;{}\<[33]%
\>[33]{}\Varid{k}\mathrel{=}\Conid{False}{}\<[E]%
\\
\>[B]{}\hsindent{5}{}\<[5]%
\>[5]{}\Conid{Member}\;(\mbox{\textquotesingle}(\Varid{k},\Varid{x})\mathrel{\,\mbox{\textquotesingle}\!:}\Varid{xs})\;{}\<[33]%
\>[33]{}\Varid{k}\mathrel{=}\Conid{True}{}\<[E]%
\\
\>[B]{}\hsindent{5}{}\<[5]%
\>[5]{}\Conid{Member}\;(\mbox{\textquotesingle}(\Varid{t},\Varid{x})\mathrel{\,\mbox{\textquotesingle}\!:}\Varid{xs})\;{}\<[33]%
\>[33]{}\Varid{k}\mathrel{=}\Conid{Member}\;\Varid{xs}\;\Varid{k}{}\<[E]%
\ColumnHook
\end{hscode}\resethooks
\caption{Some operations on type-level dictionaries.}
\label{fig:dict-operations}
\end{figure}
