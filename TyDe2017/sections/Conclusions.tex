
\section{Conclusions and Related Work}
\label{sec:conclusions}

By exploiting various recent extensions and type-level programming techniques,
we have designed a domain-specific embedded language \Edis{} which enforces
typing disciplines that keep track of available keys and their types. The
type of a command clearly specifies which keys must or must not present in
the data store, what their types ought to be, as well as how the keys and types are updated after the execution. A program can be constructed only if it does
not throw a runtime type error when it is run with a store whose status matches
its precondition and when it is the sole client interacting with the store.
The type also serves as documentation of the
commands. We believe that it is a neat case study of application of type-level programming.

\Redis{} identifies itself as a data structure store/server, rather than a
database. Nevertheless, there has been attempts to design DSELs for relational
databases that guarantee all queries made are safe. Among them, {\sc HaskellDB}~\cite{haskelldb,haskelldbimproved} dynamically generates, from monad comprehensions, SQL queries to be executed on ODBC database servers.
With the expressiveness of dependent types, Oury and Swierstra~\cite{pi}
build a DSEL in Agda for relational algebra. Eisenberg and
Weirich~\cite{singletons} ported the result to Haskell using singleton types, after GHC introduced more features facilitating type-level programming.

None of the type-level programming techniques we used in this paper are new.
Indexed monads (also called parameterized monads) have been introduced by
Atkey~\cite{indexedmonad}. McBride~\cite{kleisli} showed how to construct
indexed free-monads from Hoare Logic specifications. Kiselyov et
al.~\cite{typefun} used indexed monads to track the locks held among a given
finite set. The same paper also demonstrated implementation of a variety of
features including memorization, generic maps, session types, typed
\texttt{printf} and \texttt{sprintf}, etc., by type-level programming.
Before the introduction of type families, Kiselyov and
Shan~\cite{staticresources} used type classes and functional dependencies to
implement type-level functions, and showed that they are sufficient to track
resources in device drivers.

Lindley and McBride~\cite{phasedistinction} provided a thorough analysis and summary of the dependent-type-like features currently in Haskell, and
compared them with dependently typed languages without phase distinction such as Agda. It turns out that GHC's constraint solver works surprisingly well as an
automatic theorem prover.
