\documentclass{llncs}

\usepackage{amsmath}
%\usepackage{amsfonts}
%\usepackage{amsthm}
\usepackage{color}
\usepackage{url}
\usepackage{mathptmx}
\usepackage{doubleequals}
\usepackage{makeidx}  % allows for indexgeneration
\usepackage{textcomp} % for single quote
\usepackage{fancyvrb} % for indented verbatim

\input{Preamble}
\newcommand{\hide}[1]{}
\newcommand{\todo}[1]{\noindent\textcolor{blue}{\ifmmode \text{[#1]}\else [#1] \fi}}

\newcommand{\Redis}{{\sf Redis}}
\newcommand{\Hedis}{{\sf Hedis}}
\newcommand{\Popcorn}{{\sf Popcorn}}


\begin{document}

\title{Type Safe Redis Queries}
\subtitle{A Case Study of Type-Level Programming in Haskell}
\author{Ting-Yan Lai \inst{1}\inst{2}\and Tyng-Ruey Chuang \inst{1}
  \and Shin-Cheng Mu \inst{1}}
\institute{Institute of Information Science, Academia Sinica, Taiwan
\and Institute of Computer Science and Engineering,
National Chiao Tung University, Taiwan}

\maketitle

\begin{abstract}
\Redis{} is an in-memory data structure store, often used as a database, with a
Haskell interface \Hedis{}. \Redis{} is dynamically typed --- a key can be
discarded and re-associated to a value of a different type, and a command,
when fetching a value of a type it does not expect, signals a runtime error. We
develop a domain-specific language that, by exploiting Haskell type-level
programming techniques including indexed monad, type-level literals and closed
type families, keeps track of types of values in the database and statically
guarantees that type errors cannot happen for a class of \Redis{} programs.
\end{abstract}

%% ODER: format ==         = "\mathrel{==}"
%% ODER: format /=         = "\neq "
%
%
\makeatletter
\@ifundefined{lhs2tex.lhs2tex.sty.read}%
  {\@namedef{lhs2tex.lhs2tex.sty.read}{}%
   \newcommand\SkipToFmtEnd{}%
   \newcommand\EndFmtInput{}%
   \long\def\SkipToFmtEnd#1\EndFmtInput{}%
  }\SkipToFmtEnd

\newcommand\ReadOnlyOnce[1]{\@ifundefined{#1}{\@namedef{#1}{}}\SkipToFmtEnd}
\usepackage{amstext}
\usepackage{amssymb}
\usepackage{stmaryrd}
\DeclareFontFamily{OT1}{cmtex}{}
\DeclareFontShape{OT1}{cmtex}{m}{n}
  {<5><6><7><8>cmtex8
   <9>cmtex9
   <10><10.95><12><14.4><17.28><20.74><24.88>cmtex10}{}
\DeclareFontShape{OT1}{cmtex}{m}{it}
  {<-> ssub * cmtt/m/it}{}
\newcommand{\texfamily}{\fontfamily{cmtex}\selectfont}
\DeclareFontShape{OT1}{cmtt}{bx}{n}
  {<5><6><7><8>cmtt8
   <9>cmbtt9
   <10><10.95><12><14.4><17.28><20.74><24.88>cmbtt10}{}
\DeclareFontShape{OT1}{cmtex}{bx}{n}
  {<-> ssub * cmtt/bx/n}{}
\newcommand{\tex}[1]{\text{\texfamily#1}}	% NEU

\newcommand{\Sp}{\hskip.33334em\relax}


\newcommand{\Conid}[1]{\mathit{#1}}
\newcommand{\Varid}[1]{\mathit{#1}}
\newcommand{\anonymous}{\kern0.06em \vbox{\hrule\@width.5em}}
\newcommand{\plus}{\mathbin{+\!\!\!+}}
\newcommand{\bind}{\mathbin{>\!\!\!>\mkern-6.7mu=}}
\newcommand{\rbind}{\mathbin{=\mkern-6.7mu<\!\!\!<}}% suggested by Neil Mitchell
\newcommand{\sequ}{\mathbin{>\!\!\!>}}
\renewcommand{\leq}{\leqslant}
\renewcommand{\geq}{\geqslant}
\usepackage{polytable}

%mathindent has to be defined
\@ifundefined{mathindent}%
  {\newdimen\mathindent\mathindent\leftmargini}%
  {}%

\def\resethooks{%
  \global\let\SaveRestoreHook\empty
  \global\let\ColumnHook\empty}
\newcommand*{\savecolumns}[1][default]%
  {\g@addto@macro\SaveRestoreHook{\savecolumns[#1]}}
\newcommand*{\restorecolumns}[1][default]%
  {\g@addto@macro\SaveRestoreHook{\restorecolumns[#1]}}
\newcommand*{\aligncolumn}[2]%
  {\g@addto@macro\ColumnHook{\column{#1}{#2}}}

\resethooks

\newcommand{\onelinecommentchars}{\quad-{}- }
\newcommand{\commentbeginchars}{\enskip\{-}
\newcommand{\commentendchars}{-\}\enskip}

\newcommand{\visiblecomments}{%
  \let\onelinecomment=\onelinecommentchars
  \let\commentbegin=\commentbeginchars
  \let\commentend=\commentendchars}

\newcommand{\invisiblecomments}{%
  \let\onelinecomment=\empty
  \let\commentbegin=\empty
  \let\commentend=\empty}

\visiblecomments

\newlength{\blanklineskip}
\setlength{\blanklineskip}{0.66084ex}

\newcommand{\hsindent}[1]{\quad}% default is fixed indentation
\let\hspre\empty
\let\hspost\empty
\newcommand{\NB}{\textbf{NB}}
\newcommand{\Todo}[1]{$\langle$\textbf{To do:}~#1$\rangle$}

\EndFmtInput
\makeatother
%
%
%
%
%
%
% This package provides two environments suitable to take the place
% of hscode, called "plainhscode" and "arrayhscode". 
%
% The plain environment surrounds each code block by vertical space,
% and it uses \abovedisplayskip and \belowdisplayskip to get spacing
% similar to formulas. Note that if these dimensions are changed,
% the spacing around displayed math formulas changes as well.
% All code is indented using \leftskip.
%
% Changed 19.08.2004 to reflect changes in colorcode. Should work with
% CodeGroup.sty.
%
\ReadOnlyOnce{polycode.fmt}%
\makeatletter

\newcommand{\hsnewpar}[1]%
  {{\parskip=0pt\parindent=0pt\par\vskip #1\noindent}}

% can be used, for instance, to redefine the code size, by setting the
% command to \small or something alike
\newcommand{\hscodestyle}{}

% The command \sethscode can be used to switch the code formatting
% behaviour by mapping the hscode environment in the subst directive
% to a new LaTeX environment.

\newcommand{\sethscode}[1]%
  {\expandafter\let\expandafter\hscode\csname #1\endcsname
   \expandafter\let\expandafter\endhscode\csname end#1\endcsname}

% "compatibility" mode restores the non-polycode.fmt layout.

\newenvironment{compathscode}%
  {\par\noindent
   \advance\leftskip\mathindent
   \hscodestyle
   \let\\=\@normalcr
   \let\hspre\(\let\hspost\)%
   \pboxed}%
  {\endpboxed\)%
   \par\noindent
   \ignorespacesafterend}

\newcommand{\compaths}{\sethscode{compathscode}}

% "plain" mode is the proposed default.
% It should now work with \centering.
% This required some changes. The old version
% is still available for reference as oldplainhscode.

\newenvironment{plainhscode}%
  {\hsnewpar\abovedisplayskip
   \advance\leftskip\mathindent
   \hscodestyle
   \let\hspre\(\let\hspost\)%
   \pboxed}%
  {\endpboxed%
   \hsnewpar\belowdisplayskip
   \ignorespacesafterend}

\newenvironment{oldplainhscode}%
  {\hsnewpar\abovedisplayskip
   \advance\leftskip\mathindent
   \hscodestyle
   \let\\=\@normalcr
   \(\pboxed}%
  {\endpboxed\)%
   \hsnewpar\belowdisplayskip
   \ignorespacesafterend}

% Here, we make plainhscode the default environment.

\newcommand{\plainhs}{\sethscode{plainhscode}}
\newcommand{\oldplainhs}{\sethscode{oldplainhscode}}
\plainhs

% The arrayhscode is like plain, but makes use of polytable's
% parray environment which disallows page breaks in code blocks.

\newenvironment{arrayhscode}%
  {\hsnewpar\abovedisplayskip
   \advance\leftskip\mathindent
   \hscodestyle
   \let\\=\@normalcr
   \(\parray}%
  {\endparray\)%
   \hsnewpar\belowdisplayskip
   \ignorespacesafterend}

\newcommand{\arrayhs}{\sethscode{arrayhscode}}

% The mathhscode environment also makes use of polytable's parray 
% environment. It is supposed to be used only inside math mode 
% (I used it to typeset the type rules in my thesis).

\newenvironment{mathhscode}%
  {\parray}{\endparray}

\newcommand{\mathhs}{\sethscode{mathhscode}}

% texths is similar to mathhs, but works in text mode.

\newenvironment{texthscode}%
  {\(\parray}{\endparray\)}

\newcommand{\texths}{\sethscode{texthscode}}

% The framed environment places code in a framed box.

\def\codeframewidth{\arrayrulewidth}
\RequirePackage{calc}

\newenvironment{framedhscode}%
  {\parskip=\abovedisplayskip\par\noindent
   \hscodestyle
   \arrayrulewidth=\codeframewidth
   \tabular{@{}|p{\linewidth-2\arraycolsep-2\arrayrulewidth-2pt}|@{}}%
   \hline\framedhslinecorrect\\{-1.5ex}%
   \let\endoflinesave=\\
   \let\\=\@normalcr
   \(\pboxed}%
  {\endpboxed\)%
   \framedhslinecorrect\endoflinesave{.5ex}\hline
   \endtabular
   \parskip=\belowdisplayskip\par\noindent
   \ignorespacesafterend}

\newcommand{\framedhslinecorrect}[2]%
  {#1[#2]}

\newcommand{\framedhs}{\sethscode{framedhscode}}

% The inlinehscode environment is an experimental environment
% that can be used to typeset displayed code inline.

\newenvironment{inlinehscode}%
  {\(\def\column##1##2{}%
   \let\>\undefined\let\<\undefined\let\\\undefined
   \newcommand\>[1][]{}\newcommand\<[1][]{}\newcommand\\[1][]{}%
   \def\fromto##1##2##3{##3}%
   \def\nextline{}}{\) }%

\newcommand{\inlinehs}{\sethscode{inlinehscode}}

% The joincode environment is a separate environment that
% can be used to surround and thereby connect multiple code
% blocks.

\newenvironment{joincode}%
  {\let\orighscode=\hscode
   \let\origendhscode=\endhscode
   \def\endhscode{\def\hscode{\endgroup\def\@currenvir{hscode}\\}\begingroup}
   %\let\SaveRestoreHook=\empty
   %\let\ColumnHook=\empty
   %\let\resethooks=\empty
   \orighscode\def\hscode{\endgroup\def\@currenvir{hscode}}}%
  {\origendhscode
   \global\let\hscode=\orighscode
   \global\let\endhscode=\origendhscode}%

\makeatother
\EndFmtInput
%

\ReadOnlyOnce{Formatting.fmt}%
\makeatletter

\let\Varid\mathit
\let\Conid\mathsf

\def\commentbegin{\quad\{\ }
\def\commentend{\}}

\newcommand{\ty}[1]{\Conid{#1}}
\newcommand{\con}[1]{\Conid{#1}}
\newcommand{\id}[1]{\Varid{#1}}
\newcommand{\cl}[1]{\Varid{#1}}
\newcommand{\opsym}[1]{\mathrel{#1}}

\newcommand\Keyword[1]{\textbf{\textsf{#1}}}
\newcommand\Hide{\mathbin{\downarrow}}
\newcommand\Reveal{\mathbin{\uparrow}}


%% Paper-specific keywords


\makeatother
\EndFmtInput

\section{Introduction}
\label{sec:introduction}

\Redis{}\footnote{\url{https://redis.io}} is an open source, in-memory data structure store, often used as database, cache and message broker. A \Redis{} datatype can be think of as a set of key-value pairs, where each value is associated with a binary-safe string key to identify and manipulate with.
\Redis{} allows values of various types, including strings, hashes, lists, and sets, etc, to be stored, and provides a collection of of atomic \emph{commands} to manipulate these values.

For an example, consider the following sequence of commands, entered through the interactive interface of \Redis{}. The keys \texttt{some-set} and \texttt{another-set}
are both associated to a set. The two call to command \texttt{SADD} respectively
adds three and two values to the two sets, before \texttt{SINTER} takes their intersection:
\begin{tabbing}\tt
~redis\char62{}~SADD~some\char45{}set~a~b~c\\
\tt ~\char40{}integer\char41{}~3\\
\tt ~redis\char62{}~SADD~another\char45{}set~a~b\\
\tt ~\char40{}integer\char41{}~2\\
\tt ~redis\char62{}~SINTER~some\char45{}set~another\char45{}set\\
\tt ~1\char41{}~\char34{}a\char34{}\\
\tt ~2\char41{}~\char34{}b\char34{}
\end{tabbing}

Note that the keys \texttt{some-set} and \texttt{another-set}, if not existing before the call to \texttt{SADD}, are created on site. The call to
\texttt{SADD} returns the size of the set after completion of the command.

Many third party libraries provide interfaces that allow general purpose programming languages to access \Redis{} through its TCP protocol.
For Haskell, the most popular library is \Hedis{}\footnote{\url{https://hackage.haskell.org/package/hedis}}.
The following program implements the previous example:
\begin{hscode}\SaveRestoreHook
\column{B}{@{}>{\hspre}l<{\hspost}@{}}%
\column{5}{@{}>{\hspre}l<{\hspost}@{}}%
\column{E}{@{}>{\hspre}l<{\hspost}@{}}%
\>[B]{}\Varid{program}\mathbin{::}\Conid{Redis}\;(\Conid{Either}\;\Conid{Reply}\;[\mskip1.5mu \Conid{ByteString}\mskip1.5mu]){}\<[E]%
\\
\>[B]{}\Varid{program}\mathrel{=}\mathbf{do}{}\<[E]%
\\
\>[B]{}\hsindent{5}{}\<[5]%
\>[5]{}\Varid{sadd}\;\text{\tt \char34 some-set\char34}\;[\mskip1.5mu \text{\tt \char34 a\char34},\text{\tt \char34 b\char34}\mskip1.5mu]{}\<[E]%
\\
\>[B]{}\hsindent{5}{}\<[5]%
\>[5]{}\Varid{sadd}\;\text{\tt \char34 another-set\char34}\;[\mskip1.5mu \text{\tt \char34 a\char34},\text{\tt \char34 b\char34},\text{\tt \char34 c\char34}\mskip1.5mu]{}\<[E]%
\\
\>[B]{}\hsindent{5}{}\<[5]%
\>[5]{}\Varid{sinter}\;[\mskip1.5mu \text{\tt \char34 some-set\char34},\text{\tt \char34 another-set\char34}\mskip1.5mu]~~.{}\<[E]%
\ColumnHook
\end{hscode}\resethooks
The function \ensuremath{\Varid{sadd}\mathbin{::}\Conid{ByteString}\to [\mskip1.5mu \Conid{ByteString}\mskip1.5mu]\to \Conid{Redis}\;(\Conid{Either}\;\Conid{Reply}\;\Conid{Integer})} takes a key and a list of values as arguments, and returns
an \ensuremath{\Conid{Integer}} on success, or returns a \ensuremath{\Conid{Reply}}, a low-level representation of
replies from the Redis server, in case of failures. All wrapped in the monad
\ensuremath{\Conid{Redis}}, the context of command execution.\footnotemark

Note that keys and values, being nothing but binary strings in Redis, are
represented using Haskell \ensuremath{\Conid{ByteString}}. Values of other types must be encoded
as \ensuremath{\Conid{ByteString}}s before being written to the database, and decoded after being
read back.

\footnotetext{\Hedis{} provides another kind of context, \ensuremath{\Conid{RedisTx}}, for \emph{transactions}, united with \ensuremath{\Conid{Redis}} under the class \ensuremath{\Conid{RedisCtx}}. We
demonstrate only \ensuremath{\Conid{Redis}} in this paper.}

%\paragraph{Motivation}
%All binary strings are equal, but some binary strings are more equal than others.
%While everything in \Redis{} is essentially a binary string, these strings
%are treated differently.
\Redis{} supports many different kind of data
structures, such as strings, hashes, lists, etc. While they are all encoded as
binary strings before being written to the database, most commands only works
with data of certain types. In the following example, the key
\texttt{some-string} is associated to string \texttt{foo} --- the command
\texttt{SET} always associates a key to a string. The subsequent call to \texttt{SADD}, which adds a value to a set, thus causes a runtime error.
\begin{tabbing}\tt
~redis\char62{}~SET~some\char45{}string~foo\\
\tt ~OK\\
\tt ~redis\char62{}~SADD~some\char45{}string~bar\\
\tt ~\char40{}error\char41{}~WRONGTYPE~Operation~against~a~key~holding~the~wrong\\
\tt ~kind~of~value
\end{tabbing}

%\paragraph{Example 2} Even worse, not all strings are equal!
% The call \texttt{INCR some-string} parses the string associated with key
% \texttt{some-string} to an integer, increments it by one, and store it back as
% a string. If the string can not be parse as an integer, a runtime error
% is raised.
% \begin{verbatim}
% redis> SET some-string foo
% OK
% redis> INCR some-string
% (error) ERR value is not an integer or out of range
% \end{verbatim}

Being a simple wrapper on top of the TCP protocol of \Redis{}, \Hedis{}
inherits the problem. Executing following program yields the same error
wrapped in Haskell: \ensuremath{\Conid{Left}\;(\Conid{Error}} \texttt{"WRONGTYPE Operation against a
key holding the wrong kind of value"}\ensuremath{)}.
\begin{hscode}\SaveRestoreHook
\column{B}{@{}>{\hspre}l<{\hspost}@{}}%
\column{5}{@{}>{\hspre}l<{\hspost}@{}}%
\column{E}{@{}>{\hspre}l<{\hspost}@{}}%
\>[B]{}\Varid{program}\mathbin{::}\Conid{Redis}\;(\Conid{Either}\;\Conid{Reply}\;\Conid{Integer}){}\<[E]%
\\
\>[B]{}\Varid{program}\mathrel{=}\mathbf{do}{}\<[E]%
\\
\>[B]{}\hsindent{5}{}\<[5]%
\>[5]{}\Varid{set}\;\text{\tt \char34 some-string\char34}\;\text{\tt \char34 foo\char34}{}\<[E]%
\\
\>[B]{}\hsindent{5}{}\<[5]%
\>[5]{}\Varid{sadd}\;\text{\tt \char34 some-string\char34}\;[\mskip1.5mu \text{\tt \char34 a\char34}\mskip1.5mu]~~.{}\<[E]%
\ColumnHook
\end{hscode}\resethooks

% \paragraph{The Cause} Every key is associated with a value, and every value has
% its own type. But most commands in \Redis{} only work with a certain type of
%  value. When a command is used on a wrong type of key, a runtime error occurs.
%  The problems illustrated above arise from the absence of type checking, with
%  respects to \textbf{the type of a value that associates with a key}.
%  These problems could have been avoided, if we could know the type every key
%  associates with in advance, and prevent programs with invalid commands from
%  executing.
%
% \paragraph{Hedis as an embedded DSL}
% Haskell makes it easy to build and use {\em domain specific embedded languages} (DSELs), and \Hedis{} can be regarded as one of them. What makes \Hedis{} peculiar is that,
%  it has \emph{variable bindings} (between keys and values), but with very
%  little or no semantic checking, neither dynamically nor statically.
%
Such a programming model is certainly very error-prone. Working within Haskell,
a host language with a strong typing system, one naturally wishes to build a
a domain-specific embedded language (DSEL) that exploits the rich type system
of Haskell to ensure absence of \Redis{} type errors.

This paper discusses the techniques we used and experiences we learned from building such a language, nicknamed \Popcorn{}. We constructed an {\em indexed
monad}, on top of the monad \ensuremath{\Conid{Redis}}, which is indexed by a dictionary that
maintains the set of currently defined keys and their types. To represent
the dictionary, we need to encode variable binds with {\em type-level} lists
and strings. To summarize our contributions:
\begin{itemize}
\item We present \Popcorn{}, a statically typed domain-specific language embedded in Haskell and built on \Hedis{}.
% also makes Redis polymorphic by automatically converting back and forth from values of arbitrary types and boring ByteStrings.
%
\item We demonstrate how to model variable bindings of an embedded DSL using
 language extensions including type-level literals and data kinds.
%
\item We provide (yet another) an example of encoding effects and constraints of
in types, with indexed monad~\cite{indexedmonad}, closed type-families~\cite{closedtypefamilies} and constraints kinds~\cite{constraintskinds}.
\end{itemize}
\todo{Phrase this better.}

%% ODER: format ==         = "\mathrel{==}"
%% ODER: format /=         = "\neq "
%
%
\makeatletter
\@ifundefined{lhs2tex.lhs2tex.sty.read}%
  {\@namedef{lhs2tex.lhs2tex.sty.read}{}%
   \newcommand\SkipToFmtEnd{}%
   \newcommand\EndFmtInput{}%
   \long\def\SkipToFmtEnd#1\EndFmtInput{}%
  }\SkipToFmtEnd

\newcommand\ReadOnlyOnce[1]{\@ifundefined{#1}{\@namedef{#1}{}}\SkipToFmtEnd}
\usepackage{amstext}
\usepackage{amssymb}
\usepackage{stmaryrd}
\DeclareFontFamily{OT1}{cmtex}{}
\DeclareFontShape{OT1}{cmtex}{m}{n}
  {<5><6><7><8>cmtex8
   <9>cmtex9
   <10><10.95><12><14.4><17.28><20.74><24.88>cmtex10}{}
\DeclareFontShape{OT1}{cmtex}{m}{it}
  {<-> ssub * cmtt/m/it}{}
\newcommand{\texfamily}{\fontfamily{cmtex}\selectfont}
\DeclareFontShape{OT1}{cmtt}{bx}{n}
  {<5><6><7><8>cmtt8
   <9>cmbtt9
   <10><10.95><12><14.4><17.28><20.74><24.88>cmbtt10}{}
\DeclareFontShape{OT1}{cmtex}{bx}{n}
  {<-> ssub * cmtt/bx/n}{}
\newcommand{\tex}[1]{\text{\texfamily#1}}	% NEU

\newcommand{\Sp}{\hskip.33334em\relax}


\newcommand{\Conid}[1]{\mathit{#1}}
\newcommand{\Varid}[1]{\mathit{#1}}
\newcommand{\anonymous}{\kern0.06em \vbox{\hrule\@width.5em}}
\newcommand{\plus}{\mathbin{+\!\!\!+}}
\newcommand{\bind}{\mathbin{>\!\!\!>\mkern-6.7mu=}}
\newcommand{\rbind}{\mathbin{=\mkern-6.7mu<\!\!\!<}}% suggested by Neil Mitchell
\newcommand{\sequ}{\mathbin{>\!\!\!>}}
\renewcommand{\leq}{\leqslant}
\renewcommand{\geq}{\geqslant}
\usepackage{polytable}

%mathindent has to be defined
\@ifundefined{mathindent}%
  {\newdimen\mathindent\mathindent\leftmargini}%
  {}%

\def\resethooks{%
  \global\let\SaveRestoreHook\empty
  \global\let\ColumnHook\empty}
\newcommand*{\savecolumns}[1][default]%
  {\g@addto@macro\SaveRestoreHook{\savecolumns[#1]}}
\newcommand*{\restorecolumns}[1][default]%
  {\g@addto@macro\SaveRestoreHook{\restorecolumns[#1]}}
\newcommand*{\aligncolumn}[2]%
  {\g@addto@macro\ColumnHook{\column{#1}{#2}}}

\resethooks

\newcommand{\onelinecommentchars}{\quad-{}- }
\newcommand{\commentbeginchars}{\enskip\{-}
\newcommand{\commentendchars}{-\}\enskip}

\newcommand{\visiblecomments}{%
  \let\onelinecomment=\onelinecommentchars
  \let\commentbegin=\commentbeginchars
  \let\commentend=\commentendchars}

\newcommand{\invisiblecomments}{%
  \let\onelinecomment=\empty
  \let\commentbegin=\empty
  \let\commentend=\empty}

\visiblecomments

\newlength{\blanklineskip}
\setlength{\blanklineskip}{0.66084ex}

\newcommand{\hsindent}[1]{\quad}% default is fixed indentation
\let\hspre\empty
\let\hspost\empty
\newcommand{\NB}{\textbf{NB}}
\newcommand{\Todo}[1]{$\langle$\textbf{To do:}~#1$\rangle$}

\EndFmtInput
\makeatother
%
%
%
%
%
%
% This package provides two environments suitable to take the place
% of hscode, called "plainhscode" and "arrayhscode". 
%
% The plain environment surrounds each code block by vertical space,
% and it uses \abovedisplayskip and \belowdisplayskip to get spacing
% similar to formulas. Note that if these dimensions are changed,
% the spacing around displayed math formulas changes as well.
% All code is indented using \leftskip.
%
% Changed 19.08.2004 to reflect changes in colorcode. Should work with
% CodeGroup.sty.
%
\ReadOnlyOnce{polycode.fmt}%
\makeatletter

\newcommand{\hsnewpar}[1]%
  {{\parskip=0pt\parindent=0pt\par\vskip #1\noindent}}

% can be used, for instance, to redefine the code size, by setting the
% command to \small or something alike
\newcommand{\hscodestyle}{}

% The command \sethscode can be used to switch the code formatting
% behaviour by mapping the hscode environment in the subst directive
% to a new LaTeX environment.

\newcommand{\sethscode}[1]%
  {\expandafter\let\expandafter\hscode\csname #1\endcsname
   \expandafter\let\expandafter\endhscode\csname end#1\endcsname}

% "compatibility" mode restores the non-polycode.fmt layout.

\newenvironment{compathscode}%
  {\par\noindent
   \advance\leftskip\mathindent
   \hscodestyle
   \let\\=\@normalcr
   \let\hspre\(\let\hspost\)%
   \pboxed}%
  {\endpboxed\)%
   \par\noindent
   \ignorespacesafterend}

\newcommand{\compaths}{\sethscode{compathscode}}

% "plain" mode is the proposed default.
% It should now work with \centering.
% This required some changes. The old version
% is still available for reference as oldplainhscode.

\newenvironment{plainhscode}%
  {\hsnewpar\abovedisplayskip
   \advance\leftskip\mathindent
   \hscodestyle
   \let\hspre\(\let\hspost\)%
   \pboxed}%
  {\endpboxed%
   \hsnewpar\belowdisplayskip
   \ignorespacesafterend}

\newenvironment{oldplainhscode}%
  {\hsnewpar\abovedisplayskip
   \advance\leftskip\mathindent
   \hscodestyle
   \let\\=\@normalcr
   \(\pboxed}%
  {\endpboxed\)%
   \hsnewpar\belowdisplayskip
   \ignorespacesafterend}

% Here, we make plainhscode the default environment.

\newcommand{\plainhs}{\sethscode{plainhscode}}
\newcommand{\oldplainhs}{\sethscode{oldplainhscode}}
\plainhs

% The arrayhscode is like plain, but makes use of polytable's
% parray environment which disallows page breaks in code blocks.

\newenvironment{arrayhscode}%
  {\hsnewpar\abovedisplayskip
   \advance\leftskip\mathindent
   \hscodestyle
   \let\\=\@normalcr
   \(\parray}%
  {\endparray\)%
   \hsnewpar\belowdisplayskip
   \ignorespacesafterend}

\newcommand{\arrayhs}{\sethscode{arrayhscode}}

% The mathhscode environment also makes use of polytable's parray 
% environment. It is supposed to be used only inside math mode 
% (I used it to typeset the type rules in my thesis).

\newenvironment{mathhscode}%
  {\parray}{\endparray}

\newcommand{\mathhs}{\sethscode{mathhscode}}

% texths is similar to mathhs, but works in text mode.

\newenvironment{texthscode}%
  {\(\parray}{\endparray\)}

\newcommand{\texths}{\sethscode{texthscode}}

% The framed environment places code in a framed box.

\def\codeframewidth{\arrayrulewidth}
\RequirePackage{calc}

\newenvironment{framedhscode}%
  {\parskip=\abovedisplayskip\par\noindent
   \hscodestyle
   \arrayrulewidth=\codeframewidth
   \tabular{@{}|p{\linewidth-2\arraycolsep-2\arrayrulewidth-2pt}|@{}}%
   \hline\framedhslinecorrect\\{-1.5ex}%
   \let\endoflinesave=\\
   \let\\=\@normalcr
   \(\pboxed}%
  {\endpboxed\)%
   \framedhslinecorrect\endoflinesave{.5ex}\hline
   \endtabular
   \parskip=\belowdisplayskip\par\noindent
   \ignorespacesafterend}

\newcommand{\framedhslinecorrect}[2]%
  {#1[#2]}

\newcommand{\framedhs}{\sethscode{framedhscode}}

% The inlinehscode environment is an experimental environment
% that can be used to typeset displayed code inline.

\newenvironment{inlinehscode}%
  {\(\def\column##1##2{}%
   \let\>\undefined\let\<\undefined\let\\\undefined
   \newcommand\>[1][]{}\newcommand\<[1][]{}\newcommand\\[1][]{}%
   \def\fromto##1##2##3{##3}%
   \def\nextline{}}{\) }%

\newcommand{\inlinehs}{\sethscode{inlinehscode}}

% The joincode environment is a separate environment that
% can be used to surround and thereby connect multiple code
% blocks.

\newenvironment{joincode}%
  {\let\orighscode=\hscode
   \let\origendhscode=\endhscode
   \def\endhscode{\def\hscode{\endgroup\def\@currenvir{hscode}\\}\begingroup}
   %\let\SaveRestoreHook=\empty
   %\let\ColumnHook=\empty
   %\let\resethooks=\empty
   \orighscode\def\hscode{\endgroup\def\@currenvir{hscode}}}%
  {\origendhscode
   \global\let\hscode=\orighscode
   \global\let\endhscode=\origendhscode}%

\makeatother
\EndFmtInput
%

\ReadOnlyOnce{Formatting.fmt}%
\makeatletter

\let\Varid\mathit
\let\Conid\mathsf

\def\commentbegin{\quad\{\ }
\def\commentend{\}}

\newcommand{\ty}[1]{\Conid{#1}}
\newcommand{\con}[1]{\Conid{#1}}
\newcommand{\id}[1]{\Varid{#1}}
\newcommand{\cl}[1]{\Varid{#1}}
\newcommand{\opsym}[1]{\mathrel{#1}}

\newcommand\Keyword[1]{\textbf{\textsf{#1}}}
\newcommand\Hide{\mathbin{\downarrow}}
\newcommand\Reveal{\mathbin{\uparrow}}


%% Paper-specific keywords


\makeatother
\EndFmtInput

\section{Indexed Monads}
\label{sec:indexed-monads}

Stateful computations are often reasoned in a Hoare-logic style: each command
is labelled by a \emph{precondition} and a \emph{postcondition}. If the former
is satisfied before the command is executed, the latter is guaranteed to hold
afterwards.

In Haskell, stateful computations are represented by monads. In order to
reason about their behaviors within the type system, we wish to label a state
monad with its pre and postcondition. An \emph{indexed
monad}~\cite{indexedmonad} (also called \emph{monadish} or
\emph{parameterised monad}) is a monad that, in addition to the type of value
it computes, takes two more type arguments representing an initial state and
a final state, to be interpreted like a Hoare triple~\cite{kleisli}:
\begin{hscode}\SaveRestoreHook
\column{B}{@{}>{\hspre}l<{\hspost}@{}}%
\column{5}{@{}>{\hspre}l<{\hspost}@{}}%
\column{E}{@{}>{\hspre}l<{\hspost}@{}}%
\>[B]{}\mathbf{class}\;\Conid{IMonad}\;\Varid{m}\;\mathbf{where}{}\<[E]%
\\
\>[B]{}\hsindent{5}{}\<[5]%
\>[5]{}\Varid{unit}\mathbin{::}\Varid{a}\to \Varid{m}\;\Varid{p}\;\Varid{p}\;\Varid{a}{}\<[E]%
\\
\>[B]{}\hsindent{5}{}\<[5]%
\>[5]{}\Varid{bind}\mathbin{::}\Varid{m}\;\Varid{p}\;\Varid{q}\;\Varid{a}\to (\Varid{a}\to \Varid{m}\;\Varid{q}\;\Varid{r}\;\Varid{b})\to \Varid{m}\;\Varid{p}\;\Varid{r}\;\Varid{b}~~.{}\<[E]%
\ColumnHook
\end{hscode}\resethooks
The intention is that a computation of type \ensuremath{\Varid{m}\;\Varid{p}\;\Varid{q}\;\Varid{a}} is a stateful computation
such that if it starts execution in a state satisfying \ensuremath{\Varid{p}} and terminates, it
yields a value of type \ensuremath{\Varid{a}}, and the new state satisfies \ensuremath{\Varid{q}}.
The operator \ensuremath{\Varid{unit}} lifts a pure computation to a stateful computation that
does not alter the state. In \ensuremath{\Varid{x}\mathbin{`\Varid{bind}`}\Varid{f}}, a computation \ensuremath{\Varid{x}\mathbin{::}\Varid{m}\;\Varid{p}\;\Varid{q}\;\Varid{a}} must
be chained before \ensuremath{\Varid{f}\mathbin{::}\Varid{a}\to \Varid{m}\;\Varid{q}\;\Varid{r}\;\Varid{b}}, which expects a value of type \ensuremath{\Varid{a}} and
a state satisfying \ensuremath{\Varid{q}} and, if terminates, ends in a state satisfying \ensuremath{\Varid{r}}.
The result is a monad \ensuremath{\Varid{m}\;\Varid{p}\;\Varid{r}\;\Varid{b}} --- a computation that, if executed in a state
satisfying \ensuremath{\Varid{p}} and terminates, yields a value \ensuremath{\Varid{b}} and a state satisfying \ensuremath{\Varid{r}}.
Indexed monads have been used ~\cite{typefun,staticresources} ... \todo{for what? Some discriptions here to properly cite them.}

We define a new indexed monad \ensuremath{\Conid{Popcorn}} which, at term level, merely wraps
\ensuremath{\Conid{Redis}} in an additional constructor. The purpose is to add the pre and
postconditions at type level:
\begin{hscode}\SaveRestoreHook
\column{B}{@{}>{\hspre}l<{\hspost}@{}}%
\column{5}{@{}>{\hspre}l<{\hspost}@{}}%
\column{E}{@{}>{\hspre}l<{\hspost}@{}}%
\>[B]{}\mathbf{newtype}\;\Conid{Popcorn}\;\Varid{p}\;\Varid{q}\;\Varid{a}\mathrel{=}\Conid{Popcorn}\;\{\mskip1.5mu \Varid{unPopcorn}\mathbin{::}\Conid{Redis}\;\Varid{a}\mskip1.5mu\}~~,{}\<[E]%
\\[\blanklineskip]%
\>[B]{}\mathbf{instance}\;\Conid{IMonad}\;\Conid{Popcorn}\;\mathbf{where}{}\<[E]%
\\
\>[B]{}\hsindent{5}{}\<[5]%
\>[5]{}\Varid{unit}\mathrel{=}\Conid{Popcorn}\mathbin{\cdot}\Varid{return}{}\<[E]%
\\
\>[B]{}\hsindent{5}{}\<[5]%
\>[5]{}\Varid{bind}\;\Varid{m}\;\Varid{f}\mathrel{=}\Conid{Popcorn}\;(\Varid{unPopcorn}\;\Varid{m}\bind \Varid{unPopcorn}\mathbin{\cdot}\Varid{f})~~.{}\<[E]%
\ColumnHook
\end{hscode}\resethooks
To execute a \ensuremath{\Conid{Popcorn}} program, simply apply it to \ensuremath{\Varid{unPopcorn}} to erase the additional type information and get back an ordinary \Hedis{} program.

\paragraph{\text{PING}: A First Example}
In \Redis{}, \text{PING} does nothing but replies with
 \text{PONG} if the connection is alive. In Hedis,
 \ensuremath{\Varid{ping}} has type:

\begin{hscode}\SaveRestoreHook
\column{B}{@{}>{\hspre}l<{\hspost}@{}}%
\column{E}{@{}>{\hspre}l<{\hspost}@{}}%
\>[B]{}\Varid{ping}\mathbin{::}\Conid{Redis}\;(\Conid{Either}\;\Conid{Reply}\;\Conid{Status}){}\<[E]%
\ColumnHook
\end{hscode}\resethooks

Now with \ensuremath{\Conid{Popcorn}}, we could make our own version of
\ensuremath{\Varid{ping}}\footnotemark

\footnotetext{\ensuremath{\Varid{ping}} from Hedis is qualified with
\ensuremath{\Conid{Hedis}} to prevent function name clashing in our code.}

\begin{hscode}\SaveRestoreHook
\column{B}{@{}>{\hspre}l<{\hspost}@{}}%
\column{E}{@{}>{\hspre}l<{\hspost}@{}}%
\>[B]{}\Varid{ping}\mathbin{::}\Conid{Popcorn}\;\Varid{xs}\;\Varid{xs}\;(\Conid{Either}\;\Conid{Reply}\;\Conid{Status}){}\<[E]%
\\
\>[B]{}\Varid{ping}\mathrel{=}\Conid{Popcorn}\;\Varid{\Conid{Hedis}.ping}{}\<[E]%
\ColumnHook
\end{hscode}\resethooks

The dictionary \ensuremath{\Varid{xs}} in the type remains unaffected after the
 action, because \ensuremath{\Varid{ping}} does not affect any key-type
 bindings. To encode other commands that modifies key-type bindings, we need
 type-level functions to annotate those effects on the dictionary.

%% ODER: format ==         = "\mathrel{==}"
%% ODER: format /=         = "\neq "
%
%
\makeatletter
\@ifundefined{lhs2tex.lhs2tex.sty.read}%
  {\@namedef{lhs2tex.lhs2tex.sty.read}{}%
   \newcommand\SkipToFmtEnd{}%
   \newcommand\EndFmtInput{}%
   \long\def\SkipToFmtEnd#1\EndFmtInput{}%
  }\SkipToFmtEnd

\newcommand\ReadOnlyOnce[1]{\@ifundefined{#1}{\@namedef{#1}{}}\SkipToFmtEnd}
\usepackage{amstext}
\usepackage{amssymb}
\usepackage{stmaryrd}
\DeclareFontFamily{OT1}{cmtex}{}
\DeclareFontShape{OT1}{cmtex}{m}{n}
  {<5><6><7><8>cmtex8
   <9>cmtex9
   <10><10.95><12><14.4><17.28><20.74><24.88>cmtex10}{}
\DeclareFontShape{OT1}{cmtex}{m}{it}
  {<-> ssub * cmtt/m/it}{}
\newcommand{\texfamily}{\fontfamily{cmtex}\selectfont}
\DeclareFontShape{OT1}{cmtt}{bx}{n}
  {<5><6><7><8>cmtt8
   <9>cmbtt9
   <10><10.95><12><14.4><17.28><20.74><24.88>cmbtt10}{}
\DeclareFontShape{OT1}{cmtex}{bx}{n}
  {<-> ssub * cmtt/bx/n}{}
\newcommand{\tex}[1]{\text{\texfamily#1}}	% NEU

\newcommand{\Sp}{\hskip.33334em\relax}


\newcommand{\Conid}[1]{\mathit{#1}}
\newcommand{\Varid}[1]{\mathit{#1}}
\newcommand{\anonymous}{\kern0.06em \vbox{\hrule\@width.5em}}
\newcommand{\plus}{\mathbin{+\!\!\!+}}
\newcommand{\bind}{\mathbin{>\!\!\!>\mkern-6.7mu=}}
\newcommand{\rbind}{\mathbin{=\mkern-6.7mu<\!\!\!<}}% suggested by Neil Mitchell
\newcommand{\sequ}{\mathbin{>\!\!\!>}}
\renewcommand{\leq}{\leqslant}
\renewcommand{\geq}{\geqslant}
\usepackage{polytable}

%mathindent has to be defined
\@ifundefined{mathindent}%
  {\newdimen\mathindent\mathindent\leftmargini}%
  {}%

\def\resethooks{%
  \global\let\SaveRestoreHook\empty
  \global\let\ColumnHook\empty}
\newcommand*{\savecolumns}[1][default]%
  {\g@addto@macro\SaveRestoreHook{\savecolumns[#1]}}
\newcommand*{\restorecolumns}[1][default]%
  {\g@addto@macro\SaveRestoreHook{\restorecolumns[#1]}}
\newcommand*{\aligncolumn}[2]%
  {\g@addto@macro\ColumnHook{\column{#1}{#2}}}

\resethooks

\newcommand{\onelinecommentchars}{\quad-{}- }
\newcommand{\commentbeginchars}{\enskip\{-}
\newcommand{\commentendchars}{-\}\enskip}

\newcommand{\visiblecomments}{%
  \let\onelinecomment=\onelinecommentchars
  \let\commentbegin=\commentbeginchars
  \let\commentend=\commentendchars}

\newcommand{\invisiblecomments}{%
  \let\onelinecomment=\empty
  \let\commentbegin=\empty
  \let\commentend=\empty}

\visiblecomments

\newlength{\blanklineskip}
\setlength{\blanklineskip}{0.66084ex}

\newcommand{\hsindent}[1]{\quad}% default is fixed indentation
\let\hspre\empty
\let\hspost\empty
\newcommand{\NB}{\textbf{NB}}
\newcommand{\Todo}[1]{$\langle$\textbf{To do:}~#1$\rangle$}

\EndFmtInput
\makeatother
%
%
%
%
%
%
% This package provides two environments suitable to take the place
% of hscode, called "plainhscode" and "arrayhscode". 
%
% The plain environment surrounds each code block by vertical space,
% and it uses \abovedisplayskip and \belowdisplayskip to get spacing
% similar to formulas. Note that if these dimensions are changed,
% the spacing around displayed math formulas changes as well.
% All code is indented using \leftskip.
%
% Changed 19.08.2004 to reflect changes in colorcode. Should work with
% CodeGroup.sty.
%
\ReadOnlyOnce{polycode.fmt}%
\makeatletter

\newcommand{\hsnewpar}[1]%
  {{\parskip=0pt\parindent=0pt\par\vskip #1\noindent}}

% can be used, for instance, to redefine the code size, by setting the
% command to \small or something alike
\newcommand{\hscodestyle}{}

% The command \sethscode can be used to switch the code formatting
% behaviour by mapping the hscode environment in the subst directive
% to a new LaTeX environment.

\newcommand{\sethscode}[1]%
  {\expandafter\let\expandafter\hscode\csname #1\endcsname
   \expandafter\let\expandafter\endhscode\csname end#1\endcsname}

% "compatibility" mode restores the non-polycode.fmt layout.

\newenvironment{compathscode}%
  {\par\noindent
   \advance\leftskip\mathindent
   \hscodestyle
   \let\\=\@normalcr
   \let\hspre\(\let\hspost\)%
   \pboxed}%
  {\endpboxed\)%
   \par\noindent
   \ignorespacesafterend}

\newcommand{\compaths}{\sethscode{compathscode}}

% "plain" mode is the proposed default.
% It should now work with \centering.
% This required some changes. The old version
% is still available for reference as oldplainhscode.

\newenvironment{plainhscode}%
  {\hsnewpar\abovedisplayskip
   \advance\leftskip\mathindent
   \hscodestyle
   \let\hspre\(\let\hspost\)%
   \pboxed}%
  {\endpboxed%
   \hsnewpar\belowdisplayskip
   \ignorespacesafterend}

\newenvironment{oldplainhscode}%
  {\hsnewpar\abovedisplayskip
   \advance\leftskip\mathindent
   \hscodestyle
   \let\\=\@normalcr
   \(\pboxed}%
  {\endpboxed\)%
   \hsnewpar\belowdisplayskip
   \ignorespacesafterend}

% Here, we make plainhscode the default environment.

\newcommand{\plainhs}{\sethscode{plainhscode}}
\newcommand{\oldplainhs}{\sethscode{oldplainhscode}}
\plainhs

% The arrayhscode is like plain, but makes use of polytable's
% parray environment which disallows page breaks in code blocks.

\newenvironment{arrayhscode}%
  {\hsnewpar\abovedisplayskip
   \advance\leftskip\mathindent
   \hscodestyle
   \let\\=\@normalcr
   \(\parray}%
  {\endparray\)%
   \hsnewpar\belowdisplayskip
   \ignorespacesafterend}

\newcommand{\arrayhs}{\sethscode{arrayhscode}}

% The mathhscode environment also makes use of polytable's parray 
% environment. It is supposed to be used only inside math mode 
% (I used it to typeset the type rules in my thesis).

\newenvironment{mathhscode}%
  {\parray}{\endparray}

\newcommand{\mathhs}{\sethscode{mathhscode}}

% texths is similar to mathhs, but works in text mode.

\newenvironment{texthscode}%
  {\(\parray}{\endparray\)}

\newcommand{\texths}{\sethscode{texthscode}}

% The framed environment places code in a framed box.

\def\codeframewidth{\arrayrulewidth}
\RequirePackage{calc}

\newenvironment{framedhscode}%
  {\parskip=\abovedisplayskip\par\noindent
   \hscodestyle
   \arrayrulewidth=\codeframewidth
   \tabular{@{}|p{\linewidth-2\arraycolsep-2\arrayrulewidth-2pt}|@{}}%
   \hline\framedhslinecorrect\\{-1.5ex}%
   \let\endoflinesave=\\
   \let\\=\@normalcr
   \(\pboxed}%
  {\endpboxed\)%
   \framedhslinecorrect\endoflinesave{.5ex}\hline
   \endtabular
   \parskip=\belowdisplayskip\par\noindent
   \ignorespacesafterend}

\newcommand{\framedhslinecorrect}[2]%
  {#1[#2]}

\newcommand{\framedhs}{\sethscode{framedhscode}}

% The inlinehscode environment is an experimental environment
% that can be used to typeset displayed code inline.

\newenvironment{inlinehscode}%
  {\(\def\column##1##2{}%
   \let\>\undefined\let\<\undefined\let\\\undefined
   \newcommand\>[1][]{}\newcommand\<[1][]{}\newcommand\\[1][]{}%
   \def\fromto##1##2##3{##3}%
   \def\nextline{}}{\) }%

\newcommand{\inlinehs}{\sethscode{inlinehscode}}

% The joincode environment is a separate environment that
% can be used to surround and thereby connect multiple code
% blocks.

\newenvironment{joincode}%
  {\let\orighscode=\hscode
   \let\origendhscode=\endhscode
   \def\endhscode{\def\hscode{\endgroup\def\@currenvir{hscode}\\}\begingroup}
   %\let\SaveRestoreHook=\empty
   %\let\ColumnHook=\empty
   %\let\resethooks=\empty
   \orighscode\def\hscode{\endgroup\def\@currenvir{hscode}}}%
  {\origendhscode
   \global\let\hscode=\orighscode
   \global\let\endhscode=\origendhscode}%

\makeatother
\EndFmtInput
%

\ReadOnlyOnce{Formatting.fmt}%
\makeatletter

\let\Varid\mathit
\let\Conid\mathsf

\def\commentbegin{\quad\{\ }
\def\commentend{\}}

\newcommand{\ty}[1]{\Conid{#1}}
\newcommand{\con}[1]{\Conid{#1}}
\newcommand{\id}[1]{\Varid{#1}}
\newcommand{\cl}[1]{\Varid{#1}}
\newcommand{\opsym}[1]{\mathrel{#1}}

\newcommand\Keyword[1]{\textbf{\textsf{#1}}}
\newcommand\Hide{\mathbin{\downarrow}}
\newcommand\Reveal{\mathbin{\uparrow}}


%% Paper-specific keywords


\makeatother
\EndFmtInput

\section{Type-Level Dictionaries}
\label{sec:type-level-dict}

One of the challenges of statically ensuring type correctness of \Redis{},
which also presents in other stateful languages, is that the type of the value
associated to a key can be altered after updating. To ensure type correctness,
we have to keep track of the (\Redis{}) types of all existing keys in a
{\em dictionary} --- conceptually, a list of pairs of keys and \Redis{} types.
Each \Redis{} command is embedded in \Popcorn{} as a monadic computation. The
monad, to be presented in Section~\ref{sec:indexed-monads}, is indexed by
the dictionaries before and after the computation. In a dependently typed
programming language (without the so-called ``phase distinction'' ---
separation between types and terms), this would pose no problem. In Haskell
however, the dictionaries, to index a monad, has to be a Haskell type as well.

In this section we describe how to construct a type-level dictionary, to be
used with the indexed monad in Section~\ref{sec:indexed-monads}. More operations
on the dictionary will be presented in Section~\ref{sec:type-level-fun}.

\subsection{Datatype Promotion}

% Normally, at the term level, we could express the datatype of dictionary with
% \emph{type synonym} like this.\footnotemark
%
% \begin{spec}
% type Key = String
% type Dictionary = [(Key, TypeRep)]
% \end{spec}
%
% \footnotetext{|TypeRep| supports term-level representations
%  of datatypes, available in |Data.Typeable|}

A datatype definition such as the one below:
\begin{hscode}\SaveRestoreHook
\column{B}{@{}>{\hspre}l<{\hspost}@{}}%
\column{E}{@{}>{\hspre}l<{\hspost}@{}}%
\>[B]{}\mathbf{data}\;\Conid{List}\;\Varid{a}\mathrel{=}\Conid{Nil}\mid \Conid{Cons}\;\Varid{a}\;(\Conid{List}\;\Varid{a})~~,{}\<[E]%
\ColumnHook
\end{hscode}\resethooks
is usually understood as having defined a type constructor \ensuremath{\Conid{List}}, and two value
constructors \ensuremath{\Conid{Nil}} and \ensuremath{\Conid{Cons}}. For all lifted types \ensuremath{\Varid{a}}, \ensuremath{\Conid{List}\;\Varid{a}} is also a
lifted type. Thus the {\em kind} of \ensuremath{\Conid{List}} is \ensuremath{\mathbin{*}\to \mathbin{*}}, where \ensuremath{\mathbin{*}} is the kind of
all {\em lifted types} in Haskell. The two value constructors respectively
have types \ensuremath{\Conid{Nil}\mathbin{::}\Conid{List}\;\Varid{a}} and \ensuremath{\Conid{Cons}\mathbin{::}\Varid{a}\to \Conid{List}\;\Varid{a}\to \Conid{List}\;\Varid{a}}.

The GHC extension \emph{data kinds}~\cite{promotion}, however, automatically
promotes certain ``suitable'' types to kinds. With the extension, the \ensuremath{\mathbf{data}}
definition has an alternative reading: for all kind \ensuremath{\Varid{k}}, \ensuremath{\Conid{List}\;\Varid{k}} is also a
kind. \ensuremath{\Conid{Nil}} is a type, whose kind is \ensuremath{\Conid{List}\;\Varid{k}} for some \ensuremath{\Varid{k}}. Given a type \ensuremath{\Varid{x}}
of kind \ensuremath{\Varid{k}} and a type \ensuremath{\Varid{xs}} of kind \ensuremath{\Conid{List}\;\Varid{k}}, \ensuremath{\Conid{Cons}\;\Varid{x}\;\Varid{xs}} is again a type of
kind \ensuremath{\Conid{List}\;\Varid{k}}. Formally, for kind \ensuremath{\Varid{k}}, \ensuremath{\Conid{Nil}\mathbin{::}\Conid{List}\;\Varid{k}} and \ensuremath{\Conid{Cons}\mathbin{::}\Varid{k}\to \Conid{List}\;\Varid{k}\to \Conid{List}\;\Varid{k}}. Whether a constructor is promoted can often be inferred
from the context. To be more specific, prefixing a contructor with
a single quote, such as in \ensuremath{\thinspace'\Conid{Nil}} and \ensuremath{\thinspace'\Conid{Cons}}, denotes that it is promoted.

The built-in lists in Haskell has two constructors \ensuremath{[\mskip1.5mu \mskip1.5mu]} and \ensuremath{(\mathbin{:})}, and
\ensuremath{[\mskip1.5mu \mathrm{1},\mathrm{2},\mathrm{3}\mskip1.5mu]} is take as a syntax sugar for \ensuremath{\mathrm{1}\mathbin{:}(\mathrm{2}\mathbin{:}(\mathrm{3}\mathbin{:}[\mskip1.5mu \mskip1.5mu]))}. The type and
data constructors can also be promoted. Given kind \ensuremath{\Varid{k}}, \ensuremath{[\mskip1.5mu \Varid{k}\mskip1.5mu]} is also a kind.
The type constructor 
For example, since \ensuremath{\Conid{Int}}, \ensuremath{\Conid{Char}}, etc., all have kind \ensuremath{\mathbin{*}}, \ensuremath{\Conid{Int}\mathbin{\thinspace':}(\Conid{Char}\mathbin{\thinspace':}(\Conid{Bool}\mathbin{\thinspace':}\thinspace'[]))} is a type having kind \ensuremath{[\mskip1.5mu \mathbin{*}\mskip1.5mu]} -- a list of types. The same list can be denoted by a syntax
sugar \ensuremath{\thinspace'[\!\;\Conid{Int},\Conid{Char},\Conid{Bool}\;\!]}.

sugar for \ensuremath{\mathrm{1}\mathbin{\thinspace':}(\mathrm{2}\mathbin{\thinspace':}(\mathrm{3}\mathbin{\thinspace':}\thinspace'[]))}
Haskell sugars lists \ensuremath{[\mskip1.5mu \mathrm{1},\mathrm{2},\mathrm{3}\mskip1.5mu]} and tuples
 \ensuremath{(\mathrm{1},\text{\tt 'a'})} with brackets and parentheses.
 We could also express promoted lists and tuples in types like this with
 a single quote prefixed. For example:
 \ensuremath{\thinspace'[\!\;\Conid{Int},\Conid{Char}\;\!]}, \ensuremath{\thinspace'(\!\;\Conid{Int},\Conid{Char}\;\!)}.

\subsection{Type-level literals}

Now we have type-level lists and tuples to construct the dictionary.
For keys, \ensuremath{\Conid{String}} also has a type-level correspondence:
\ensuremath{\Conid{Symbol}}.

\begin{hscode}\SaveRestoreHook
\column{B}{@{}>{\hspre}l<{\hspost}@{}}%
\column{E}{@{}>{\hspre}l<{\hspost}@{}}%
\>[B]{}\mathbf{data}\;\Conid{Symbol}{}\<[E]%
\ColumnHook
\end{hscode}\resethooks

Symbol is defined without a value constructor, because it's intended to be used
 as a promoted kind.

\begin{hscode}\SaveRestoreHook
\column{B}{@{}>{\hspre}l<{\hspost}@{}}%
\column{E}{@{}>{\hspre}l<{\hspost}@{}}%
\>[B]{}\text{\tt \char34 this~is~a~type-level~string~literal\char34}\mathbin{::}\Conid{Symbol}{}\<[E]%
\ColumnHook
\end{hscode}\resethooks
%
% Nonetheless, it's still useful to have a term-level value that links with a
%  Symbol, when we want to retrieve type-level information at runtime (but not the
%  other way around!).

\subsection{Putting Everything Together}

With all of these ingredients ready, let's build some dictionaries!

\begin{hscode}\SaveRestoreHook
\column{B}{@{}>{\hspre}l<{\hspost}@{}}%
\column{E}{@{}>{\hspre}l<{\hspost}@{}}%
\>[B]{}\mathbf{type}\;\Conid{DictEmpty}\mathrel{=}\text{\tt '[]\;type~Dict0~=~'}{}\<[E]%
\\
\>[B]{}[\mskip1.5mu \text{\tt '(\char34 key\char34 ,~Bool)~]\;type~Dict1~=~'}{}\<[E]%
\\
\>[B]{}[\mskip1.5mu \text{\tt '(\char34 A\char34 ,~Int),~'}\;(\text{\tt \char34 B\char34},\text{\tt \char34 A\char34})\mskip1.5mu]{}\<[E]%
\ColumnHook
\end{hscode}\resethooks

These dictionaries are defined with \emph{type synonym}, since they are
 \emph{types}, not \emph{terms}. If we ask \text{GHCi} what is the
 kind of \ensuremath{\Conid{Dict1}}, we will get \ensuremath{\Conid{Dict1}\mathbin{::}[\mskip1.5mu (\Conid{Symbol},\mathbin{*})\mskip1.5mu]}

The kind \ensuremath{\mathbin{*}} (pronounced ``star'') stands for the set of all
 concrete type expressions, such as \ensuremath{\Conid{Int}},
 \ensuremath{\Conid{Char}} or even a symbol \ensuremath{\text{\tt \char34 symbol\char34}},
 while \ensuremath{\Conid{Symbol}} is restricted to all symbols only.

%% ODER: format ==         = "\mathrel{==}"
%% ODER: format /=         = "\neq "
%
%
\makeatletter
\@ifundefined{lhs2tex.lhs2tex.sty.read}%
  {\@namedef{lhs2tex.lhs2tex.sty.read}{}%
   \newcommand\SkipToFmtEnd{}%
   \newcommand\EndFmtInput{}%
   \long\def\SkipToFmtEnd#1\EndFmtInput{}%
  }\SkipToFmtEnd

\newcommand\ReadOnlyOnce[1]{\@ifundefined{#1}{\@namedef{#1}{}}\SkipToFmtEnd}
\usepackage{amstext}
\usepackage{amssymb}
\usepackage{stmaryrd}
\DeclareFontFamily{OT1}{cmtex}{}
\DeclareFontShape{OT1}{cmtex}{m}{n}
  {<5><6><7><8>cmtex8
   <9>cmtex9
   <10><10.95><12><14.4><17.28><20.74><24.88>cmtex10}{}
\DeclareFontShape{OT1}{cmtex}{m}{it}
  {<-> ssub * cmtt/m/it}{}
\newcommand{\texfamily}{\fontfamily{cmtex}\selectfont}
\DeclareFontShape{OT1}{cmtt}{bx}{n}
  {<5><6><7><8>cmtt8
   <9>cmbtt9
   <10><10.95><12><14.4><17.28><20.74><24.88>cmbtt10}{}
\DeclareFontShape{OT1}{cmtex}{bx}{n}
  {<-> ssub * cmtt/bx/n}{}
\newcommand{\tex}[1]{\text{\texfamily#1}}	% NEU

\newcommand{\Sp}{\hskip.33334em\relax}


\newcommand{\Conid}[1]{\mathit{#1}}
\newcommand{\Varid}[1]{\mathit{#1}}
\newcommand{\anonymous}{\kern0.06em \vbox{\hrule\@width.5em}}
\newcommand{\plus}{\mathbin{+\!\!\!+}}
\newcommand{\bind}{\mathbin{>\!\!\!>\mkern-6.7mu=}}
\newcommand{\rbind}{\mathbin{=\mkern-6.7mu<\!\!\!<}}% suggested by Neil Mitchell
\newcommand{\sequ}{\mathbin{>\!\!\!>}}
\renewcommand{\leq}{\leqslant}
\renewcommand{\geq}{\geqslant}
\usepackage{polytable}

%mathindent has to be defined
\@ifundefined{mathindent}%
  {\newdimen\mathindent\mathindent\leftmargini}%
  {}%

\def\resethooks{%
  \global\let\SaveRestoreHook\empty
  \global\let\ColumnHook\empty}
\newcommand*{\savecolumns}[1][default]%
  {\g@addto@macro\SaveRestoreHook{\savecolumns[#1]}}
\newcommand*{\restorecolumns}[1][default]%
  {\g@addto@macro\SaveRestoreHook{\restorecolumns[#1]}}
\newcommand*{\aligncolumn}[2]%
  {\g@addto@macro\ColumnHook{\column{#1}{#2}}}

\resethooks

\newcommand{\onelinecommentchars}{\quad-{}- }
\newcommand{\commentbeginchars}{\enskip\{-}
\newcommand{\commentendchars}{-\}\enskip}

\newcommand{\visiblecomments}{%
  \let\onelinecomment=\onelinecommentchars
  \let\commentbegin=\commentbeginchars
  \let\commentend=\commentendchars}

\newcommand{\invisiblecomments}{%
  \let\onelinecomment=\empty
  \let\commentbegin=\empty
  \let\commentend=\empty}

\visiblecomments

\newlength{\blanklineskip}
\setlength{\blanklineskip}{0.66084ex}

\newcommand{\hsindent}[1]{\quad}% default is fixed indentation
\let\hspre\empty
\let\hspost\empty
\newcommand{\NB}{\textbf{NB}}
\newcommand{\Todo}[1]{$\langle$\textbf{To do:}~#1$\rangle$}

\EndFmtInput
\makeatother
%
%
%
%
%
%
% This package provides two environments suitable to take the place
% of hscode, called "plainhscode" and "arrayhscode". 
%
% The plain environment surrounds each code block by vertical space,
% and it uses \abovedisplayskip and \belowdisplayskip to get spacing
% similar to formulas. Note that if these dimensions are changed,
% the spacing around displayed math formulas changes as well.
% All code is indented using \leftskip.
%
% Changed 19.08.2004 to reflect changes in colorcode. Should work with
% CodeGroup.sty.
%
\ReadOnlyOnce{polycode.fmt}%
\makeatletter

\newcommand{\hsnewpar}[1]%
  {{\parskip=0pt\parindent=0pt\par\vskip #1\noindent}}

% can be used, for instance, to redefine the code size, by setting the
% command to \small or something alike
\newcommand{\hscodestyle}{}

% The command \sethscode can be used to switch the code formatting
% behaviour by mapping the hscode environment in the subst directive
% to a new LaTeX environment.

\newcommand{\sethscode}[1]%
  {\expandafter\let\expandafter\hscode\csname #1\endcsname
   \expandafter\let\expandafter\endhscode\csname end#1\endcsname}

% "compatibility" mode restores the non-polycode.fmt layout.

\newenvironment{compathscode}%
  {\par\noindent
   \advance\leftskip\mathindent
   \hscodestyle
   \let\\=\@normalcr
   \let\hspre\(\let\hspost\)%
   \pboxed}%
  {\endpboxed\)%
   \par\noindent
   \ignorespacesafterend}

\newcommand{\compaths}{\sethscode{compathscode}}

% "plain" mode is the proposed default.
% It should now work with \centering.
% This required some changes. The old version
% is still available for reference as oldplainhscode.

\newenvironment{plainhscode}%
  {\hsnewpar\abovedisplayskip
   \advance\leftskip\mathindent
   \hscodestyle
   \let\hspre\(\let\hspost\)%
   \pboxed}%
  {\endpboxed%
   \hsnewpar\belowdisplayskip
   \ignorespacesafterend}

\newenvironment{oldplainhscode}%
  {\hsnewpar\abovedisplayskip
   \advance\leftskip\mathindent
   \hscodestyle
   \let\\=\@normalcr
   \(\pboxed}%
  {\endpboxed\)%
   \hsnewpar\belowdisplayskip
   \ignorespacesafterend}

% Here, we make plainhscode the default environment.

\newcommand{\plainhs}{\sethscode{plainhscode}}
\newcommand{\oldplainhs}{\sethscode{oldplainhscode}}
\plainhs

% The arrayhscode is like plain, but makes use of polytable's
% parray environment which disallows page breaks in code blocks.

\newenvironment{arrayhscode}%
  {\hsnewpar\abovedisplayskip
   \advance\leftskip\mathindent
   \hscodestyle
   \let\\=\@normalcr
   \(\parray}%
  {\endparray\)%
   \hsnewpar\belowdisplayskip
   \ignorespacesafterend}

\newcommand{\arrayhs}{\sethscode{arrayhscode}}

% The mathhscode environment also makes use of polytable's parray 
% environment. It is supposed to be used only inside math mode 
% (I used it to typeset the type rules in my thesis).

\newenvironment{mathhscode}%
  {\parray}{\endparray}

\newcommand{\mathhs}{\sethscode{mathhscode}}

% texths is similar to mathhs, but works in text mode.

\newenvironment{texthscode}%
  {\(\parray}{\endparray\)}

\newcommand{\texths}{\sethscode{texthscode}}

% The framed environment places code in a framed box.

\def\codeframewidth{\arrayrulewidth}
\RequirePackage{calc}

\newenvironment{framedhscode}%
  {\parskip=\abovedisplayskip\par\noindent
   \hscodestyle
   \arrayrulewidth=\codeframewidth
   \tabular{@{}|p{\linewidth-2\arraycolsep-2\arrayrulewidth-2pt}|@{}}%
   \hline\framedhslinecorrect\\{-1.5ex}%
   \let\endoflinesave=\\
   \let\\=\@normalcr
   \(\pboxed}%
  {\endpboxed\)%
   \framedhslinecorrect\endoflinesave{.5ex}\hline
   \endtabular
   \parskip=\belowdisplayskip\par\noindent
   \ignorespacesafterend}

\newcommand{\framedhslinecorrect}[2]%
  {#1[#2]}

\newcommand{\framedhs}{\sethscode{framedhscode}}

% The inlinehscode environment is an experimental environment
% that can be used to typeset displayed code inline.

\newenvironment{inlinehscode}%
  {\(\def\column##1##2{}%
   \let\>\undefined\let\<\undefined\let\\\undefined
   \newcommand\>[1][]{}\newcommand\<[1][]{}\newcommand\\[1][]{}%
   \def\fromto##1##2##3{##3}%
   \def\nextline{}}{\) }%

\newcommand{\inlinehs}{\sethscode{inlinehscode}}

% The joincode environment is a separate environment that
% can be used to surround and thereby connect multiple code
% blocks.

\newenvironment{joincode}%
  {\let\orighscode=\hscode
   \let\origendhscode=\endhscode
   \def\endhscode{\def\hscode{\endgroup\def\@currenvir{hscode}\\}\begingroup}
   %\let\SaveRestoreHook=\empty
   %\let\ColumnHook=\empty
   %\let\resethooks=\empty
   \orighscode\def\hscode{\endgroup\def\@currenvir{hscode}}}%
  {\origendhscode
   \global\let\hscode=\orighscode
   \global\let\endhscode=\origendhscode}%

\makeatother
\EndFmtInput
%

\ReadOnlyOnce{Formatting.fmt}%
\makeatletter

\let\Varid\mathit
\let\Conid\mathsf

\def\commentbegin{\quad\{\ }
\def\commentend{\}}

\newcommand{\ty}[1]{\Conid{#1}}
\newcommand{\con}[1]{\Conid{#1}}
\newcommand{\id}[1]{\Varid{#1}}
\newcommand{\cl}[1]{\Varid{#1}}
\newcommand{\opsym}[1]{\mathrel{#1}}

\newcommand\Keyword[1]{\textbf{\textsf{#1}}}
\newcommand\Hide{\mathbin{\downarrow}}
\newcommand\Reveal{\mathbin{\uparrow}}


%% Paper-specific keywords


\makeatother
\EndFmtInput

\section{Embedding \Hedis{} Commands}
\label{sec:embedding-commands}

Having the indexed monads and type-level dictionaries, in this section we
present our embedding of \Hedis{} commands into \Edis{}, while introducing
necessary concepts when they are used.

\subsection{Proxies and Singleton Types}

The \Hedis{} function \ensuremath{\Varid{del}\mathbin{::}[\mskip1.5mu \Conid{ByteString}\mskip1.5mu]\to \Conid{Either}\;\Conid{Reply}\;\Conid{Integer}} takes a list
of keys (encoded to \ensuremath{\Conid{ByteString}}) and removes the entries having those keys in
the database. For simplicity, we consider creating a \Edis{} counterpart
that takes only one key. A first attempt may lead to something like the
following:
\begin{hscode}\SaveRestoreHook
\column{B}{@{}>{\hspre}l<{\hspost}@{}}%
\column{E}{@{}>{\hspre}l<{\hspost}@{}}%
\>[B]{}\Varid{del}\mathbin{::}\Conid{String}\to \Conid{Edis}\;\Varid{xs}\;(\Conid{Del}\;\Varid{xs}~\mathbin{?})\;(\Conid{Either}\;\Conid{Reply}\;\Conid{Integer}){}\<[E]%
\\
\>[B]{}\Varid{del}\;\Varid{key}\mathrel{=}\Conid{Edis}\mathbin{\$}\Varid{\Conid{Hedis}.del}\;[\mskip1.5mu \Varid{encode}\;\Varid{key}\mskip1.5mu]~~,{}\<[E]%
\ColumnHook
\end{hscode}\resethooks
where the function \ensuremath{\Varid{encode}} converts \ensuremath{\Conid{String}} to \ensuremath{\Conid{ByteString}}. At term-level,
our \ensuremath{\Varid{del}} merely calls \ensuremath{\Varid{\Conid{Hedis}.del}}. At
type-level, if the status of the database before \ensuremath{\Varid{del}} is called is specified
by the dictionary \ensuremath{\Varid{xs}}, the status afterwards should be specified by
\ensuremath{\Conid{Del}\;\Varid{xs}~\mathbin{?}}. The question, however, is what to fill in place of the
question mark. It cannot be \ensuremath{\Conid{Del}\;\Varid{xs}\;\Varid{key}}, since \ensuremath{\Varid{key}} is a runtime value and
not a type. How do we smuggle a runtime value to type-level?

In a language with phase distinction like Haskell, it is certainly impossible
to pass the value of \ensuremath{\Varid{key}} to the type checker if it truly is a runtime value,
for example, a string read from the user. If the value of \ensuremath{\Varid{key}} can be
determined statically, however, {\em singleton types} can be used to represent a
type as a value, thus build a connection between the two realms.

A singleton type is a type that has only one term. When the term is built, it
carries a type that can be inspected by the type checker. The term can be think
of as a representative of its type at the realm of runtime values. For our
purpose, we will use the following type \ensuremath{\Conid{Proxy}}:
\begin{hscode}\SaveRestoreHook
\column{B}{@{}>{\hspre}l<{\hspost}@{}}%
\column{E}{@{}>{\hspre}l<{\hspost}@{}}%
\>[B]{}\mathbf{data}\;\Conid{Proxy}\;\Varid{t}\mathrel{=}\Conid{Proxy}~~.{}\<[E]%
\ColumnHook
\end{hscode}\resethooks
For every type \ensuremath{\Varid{t}}, \ensuremath{\Conid{Proxy}\;\Varid{t}} is a type that has only one term: \ensuremath{\Conid{Proxy}}.%
\footnote{While giving the same name to both the type and the term can be very
confusing, it is unfortunately a common practice in the Haskell community.}
To call \ensuremath{\Varid{del}}, instead of passing a key as a \ensuremath{\Conid{String}}, we give it a proxy with
a specified type:
\begin{hscode}\SaveRestoreHook
\column{B}{@{}>{\hspre}l<{\hspost}@{}}%
\column{E}{@{}>{\hspre}l<{\hspost}@{}}%
\>[B]{}\Varid{del}\;(\Conid{Proxy}\mathbin{::}\Conid{Proxy}\;\text{\tt \char34 A\char34})~~,{}\<[E]%
\ColumnHook
\end{hscode}\resethooks
where \ensuremath{\text{\tt \char34 A\char34}} is not a value, but a string lifted to a type (of kind \ensuremath{\Conid{Symbol}}).
Now that the type checker has access to the key, the type of \ensuremath{\Varid{del}} could be
something alone the line of \ensuremath{\Varid{del}\mathbin{::}\Conid{Proxy}\;\Varid{s}\to \Conid{Edis}\;\Varid{xs}\;(\Conid{Del}\;\Varid{xs}\;\Varid{s})\mathbin{...}}.

The next problem is that, \ensuremath{\Varid{del}}, at term level, gets only a value constructor
\ensuremath{\Conid{Proxy}} without further information, while it needs to pass a \ensuremath{\Conid{ByteString}} key
to \ensuremath{\Varid{\Conid{Hedis}.del}}. Every concrete string literal lifted to a type, for example
\ensuremath{\text{\tt \char34 A\char34}}, belongs to a type class \ensuremath{\Conid{KnownSymbol}}. For all type \ensuremath{\Varid{n}} in \ensuremath{\Conid{KnownSymbol}},
the function \ensuremath{\Varid{symbolVal}}:\begin{hscode}\SaveRestoreHook
\column{B}{@{}>{\hspre}l<{\hspost}@{}}%
\column{3}{@{}>{\hspre}l<{\hspost}@{}}%
\column{E}{@{}>{\hspre}l<{\hspost}@{}}%
\>[3]{}\Varid{symbolVal}\mathbin{::}\Conid{KnownSymbol}\;\Varid{n}\Rightarrow \Varid{proxy}\;\Varid{n}\to \Conid{String}~~,{}\<[E]%
\ColumnHook
\end{hscode}\resethooks
retrieves the string associated with a type-level literal that is known at
compile time. In summary, \ensuremath{\Varid{del}} can be implemented as:
\begin{hscode}\SaveRestoreHook
\column{B}{@{}>{\hspre}l<{\hspost}@{}}%
\column{6}{@{}>{\hspre}l<{\hspost}@{}}%
\column{45}{@{}>{\hspre}l<{\hspost}@{}}%
\column{E}{@{}>{\hspre}l<{\hspost}@{}}%
\>[B]{}\Varid{del}{}\<[6]%
\>[6]{}\mathbin{::}\Conid{KnownSymbol}\;\Varid{s}{}\<[E]%
\\
\>[6]{}\Rightarrow \Conid{Proxy}\;\Varid{s}\to \Conid{Edis}\;\Varid{xs}\;(\Conid{Del}\;\Varid{xs}\;\Varid{s})\;(\Conid{Either}\;\Conid{Reply}\;\Conid{Integer}){}\<[E]%
\\
\>[B]{}\Varid{del}\;\Varid{key}\mathrel{=}\Conid{Edis}\;(\Varid{\Conid{Hedis}.del}\;[\mskip1.5mu \Varid{encodeKey}\;\Varid{key}\mskip1.5mu]){}\<[45]%
\>[45]{}~~,{}\<[E]%
\ColumnHook
\end{hscode}\resethooks
where \ensuremath{\Varid{encodeKey}\mathrel{=}\Varid{encode}\mathbin{\cdot}\Varid{symbolVal}}.

A final note: the function \ensuremath{\Varid{encode}}, from the Haskell library {\sc cereal},
helps to convert certain datatype that are {\em serializable} into \ensuremath{\Conid{ByteString}}.
The function and its dual \ensuremath{\Varid{decode}} will be use more later.
\begin{hscode}\SaveRestoreHook
\column{B}{@{}>{\hspre}l<{\hspost}@{}}%
\column{9}{@{}>{\hspre}l<{\hspost}@{}}%
\column{E}{@{}>{\hspre}l<{\hspost}@{}}%
\>[B]{}\Varid{encode}{}\<[9]%
\>[9]{}\mathbin{::}\Conid{Serialize}\;\Varid{a}\Rightarrow \Varid{a}\to \Conid{ByteString}~~,{}\<[E]%
\\
\>[B]{}\Varid{decode}{}\<[9]%
\>[9]{}\mathbin{::}\Conid{Serialize}\;\Varid{a}\Rightarrow \Conid{ByteString}\to \Conid{Either}\;\Conid{String}\;\Varid{a}~~.{}\<[E]%
\ColumnHook
\end{hscode}\resethooks

\subsection{Storing Primitive Datatypes Other Than Strings}
\label{sec:polymorphic-redis}

As mentioned before, while \Redis{} provide a number of container types
including lists, sets, and hash, etc., the primitive type is string.
\Hedis{} programmers manually convert other types of data to strings before
saving them into the data store. In \Edis{}, we wish to save some of the
effort for the programmers, as well as keeping a careful record of the intended
types of the strings in the data store.

To keep track of intended types of strings in the data store, we define the
following types (that have no terms):
\begin{hscode}\SaveRestoreHook
\column{B}{@{}>{\hspre}l<{\hspost}@{}}%
\column{E}{@{}>{\hspre}l<{\hspost}@{}}%
\>[B]{}\mathbf{data}\;\Conid{StringOf}\;\Varid{x}~~,{}\<[E]%
\\
\>[B]{}\mathbf{data}\;\Conid{ListOf}\;\Varid{x}~~,{}\<[E]%
\\
\>[B]{}\mathbf{data}\;\Conid{SetOf}\;\Varid{x}~~...{}\<[E]%
\ColumnHook
\end{hscode}\resethooks
If an key is associated with, for example, \ensuremath{\Conid{StringOf}\;\Conid{Int}} in
our dictionary, we mean that its associated value in the data store was
serialized from an \ensuremath{\Conid{Int}} and should be used as an \ensuremath{\Conid{Int}}. Types
\ensuremath{\Conid{ListOf}\;\Varid{x}} and \ensuremath{\Conid{SetOf}\;\Varid{x}}, respectively, denotes that the value is a list
or a set of the given type.

While the \ensuremath{\Varid{set}} command in \Hedis{} always writes a string to the data store,
the corresponding \ensuremath{\Varid{set}} in \Redis{} applies to any serializable type (those
in the class \ensuremath{\Conid{Serialize}}), and performs the encoding for the user:
\begin{hscode}\SaveRestoreHook
\column{B}{@{}>{\hspre}l<{\hspost}@{}}%
\column{6}{@{}>{\hspre}l<{\hspost}@{}}%
\column{E}{@{}>{\hspre}l<{\hspost}@{}}%
\>[B]{}\Varid{set}{}\<[6]%
\>[6]{}\mathbin{::}(\Conid{KnownSymbol}\;\Varid{s},\Conid{Serialize}\;\Varid{x}){}\<[E]%
\\
\>[6]{}\Rightarrow \Conid{Proxy}\;\Varid{s}\to \Varid{x}\to \Conid{Edis}\;\Varid{xs}\;(\Conid{Set}\;\Varid{xs}\;\Varid{s}\;(\Conid{StringOf}\;\Varid{x}))\;(\Conid{Either}\;\Conid{Reply}\;\Conid{Status}){}\<[E]%
\\
\>[B]{}\Varid{set}\;\Varid{key}\;\Varid{val}\mathrel{=}\Conid{Edis}\mathbin{\$}\Varid{\Conid{Hedis}.set}\;(\Varid{encodeKey}\;\Varid{key})\;(\Varid{encode}\;\Varid{val})~~,{}\<[E]%
\ColumnHook
\end{hscode}\resethooks
For example, executing \ensuremath{\Varid{set}\;(\Conid{Proxy}\mathbin{::}\Conid{Proxy}\;\text{\tt \char34 A\char34})\;\Conid{True}} updates the dictionary
with an entry \ensuremath{\mbox{\textquotesingle}(\text{\tt \char34 A\char34},\Conid{StringOf}\;\Conid{Bool})}. If \ensuremath{\text{\tt \char34 A\char34}} is not in the dictionary,
this entry is added; otherwise the old type of \ensuremath{\text{\tt \char34 A\char34}} is updated to
\ensuremath{\Conid{StringOf}\;\Conid{Bool}}.

\Redis{} command \texttt{INCR} reads the string associated with the given key,
parse it as an integer, and increments it, before storing it back. The command
\texttt{INCRBYFLOAT} increments the floating number, associated to the given
key. They can be
\begin{hscode}\SaveRestoreHook
\column{B}{@{}>{\hspre}l<{\hspost}@{}}%
\column{7}{@{}>{\hspre}l<{\hspost}@{}}%
\column{14}{@{}>{\hspre}l<{\hspost}@{}}%
\column{E}{@{}>{\hspre}l<{\hspost}@{}}%
\>[B]{}\Varid{incr}{}\<[7]%
\>[7]{}\mathbin{::}(\Conid{KnownSymbol}\;\Varid{s},\Conid{Get}\;\Varid{xs}\;\Varid{s}\mathord{\sim}\Conid{Just}\;(\Conid{StringOf}\;\Conid{Integer})){}\<[E]%
\\
\>[7]{}\Rightarrow \Conid{Proxy}\;\Varid{s}\to \Conid{Edis}\;\Varid{xs}\;\Varid{xs}\;(\Conid{Either}\;\Conid{Reply}\;\Conid{Integer}){}\<[E]%
\\
\>[B]{}\Varid{incr}\;\Varid{key}\mathrel{=}\Conid{Edis}\mathbin{\$}\Varid{\Conid{Hedis}.incr}\;(\Varid{encodeKey}\;\Varid{key})~~,{}\<[E]%
\\[\blanklineskip]%
\>[B]{}\Varid{incrbyfloat}{}\<[14]%
\>[14]{}\mathbin{::}(\Conid{KnownSymbol}\;\Varid{s},\Conid{Get}\;\Varid{xs}\;\Varid{s}\mathord{\sim}\Conid{Just}\;(\Conid{StringOf}\;\Conid{Double})){}\<[E]%
\\
\>[14]{}\Rightarrow \Conid{Proxy}\;\Varid{s}\to \Conid{Double}\to \Conid{Edis}\;\Varid{xs}\;\Varid{xs}\;(\Conid{Either}\;\Conid{Reply}\;\Conid{Double}){}\<[E]%
\\
\>[B]{}\Varid{incrbyfloat}\;\Varid{key}\;\Varid{eps}\mathrel{=}\Conid{Edis}\mathbin{\$}\Varid{\Conid{Hedis}.incrbyfloat}\;(\Varid{encodeKey}\;\Varid{key})\;\Varid{eps}~~.{}\<[E]%
\ColumnHook
\end{hscode}\resethooks
Notice the use of (\ensuremath{\mathord{\sim}}), \emph{equality constraints}~\cite{typeeq}, to enforce
that the intended type of value associated with key \ensuremath{\Varid{s}} must respectively be
\ensuremath{\Conid{Integer}} and \ensuremath{\Conid{Double}}. The function \ensuremath{\Varid{incr}} is allowed to be called only
in a context where the type checker is able to reduce \ensuremath{\Conid{Get}\;\Varid{xs}\;\Varid{s}} to
\ensuremath{\Conid{Just}\;(\Conid{StringOf}\;\Conid{Integer})}. Similarly with \ensuremath{\Varid{incrbyfloat}}.

\todo{Some more interesting functions that uses IF and FromJust, etc.
What about \ensuremath{\Varid{lset}}?}

\subsection{Constraint Disjunctions}

Recall, from Section \ref{sec:introduction}, that commands \texttt{LPUSH key
val} and \texttt{LLEN key} returns normally either when \ensuremath{\Varid{key}} does not present
in the data store, or when \ensuremath{\Varid{key}} presents and is associated to a list.
What we wish to have in their constraint is thus a predicate equivalent to \ensuremath{\Conid{Get}\;\Varid{xs}\;\Varid{s}\doubleequals\Conid{Just}\;(\Conid{ListOf}\;\Varid{x})\mathrel{\vee}\neg \;(\Conid{Member}\;\Varid{xs}\;\Varid{s})}.

To impose a conjunctive constraint \ensuremath{\Conid{P}\mathrel{\wedge}\Conid{Q}}, one may simply put them both in the
type: \ensuremath{(\Conid{P},\Conid{Q})\Rightarrow \mathbin{...}}. Expressing disjunctive constraints is only slightly
harder, thanks to our type-level functions. Various operators for type-level
boolean and equality are defined in \text{Data.Type.Bool} and
\text{Data.Type.Equality}, like how we defined \ensuremath{\Conid{Or}} in Section
\ref{sec:type-fun}. We may thus write the predicate as:
\begin{hscode}\SaveRestoreHook
\column{B}{@{}>{\hspre}l<{\hspost}@{}}%
\column{E}{@{}>{\hspre}l<{\hspost}@{}}%
\>[B]{}\Conid{Get}\;\Varid{xs}\;\Varid{s}\doubleequals\Conid{Just}\;(\Conid{ListOf}\;\Varid{x})\mathbin{`\Conid{Or}`}\Conid{Not}\;(\Conid{Member}\;\Varid{xs}\;\Varid{s})~~.{}\<[E]%
\ColumnHook
\end{hscode}\resethooks
To avoid referring to \ensuremath{\Varid{x}}, which might not exist, we define an auxiliary predicate \ensuremath{\Conid{IsList}\mathbin{::}\Conid{Maybe}\mathbin{*}\to \Conid{Bool}}:
\begin{hscode}\SaveRestoreHook
\column{B}{@{}>{\hspre}l<{\hspost}@{}}%
\column{5}{@{}>{\hspre}l<{\hspost}@{}}%
\column{23}{@{}>{\hspre}l<{\hspost}@{}}%
\column{E}{@{}>{\hspre}l<{\hspost}@{}}%
\>[B]{}\mathbf{type}\;\Varid{family}\;\Conid{IsList}\;(\Varid{x}\mathbin{::}\mathbin{*})\mathbin{::}\Conid{Bool}\;\mathbf{where}{}\<[E]%
\\
\>[B]{}\hsindent{5}{}\<[5]%
\>[5]{}\Conid{IsList}\;(\Conid{ListOf}\;\Varid{n})\mathrel{=}\mbox{\textquotesingle}\Conid{True}{}\<[E]%
\\
\>[B]{}\hsindent{5}{}\<[5]%
\>[5]{}\Conid{IsList}\;\Varid{x}{}\<[23]%
\>[23]{}\mathrel{=}\mbox{\textquotesingle}\Conid{False}~~.{}\<[E]%
\ColumnHook
\end{hscode}\resethooks
As many other list-related commands also have this ``List or nothing'' semantics, we give the type constraint a name:
\begin{hscode}\SaveRestoreHook
\column{B}{@{}>{\hspre}l<{\hspost}@{}}%
\column{E}{@{}>{\hspre}l<{\hspost}@{}}%
\>[B]{}\Conid{ListOrNX}\;\Varid{xs}\;\Varid{s}\mathrel{=}(\Conid{IsList}\;(\Conid{Get}\;\Varid{xs}\;\Varid{s})\mathrel{\vee}\Conid{Not}\;(\Conid{Member}\;\Varid{xs}\;\Varid{s}))\mathord{\sim}\Conid{True}~~.{}\<[E]%
\ColumnHook
\end{hscode}\resethooks
\noindent The complete implementation of \text{LLEN} with
\ensuremath{\Conid{ListOrNX}} is therefore:
\begin{hscode}\SaveRestoreHook
\column{B}{@{}>{\hspre}l<{\hspost}@{}}%
\column{7}{@{}>{\hspre}l<{\hspost}@{}}%
\column{E}{@{}>{\hspre}l<{\hspost}@{}}%
\>[B]{}\Varid{lpush}\mathbin{::}(\Conid{KnownSymbol}\;\Varid{s},\Conid{Serialize}\;\Varid{x},\Conid{ListOrNX}\;\Varid{xs}\;\Varid{s}){}\<[E]%
\\
\>[B]{}\hsindent{7}{}\<[7]%
\>[7]{}\Rightarrow \Conid{Proxy}\;\Varid{s}\to \Varid{x}\to \Conid{Edis}\;\Varid{xs}\;(\Conid{Set}\;\Varid{xs}\;\Varid{s}\;(\Conid{ListOf}\;\Varid{x}))\;(\Conid{Either}\;\Conid{Reply}\;\Conid{Integer}){}\<[E]%
\\
\>[B]{}\Varid{lpush}\;\Varid{key}\;\Varid{val}\mathrel{=}\Conid{Edis}\mathbin{\$}\Varid{\Conid{Redis}.lpush}\;(\Varid{encodeKey}\;\Varid{key})\;[\mskip1.5mu \Varid{encode}\;\Varid{val}\mskip1.5mu]~~,{}\<[E]%
\\[\blanklineskip]%
\>[B]{}\Varid{llen}{}\<[7]%
\>[7]{}\mathbin{::}(\Conid{KnownSymbol}\;\Varid{s},\Conid{ListOrNX}\;\Varid{xs}\;\Varid{s}){}\<[E]%
\\
\>[7]{}\Rightarrow \Conid{Proxy}\;\Varid{s}\to \Conid{Edis}\;\Varid{xs}\;\Varid{xs}\;(\Conid{Either}\;\Conid{Reply}\;\Conid{Integer}){}\<[E]%
\\
\>[B]{}\Varid{llen}\;\Varid{key}\mathrel{=}\Conid{Edis}\mathbin{\$}\Varid{\Conid{Hedis}.llen}\;(\Varid{encodeKey}\;\Varid{key})~~.{}\<[E]%
\ColumnHook
\end{hscode}\resethooks



\todo{why cite \cite{singletons} here?}

% The type expression above has kind |Bool|, we could make it
%  a type constraint by asserting equality.
%
% With \emph{constraint kind}, a recent addition to GHC, type constraints now has
%  its own kind: |Constraint|. That means type constraints
%  are not restricted to the left side of a |=>| anymore,
%  they could appear in anywhere that accepts something of kind
%  |Constraint|, and any type that has kind
%  |Constraint| can also be used as a type constraint.
%  \footnote{See \url{https://downloads.haskell.org/~ghc/7.4.1/docs/html/users_guide/constraint-kind.html}.}

\subsection{Assertions}
\label{sec:assertions}

Finally, the creation/update behavior of \Redis{} functions is, in our opinion,
very error-prone. It might be preferable if we can explicit declare some new
keys, after ensure that they do not already exist (in our types), and renounce
them and forbid further access when we are sure that they shall not be referred
to anymore. This can be done below:
\begin{hscode}\SaveRestoreHook
\column{B}{@{}>{\hspre}l<{\hspost}@{}}%
\column{9}{@{}>{\hspre}l<{\hspost}@{}}%
\column{E}{@{}>{\hspre}l<{\hspost}@{}}%
\>[B]{}\Varid{declare}\mathbin{::}(\Conid{KnownSymbol}\;\Varid{s},\Conid{Member}\;\Varid{xs}\;\Varid{s}\mathord{\sim}\Conid{False}){}\<[E]%
\\
\>[B]{}\hsindent{9}{}\<[9]%
\>[9]{}\Rightarrow \Conid{Proxy}\;\Varid{s}\to \Conid{Proxy}\;\Varid{x}\to \Conid{Edis}\;\Varid{xs}\;(\Conid{Set}\;\Varid{xs}\;\Varid{s}\;\Varid{x})\;(){}\<[E]%
\\
\>[B]{}\Varid{declare}\;\Varid{s}\;\Varid{x}\mathrel{=}\Conid{Edis}\mathbin{\$}\Varid{return}\;()~~,{}\<[E]%
\\[\blanklineskip]%
\>[B]{}\Varid{renounce}\mathbin{::}(\Conid{KnownSymbol}\;\Varid{s},\Conid{Member}\;\Varid{xs}\;\Varid{s}\mathord{\sim}\Conid{True}){}\<[E]%
\\
\>[B]{}\hsindent{9}{}\<[9]%
\>[9]{}\Rightarrow \Conid{Proxy}\;\Varid{s}\to \Conid{Edis}\;\Varid{xs}\;(\Conid{Del}\;\Varid{xs}\;\Varid{s})\;(){}\<[E]%
\\
\>[B]{}\Varid{renounce}\;\Varid{s}\mathrel{=}\Conid{Edis}\mathbin{\$}\Varid{return}\;()~~.{}\<[E]%
\ColumnHook
\end{hscode}\resethooks
The command \ensuremath{\Varid{declare}\;\Varid{s}\;\Varid{x}} adds a new key \ensuremath{\Varid{s}} with type \ensuremath{\Varid{x}} to the dictionary,
if it does not already exist. Dually, \ensuremath{\Varid{renounce}} removes a key from the
dictionary. Even though it may still exist in the data store, it is not allowed
to be referred to. The command \ensuremath{\Varid{start}} initializes the dictionary to \ensuremath{\mbox{\textquotesingle}[~]}:
\begin{hscode}\SaveRestoreHook
\column{B}{@{}>{\hspre}l<{\hspost}@{}}%
\column{E}{@{}>{\hspre}l<{\hspost}@{}}%
\>[B]{}\Varid{start}\mathbin{::}\Conid{Edis}\;\mbox{\textquotesingle}[~]\;\mbox{\textquotesingle}[~]\;(){}\<[E]%
\\
\>[B]{}\Varid{start}\mathrel{=}\Conid{Edis}\mathbin{\$}\Varid{return}\;()~~.{}\<[E]%
\ColumnHook
\end{hscode}\resethooks

\subsection{A Larger Example}

The following program increases the value of \ensuremath{\text{\tt \char34 A\char34}} as an integer, push the result of the increment to list \ensuremath{\text{\tt \char34 L\char34}}, and then pops it out:
\begin{hscode}\SaveRestoreHook
\column{B}{@{}>{\hspre}l<{\hspost}@{}}%
\column{5}{@{}>{\hspre}l<{\hspost}@{}}%
\column{9}{@{}>{\hspre}l<{\hspost}@{}}%
\column{13}{@{}>{\hspre}l<{\hspost}@{}}%
\column{19}{@{}>{\hspre}l<{\hspost}@{}}%
\column{23}{@{}>{\hspre}l<{\hspost}@{}}%
\column{24}{@{}>{\hspre}l<{\hspost}@{}}%
\column{33}{@{}>{\hspre}l<{\hspost}@{}}%
\column{E}{@{}>{\hspre}l<{\hspost}@{}}%
\>[B]{}\Varid{main}\mathbin{::}\Conid{IO}\;(){}\<[E]%
\\
\>[B]{}\Varid{main}\mathrel{=}\mathbf{do}{}\<[E]%
\\
\>[B]{}\hsindent{5}{}\<[5]%
\>[5]{}\Varid{conn}{}\<[13]%
\>[13]{}\leftarrow \Varid{connect}\;\Varid{defaultConnectInfo}{}\<[E]%
\\
\>[B]{}\hsindent{5}{}\<[5]%
\>[5]{}\Varid{result}{}\<[13]%
\>[13]{}\leftarrow \Varid{runRedis}\;\Varid{conn}\mathbin{\$}\Varid{unEdis}\mathbin{\$}\Varid{start}{}\<[E]%
\\
\>[5]{}\hsindent{4}{}\<[9]%
\>[9]{}\mathbin{`\Varid{bind}`}\lambda \anonymous \to {}\<[24]%
\>[24]{}\Varid{declare}\;(\Conid{Proxy}\mathbin{::}\Conid{Proxy}\;\text{\tt \char34 A\char34})\;(\Conid{Proxy}\mathbin{::}\Conid{Proxy}\;\Conid{Integer}){}\<[E]%
\\
\>[5]{}\hsindent{4}{}\<[9]%
\>[9]{}\mathbin{`\Varid{bind}`}\lambda \anonymous \to {}\<[24]%
\>[24]{}\Varid{incr}\;(\Conid{Proxy}\mathbin{::}\Conid{Proxy}\;\text{\tt \char34 A\char34}){}\<[E]%
\\
\>[5]{}\hsindent{4}{}\<[9]%
\>[9]{}\mathbin{`\Varid{bind}`}\lambda \Varid{n}\to {}\<[23]%
\>[23]{}\mathbf{case}\;\Varid{n}\;\mathbf{of}{}\<[E]%
\\
\>[9]{}\hsindent{4}{}\<[13]%
\>[13]{}\Conid{Left}\;{}\<[19]%
\>[19]{}\Varid{err}{}\<[24]%
\>[24]{}\to \Varid{lpush}\;(\Conid{Proxy}\mathbin{::}\Conid{Proxy}\;\text{\tt \char34 L\char34})\;\mathrm{0}{}\<[E]%
\\
\>[9]{}\hsindent{4}{}\<[13]%
\>[13]{}\Conid{Right}\;\Varid{n}{}\<[24]%
\>[24]{}\to \Varid{lpush}\;(\Conid{Proxy}\mathbin{::}\Conid{Proxy}\;\text{\tt \char34 L\char34})\;\Varid{n}{}\<[E]%
\\
\>[5]{}\hsindent{4}{}\<[9]%
\>[9]{}\mathbin{`\Varid{bind}`}\lambda \anonymous \to {}\<[24]%
\>[24]{}\Varid{lpop}\;{}\<[33]%
\>[33]{}(\Conid{Proxy}\mathbin{::}\Conid{Proxy}\;\text{\tt \char34 L\char34}){}\<[E]%
\\
\>[B]{}\hsindent{5}{}\<[5]%
\>[5]{}\Varid{print}\;\Varid{result}{}\<[E]%
\ColumnHook
\end{hscode}\resethooks

The syntax is pretty heavy, like the old days when there's no
 \emph{do-notation}\cite{history}. But if we don't need any variable bindings
 between operations, we could compose these commands with a sequencing operator
 \ensuremath{(\mathbin{>>>})}.

\begin{hscode}\SaveRestoreHook
\column{B}{@{}>{\hspre}l<{\hspost}@{}}%
\column{E}{@{}>{\hspre}l<{\hspost}@{}}%
\>[B]{}(\mathbin{>>>})\mathbin{::}\Conid{IMonad}\;\Varid{m}\Rightarrow \Varid{m}\;\Varid{p}\;\Varid{q}\;\Varid{a}\to \Varid{m}\;\Varid{q}\;\Varid{r}\;\Varid{b}\to \Varid{m}\;\Varid{p}\;\Varid{r}\;\Varid{b}{}\<[E]%
\ColumnHook
\end{hscode}\resethooks
\begin{hscode}\SaveRestoreHook
\column{B}{@{}>{\hspre}l<{\hspost}@{}}%
\column{5}{@{}>{\hspre}l<{\hspost}@{}}%
\column{17}{@{}>{\hspre}l<{\hspost}@{}}%
\column{E}{@{}>{\hspre}l<{\hspost}@{}}%
\>[B]{}\Varid{program}\mathrel{=}\Varid{start}{}\<[E]%
\\
\>[B]{}\hsindent{5}{}\<[5]%
\>[5]{}\mathbin{>>>}\Varid{declare}\;(\Conid{Proxy}\mathbin{::}\Conid{Proxy}\;\text{\tt \char34 A\char34})\;(\Conid{Proxy}\mathbin{::}\Conid{Proxy}\;\Conid{Integer}){}\<[E]%
\\
\>[B]{}\hsindent{5}{}\<[5]%
\>[5]{}\mathbin{>>>}\Varid{incr}\;{}\<[17]%
\>[17]{}(\Conid{Proxy}\mathbin{::}\Conid{Proxy}\;\text{\tt \char34 A\char34}){}\<[E]%
\\
\>[B]{}\hsindent{5}{}\<[5]%
\>[5]{}\mathbin{>>>}\Varid{lpush}\;{}\<[17]%
\>[17]{}(\Conid{Proxy}\mathbin{::}\Conid{Proxy}\;\text{\tt \char34 L\char34})\;\mathrm{0}{}\<[E]%
\\
\>[B]{}\hsindent{5}{}\<[5]%
\>[5]{}\mathbin{>>>}\Varid{lpop}\;{}\<[17]%
\>[17]{}(\Conid{Proxy}\mathbin{::}\Conid{Proxy}\;\text{\tt \char34 L\char34}){}\<[E]%
\ColumnHook
\end{hscode}\resethooks

\input{sections/Discussions}

\section{Conclusions}
\label{sec:conclusions}

By exploiting various recent extensions and type-level programming techniques,
we have designed a domain-specific embedded language \Edis{} which, at
term-level, is simply a wrapper of \Hedis{}, but enforces typing disciplines that keeps track of available keys and their types and allows a program to be
constructed only if it does not throw a runtime type error. We believe that it
is a neat case study of application of type-level programming.



% With more and more extensions added to the language, Haskell is gradually
%  becoming a dependently-typed language,\footnote{Why not be dependently typed? \url{http://stackoverflow.com/questions/12961651/why-not-be-dependently-typed}}
%  but it's not that dreadful as many (including us) would have thought.


\bibliographystyle{splncs03}
\bibliography{cites}

\end{document}
