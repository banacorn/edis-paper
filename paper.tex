%
% LaTeX template for preparation of submissions to PLDI'16
%
% Requires temporary version of sigplanconf style file provided on
% PLDI'16 web site.
%
\documentclass[pldi]{sigplanconf-pldi16}
% \documentclass[pldi-cameraready]{sigplanconf-pldi16}

%
% the following standard packages may be helpful, but are not required
%
\usepackage{SIunits}            % typset units correctly
\usepackage{courier}            % standard fixed width font
\usepackage[scaled]{helvet}     % see www.ctan.org/get/macros/latex/required/psnfss/psnfss2e.pdf
\usepackage{url}                % format URLs
\usepackage{listings}           % format code
\usepackage{enumitem}           % adjust spacing in enums
\usepackage[colorlinks=true,allcolors=blue,breaklinks,draft=false]{hyperref}   % hyperlinks, including DOIs and URLs in bibliography
% known bug: http://tex.stackexchange.com/questions/1522/pdfendlink-ended-up-in-different-nesting-level-than-pdfstartlink
\newcommand{\doi}[1]{doi:~\href{http://dx.doi.org/#1}{\Hurl{#1}}}   % print a hyperlinked DOI

% code highlighting
\usepackage{minted}
\usemintedstyle{tango}

% no red boxes on parser error:
\AtBeginEnvironment{minted}{%
  \renewcommand{\fcolorbox}[4][]{#4}}


% quotes
\usepackage{dirtytalk}

% for symbols
\usepackage[T1]{fontenc}
\usepackage{lmodern}
\usepackage{upquote}

% use arbitrary fonts
\usepackage{fontspec}
\setmonofont{Inconsolata}


\begin{document}

\title{Preventing runtime errors of Redis at compile time}

%
% any author declaration will be ignored  when using 'pldi' option (for double blind review)
%

\authorinfo{Person 1 \and Person 2}
{\makebox{A Department} \\
\makebox{A University}  \\
\makebox{A Place, AS 12345}}
{\{person1,person2\}@cs.auniv.edu}

\maketitle

\begin{abstract}
Programmers often interact with database systems, by sending queries through
libraries or packages in some programming languages. But people make mistakes,
while some of the syntactic and semantic errors can be prevented by the language
at compile time, or caught by the package at runtime, most semantic errors are
just being ignored, causing problems at the database system.

In this paper, we demonstrates how to prevent those runtime errors at compile
time, allowing users to write more reliable database queries without runtime
overhead, by exploiting type-level programming techniques such as indexed
monad, type-level literals and closed type families in Haskell.

The database system and the package we are targeting
are \emph{Redis} and \emph{Hedis} respectively, and our implementation is
available as \emph{Edis} on \emph{Hackage}.

\paragraph{Categories and Subject Descriptors}
\paragraph{Keywords}
type-level programming; Haskell; database query language; Redis;
\end{abstract}

\section{Introduction}

\subsection{Redis}

Redis is an open source, in-memory data structure store, used as
database, cache and message broker. Each value is associated with a binary-safe
string key to identify and manipulate with. Redis supports many different
kind of values, such as strings, hashes, lists and sets.

To manipulate these values, Redis comes with a set of atomic operations,
called \emph{commands}.

For example, if we want to add a bunch of strings to 2 sets, and intersect them,
we could input the following sequence of commands into Redis's client
interactively.

\begin{minted}[xleftmargin=1em]{text}
redis> SADD some-set a b c
(integer) 3
redis> SADD another-set a b
(integer) 2
redis> SINTER some-set another-set
1) "a"
2) "b"
\end{minted}

Keys such as \mintinline{text}{some-set} and \mintinline{text}{another-set}
are created on site, with the command \mintinline{text}{SADD}, which returns
the size of set after completed.

\subsection{Hedis}
Redis can also be used in most programming languages, with 3rd party libraries
or packages that talk with Redis' TCP protocol. In Haskell, the most popular
package is Hedis.

The previous example in Redis's client would look something like this with
Hedis in Haskell.
\begin{minted}[xleftmargin=1em]{haskell}
program :: Redis (Either Reply [ByteString])
program = do
    sadd "some-set" ["a", "b"]
    sadd "another-set" ["a", "b", "c"]
    sinter ["some-set", "another-set"]
\end{minted}

Function \mintinline{haskell}{sadd} takes a key and a list of values as
arguments, and returns an \mintinline{haskell}{Integer}, wrapped in
\mintinline{haskell}{Either} to indicate possible failures in the context
\mintinline{haskell}{Redis}\footnotemark of command execution.

\begin{minted}[xleftmargin=1em]{haskell}
sadd :: ByteString      -- key
    -> [ByteString]     -- values
    -> Redis (Either Reply Integer)
\end{minted}

Keys and values are of type \mintinline{haskell}{ByteString}, since they are all
just binary strings in Redis. If a user wants to store values of
arbitrary types, he or she will have to encode and decode them as
 \mintinline{haskell}{ByteString}s.

\footnotetext{Hedis provides another kind of context, \mintinline{haskell}{RedisTx}, for \emph{transactions}, united with \mintinline{haskell}{Redis} under the class of \mintinline{haskell}{RedisCtx}. We use \mintinline{haskell}{Redis} only here for brevity. }

\subsection{Problems}

\say{All binary strings are equal, but some binary strings are more equal than others.}

Although everything in Redis is essentially a binary string, they are treated
differently. Redis supports many different kind of data structures, such as
strings, hashes, lists, etc. These values are of different data types, and some
commands only work on certain types of them, much like how language C treats
a piece of data.

\paragraph{Problem 1} Consider the following example. \mintinline{text}{SET} is
a command that only works on strings. The key \mintinline{text}{some-string} is
now related with a string. If we treat it as a set and add an element to it with
\mintinline{text}{SADD}, a runtime error arises.

\begin{minted}[xleftmargin=1em]{text}
redis> SET some-string foo
OK
redis> SADD some-string bar
(error) WRONGTYPE Operation against a key
 holding the wrong kind of value
\end{minted}

\paragraph{Problem 2} Even worse, not all strings are equal!
\mintinline{text}{INCR} parses the string value as an integer, increments it by
one, and store it back as a string. If a string value can't be parse as an
integer, another runtime error arises

\begin{minted}[xleftmargin=1em]{text}
redis> SET some-string foo
OK
redis> INCR some-string
(error) ERR value is not an integer or out
 of range
\end{minted}

\paragraph{In Hedis} The same goes for Hedis, since it's only a simple wrapping
 on top of Redis's protocol written in Haskell.

\begin{minted}[xleftmargin=1em]{haskell}
program :: Redis (Either Reply Integer)
program = do
    set "some-string" "foo"
    sadd "some-string" ["a"]
\end{minted}

It yields the same error as in Redis's client.

\begin{minted}[xleftmargin=1em]{haskell}
Left (Error "WRONGTYPE Operation against a
 key holding the wrong kind of value")
\end{minted}

\paragraph{The Cause} These problems arise from the absence of type checking,
with respects to \textbf{the type of a value that associates with a key}.

\subsection{Hedis as an embedded DSL}

Haskell makes it easy to build and use Domain Specific Languages (DSLs),
and Hedis can be regarded as one of them. What makes Hedis special is that,
it has variable bindings (between keys and values), but with very
little or no semantic checking, neither dynamically nor statically.

We began with making Hedis a dynamically checked embedded DSL, and implemented a
runtime type checker that keeps track of everything. But then we found that
things can be a lot easier, by leveraging the host language's type checker,
which also make it statically type-checked!

\subsection{Contributions}

To summarize our contributions:

\begin{itemize}[noitemsep]
\item We make Hedis statically type-checked, without runtime overhead.
\item We demonstrates how to model variable bindings of an embedded DSL with
language extensions like type-level literals and data kinds.
\item We provide (yet another) an example of encoding effects and
 constraints of an action in types, with indexed monad\cite{indexedmonad}
  and other language extensions such as closed
  type-families\cite{closedtypefamilies} and constraints
  kinds\cite{constraintskinds}.
\item Edis, a package we built for programmers. This package helps programmers
 to write more reliable Redis codes, and also makes Redis polymorphic by
 automatically converting back and forth from values of any types and boring
 ByteStrings.
\end{itemize}

\section{Type-level Dictionary}

To check the bindings between keys and values, we need a \emph{dictionary-like}
structure, and encode it as a \emph{type} somehow.

\subsection{Datatype promotion}
Normally, at the term level, we could express data type of the dictionary with
\emph{type synonym} like this.

\begin{minted}[xleftmargin=1em]{haskell}
type Key = String
type Dictionary = [(Key, TypeRep)]
\end{minted}

To encode this in the type level, everything has to be
\emph{promoted}\cite{promotion} one level up.
From terms to types, and from types to kinds.

Luckily, with recently added GHC extension \emph{data kinds}. With data kinds
we can define our own kinds just like we can with datatypes, and every suitable
datatype will be promoted to be a kind, and its value constructors promoted
to be type constructors, automatically.

\begin{minted}[xleftmargin=1em]{haskell}
data List a = Nil | Cons a (List a)
\end{minted}

Give rise to the following kinds and type constructors:\footnotemark\footnotemark

\footnotetext{To distinguish between types and promoted constructors that have
 ambiguous names, prefix promoted constructor with a single quote like
 \mintinline{text}{'Nil} and \mintinline{text}{'Cons}}
\footnotetext{All kinds have \emph{sort} BOX in Haskell\cite{sorts}}

\begin{minted}[xleftmargin=1em]{haskell}
List k :: BOX
Nil  :: List k
Cons :: k -> List k -> List k
\end{minted}

Haskell sugars lists \mintinline{haskell}{[1, 2, 3]} and
tuples \mintinline{text}{(1, 'a')} with brackets and parentheses.
Promoted lists and tuples are also sugared with single quote prefixed. For
example \mintinline{text}{'}\mintinline{haskell}{[Int, Char]},
\mintinline{text}{'}\mintinline{haskell}{(Int, Char)}.

\subsection{Type-level literals}

Now we have type-level lists and tuples to construct the dictionary.
For keys, \mintinline{haskell}{String} also has a type-level correspondence:
\mintinline{haskell}{Symbol}.

\begin{minted}[xleftmargin=1em]{haskell}
data Symbol
\end{minted}

Symbol is defined without a value constructor, because it's
intended to be used as a promoted kind.

\begin{minted}[xleftmargin=1em]{haskell}
"this is a type-level string literal" :: Symbol
\end{minted}

Nonetheless, it's still useful to have a term-level value that links with a
Symbol, when we want to retreive type-level informations at runtime (but not the
 other way around!).

\subsection{Putting everything together}

With all of these ingrediants ready, let's build some dictionaries!

\begin{minted}[xleftmargin=1em]{haskell}
type DictEmpty = '[]
type Dict0 = '[ '("key", Bool) ]
type Dict1 = '[ '("A", Int), '("B", "A") ]
\end{minted}

They are defined with \emph{type synonym}, since they are \emph{types}
, not \emph{terms}. If we ask \mintinline{text}{GHCi} what is the kind of
\mintinline{haskell}{Dict1}, we get \mintinline{haskell}{Dict1 :: [ (Symbol, *) ]}

The star \mintinline{haskell}{*} is a kind, it stands for any kind of types,
such as \mintinline{haskell}{Int}, \mintinline{haskell}{Char} or even a symbol
\mintinline{haskell}{"symbol"}, while \mintinline{haskell}{Symbol} stands only
for all symbols.

\section{Indexed Monad}

Redis commands are \emph{actions}.
We could capture the effects caused by an action, by expressing the states it
affects, before and after. That is, the \emph{preconditions} and
\emph{postconditions} of an action. In such way, we could also impose
constraints on the preconditions.

\emph{Indexed monads} (or \emph{monadish},
\emph{parameterised monad})\cite{indexedmonad}
can be used\cite{typefun}\cite{staticresources} to model such preconditions and
postconditions in types. An indexed monad is a type constructor that takes three
arguments: an intitial state, a final state, and the type of a value that an
action computes, which can be read like a Hoare triple\cite{kleisli}.

\begin{minted}[xleftmargin=1em]{haskell}
class IMonad m where
    unit :: a -> m p p a
    bind :: m p q a -> (a -> m q r b) -> m p r b
\end{minted}

Class \mintinline{haskell}{IMonad} comes with two operations:
\mintinline{haskell}{unit} for identities and \mintinline{haskell}{bind} for
compositions, as in Monad.

We define a new datatype \mintinline{haskell}{Edis}, which is basically just
\mintinline{haskell}{Redis} indexed with more informations in types. And we make
 it an instance of \mintinline{haskell}{IMonad}.

\begin{minted}[xleftmargin=1em]{haskell}
newtype Edis p q a = Edis { unEdis :: Redis a }

instance IMonad Edis where
    unit = Edis . return
    bind m f = Edis (unEdis m >>= unEdis . f )
\end{minted}

The first and second argument of type \mintinline{haskell}{Edis} is where the
dictionaries going to stay.

\subsection{The first attempt}

In Redis, \mintinline{text}{PING} does nothing but replies with
\mintinline{text}{PONG} if the connection is alive. In Hedis,
\mintinline{text}{PING} has type:

\begin{minted}[xleftmargin=1em]{haskell}
ping :: Redis (Either Reply Status)
\end{minted}

Now we have \mintinline{haskell}{Edis}, let's make our own version of
\mintinline{haskell}{ping}\footnotemark

\begin{minted}[xleftmargin=1em]{haskell}
ping :: Edis xs xs (Either Reply Status)
ping = Edis Hedis.ping
\end{minted}

Dicionary \mintinline{haskell}{xs} in the type remain unaffected after the
action, because \mintinline{haskell}{ping} does not affect any key-value
bindings. To encode other commands that alters key-value bindings, we need
 type-level functions to annotate those effects on the dictionary.

\section{Bookkeeping with type families}
\subsection{Closed type families}

Type families have a wide variaty of applications. They can appear inside type
classes\cite{tfclass}\cite{tfsynonym}, or at toplevel. Toplevel type families
can be used to compute over types, they come in open\cite{tfopen} or closed
forms\cite{tfclosed}. We are using \emph{closed type families} since we need
none of the extensibility provided by open type families, and we may need to
overlap instances, like what we do in defining function cases.

%  are
% permitted as well.
%
%
%  open type synonym families, and
% closed type synonym families
% One of which is to compute types, we are using closed type families, since all
% we want is to use it as a function on the type level, no extensibility needed,
% everything is defined in one-piece


\footnotetext{\mintinline{haskell}{ping} from Hedis is qualified with
\mintinline{haskell}{Hedis} to prevent function name clashing}
%
% \mintinline{haskell}{IMonad} \mintinline{haskell}{IMonad} takes three arguments: an intitial state \mintinline{haskell}{p}, a final
%  state, and computes a value of type a



\subsection{Proxies and Kind polymorphism}

Since Haskell has a phase distinction\cite{phasedistinction} between

To link a symbol with a term-level value

What we need is a \emph{singleton}\cite{singletons}, a type
that has only one instance at the term level.

With \emph{kind polymorphism}\cite{promotion}, we have parametric polymorphism
at the kind level.

\begin{minted}[xleftmargin=1em]{haskell}
Proxy :: Proxy "a"
\end{minted}

\subsection{Closed type families}

\subsection{Conclusions and Related Work}


% references
\bibliographystyle{abbrvnat}
\bibliography{cites}

% \subsection{Model bindings with Type-level Dictionaries}
% \subsection{Bookkeeping with Type families}
% \subsection{Capture effects with Index Monad}
% \paragraph{Type-level literals}


%
% Values have types, but keys don't. In other words, if we treat keys like
% variables, then Redis behaves likes a dynamically and strongly typed language.


% We can only reuse this key as a \emph{set} after it is removed with \mintinline{text}{DEL}
% explicitly.
%
% \begin{minted}{text}
% redis> DEL some-string
% (integer) 1
% redis> SADD some-string bar
% (integer) 1
% \end{minted}


\end{document}
