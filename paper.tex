%
% LaTeX template for preparation of submissions to PLDI'16
%
% Requires temporary version of sigplanconf style file provided on
% PLDI'16 web site.
%
\documentclass[pldi]{sigplanconf-pldi16}
% \documentclass[pldi-cameraready]{sigplanconf-pldi16}

%
% the following standard packages may be helpful, but are not required
%
\usepackage{SIunits}            % typset units correctly
\usepackage{courier}            % standard fixed width font
\usepackage[scaled]{helvet}     % see www.ctan.org/get/macros/latex/required/psnfss/psnfss2e.pdf
\usepackage{url}                % format URLs
\usepackage{listings}           % format code
\usepackage{enumitem}           % adjust spacing in enums
\usepackage[colorlinks=true,allcolors=blue,breaklinks,draft=false]{hyperref}   % hyperlinks, including DOIs and URLs in bibliography
% known bug: http://tex.stackexchange.com/questions/1522/pdfendlink-ended-up-in-different-nesting-level-than-pdfstartlink
\newcommand{\doi}[1]{doi:~\href{http://dx.doi.org/#1}{\Hurl{#1}}}   % print a hyperlinked DOI

% code highlighting
\usepackage{minted}
\usemintedstyle{tango}

% no red boxes on parser error:
\AtBeginEnvironment{minted}{%
  \renewcommand{\fcolorbox}[4][]{#4}}


% quotes
\usepackage{dirtytalk}

% for symbols
\usepackage[T1]{fontenc}
\usepackage{lmodern}
\usepackage{upquote}

% use arbitrary fonts
\usepackage{fontspec}
\setmonofont{Inconsolata}

% for clickable footnotes
\usepackage{hyperref}


\newcommand{\todo}[2]{{\bf [#1]}: #2}

\begin{document}

\title{Preventing Runtime Errors of Redis at Compile Time}

%
% any author declaration will be ignored  when using 'pldi' option (for double blind review)
%

\authorinfo{Ting-Yan Lai}
{\makebox{Institute of Computer Science and Engineering} \\
\makebox{National Chiao Tung University, Taiwan}}
{banacorn.cs03g@nctu.edu.tw}

\authorinfo{Tyng-Ruey Chuang \and Shin-Cheng Mu}
{\makebox{Institute of Information Science} \\
\makebox{Academia Sinica, Taiwan}}
{\{trc,scm\}@iis.sinica.edu.tw}

\maketitle

\begin{abstract}
Programmers often interact with database systems by sending queries through libraries or packages in some programming languages. People, however, make mistakes.
While some of the syntactic and semantic errors can be prevented by the
 language at compile time, or caught by the package at runtime, most semantic
 errors are just being ignored, causing problems at the database system.

In this paper, we demonstrate how to prevent those runtime errors at compile
 time, thus allowing users to write more reliable database queries without
 runtime overhead, by exploiting type-level programming techniques such as
 indexed monad, type-level literals and closed type families in Haskell.

The database system and the package we are targeting are \emph{Redis} and
 \emph{Hedis} respectively, and our implementation is available as \emph{Popcorn}\footnotemark
 on \emph{Hackage}.

\footnotetext{Popcorn, a temporary codename for the purposes of double-blind reviewing}

\category{D.3.3}{Programming Languages}{Language Constructs and Features}
\category{F.3.1}{Specifying and Verifying and Reasoning about Programs}{Specification techniques}
\category{H.2.3}{Database Management}{Languages}
\keywords{Haskell; type families; type-level literals; indexed monad; Redis; database query language}
\end{abstract}

\section{Introduction}

\subsection{Redis}

Redis\footnote{\url{https://redis.io}} is an open source, in-memory data structure store, often used as
database, cache and message broker. A Redis datatype can be think of as a set of key-value pairs, where each value is associated with a binary-safe string key to identify and manipulate with. Redis supports many different kind of values, such as strings, hashes, lists, and sets, etc, and provides a collection of of atomic \emph{commands} to manipulate these values.

For an example, consider the following sequence of commands, entered through the interactive interface of Redis. The keys
\mintinline{text}{some-set} abd \mintinline{text}{another-set} are
both associated to a set. The two call to command \mintinline{text}{SADD} respectively adds three and two values to the two sets, before \mintinline{text}{SINTER} takes their intersection:
\begin{minted}[xleftmargin=1em]{text}
redis> SADD some-set a b c
(integer) 3
redis> SADD another-set a b
(integer) 2
redis> SINTER some-set another-set
1) "a"
2) "b"
\end{minted}

Note that the keys \mintinline{text}{some-set} and \mintinline{text}{another-set}, if not existing before the call to \mintinline{text}{SADD}, are created on site. The call to
\mintinline{text}{SADD} returns the size of the set after
completion of the command.

\subsection{Hedis}

Many third party library exist to allow general purpose programmings
to access Redis databases through its TCP protocol. The most popular
such library for Haskell is Hedis\footnote{\url{https://hackage.haskell.org/package/hedis}}.

The following program is how the previous example looks like in
Hedis:

\begin{minted}[xleftmargin=1em]{haskell}
program :: Redis (Either Reply [ByteString])
program = do
    sadd "some-set" ["a", "b"]
    sadd "another-set" ["a", "b", "c"]
    sinter ["some-set", "another-set"]
\end{minted}

The function \mintinline{haskell}{sadd} takes a key and a list of values as
 arguments, and returns an \mintinline{haskell}{Integer} on success, or
 returns a \mintinline{haskell}{Reply}, a low-level representation of replies
 from the Redis server, in case of failures. All wrapped in
 \mintinline{haskell}{Redis}\footnotemark, the context of command execution.

\begin{minted}[xleftmargin=1em]{haskell}
sadd :: ByteString      -- key
    -> [ByteString]     -- values
    -> Redis (Either Reply Integer)
\end{minted}

Note that keys and values, being nothing but binary strings in Redis,
are represented using Haskell \mintinline{haskell}{ByteString}.
Values of other types must be encoded as \mintinline{haskell}{ByteString}s before being written to the database,
and decoded after being read back.

\footnotetext{Hedis provides another kind of context, \mintinline{haskell}{RedisTx}, for \emph{transactions}, united with \mintinline{haskell}{Redis} under the class of \mintinline{haskell}{RedisCtx}. For brevity, we demonstrate only \mintinline{haskell}{Redis} in this paper. }

\section{Motivation}

All binary strings are equal, but some binary strings are more equal than others.

Although everything in Redis is essentially a binary string, these strings are treated differently. Redis supports many different
kind of data structures, such as strings, hashes, lists, etc. While
they are all encoded as binary strings before being written to the databse, most commands, much like how the C language treats a piece of data, only works with data of certain types.

\paragraph{Problem 1} The command \mintinline{text}{SET}, by
its definition, associates a key to a string. In the following
example, the key \mintinline{text}{some-string} is associated
to string \mintinline{text}{foo}. Subsequent calls to
\mintinline{text}{SADD} causes runtime errors, since the value
of \mintinline{text}{some-string} is not a set, but a string.

\begin{minted}[xleftmargin=1em]{text}
redis> SET some-string foo
OK
redis> SADD some-string bar
(error) WRONGTYPE Operation against a key
 holding the wrong kind of value
\end{minted}

\paragraph{Problem 2} Even worse, not all strings are equal!
The call \mintinline{text}{INCR some-string} parses the string
associated with key \mintinline{text}{some-string}
to an integer, increments it by one, and store it back as a string.
If the string can not be parse as an integer, a runtime error
is raised.

\begin{minted}[xleftmargin=1em]{text}
redis> SET some-string foo
OK
redis> INCR some-string
(error) ERR value is not an integer or out
 of range
\end{minted}

\paragraph{In Hedis} Hedis, being only a simple wrapper on top
of the TCP protocol of Redis, inherits all the problems mentioned
above. The following program yields the same error as that in
the Redis client.

\begin{minted}[xleftmargin=1em]{haskell}
program :: Redis (Either Reply Integer)
program = do
    set "some-string" "foo"
    sadd "some-string" ["a"]
\end{minted}
\begin{minted}[xleftmargin=1em]{haskell}
Left (Error "WRONGTYPE Operation against a
 key holding the wrong kind of value")
\end{minted}

\paragraph{The Cause} Every key is associated with a value, and every value has
 it's own type. But most commands in Redis only work with a certain type of
 value. When a command is used on a wrong type of key, a runtime error occurs.
 The problems illustrated above arise from the absence of type checking, with
 respects to \textbf{the type of a value that associates with a key}.
 These problems could have been avoided, if we could know the type every key
 associates with in advance, and prevent programs with invalid commands from
 executing.

\subsection{Hedis as an embedded DSL}

Haskell makes it easy to build and use Domain Specific Languages (DSLs),
 and Hedis can be regarded as one of them. What makes Hedis peculiar is that,
 it has \emph{variable bindings} (between keys and values), but with very
 little or no semantic checking, neither dynamically nor statically.

We began with making Hedis a dynamically typechecked embedded DSL, and implemented a
 runtime type checker that keeps track of types of all the variable. But then we
 found that things can be a lot easier, by leveraging the host language's type
 checker. We encode variable bindings with \emph{type-level lists} and
 \emph{strings}, and control the effects on the bindings with
 \emph{indexed monad}. In contrast to the former approach, we \textbf{embedded}
 our type checker into Haskell's type system, without having to build a
 \textbf{standalone} one on the term level.

\subsection{Contributions}

To summarize our contributions:

\begin{itemize}[noitemsep]
\item We make Hedis statically type-checked, without runtime overhead.
\item We demonstrates how to model variable bindings of an embedded DSL with
 language extensions like type-level literals and data kinds.
\item We provide (yet another) an example of encoding effects and constraints of
 an action in types, with indexed monad\cite{indexedmonad} and other language
 extensions such as closed type-families\cite{closedtypefamilies} and
 constraints kinds\cite{constraintskinds}.
\item Popcorn, a package we built for programmers. This package helps programmers
 to write more reliable Redis programs, and also makes Redis polymorphic by
 automatically converting back and forth from values of arbitrary types and
 boring ByteStrings.
\end{itemize}

\section{Type-level dictionaries}

To check the bindings between keys and values, we need a \emph{dictionary-like}
 structure, and encode it as a \emph{type} somehow.

\subsection{Datatype promotion}
Normally, at the term level, we could express the datatype of dictionary with
\emph{type synonym} like this.\footnotemark

\begin{minted}[xleftmargin=1em]{haskell}
type Key = String
type Dictionary = [(Key, TypeRep)]
\end{minted}

\footnotetext{\mintinline{haskell}{TypeRep} supports term-level representations
 of datatypes, available in \mintinline{haskell}{Data.Typeable}}

To encode this in the type level, everything has to be
 \emph{promoted}\cite{promotion} one level up.
 From terms to types, and from types to kinds.

Luckily, with recently added GHC extension \emph{data kinds}, suitable
 datatype will be automatically promoted to be a kind, and its value
 constructors to be type constructors. The following type {\mintinline{haskell}{List}

\begin{minted}[xleftmargin=1em]{haskell}
data List a = Nil | Cons a (List a)
\end{minted}

Give rise to the following kinds and type constructors:\footnote{To distinguish
 between types and promoted constructors that have
 ambiguous names, prefix promoted constructor with a single quote like
 \mintinline{text}{'Nil} and \mintinline{text}{'Cons}}
\footnote{All kinds have \emph{sort} BOX in Haskell\cite{sorts}}


\begin{minted}[xleftmargin=1em]{haskell}
List k :: BOX
Nil  :: List k
Cons :: k -> List k -> List k
\end{minted}

Haskell sugars lists \mintinline{haskell}{[1, 2, 3]} and tuples
 \mintinline{text}{(1, 'a')} with brackets and parentheses.
 We could also express promoted lists and tuples in types like this with
 a single quote prefixed. For example:
 \mintinline{text}{'}\mintinline{haskell}{[Int, Char]},
 \mintinline{text}{'}\mintinline{haskell}{(Int, Char)}.

\subsection{Type-level literals}

Now we have type-level lists and tuples to construct the dictionary.
For keys, \mintinline{haskell}{String} also has a type-level correspondence:
\mintinline{haskell}{Symbol}.

\begin{minted}[xleftmargin=1em]{haskell}
data Symbol
\end{minted}

Symbol is defined without a value constructor, because it's intended to be used
 as a promoted kind.

\begin{minted}[xleftmargin=1em]{haskell}
"this is a type-level string literal" :: Symbol
\end{minted}
%
% Nonetheless, it's still useful to have a term-level value that links with a
%  Symbol, when we want to retrieve type-level information at runtime (but not the
%  other way around!).

\subsection{Putting everything together}

With all of these ingredients ready, let's build some dictionaries!

\begin{minted}[xleftmargin=1em]{haskell}
type DictEmpty = '[]
type Dict0 = '[ '("key", Bool) ]
type Dict1 = '[ '("A", Int), '("B", "A") ]
\end{minted}

These dictionaries are defined with \emph{type synonym}, since they are
 \emph{types}, not \emph{terms}. If we ask \mintinline{text}{GHCi} what is the
 kind of \mintinline{haskell}{Dict1}, we will get
 \mintinline{haskell}{Dict1 :: [ (Symbol, *) ]}

The kind \mintinline{haskell}{*} (pronounced "star") stands for the set of all
 concrete type expressions, such as \mintinline{haskell}{Int},
 \mintinline{haskell}{Char} or even a symbol \mintinline{haskell}{"symbol"},
 while \mintinline{haskell}{Symbol} is restricted to all symbols only.

\section{Indexed monads}

Redis commands are \emph{actions}.
 We could capture the effects caused by an action, by expressing the states it
 affects, before and after. That is, the \emph{preconditions} and
 \emph{postconditions} of an action. In such way, we could also impose
 constraints on the preconditions.

\emph{Indexed monads} (or \emph{monadish},
 \emph{parameterised monad})\cite{indexedmonad}
 can be used\cite{typefun, staticresources} to model such preconditions and
 postconditions in types. An indexed monad is a type constructor that takes
 three arguments: an initial state, a final state, and the type of a value that
 an action computes, which can be read like a Hoare triple\cite{kleisli}.

\begin{minted}[xleftmargin=1em]{haskell}
class IMonad m where
    unit :: a -> m p p a
    bind :: m p q a -> (a -> m q r b) -> m p r b
\end{minted}

Class \mintinline{haskell}{IMonad} comes with two operations:
 \mintinline{haskell}{unit} for identities and \mintinline{haskell}{bind} for
 compositions, as in monads.

We define a new datatype \mintinline{haskell}{Popcorn}, which is basically just
 the context \mintinline{haskell}{Redis} indexed with more information in types.
 We make it an instance of \mintinline{haskell}{IMonad}.

\begin{minted}[xleftmargin=1em]{haskell}
newtype Popcorn p q a =
    Popcorn { unPopcorn :: Redis a }

instance IMonad Popcorn where
    unit = Popcorn . return
    bind m f =
        Popcorn (unPopcorn m >>= unPopcorn . f )
\end{minted}

The first and second argument of type \mintinline{haskell}{Popcorn} is where the
dictionaries going to be.

To execute a \mintinline{haskell}{Popcorn} program, simply apply it to
 \mintinline{haskell}{unPopcorn} to get an ordinary Hedis program back,
 with type information erased.

\subsection{\mintinline{text}{PING}: the first attempt}

In Redis, \mintinline{text}{PING} does nothing but replies with
 \mintinline{text}{PONG} if the connection is alive. In Hedis,
 \mintinline{haskell}{ping} has type:

\begin{minted}[xleftmargin=1em]{haskell}
ping :: Redis (Either Reply Status)
\end{minted}

Now with \mintinline{haskell}{Popcorn}, we could make our own version of
\mintinline{haskell}{ping}\footnotemark

\footnotetext{\mintinline{haskell}{ping} from Hedis is qualified with
\mintinline{haskell}{Hedis} to prevent function name clashing in our code.}

\begin{minted}[xleftmargin=1em]{haskell}
ping :: Popcorn xs xs (Either Reply Status)
ping = Popcorn Hedis.ping
\end{minted}

The dictionary \mintinline{haskell}{xs} in the type remains unaffected after the
 action, because \mintinline{haskell}{ping} does not affect any key-type
 bindings. To encode other commands that modifies key-type bindings, we need
 type-level functions to annotate those effects on the dictionary.

\section{Type-level functions}
\subsection{Closed type families}

Type families have a wide variety of applications. They can appear inside type
 classes\cite{tfclass, tfsynonym}, or at toplevel. Toplevel type families
 can be used to compute over types, they come in two forms: open\cite{tfopen}
 and closed \cite{tfclosed}.

We choose \emph{closed type families}, because it allows overlapping instances,
 and we need none of the extensibility provided by open type families.
For example, consider both term-level and type-level \mintinline{haskell}{&&}:

\begin{minted}[xleftmargin=1em,fontsize=\footnotesize]{haskell}
(&&) :: Bool -> Bool -> Bool
True && True = True
a    && b    = False

type family And (a :: Bool) (b :: Bool) :: Bool where
    And True True = True
    And a    b    = False
\end{minted}

The first instance of \mintinline{haskell}{And} could be subsumed under a more
 general instance, \mintinline{haskell}{And a b}.
But the closedness allows these instances to be resolved in order, just like
 how cases are resolved in term-level functions. Also notice that how much
 \mintinline{haskell}{And} resembles to it's term-level sibling.

\subsection{Functions on type-level dictionaries}

With closed type families, we could define functions on the type level.
 Let's begin with dictionary lookup.

\begin{minted}[xleftmargin=1em]{haskell}
type family Get
    (xs :: [(Symbol, *)])   -- dictionary
    (s :: Symbol)           -- key
    :: * where              -- type

    Get (’(s, x) ’: xs) s = x
    Get (’(t, x) ’: xs) s = Get xs s
\end{minted}

Another benefit of closed type families is that type-level equality can be
expressed by unifying type variables with the same name.
\mintinline{haskell}{Get} takes two type arguments, a dictionary and a symbol.
If the key we are looking for unifies with the symbol of an entry, then
 \mintinline{haskell}{Get} returns the corresponding type, else it keeps
 searching down the rest of the dictionary.

\mintinline{haskell}{Get }\mintinline{text}{ '}\mintinline{haskell}{[ }
\mintinline{text}{'}\mintinline{haskell}{("A", Int) ] "A"} evaluates to
\mintinline{haskell}{Int}.

But \mintinline{haskell}{Get }
\mintinline{text}{'}\mintinline{haskell}{[ }
\mintinline{text}{'}\mintinline{haskell}{("A", Int) ] "B"} would get stuck.
That's because \mintinline{haskell}{Get} is a partial function on types,
 and these types are computed at compile-time. It wouldn't make
 much sense for a type checker to crash and throw a "Non-exhaustive" error or
 be non-terminating.

We could make \mintinline{haskell}{Get} total, as we would at the term level,
 with \mintinline{haskell}{Maybe}.

\begin{minted}[xleftmargin=1em]{haskell}
type family Get
    (xs :: [(Symbol, *)])   -- dictionary
    (s :: Symbol)           -- key
    :: Maybe * where        -- type

    Get ’[]             s = Nothing
    Get (’(s, x) ’: xs) s = Just x
    Get (’(t, x) ’: xs) s = Get xs s
\end{minted}
%
Other dictionary-related functions are defined in a similar fashion.

\begin{minted}[xleftmargin=1em]{haskell}
-- inserts or updates an entry
type family Set
    (xs :: [(Symbol, *)])   -- old dictionary
    (s :: Symbol)           -- key
    (x :: *)                -- type
    :: [(Symbol, *)] where  -- new dictionary

    Set ’[]             s x = ’[ ’(s, x) ]
    Set (’(s, y) ’: xs) s x = (’(s, x) ’: xs)
    Set (’(t, y) ’: xs) s x =
        ’(t, y) ’: (Set xs s x)

-- removes an entry
type family Del
    (xs :: [(Symbol, *)])   -- old dictionary
    (s :: Symbol)           -- key
    :: [(Symbol, *)] where  -- new dictionary

    Del ’[] s             = ’[]
    Del (’(s, y) ’: xs) s = xs
    Del (’(t, y) ’: xs) s = ’(t, y) ’: (Del xs s)

-- membership
type family Member
    (xs :: [(Symbol, *)])   -- dictionary
    (s :: Symbol)           -- key
    :: Bool where           -- exists?

    Member ’[]             s = False
    Member (’(s, x) ’: xs) s = True
    Member (’(t, x) ’: xs) s = Member xs s
\end{minted}

\subsection{Proxies and singleton types}

Now we could annotate the effects of a command in types. \mintinline{text}{DEL}
 removes a key from the current database, regardless of its type.

\begin{minted}[xleftmargin=1em,fontsize=\footnotesize]{haskell}
del :: KnownSymbol s
    => Proxy s
    -> Popcorn xs (Del xs s) (Either Reply Integer)
del key = Popcorn $ Hedis.del (encodeKey key)
\end{minted}

\mintinline{haskell}{KnownSymbol} is a class that gives the string associated
 with a concrete type-level symbol, which can be retrieved with
 \mintinline{haskell}{symbolVal}.\footnotemark
 Where \mintinline{haskell}{encodeKey} converts \mintinline{haskell}{Proxy s} to
 \mintinline{haskell}{ByteString}.
\footnotetext{They are defined in \mintinline{haskell}{GHC.TypeLits}.}

\begin{minted}[xleftmargin=1em,fontsize=\footnotesize]{haskell}
encodeKey :: KnownSymbol s => Proxy s -> ByteString
encodeKey = encode . symbolVal
\end{minted}

Since Haskell has a \emph{phase distinction, phasedistinction}, types are
 erased before runtime. It's impossible to obtain information directly from
 types, we can only do this indirectly, with
 \emph{singleton types, singletons}.

A singleton type is a type that has only one instance, and the instance can be
 think of as the representative of the type at the realm of runtime values.

\mintinline{haskell}{Proxy}, as its name would suggest, can be used as
 singletons. It's a phantom type that could be indexed with any type.

\begin{minted}[xleftmargin=1em]{haskell}
data Proxy t = Proxy
\end{minted}

In the type of \mintinline{haskell}{del}, the type variable
 \mintinline{haskell}{s} is a \mintinline{haskell}{Symbol} that is decided by
 the argument of type \mintinline{haskell}{Proxy s}.
 To use \mintinline{haskell}{del}, we would have to apply it with a clumsy
 term-level proxy like this:

\begin{minted}[xleftmargin=1em]{haskell}
del (Proxy :: Proxy "A")
\end{minted}

\section{Making Redis polymorphic}

Redis supports many different datatypes, these datatypes as can be viewed as
 \emph{containers} of strings. For example, lists (of strings),
 sets (of strings), and strings themselves.

\subsection{Denoting containers}
Most Redis commands only work with a certain type of these containers. To
 annotate what container a key is associated with, we introduce these types for
 the universe of containers.

\begin{minted}[xleftmargin=1em]{haskell}
data Strings
data Lists
data Sets
...
\end{minted}

\mintinline{text}{SET} stores a string, regardless the datatype the key was
 associated with. Now we could implement \mintinline{text}{SET} like this:

\begin{minted}[xleftmargin=1em,fontsize=\footnotesize]{haskell}
set :: KnownSymbol s
    => Proxy s
    -> ByteString       -- data to store
    -> Popcorn xs (Set xs s Strings) (Either Reply Status)
set key val = Popcorn $ Hedis.set (encodeKey key) val
\end{minted}

After \mintinline{text}{SET}, the key will be associated with
 \mintinline{haskell}{Strings} in the dictionary, indicating that it's a string.

\subsection{Automatic data serialization}
But in the real world, raw binary strings are hardly useful, people would
 usually serialize their data into strings before storing them, and deserialize
 them back when in need.

Instead of letting users writing these boilerplates, we can do these
 serializations/deserializations for them. With the help from
 \mintinline{text}{cereal}, a binary serialization library.
 \mintinline{text}{cereal} comes with these two functions:

\begin{minted}[xleftmargin=1em]{haskell}
encode :: Serialize a
       => a -> ByteString Source

decode :: Serialize a
       => ByteString -> Either String a
\end{minted}

Which would do all the works for us, as long as the datatype it's handling is
 an instance of class \mintinline{haskell}{Serialize}.\footnotemark

\footnotetext{The methods of \mintinline{haskell}{Serialize} will have default
 generic implementations for all datatypes with some language extensions
 enabled, no sweat!}

\subsection{Extending container types}
We rename container types and extend it with an extra type argument,
 to indicate what kind of encoded value it's holding.

\begin{minted}[xleftmargin=1em]{haskell}
data StringOf x
data ListOf x
data SetOf x
...
\end{minted}

\mintinline{haskell}{set} reimplemented with extended container types:

\begin{minted}[xleftmargin=1em,fontsize=\footnotesize]{haskell}
set :: (KnownSymbol s, Serialize x)
    => Proxy s
    -> x       -- can be anything, as long as it's serializable
    -> Popcorn xs (Set xs s (StringOf x)) (Either Reply Status)
set key val = Popcorn $ Hedis.set (encodeKey key) (encode val)
\end{minted}

For example, if we execute \mintinline{haskell}{set (Proxy :: Proxy "A") True},
 a new entry \mintinline{text}{'}\mintinline{haskell}{("A", StringOf Bool)} will
 be inserted to the dictionary.

\subsection{Handling \mintinline{text}{INCR}}

Commands such as \mintinline{text}{INCR} and \mintinline{text}{INCRBYFLOAT}, are
 not only container-specific, they also have some requirements on what types of
 value they could operate with.

We could handle this by mapping Redis's strings of integers and floats to
 Haskell's \mintinline{text}{Integer} and \mintinline{text}{Double}.

\begin{minted}[xleftmargin=1em]{haskell}
incr :: (KnownSymbol s
      , Get xs s ~ Just (StringOf Integer))
     => Proxy s
     -> Popcorn xs xs (Either Reply Integer)
incr key = Popcorn $ Hedis.incr (encodeKey key)

incrbyfloat :: (KnownSymbol s
     , Get xs s ~ Just (StringOf Double))
    => Proxy s
    -> Double
    -> Popcorn xs xs (Either Reply Double)
incrbyfloat key n =
    Popcorn $ Hedis.incrbyfloat (encodeKey key) n
\end{minted}

\section{Imposing constraints}

To rule out programs with undesired properties, certain constraints must be
 imposed, on what arguments they can take, or what preconditions they must hold.

Consider the following example: \mintinline{text}{LLEN} returns the length of
 the list associated with a key, else raises a type error.

\begin{minted}[xleftmargin=1em]{text}
redis> LPUSH some-list bar
(integer) 1
redis> LLEN some-list
(integer) 1
redis> SET some-string foo
OK
redis> LLEN some-string
(error) WRONGTYPE Operation against a key
holding the wrong kind of value
\end{minted}

Such constraint could be expressed in types with
\emph{equality constraints}\cite{typeeq}.

\begin{minted}[xleftmargin=1em]{haskell}
llen :: (KnownSymbol s
      , Get xs s ~ Just (ListOf x))
     => Proxy s
     -> Popcorn xs xs (Either Reply Integer)
llen key =
     Popcorn $ Hedis.llen (encodeKey key)
\end{minted}

Where \mintinline{haskell}{(~)} denotes that \mintinline{haskell}{Get xs s}
and \mintinline{haskell}{Just (ListOf x)} needs to be the same.

The semantics of \mintinline{text}{LLEN} defined above is actually not
complete. \mintinline{text}{LLEN} also accepts keys that do not exist, and
 replies with \mintinline{text}{0}.

\begin{minted}[xleftmargin=1em]{text}
redis> LLEN nonexistent
(integer) 0
\end{minted}

In other words, we require that the key to be associated with a list,
 \textbf{unless} it doesn't exist at all.

\subsection{Expressing constraint disjunctions}

Unfortunately, expressing disjunctions in constraints is much more difficult
 than expressing conjunctions, since the latter could be easily done by placing
 constraints in a tuple (at the left side of \mintinline{haskell}{=>}).

There are at least three ways to express type-level constraints
\cite{singletons}. Luckily we could express constraint disjunctions with type
 families in a modular way.

The semantics we want could be expressed informally like this:

\mintinline{haskell}{Get xs s} $\equiv$ \mintinline{haskell}{Just (ListOf x)}
$\vee$ $\neg$ \mintinline{haskell}{(Member xs s)}

We could achieve this simply by translating the semantics we want to the
 domain of Boolean, with type-level boolean functions such as
\mintinline{haskell}{(&&)},
\mintinline{haskell}{(||)}, \mintinline{haskell}{Not},
\mintinline{haskell}{(==)}, etc.\footnotemark To avoid

\footnotetext{Available in \mintinline{text}{Data.Type.Bool} and
 \mintinline{text}{Data.Type.Equality}}

\begin{minted}[xleftmargin=1em,fontsize=\footnotesize]{haskell}
Get xs s == Just (ListOf x) || Not (Member xs s)
\end{minted}

To avoid addressing the type of value (as it may not exist at all), we defined
 an auxiliary predicate \mintinline{haskell}{IsList :: Maybe * -> Bool} to
 replace the former part.

\begin{minted}[xleftmargin=1em,fontsize=\footnotesize]{haskell}
IsList (Get xs s) || Not (Member xs s)
\end{minted}

The type expression above has kind \mintinline{haskell}{Bool}, we could make it
 a type constraint by asserting equality.

\begin{minted}[xleftmargin=1em,fontsize=\footnotesize]{haskell}
(IsList (Get xs s) || Not (Member xs s)) ~ True
\end{minted}

With \emph{constraint kind}, a recent addition to GHC, type constraints now has
 its own kind: \mintinline{haskell}{Constraint}. That means type constraints
 are not restricted to the left side of a \mintinline{haskell}{=>} anymore,
 they could appear in anywhere that accepts something of kind
 \mintinline{haskell}{Constraint}, and any type that has kind
 \mintinline{haskell}{Constraint} can also be used as a type constraint.
 \footnote{See \url{https://downloads.haskell.org/~ghc/7.4.1/docs/html/users_guide/constraint-kind.html}.}

As many other list-related commands also have this "List or nothing" semantics,
 we could abstract the lengthy type constraint above and give it an alias with
 type synonym.

\begin{minted}[xleftmargin=1em,fontsize=\footnotesize]{haskell}
ListOrNX xs s =
    (IsList (Get xs s) || Not (Member xs s)) ~ True
\end{minted}

The complete implementation of \mintinline{text}{LLEN} with
\mintinline{haskell}{ListOrNX} would become:

\begin{minted}[xleftmargin=1em]{haskell}
llen :: (KnownSymbol s, ListOrNX xs s)
        => Proxy s
        -> Popcorn xs xs (Either Reply Integer)
llen key = Popcorn $ Hedis.llen (encodeKey key)
\end{minted}

\section{Assertions}

Users may need to make assertions about the status of some key-type bindings in
 a Redis program. Specifically, when declaring or renouncing the existence of
 a key and the type of its associating value. We provide these functions,
 which do nothing but fiddle with types.

\begin{minted}[xleftmargin=1em]{haskell}
declare :: (KnownSymbol s, Member xs s ~ False)
        => Proxy s
        -> Proxy x  -- type of value
        -> Popcorn xs (Set xs s x) ()
declare s x = Popcorn $ return ()

renounce :: (KnownSymbol s, Member xs s ~ True)
        => Proxy s -> Popcorn xs (Del xs s) ()
renounce s = Popcorn $ return ()

-- to be used at the beginning of a program
start :: Popcorn ’[] ’[] ()
start = Popcorn $ return ()
\end{minted}

\subsection{A complete example}

The following program increases the value of "A" as an integer, push the result
 of the increment to list "L", and then pops it out.

\begin{minted}[xleftmargin=1em,fontsize=\footnotesize]{haskell}
main :: IO ()
main = do
    conn <- connect defaultConnectInfo
    result <- runRedis conn $ unPopcorn $ start
        `bind` \_ ->  declare
                        (Proxy :: Proxy "A")
                        (Proxy :: Proxy Integer)
        `bind` \_ ->  incr     (Proxy :: Proxy "A")
        `bind` \n ->  case n of
            Left  err -> lpush (Proxy :: Proxy "L") 0
            Right n   -> lpush (Proxy :: Proxy "L") n
        `bind` \_ ->  lpop     (Proxy :: Proxy "L")
    print result
\end{minted}

The syntax is pretty heavy, like the old days when there's no
 \emph{do-notation}\cite{history}. But if we don't need any variable bindings
 between operations, we could compose these commands with a sequencing operator
 \mintinline{haskell}{(>>>)}.

\begin{minted}[xleftmargin=1em,fontsize=\footnotesize]{haskell}
(>>>) :: IMonad m => m p q a -> m q r b -> m p r b
\end{minted}
\begin{minted}[xleftmargin=1em]{haskell}
program = start
    >>> declare
            (Proxy :: Proxy "A")
            (Proxy :: Proxy Integer)
    >>> incr    (Proxy :: Proxy "A")
    >>> lpush   (Proxy :: Proxy "L") 0
    >>> lpop    (Proxy :: Proxy "L")
\end{minted}

\section{Discussions}
\paragraph{Syntax}
No one could ignore the glaring shortcoming of the syntax, which occurs mainly
 in two places: \emph{symbol singletons} and \emph{indexed monad}. We are hoping
 that these issues could be resolved with future syntactic extensions.

\paragraph{Returns only determined datatypes}

All other data structures in Redis also follow the semantics similar to
 "List or nothing". Take the case of \mintinline{text}{GET}, which also shares a
 "String or nothing" semantics that should be typed:

\begin{minted}[xleftmargin=1em,fontsize=\footnotesize]{haskell}
get :: (KnownSymbol s, Serialize x, StringOrNX xs s)
    => Proxy s -> Popcorn xs xs (Either Reply (Maybe x))
\end{minted}

Since the key may not exist, we don't know what \mintinline{haskell}{x} would
be. We could left \mintinline{haskell}{x} ambiguous, and let it be decided by
 the caller. But users will then be forced to spell out the complete
type signature of everything, including the dictionaries, only to specify
the desired resulting type.

Instead of allowing the key to be non-existent, we require that the key must
 exist and it's associating type to be determined at compile time. So our
 version of \mintinline{haskell}{get} has a stricter semantics:

\begin{minted}[xleftmargin=1em,fontsize=\footnotesize]{haskell}
get :: (KnownSymbol s, Serialize x
     , Just (StringOf x) ~ Get xs s)
    => Proxy s -> Popcorn xs xs (Either Reply (Maybe x))
\end{minted}

\paragraph{Commands with multiple inputs or outputs}

Some command may take a variable number of arguments as inputs, and returns more
 than one value as outputs. To illustrate this, consider
 \mintinline{haskell}{sinter} in Hedis:

\begin{minted}[xleftmargin=1em,fontsize=\footnotesize]{haskell}
Hedis.sinter :: [ByteString]                      -- keys
             -> Redis (Either Reply [ByteString]) -- values
\end{minted}

In Hedis such command could easily be expressed with lists of
 \mintinline{haskell}{ByteString}s. But in Popcorn, things escalate quickly, as
 the keys and values will have to be expressed with \emph{heterogeneous
 lists}\cite{hetero}, which would be pratically infeasible, considering the cost,
 if not impossible.

Most importantly, the keys will all have to be constrained by
 \mintinline{haskell}{KnownSymbol}, which enforces these type literals to be
 concrete and known at compile time.
 It's still unclear whether this is possible.

So instead, we are offering commands that only has a single input and output.

\begin{minted}[xleftmargin=1em,fontsize=\footnotesize]{haskell}
sinter :: ByteString                      -- single key
       -> Redis (Either Reply ByteString) -- single value
\end{minted}

\textbf{Not all Redis programs can be typechecked} (even if they
 might turn out to be type safe). We opted for type safety rather than
 expressiveness.

\paragraph{Redis Transactions}

Redis has \emph{transactions}, another context for executing commands.
Redis transactions are atomic in the sense that, all commands in a transaction
 will executed sequentially, and no other requests issued by other clients will
 be served \textbf{in the middle}.\footnotemark In contrast, we cannot make such
 a guarantee in the ordinary context, which may destroy the assertions we made
 in types.

At this point of writing, transactions are not supported in our implementation.
 We are planning to add it in the future, and we are expecting that there
 wouldn't be much difficulty, since we've implemented a runtime type checker
 specifically targeting Redis transactions once, before we moved on to the
 types.

\section{Related Work}

While Redis can be viewed as a non-relational database system (although Redis
 seems reluctant to admit this in recent years), there also are similar goals on
 relational database systems, trying to achieve a safer database interface by
 making them statically checked.
 \emph{HaskellDB}\cite{haskelldb, haskelldbimproved} expresses quires
 with relational algebra-like combinators, wrapped in phantom types. Or as
 examples, at first to demonstrate the power of dependent types in
 Agda\cite{pi}, and then with singletons in Haskell\cite{singletons}.

\section{Conclusions}

We have demonstrated how a DSL with variable bindings could be embedded in types.
With more and more extensions added to the language, Haskell is gradually
 becoming a dependently-typed language,\footnote{Why not be dependently typed? \url{http://stackoverflow.com/questions/12961651/why-not-be-dependently-typed}}
 but it's not that dreadful as many (including us) would have thought.

\paragraph{Acknowledgements}
\footnote{Hidden for the purposes of double-blind reviewing}

% references
\bibliographystyle{abbrvnat}
\bibliography{cites}

\end{document}
