%
% LaTeX template for preparation of submissions to PLDI'16
%
% Requires temporary version of sigplanconf style file provided on
% PLDI'16 web site.
%
\documentclass[pldi]{sigplanconf-pldi16}
% \documentclass[pldi-cameraready]{sigplanconf-pldi16}

%
% the following standard packages may be helpful, but are not required
%
\usepackage{SIunits}            % typset units correctly
\usepackage{courier}            % standard fixed width font
\usepackage[scaled]{helvet}     % see www.ctan.org/get/macros/latex/required/psnfss/psnfss2e.pdf
\usepackage{url}                % format URLs
\usepackage{listings}           % format code
\usepackage{enumitem}           % adjust spacing in enums
\usepackage[colorlinks=true,allcolors=blue,breaklinks,draft=false]{hyperref}   % hyperlinks, including DOIs and URLs in bibliography
% known bug: http://tex.stackexchange.com/questions/1522/pdfendlink-ended-up-in-different-nesting-level-than-pdfstartlink
\newcommand{\doi}[1]{doi:~\href{http://dx.doi.org/#1}{\Hurl{#1}}}   % print a hyperlinked DOI

% code highlighting
\usepackage{minted}
\usemintedstyle{tango}

% no red boxes on parser error:
\AtBeginEnvironment{minted}{%
  \renewcommand{\fcolorbox}[4][]{#4}}


% quotes
\usepackage{dirtytalk}

% for symbols
\usepackage[T1]{fontenc}
\usepackage{lmodern}
\usepackage{upquote}

% use arbitrary fonts
\usepackage{fontspec}
\setmonofont{Inconsolata}

% for clickable footnotes
\usepackage{hyperref}


\begin{document}

\title{Preventing runtime errors of Redis at compile time}

%
% any author declaration will be ignored  when using 'pldi' option (for double blind review)
%

\authorinfo{Person 1 \and Person 2}
{\makebox{A Department} \\
\makebox{A University}  \\
\makebox{A Place, AS 12345}}
{\{person1,person2\}@cs.auniv.edu}

\maketitle

\begin{abstract}
Programmers often interact with database systems, by sending queries through
 libraries or packages in some programming languages. But people make mistakes,
 while some of the syntactic and semantic errors can be prevented by the
 language at compile time, or caught by the package at runtime, most semantic
 errors are just being ignored, causing problems at the database system.

In this paper, we demonstrate how to prevent those runtime errors at compile
 time, thus allowing users to write more reliable database queries without
 runtime overhead, by exploiting type-level programming techniques such as
 indexed monad, type-level literals and closed type families in Haskell.

The database system and the package we are targeting are \emph{Redis} and
 \emph{Hedis} respectively, and our implementation is available as \emph{Edis}
 on \emph{Hackage}.

\paragraph{Categories and Subject Descriptors}
\paragraph{Keywords}
Haskell; type-level programming; database query language; Redis;
\end{abstract}

\section{Introduction}

\subsection{Redis}

Redis is an open source, in-memory data structure store, used as
 database, cache and message broker. Each value is associated with a binary-safe
 string key to identify and manipulate with. Redis supports many different
 kind of values, such as strings, hashes, lists, sets, etc.

To manipulate these values, Redis comes with a set of atomic operations, called
 \emph{commands}.

For example, if we want to add a bunch of strings to 2 sets, and intersect them,
 we could input the following sequence of commands into Redis's client
 interactively.

\begin{minted}[xleftmargin=1em]{text}
redis> SADD some-set a b c
(integer) 3
redis> SADD another-set a b
(integer) 2
redis> SINTER some-set another-set
1) "a"
2) "b"
\end{minted}

Keys such as \mintinline{text}{some-set} and \mintinline{text}{another-set}
 are created on site, with the command \mintinline{text}{SADD}, which returns
 the size of set after completed.

\subsection{Hedis}
Redis can also be used in most programming languages, with 3rd party libraries
 or packages that talk with Redis' TCP protocol. In Haskell, the most popular
 package is Hedis.

The previous example in Redis's client would look something like this with
 Hedis in Haskell.

\begin{minted}[xleftmargin=1em]{haskell}
program :: Redis (Either Reply [ByteString])
program = do
    sadd "some-set" ["a", "b"]
    sadd "another-set" ["a", "b", "c"]
    sinter ["some-set", "another-set"]
\end{minted}

\mintinline{haskell}{sadd} takes a key and a list of values as arguments, and
 returns an \mintinline{haskell}{Integer}, wrapped in
 \mintinline{haskell}{Either} to indicate possible failures in the context
 \mintinline{haskell}{Redis}\footnotemark of command execution.

\begin{minted}[xleftmargin=1em]{haskell}
sadd :: ByteString      -- key
    -> [ByteString]     -- values
    -> Redis (Either Reply Integer)
\end{minted}

Keys and values are of type \mintinline{haskell}{ByteString}, since they are all
 just binary strings in Redis. If a user wants to store values of
 arbitrary types, he or she will have to encode and decode them as
 \mintinline{haskell}{ByteString}s.

\footnotetext{Hedis provides another kind of context, \mintinline{haskell}{RedisTx}, for \emph{transactions}, united with \mintinline{haskell}{Redis} under the class of \mintinline{haskell}{RedisCtx}. We demonstrate \mintinline{haskell}{Redis} only for brevity. }

\subsection{Problems}

\say{All binary strings are equal, but some binary strings are more equal than others.}

Although everything in Redis is essentially a binary string, these strings are
 treated differently. Redis supports many different kind of data structures,
 such as strings, hashes, lists, etc. These values have different datatypes,
 and most commands only work with certain types of them, much like how C
 language treats a piece of data.

\paragraph{Problem 1} Consider the following case. \mintinline{text}{SET} is
 a command that only works with strings, it associates the key
 \mintinline{text}{some-string} with a value of string. If we treat the like a
 set and add an element to it with \mintinline{text}{SADD}, a runtime error
 arises.

\begin{minted}[xleftmargin=1em]{text}
redis> SET some-string foo
OK
redis> SADD some-string bar
(error) WRONGTYPE Operation against a key
 holding the wrong kind of value
\end{minted}

\paragraph{Problem 2} Even worse, not all strings are equal!
 \mintinline{text}{INCR} parses the string value as an integer, increments it by
 one, and store it back as a string. If a string value can't be parse as an
 integer, another runtime error arises

\begin{minted}[xleftmargin=1em]{text}
redis> SET some-string foo
OK
redis> INCR some-string
(error) ERR value is not an integer or out
 of range
\end{minted}

\paragraph{In Hedis} The same goes for Hedis, since it's only a simple wrapping
 on top of Redis's protocol written in Haskell.

\begin{minted}[xleftmargin=1em]{haskell}
program :: Redis (Either Reply Integer)
program = do
    set "some-string" "foo"
    sadd "some-string" ["a"]
\end{minted}

It yields the same error as in Redis's client.

\begin{minted}[xleftmargin=1em]{haskell}
Left (Error "WRONGTYPE Operation against a
 key holding the wrong kind of value")
\end{minted}

\paragraph{The Cause} These problems arise from the absence of type checking,
 with respects to \textbf{the type of a value that associates with a key}.

\subsection{Hedis as an embedded DSL}

Haskell makes it easy to build and use Domain Specific Languages (DSLs),
 and Hedis can be regarded as one of them. What makes Hedis special is that,
 it has \emph{variable bindings} (between keys and values), but with very
 little or no semantic checking, neither dynamically nor statically.

We began with making Hedis a dynamically checked embedded DSL, and implemented a
 runtime type checker that keeps track of everything. But then we found that
 things can be a lot easier, by leveraging the host language's type checker,
 which also make it statically type-checked!

\subsection{Contributions}

To summarize our contributions:

\begin{itemize}[noitemsep]
\item We make Hedis statically type-checked, without runtime overhead.
\item We demonstrates how to model variable bindings of an embedded DSL with
 language extensions like type-level literals and data kinds.
\item We provide (yet another) an example of encoding effects and constraints of
 an action in types, with indexed monad\cite{indexedmonad} and other language
 extensions such as closed type-families\cite{closedtypefamilies} and
 constraints kinds\cite{constraintskinds}.
\item Edis, a package we built for programmers. This package helps programmers
 to write more reliable Redis programs, and also makes Redis polymorphic by
 automatically converting back and forth from values of arbitrary types and
 boring ByteStrings.
\end{itemize}

\section{Type-level dictionaries}

To check the bindings between keys and values, we need a \emph{dictionary-like}
 structure, and encode it as a \emph{type} somehow.

\subsection{Datatype promotion}
Normally, at the term level, we could express the datatype of dictionary with
\emph{type synonym} like this.\footnotemark

\begin{minted}[xleftmargin=1em]{haskell}
type Key = String
type Dictionary = [(Key, TypeRep)]
\end{minted}

\footnotetext{\mintinline{haskell}{TypeRep} supports term-level representations
 of datatypes, available in \mintinline{haskell}{Data.Typeable}}

To encode this in the type level, everything has to be
 \emph{promoted}\cite{promotion} one level up.
 From terms to types, and from types to kinds.

Luckily, with recently added GHC extension \emph{data kinds}, suitable
 datatype will be automatically promoted to be a kind, and its value
 constructors to be type constructors.

\begin{minted}[xleftmargin=1em]{haskell}
data List a = Nil | Cons a (List a)
\end{minted}

Give rise to the following kinds and type constructors:\footnotemark\footnotemark

\footnotetext{To distinguish between types and promoted constructors that have
 ambiguous names, prefix promoted constructor with a single quote like
 \mintinline{text}{'Nil} and \mintinline{text}{'Cons}}
\footnotetext{All kinds have \emph{sort} BOX in Haskell\cite{sorts}}

\begin{minted}[xleftmargin=1em]{haskell}
List k :: BOX
Nil  :: List k
Cons :: k -> List k -> List k
\end{minted}

Haskell sugars lists \mintinline{haskell}{[1, 2, 3]} and tuples
 \mintinline{text}{(1, 'a')} with brackets and parentheses.
 We could also express promoted lists and tuples in types like this with
 a single quote prefixed. For example:
 \mintinline{text}{'}\mintinline{haskell}{[Int, Char]},
 \mintinline{text}{'}\mintinline{haskell}{(Int, Char)}.

\subsection{Type-level literals}

Now we have type-level lists and tuples to construct the dictionary.
For keys, \mintinline{haskell}{String} also has a type-level correspondence:
\mintinline{haskell}{Symbol}.

\begin{minted}[xleftmargin=1em]{haskell}
data Symbol
\end{minted}

Symbol is defined without a value constructor, because it's intended to be used
 as a promoted kind.

\begin{minted}[xleftmargin=1em]{haskell}
"this is a type-level string literal" :: Symbol
\end{minted}
%
% Nonetheless, it's still useful to have a term-level value that links with a
%  Symbol, when we want to retrieve type-level information at runtime (but not the
%  other way around!).

\subsection{Putting everything together}

With all of these ingredients ready, let's build some dictionaries!

\begin{minted}[xleftmargin=1em]{haskell}
type DictEmpty = '[]
type Dict0 = '[ '("key", Bool) ]
type Dict1 = '[ '("A", Int), '("B", "A") ]
\end{minted}

These dictionaries are defined with \emph{type synonym}, since they are
 \emph{types}, not \emph{terms}. If we ask \mintinline{text}{GHCi} what is the
 kind of \mintinline{haskell}{Dict1}, we will get
 \mintinline{haskell}{Dict1 :: [ (Symbol, *) ]}

The kind \mintinline{haskell}{*} (pronounced "star") stands for the set of all
 concrete type expressions, such as \mintinline{haskell}{Int},
 \mintinline{haskell}{Char} or even a symbol \mintinline{haskell}{"symbol"},
 while \mintinline{haskell}{Symbol} is restricted to all symbols only.

\section{Indexed monads}

Redis commands are \emph{actions}.
 We could capture the effects caused by an action, by expressing the states it
 affects, before and after. That is, the \emph{preconditions} and
 \emph{postconditions} of an action. In such way, we could also impose
 constraints on the preconditions.

\emph{Indexed monads} (or \emph{monadish},
 \emph{parameterised monad})\cite{indexedmonad}
 can be used\cite{typefun}\cite{staticresources} to model such preconditions and
 postconditions in types. An indexed monad is a type constructor that takes
 three arguments: an initial state, a final state, and the type of a value that
 an action computes, which can be read like a Hoare triple\cite{kleisli}.

\begin{minted}[xleftmargin=1em]{haskell}
class IMonad m where
    unit :: a -> m p p a
    bind :: m p q a -> (a -> m q r b) -> m p r b
\end{minted}

Class \mintinline{haskell}{IMonad} comes with two operations:
 \mintinline{haskell}{unit} for identities and \mintinline{haskell}{bind} for
 compositions, as in monads.

We define a new datatype \mintinline{haskell}{Edis}, which is basically just
 the context \mintinline{haskell}{Redis} indexed with more information in types.
 We make it an instance of \mintinline{haskell}{IMonad}.

\begin{minted}[xleftmargin=1em]{haskell}
newtype Edis p q a = Edis { unEdis :: Redis a }

instance IMonad Edis where
    unit = Edis . return
    bind m f = Edis (unEdis m >>= unEdis . f )
\end{minted}

The first and second argument of type \mintinline{haskell}{Edis} is where the
dictionaries going to stay.

To execute a \mintinline{haskell}{Edis} program, simply apply it to
 \mintinline{haskell}{unEdis} to get an ordinary program of type
 \mintinline{haskell}{Redis}, with type information erased.

\subsection{\mintinline{text}{PING}: the first attempt}

In Redis, \mintinline{text}{PING} does nothing but replies with
 \mintinline{text}{PONG} if the connection is alive. In Hedis,
 \mintinline{text}{PING} has type:

\begin{minted}[xleftmargin=1em]{haskell}
ping :: Redis (Either Reply Status)
\end{minted}

Now we have \mintinline{haskell}{Edis}, let's make our own version of
\mintinline{haskell}{ping}\footnotemark

\footnotetext{\mintinline{haskell}{ping} from Hedis is qualified with
\mintinline{haskell}{Hedis} to prevent function name clashing in our code.}

\begin{minted}[xleftmargin=1em]{haskell}
ping :: Edis xs xs (Either Reply Status)
ping = Edis Hedis.ping
\end{minted}

The dictionary \mintinline{haskell}{xs} in the type remains unaffected after the
 action, because \mintinline{haskell}{ping} does not affect any key-value
 bindings. To encode other commands that modifies key-value bindings, we need
 type-level functions to annotate those effects on the dictionary.

\section{Type-level functions}
\subsection{Closed type families}

Type families have a wide variety of applications. They can appear inside type
 classes\cite{tfclass}\cite{tfsynonym}, or at toplevel. Toplevel type families
 can be used to compute over types, they come in two forms: open\cite{tfopen}
 and closed \cite{tfclosed}.

We choose \emph{closed type families}, because it allows overlapping instances,
 and we need none of the extensibility provided by open type families.
For example, consider both term-level and type-level \mintinline{haskell}{&&}:

\begin{minted}[xleftmargin=1em,fontsize=\footnotesize]{haskell}
(&&) :: Bool -> Bool -> Bool
True && True = True
a    && b    = False

type family And (a :: Bool) (b :: Bool) :: Bool where
    And True True = True
    And a    b    = False
\end{minted}

The first instance of \mintinline{haskell}{And} could be subsumed under a more
 general instance, \mintinline{haskell}{And a b}.
But the closedness allows these instances to be resolved in order, just like
 how cases are resolved in term-level functions. Also notice that how much
 \mintinline{haskell}{And} resembles to it's term-level sibling.

\subsection{Functions on type-level dictionaries}

With closed type families, we could define functions on the type level.
 Let's begin with dictionary lookup.

\begin{minted}[xleftmargin=1em]{haskell}
type family Get
    (xs :: [(Symbol, *)])   -- dictionary
    (s :: Symbol)           -- key
    :: * where              -- type

    Get (’(s, x) ’: xs) s = x
    Get (’(t, x) ’: xs) s = Get xs s
\end{minted}

Another benefit of closed type families is that type-level equality can be
expressed by unifying type variables with the same name.
\mintinline{haskell}{Get} takes two type arguments, a dictionary and a symbol.
If the key we are looking for unifies with the symbol of an entry, then
 \mintinline{haskell}{Get} returns the corresponding type, else it keeps
 searching down the rest of the dictionary.

\mintinline{haskell}{Get }\mintinline{text}{ '}\mintinline{haskell}{[ }
\mintinline{text}{'}\mintinline{haskell}{("A", Int) ] "A"} evaluates to
\mintinline{haskell}{Int}.

But \mintinline{haskell}{Get }
\mintinline{text}{'}\mintinline{haskell}{[ }
\mintinline{text}{'}\mintinline{haskell}{("A", Int) ] "B"} would get stuck.
That's because \mintinline{haskell}{Get} is a partial function on types,
 and these types are computed at compile-time. It wouldn't make
 much sense for a type checker to crash and throw a "Non-exhaustive" error or
 be non-terminating.

We could make \mintinline{haskell}{Get} total, as we would at the term level,
 with \mintinline{haskell}{Maybe}.

\begin{minted}[xleftmargin=1em]{haskell}
type family Get
    (xs :: [(Symbol, *)])   -- dictionary
    (s :: Symbol)           -- key
    :: Maybe * where        -- type

    Get ’[]             s = Nothing
    Get (’(s, x) ’: xs) s = Just x
    Get (’(t, x) ’: xs) s = Get xs s
\end{minted}
%
Other dictionary-related functions are defined in a similar fashion.

\begin{minted}[xleftmargin=1em]{haskell}
-- inserts or updates an entry
type family Set
    (xs :: [(Symbol, *)])   -- old dictionary
    (s :: Symbol)           -- key
    (x :: *)                -- type
    :: [(Symbol, *)] where  -- new dictionary

    Set ’[]             s x = ’[ ’(s, x) ]
    Set (’(s, y) ’: xs) s x = (’(s, x) ’: xs)
    Set (’(t, y) ’: xs) s x =
        ’(t, y) ’: (Set xs s x)

-- removes an entry
type family Del
    (xs :: [(Symbol, *)])   -- old dictionary
    (s :: Symbol)           -- key
    :: [(Symbol, *)] where  -- new dictionary

    Del ’[] s             = ’[]
    Del (’(s, y) ’: xs) s = xs
    Del (’(t, y) ’: xs) s = ’(t, y) ’: (Del xs s)

-- membership
type family Member
    (xs :: [(Symbol, *)])   -- dictionary
    (s :: Symbol)           -- key
    :: Bool where           -- exists?

    Member ’[]             s = False
    Member (’(s, x) ’: xs) s = True
    Member (’(t, x) ’: xs) s = Member xs s
\end{minted}

\subsection{Proxies and singleton types}

Now we could annotate the effects of a command in types. \mintinline{text}{DEL}
 removes a key from the current database, regardless of its type.

\begin{minted}[xleftmargin=1em]{haskell}
del :: KnownSymbol s
    => Proxy s
    -> Edis xs (Del xs s) (Either Reply Integer)
del key = Edis $ Hedis.del (symbolVal key)
\end{minted}

\mintinline{haskell}{KnownSymbol} is a class that gives the string associated
 with a concrete type-level symbol, which can be retrieved with
 \mintinline{haskell}{symbolVal}.\footnotemark

\footnotetext{They are defined in \mintinline{haskell}{GHC.TypeLits}.}

Since Haskell has a \emph{phase distinction}\cite{phasedistinction}, types are
 erased before runtime. It's impossible to obtain information directly from
 types, we can only do this indirectly, with
 \emph{singleton types}\cite{singletons}.

A singleton type is a type that has only one instance, and the instance can be
 think of as the representative of the type at the realm of runtime values.

\mintinline{haskell}{Proxy}, as its name would suggest, can be used as
 singletons. It's a phantom type that could be indexed with any type.

\begin{minted}[xleftmargin=1em]{haskell}
data Proxy t = Proxy
\end{minted}

In the type of \mintinline{haskell}{del}, the type variable
 \mintinline{haskell}{s} is a \mintinline{haskell}{Symbol} that is decided by
 the argument of type \mintinline{haskell}{Proxy s}.
 To use \mintinline{haskell}{del}, we would have to apply it with a clumsy
 term-level proxy like this:

\begin{minted}[xleftmargin=1em]{haskell}
del (Proxy :: Proxy "A")
\end{minted}

\section{Making Redis polymorphic}

Redis supports many different datatypes, these datatypes as can be viewed as
 \emph{containers} of strings. For example, lists (of strings),
 sets (of strings), and strings themselves.

\subsection{Denoting containers}
Most Redis commands only work with a certain type of these containers. To
 annotate what container a key is associated with, we introduce these container
 types.

\begin{minted}[xleftmargin=1em]{haskell}
data Strings
data Lists
data Sets
...
\end{minted}

\mintinline{text}{SET} stores a string, regardless the datatype the key was
 associated with. Now we could implement \mintinline{text}{SET} like this:

\begin{minted}[xleftmargin=1em,fontsize=\footnotesize]{haskell}
set :: KnownSymbol s
    => Proxy s
    -> ByteString       -- data to store
    -> Edis xs (Set xs s Strings) (Either Reply Status)
set key val = Edis $ Hedis.set (symbolVal key) val
\end{minted}

After \mintinline{text}{SET}, the key will be associated with
 \mintinline{haskell}{Strings} in the dictionary, indicating that it's a string.

\subsection{Automatic data serialization}
But in the real world, raw binary strings are hardly useful, people would
 usually serialize their data into strings before storing them, and deserialize
 them back when in need.

Instead of letting users writing these boilerplates, we will be doing these
 serializations/deserializations for them. With the help from
 \mintinline{text}{cereal}, a binary serialization library.
 \mintinline{text}{cereal} comes with these two functions:

\begin{minted}[xleftmargin=1em]{haskell}
encode :: Serialize a
       => a -> ByteString Source

decode :: Serialize a
       => ByteString -> Either String a
\end{minted}

Which would do all the works for us, as long as the datatype it's handling is
 an instance of class \mintinline{haskell}{Serialize}.\footnotemark

\footnotetext{The methods of \mintinline{haskell}{Serialize} will have default
 generic implementations for all datatypes with some language extensions
 enabled, no sweat!}

\subsection{Extending container types}
We rename container types and extend it with an extra type argument,
 to indicate what kind of encoded value it's holding.

\begin{minted}[xleftmargin=1em]{haskell}
data StringOf x
data ListOf x
data SetOf x
...
\end{minted}

\mintinline{haskell}{set} reimplemented with extended container types:

\begin{minted}[xleftmargin=1em,fontsize=\footnotesize]{haskell}
set :: (KnownSymbol s, Serialize x)
    => Proxy s
    -> x       -- can be anything, as long as it's serializable
    -> Edis xs (Set xs s (StringOf x)) (Either Reply Status)
set key val = Edis $ Hedis.set (symbolVal key) (encode val)
\end{minted}

For example, if we execute \mintinline{haskell}{set (Proxy :: Proxy "A") True},
 a new entry \mintinline{text}{'}\mintinline{haskell}{("A", StringOf Bool)} will
 be inserted to the dictionary.

\subsection{Handling \mintinline{text}{INCR}}

Commands such as \mintinline{text}{INCR} and \mintinline{text}{INCRBYFLOAT}, are
 not only container-specific, they also have some requirements on what types of
 value they could operate on.

We could handle this by mapping Redis's strings of integers and floats to
 Haskell's \mintinline{text}{Integer} and \mintinline{text}{Double}.

\begin{minted}[xleftmargin=1em,fontsize=\footnotesize]{haskell}
incr :: (KnownSymbol s, Get xs s ~ Just (StringOf Integer))
     => Proxy s -> Edis xs xs (Either Reply Integer)
incr key = Edis $ Hedis.incr (symbolVal key)

incrbyfloat :: (KnownSymbol s, Get xs s ~ Just (StringOf Double))
     => Proxy s -> Double -> Edis xs xs (Either Reply Double)
incrbyfloat key n = Edis $ Hedis.incrbyfloat (symbolVal key) n
\end{minted}

\section{Imposing constraints}

To rule out programs with undesired properties, certain constraints must be
 imposed, on what arguments they can take, or what preconditions they must hold.

Consider the following example: \mintinline{text}{LLEN} returns the length of
 the list associated with a key, else raises a type error.

\begin{minted}[xleftmargin=1em]{text}
redis> LPUSH some-list bar
(integer) 1
redis> LLEN some-list
(integer) 1
redis> SET some-string foo
OK
redis> LLEN some-string
(error) WRONGTYPE Operation against a key
holding the wrong kind of value
\end{minted}

Such constraint could be expressed in types with
\emph{equality constraints}\cite{typeeq}.

\begin{minted}[xleftmargin=1em]{haskell}
llen :: (KnownSymbol s, Get xs s ~ Just (ListOf x))
        => Proxy s
        -> Edis xs xs (Either Reply Integer)
llen key = Edis $ Hedis.llen (symbolVal key)
\end{minted}

Where \mintinline{haskell}{(~)} denotes that \mintinline{haskell}{Get xs s}
and \mintinline{haskell}{Just (ListOf x)} needs to be the same.

The semantics of \mintinline{text}{LLEN} defined above is actually not
complete. \mintinline{text}{LLEN} also accepts keys that do not exist, and
 replies with \mintinline{text}{0}.

\begin{minted}[xleftmargin=1em]{text}
redis> LLEN nonexistent
(integer) 0
\end{minted}

In other words, we require that the key to be associated with a list,
 textbf{unless} it doesn't exist at all.

\subsection{Expressing constraint disjunctions}

Unfortunately, expressing disjunctions in constraints is much more difficult
 than expressing conjunctions, since the latter could be easily done by placing
 constraints in a tuple (at the left side of \mintinline{haskell}{=>}).

There are at least three ways to express type-level constraints
\cite{singletons}. Luckily we could express constraint disjunctions with type
 families in a modular way.

The semantics we want could be expressed informally like this:

\mintinline{haskell}{Get xs s} $\equiv$ \mintinline{haskell}{Just (ListOf x)}
$\vee$ $\neg$ \mintinline{haskell}{(Member xs s)}

We could achieve this simply by translating the semantics we want to the
 domain of Boolean, with type-level boolean functions such as
\mintinline{haskell}{(&&)},
\mintinline{haskell}{(||)}, \mintinline{haskell}{Not},
\mintinline{haskell}{(==)}, etc.\footnotemark To avoid

\footnotetext{Available in \mintinline{text}{Data.Type.Bool} and
 \mintinline{text}{Data.Type.Equality}}

\begin{minted}[xleftmargin=1em,fontsize=\footnotesize]{haskell}
Get xs s == Just (ListOf x) || Not (Member xs s)
\end{minted}

To avoid addressing the type of value (as it may not exist at all), we defined
 an auxiliary predicate \mintinline{haskell}{IsList :: Maybe * -> Bool} to
 replace the former part.

\begin{minted}[xleftmargin=1em,fontsize=\footnotesize]{haskell}
IsList (Get xs s) || Not (Member xs s)
\end{minted}

The type expression above has kind \mintinline{haskell}{Bool}, we could make it
 a type constraint by asserting equality.

\begin{minted}[xleftmargin=1em,fontsize=\footnotesize]{haskell}
(IsList (Get xs s) || Not (Member xs s)) ~ True
\end{minted}

With \emph{constraint kind}, a recent addition to GHC. Type constraints now has
 its own kind: \mintinline{haskell}{Constraint}. That means type constraints
 are not restricted to the left side of a \mintinline{haskell}{=>} anymore,
 they could appear in anywhere that accepts something of kind
 \mintinline{haskell}{Constraint}, and any type that has kind
 \mintinline{haskell}{Constraint} can also be used as a type constraint.
 \footnote{See \url{https://downloads.haskell.org/~ghc/7.4.1/docs/html/users_guide/constraint-kind.html}.}

As many other list-related commands also have this "List or Nothing" semantics,
 we could abstract the lengthy type constraint above and give it an alias with
 type synonym.

\begin{minted}[xleftmargin=1em,fontsize=\footnotesize]{haskell}
ListOrNX xs s =
    (IsList (Get xs s) || Not (Member xs s)) ~ True
\end{minted}

The complete implementation of \mintinline{text}{LLEN} with
\mintinline{haskell}{ListOrNX} would become:

\begin{minted}[xleftmargin=1em]{haskell}
llen :: (KnownSymbol s, ListOrNX xs s)
        => Proxy s
        -> Edis xs xs (Either Reply Integer)
llen key = Edis $ Hedis.llen (symbolVal key)
\end{minted}

\section{Assertions}

Users may need to make assertions about the status of some key-type bindings in
 a Redis program. For example, declaring the existence of a key and the type
 of its associating value. We provide these functions, which do nothing but
 fiddling with types.

\begin{minted}[xleftmargin=1em]{haskell}
declare :: (KnownSymbol s, Member xs s ~ False)
        => Proxy s
        -> Proxy x  -- type of value
        -> Edis xs (Set xs s x) ()
declare s x = Edis $ return ()

renounce :: (KnownSymbol s, Member xs s ~ True)
        => Proxy s -> Edis xs (Del xs s) ()
renounce s = Edis $ return ()

-- empty precondition
start :: Edis ’[] ’[] ()
start = Edis $ return ()
\end{minted}

\subsection{A complete example}

The following program increases the value of "A" as an integer, push the result
 of the increment to list "L", and then pops it out.

\begin{minted}[xleftmargin=1em]{haskell}
main :: IO ()
main = do
    conn <- connect defaultConnectInfo
    result <- runRedis conn $ unEdis $ start
        `bind` \_ ->  declare
                        (Proxy :: Proxy "A")
                        (Proxy :: Proxy Integer)
        `bind` \_ ->  incr     (Proxy :: Proxy "A")
        `bind` \n ->  case n of
            Left  err -> lpush (Proxy :: Proxy "L") 0
            Right n   -> lpush (Proxy :: Proxy "L") n
        `bind` \_ ->  lpop     (Proxy :: Proxy "L")
    print result
\end{minted}

The syntax is pretty heavy, like the old days when there's no
 \emph{do-notation}\cite{history}. But if we don't need any variable bindings
 between operations, we could compose these commands with a sequencing operator
 \mintinline{haskell}{(>>>)}.

\begin{minted}[xleftmargin=1em]{haskell}
(>>>) :: IMonad m => m p q a -> m q r b -> m p r b

program = start
    >>> declare
            (Proxy :: Proxy "A")
            (Proxy :: Proxy Integer)
    >>> incr    (Proxy :: Proxy "A")
    >>> lpush   (Proxy :: Proxy "L") 0
    >>> lpop    (Proxy :: Proxy "L")
\end{minted}

\section{Discussions}
\paragraph{Syntax}
\paragraph{Commands with multiple inputs or outputs}
\paragraph{Redis Transactions}
\section{Conclusion and Related Work}
\paragraph{Acknowledgements}


% references
\bibliographystyle{abbrvnat}
\bibliography{cites}

\end{document}
