%
% LaTeX template for prepartion of submissions to PLDI'16
%
% Requires temporary version of sigplanconf style file provided on
% PLDI'16 web site.
%
\documentclass[pldi]{sigplanconf-pldi16}
% \documentclass[pldi-cameraready]{sigplanconf-pldi16}

%
% the following standard packages may be helpful, but are not required
%
\usepackage{SIunits}            % typset units correctly
\usepackage{courier}            % standard fixed width font
\usepackage[scaled]{helvet}     % see www.ctan.org/get/macros/latex/required/psnfss/psnfss2e.pdf
\usepackage{url}                % format URLs
\usepackage{listings}           % format code
\usepackage{enumitem}           % adjust spacing in enums
\usepackage[colorlinks=true,allcolors=blue,breaklinks,draft=false]{hyperref}   % hyperlinks, including DOIs and URLs in bibliography
% known bug: http://tex.stackexchange.com/questions/1522/pdfendlink-ended-up-in-different-nesting-level-than-pdfstartlink
\newcommand{\doi}[1]{doi:~\href{http://dx.doi.org/#1}{\Hurl{#1}}}   % print a hyperlinked DOI

% \usepackage[utf8]{inputenc}
% \usepackage[english]{babel}
\usepackage{minted}     % code highlighting
\usemintedstyle{tango}

\begin{document}

\title{Catching runtime errors of Redis at compile time}

%
% any author declaration will be ignored  when using 'pldi' option (for double blind review)
%

\authorinfo{Person 1 \and Person 2}
{\makebox{A Department} \\
\makebox{A University}  \\
\makebox{A Place, AS 12345}}
{\{person1,person2\}@cs.auniv.edu}

\maketitle

\begin{abstract}
We often interacts with database systems through dedicated libraries or packages
implemented in some programming languages. People make mistakes, while some of
the syntactic and semantic errors can be prevented by the language at compile
time, or caught by the package at runtime, most semantic errors are just
being ignored, causing problems at the database system.

In this paper, we demonstrates how to prevent those runtime errors at compile
time, and allow users to write reliable database queries without runtime overhead,
by exploiting type-level programming techniques such as parametedzied monad,
type-level literals and type families in Haskell.

The database system and the package we are targeting
are \emph{Redis} and \emph{Hedis} respectively, and our implementation is
available as \emph{Edis} on \emph{Hackage}.

\paragraph{Categories and Subject Descriptors}
\paragraph{Keywords}
type-level programming; Haskell; database query language; Redis;
\end{abstract}

\section{Introduction}

\subsection{Redis}

Redis is an open source, in-memory data structure store, used as
database, cache and message broker. Each value is associated with a binary-safe
string key to identify and manipulate with. Redis supports many different
kind of values, such as strings, hashes, lists and sets.


To manipulate these values, Redis comes with a set of atomic operations,
called \emph{commands}.

For example, if we want to add a bunch of strings to 2 sets, and intersect them,
we could input the following sequence of commands into Redis's client interactively.

\begin{minted}{text}
redis> SADD some-set a b c
(integer) 3
redis> SADD another-set a b
(integer) 2
redis> SINTER some-set another-set
1) "a"
2) "b"
\end{minted}

Keys such as \mintinline{text}{some-set} and \mintinline{text}{another-set}
are created on site, with the command \mintinline{text}{SADD}, which returns
the size of set after completed.

\subsection{Hedis}
Redis can also be used in most programming languages, with 3rd party libraries
or packages that talk with Redis' TCP protocol. In Haskell, the most popular
package is \emph{Hedis}.

The previous example in Redis's client would look something like this with \emph{Hedis}
in Haskell.
\begin{minted}{haskell}
program :: Redis (Either Reply [ByteString])
program = do
    sadd "some-set" ["a", "b"]
    sadd "another-set" ["a", "b", "C"]
    sinter ["some-set", "another-set"]
\end{minted}

Function \mintinline{haskell}{sadd} takes a key and a list of values as arguments
, and returns an \mintinline{haskell}{Integer}, wrapped in
\mintinline{haskell}{Either} to indicate possible failures in
the context \mintinline{haskell}{Redis} of command execution.

\begin{minted}{haskell}
sadd :: ByteString      -- key
    -> [ByteString]     -- values
    -> Redis (Either Reply Integer)
\end{minted}

Keys and values are of type \mintinline{haskell}{ByteString}, since they are all
just binary strings in Redis. That means if a user wants to store values of
arbitrary types, he or she will have to encode and decode them as
 \mintinline{haskell}{ByteString}s.



\end{document}
